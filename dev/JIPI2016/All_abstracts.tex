\documentclass[a4paper]{article}
\usepackage{a4wide}
\usepackage{type1cm}
\usepackage{graphicx}
\usepackage{textcomp}
\usepackage[USenglish]{datetime}
\usepackage[utf8]{inputenc}
\usepackage{color}
\definecolor{gray}{rgb}{0.5,0.5,0.5}
\definecolor{lgray}{rgb}{0.75,0.75,0.75}
\makeatletter
\renewcommand{\maketitle}{%
	\thispagestyle{empty}
	\includegraphics[height=25mm]{logo.png}
	\hfill
	{\fontsize{1.6cm}{1.6cm}\selectfont \sffamily 
	 \textcolor{lgray}{4{{\fontsize{1.2cm}{1.2cm}\selectfont $^\textsf{th}$}}}
	 \textcolor{gray}{JIPI}
	 }
	\rule{\textwidth}{0.1mm} \\
	\begin{flushright}
	{\LARGE {\sffamily \emph{\textcolor{gray}{Jornada d'Investigadors Predoctorals Interdisciplin\`{a}ria}}}} 
	\end{flushright} 
	\vspace{-8pt}
	{\sffamily \emph{\textcolor{gray}{Barcelona, \today.}}}
	\vskip 2em
	\begin{center} {\sffamily {\huge \@title}} \end{center}
	}
\newenvironment{abstrct}
	{\noindent\begin{minipage}{\textwidth}\setlength\parindent{10pt}}
	{\end{minipage}\vskip 2em}

\begin{document}

\title{Collected abstracts}

\maketitle

\vfill\noindent\textbf{Stats}\newline
 
\begin{tabular}{lclc} 

\multicolumn{4}{l}{Total participants: 201} \\ 

\multicolumn{4}{l}{Total registered for dinner: 67 (33.33\%)} \\ 

\multicolumn{4}{l}{Total registered as participants: 104 } \\ 

\multicolumn{4}{l}{Total abstracts for flashtalks: 63 } \\ 

\multicolumn{4}{l}{Total abstracts for posters: 19 } \\ 

Duplicated participants deleted: & 3 &\quad Incomplete participants deleted: & 5 \\ 

Duplicated flashtalks deleted: & 3 &\quad Incomplete flashtalks deleted: & 3 \\ 

Duplicated posters deleted:    & 1 &\quad Incomplete posters deleted:    & 3 

\end{tabular}
\vfill\vfill


\newpage\section{Flashtalks}

\begin{abstrct}
    \begin{center}\textbf{\subsection{Unconventional behaviour of the anticancer drug Dasatinib. A vision from a computational and experimental point of view.}}

carlos heras

\emph{UB}
\end{center}

\textbf{Keywords: }Anticancer, dirradical behaviour..

In this work we show a new way by which purely organic molecules can act that is not known so far, in this case the powerful second generation anticancer drug Dasatinib. It is possible to differentiate the traditional electronic state, in which all electrons are paired, and the diradical state, which is thermally reached without irradiating. What we propose is an accessible highly reactive diradical state of this molecule and its reactive implications.  First, new experimental evidences claim this new behavior. Also, we have made a theoretical and computational analysis of this very reactive state and its stability. Finally, we present a view of global processes and crystal structures in the involved mechanisms. In an overview of the whole process, we compare the general structure of Dasatinb with other drugs and extrapolate its new phenomenon to other methabolic processes.
\end{abstrct}

\begin{abstrct}
\begin{center}\textbf{The difusse key of the informational city. The innovative milieu. Aproximation to 22@Barcelona}

Sonia Cueva Ortiz

\emph{Polytechnic University of Catalonia UPC}
\end{center}

\textbf{Keywords: }innovative milieu, urban planification, informational city.

In the building of the so-called informational city, there is a willingness to channel the social interaction towards innovation. It says that the exchange of information, make easy and promotes creativity, innovation.



In fact the information has come to be seen as the raw material to transform to thereby achieve a final product, a  product of value, called knowledge, this being the most valuable product of the new era. Hence the economy that aims to move from the industrial age to the informational, worry about creating the right environment in which innovation occurs with maximum fluidity and frequency, that has been called the innovative milieu. Under the characteristics of places considered innovatives, now they try to build innovative milieux , contained in the urban plans, under the title refers to a new kind of town: informational city, knowledge city, technopolis, and others. In the case of Barcelona, under the concept of digital city (MPGM 2000).



But when it channel the interaction towards innovation it could be invading or removing really spontaneous interaction spaces , where all actors are on equal rights. The question that tries to reveal is: Is social inclusion warranted when promote interaction in the areas of the so-called digital city? revealing why: If the informational city values the spontaneous interaction between different actors, why is it indifferent to the use of public space? What kind of interactions looks for? What are the actors that promote interaction: what is their role and their interests? What is the role of society in those areas of interaction? What changes does the innovative milieu over the city? Really it demand changes?.  Questions it will solve under two entries interest of the city: political / social and physical / spatial.
\end{abstrct}

\begin{abstrct}
\begin{center}\textbf{Shallow groundwater dynamics during storm events (Vallcebre research catchment)}

Maria Roig Planasdemunt

\emph{PHD student}
\end{center}

\textbf{Keywords: }Shallow groundwater, water table, runoff, Mediterranean, Vallcebre.

This work was performed with the colaboration of Jérôme Latron and Pilar Llorens (Institute of Environmental Assessment and Water Research).

Hydrologist recognize the important role of groundwater into the hydrological catchment functioning. Knowledge of the spatial-temporal variability of depth to water table, and of the factors controlling the variability, is required in order to improve the hydrological understanding. Despite the amount of studies about of groundwater dynamics, there are few focused at event scale and in Mediterranean areas. These areas, considered as one of the most vulnerable areas to global change, have received few attentions. Water resources of Mediterranean regions mainly depend on runoff generated in mountain areas. Therefore, the study of the depth to water table dynamics in Mediterranean mountains is important for water resources management as it has implications on runoff generation.

With the aim of improving the knowledge of the hydrological functioning of Mediterranean mountain areas, this work investigates the spatial and temporal dynamics of the depth to water table during rainfall-runoff events in the Vallcebre Research Catchments (NE Spain, 42$^o$ 12’N, 1$^o$ 49’E). In combination with rainfall and runoff measurements, the depth to the water table was monitored at 13 locations within the Can Vila catchment (0.56 km2) during 19 rainfall-runoff events. The distribution of piezometers in the catchment allows examining the effect of topography and distance from the stream on the spatial and temporal distribution of depth to water table. On the other hand, the analysis of different rainfall-runoff events allows investigating the role of antecedent wetness conditions on the shallow groundwater dynamics associated to the streamflow response.

Results show that the depth to water table did not rise in unison throughout the catchment during rainfall-runoff events. The shallow groundwater response was clearly different between lo
\end{abstrct}

\begin{abstrct}
\begin{center}\textbf{Improving cognitive and non-cognitive abilities:  Impact evaluation of the AGE program in Mexico}

Jorge Cimentada

\emph{PhD Student}
\end{center}

\textbf{Keywords: }School-based management, impact evaluation, decentralization, educational policy.

This article evaluates a World Bank policy named AGE that aims to involve parents in the decision making process of schools. Using data from an experimental design, we explore if parents involvement in the schools decision making in four rural provinces in Mexico improved math test scores and survey response rates, a new and innovative measure of conscientiousness. With the 2007 baseline survey, which contained 8,723 students, and the 2009 survey, which contained 7,311 respondents, we investigated differences between treated and control students, indigenous and non-indigenous students and indigenous students in the treated and control schools. We find that the application of the treatment has no significant impact in both our measures of interest. We do find that indigenous students have higher response rates than in the baseline survey. This, however, is not due to the treatment of interest. This is likely to be related to the inclusion and acceptance of this marginalized group into society.
\end{abstrct}

\begin{abstrct}
\begin{center}\textbf{Wondering materials return to 2 dimensions}

Arevik Musheghyan; Stefanos Chaitoglou

\emph{PhD student}
\end{center}

\textbf{Keywords: }graphene, CVD synthesis, electrical properties, applications.

Single layer or 2D materials are crystalline materials consisting of a single layer of atoms. Graphene  a single layer of graphite, is the 'superstar' in this category as it was the first of these materials to be discovered and isolated, resulted in the nomination of the Nobel in Physics of 2010 to the people who contributed to this work. Graphene is an allotrope of carbon in the form of a two-dimensional structure. Carbon is the second most abundant mass within the human body and the fourth most abundant element in the universe (by mass), after hydrogen, helium and oxygen, so therefore graphene could well be an ecologically friendly, sustainable solution for an almost limitless number of applications. It is one of the thinnest compound known to man, one atom thick, the lightest material known, the strongest compound discovered, the best conductor of heat at room temperature and also the best conductor of electricity . Today graphene is the number one topic considering publications and founding in European Union. In our talk we will present the outstanding properties, considering mechanical strength, heat and electrical conductivity and transparency, and the great potential use in applications, in electronics, biological engineering, filtration, lightweight/strong composite materials, photovoltaics and energy storage. Between the different approaches to obtain graphene, the bottom up methods and in specific the chemical vapor deposition growth is the one which permits fast and large scale growth. We present the basics behind the method and the motivation of our research focusing in alternative paths to rule the graphene growth.
\end{abstrct}

\begin{abstrct}
\begin{center}\textbf{Teacher Education from a Gender Perspective in Catalonia and Finland}

Laia Arias Solé

\emph{Universitat de Barcelona}
\end{center}

\textbf{Keywords: }teacher education; sexism; gender equality; gender perspective; higher education..

The present doctoral thesis is based on the research into the teacher education regarding the degrees in Primary Education from a gender perspective in a Catalan university and a Finnish one, and the way these universities promote the gender equality. According to several international indexes, Finland is one of the best countries in the world when it comes to education and gender equality, although sexism and gender inequalities can be found in both Catalan and Finnish societies. In this research it is considered that the introduction and implementation of the gender perspective within an education system can be a way to eradicate the gender inequalities and the gender-based discrimination in a society. Taking into consideration the two universities where this research is being conducted, the present thesis has two main aims: firstly, to know if there is sexism within the primary teacher education and, if so, to analyse how it is, and secondly, to know the practices that are carried out from a gender perspective in both universities. Therefore, this research analyses how the primary teacher education studies are from a gender perspective, what it is done by the universities to promote the gender equality, and also if the education systems of Catalonia and Finland can be improved in these matters. A case study was chosen as a research method, and two methodological techniques are being used: on the one hand, semi-structured interviews to students and university teachers, lecturers and professors of the faculties of Education of the universities chosen; on the other hand, an analysis of documents that might be relevant within the teacher education and gender equality fields, such as the syllabus within teacher education programmes or the plans of equality of the universities studied.
\end{abstrct}

\begin{abstrct}
\begin{center}\textbf{Saltscapes and salt heritage as a tool for conservation, education and local development}

Katia Hueso Kortekaas

\emph{IPAISAL}
\end{center}

\textbf{Keywords: }heritage, conservation, local development, cultural landscape, salinas.

Salinas are cultural landscapes with distinct features associated to the presence of salt in them. The salt making activity has also created a diverse cultural tangible and intangible heritage that needs to be preserved for future generations. The widespread abandonment of the traditional salt making activity in Europe over the last decades, especially in isolated rural areas, risks that this heritage is completely lost. However, in some locations, the artisanal salt production has been recovered and now constitutes a pole for socioeconomic development at local or even regional scale. In this research, I will try to gain insight into the process of recovery in four different sites in Europe (Salinas de Añana in Spain, Sečovlje, in Slovenia, Guérande in France and Læsø in Denmark) and how salt making at artisanal scale can contribute to the generation of employment opportunities by using a broad range of products and services a saltscape can offer. In addition, I will try to deepen into how these processes contribute to the conservation of the cultural landscape, the heritage it hosts, and to the awareness raising and knowledge of these assets by the general public. The ultimate aim of this work is to understand the general patterns of successful recovery, so that other saltscapes and similar cultural landscapes may benefit from their experience.
\end{abstrct}

\begin{abstrct}
\begin{center}\textbf{Contributions to micro climate comfort of the tree species Platanus x hispanica}

Gilkauris María Rojas Cortorreal

\emph{Phd Student}
\end{center}

\textbf{Keywords: }Thermal Comfort, Urban vegetation, Surface Radiant temperature, Mediterranean climate, Vegetation.

Vegetation is one of the tools that allows the mitigation of Urban Heat Island, previous studies have analyzed how the vegetation improves these urban areas. Therefore this study has focused on the evaluation of five (5) urban tree species of different densities and higher urban use in Barcelona. These species are Parkinsonia, Schinus molle, Platanus x hispanica, Celtis australis and Quercus ilex. In this instance, we will discuss the tree species Platanus x hispanica.



The objective of this research is to examine quantitatively and qualitatively how these tree species improve the urban microclimate through their morphological characteristics.



The methodology used is divided into three stages. The first stage was the selection of measurement points, the indicators being the value and use of urban pedestrian and vehicular routes and the types of tree species used. The second stage was performing in situ measurements, the indicators being thermal comfort parameters, the parameters of the tree species and collecting weather data. The third stage was analyzing the results and drawing conclusions.



The quantitative results obtained in this research were the values of the surface radiant Temperature of the soil, a difference of up to 7.9 $^o$C was recorded between the surface radiant temperature of the soil under shadow and the radiant surface temperature of the soil outside the shadow of a tree. This protection also occurs on the facades of the buildings which are in immediate exchange with this species, achieving less contributions to the environment in both the exterior and interior of the building.



Measuredly, less contributions to the radiant temperature of the environment and less radiation exchange with the human body is achieved. This is reflected in the comfort levels of each person when navigating these urban areas. 



The main tool for the protection of these urban areas and to mitigate the Urban Heat Island is vegetation.
\end{abstrct}

\begin{abstrct}
\begin{center}\textbf{Under pressure? Don’t lose direction! Smart Sensors: Development of CMOS and MEMS on the same platform}

Josep Maria Sánchez Chiva; Saoni Banerji

\emph{Universitat Politècnica de Catalunya, departament d'Enginyeria Electrònica}
\end{center}

\textbf{Keywords: }MEMS, CMOS-MEMS, resonant sensor, pressure sensor, magnetometer, electronic compass.

Micro Electro Mechanical Systems (MEMS) are movable structures fabricated on the surface of silicon chips by selectively depositing and etching away materials and silicon. Such movement allows the measurement of a wide range of parameters. In the recent years, the integration of MEMS devices and electronics on the same chip i.e. CMOS-MEMS integration, has allowed the improvement of sensors’ performance and fabrication costs, extending their integration in all types of devices. In the Advanced Hardware Architectures research group at the UPC, we are currently conducting research on system-on-chip CMOS-MEMS magnetic field and pressure sensors.

 

Contrary to pressure sensors using other technologies, resonant pressure sensors allow direct coupling to digital electronics without requiring analog to digital converters (ADCs). This feature enhances their resolution and reliability by providing more immunity to noise and interference. Recently, monolithically integrated CMOS-MEMS resonant pressure sensors have been extensively used in atmospheric pressure monitoring and altitude sensing due to their low cost, small size and high reliability. Presently, the pressure sensors integrated in the smartphones and wearable devices suffer from poor sensitivity. The primary purpose of our study is to develop an optimized CMOS-MEMS resonant pressure sensor with enhanced sensitivity at atmospheric pressure which can be utilized in a vertical GPS enhancement system.

 

Magnetic field sensors are key in the development of electronic compasses integrated in smartphones and other devices. Currently, magnetic sensors used in smartphones do not use MEMS technology but sensors that require materials incompatible with the fabrication of CMOS electronics. Our objective is to improve the performance of MEMS based electronic compasses by using Lorentz force based MEMS magnetic sensors. Such sensors, are compatible with CMOS process and could substitute actual sensors by reducing fabrication cost.
\end{abstrct}

\begin{abstrct}
\begin{center}\textbf{THE TRANSFORMATION OF THE SPANISH TELEVISION INDUSTRY: TECHNOLOGICAL TRANSITION, DIGITAL REGULATION AND REDEFINITION OF THE AUDIOVISUAL MARKET (2010-2016)}

Marta Albújar Villarrubia

\emph{Universitat Autònoma de Barcelona}
\end{center}

\textbf{Keywords: }TELEVISION, AUDIOVISUAL MARKET, DTT, COMMUNICATION, SPAIN.

The implementation of Digital Terrestrial Television in Spain has allowed free access to a higher number of TV channels. Technological development and the digitalisation of the transmissions have brought new ways of getting content and the arrival of new players.

This research project focuses on the transformation that the Spanish television industry has undergone, since the analogic switch-off in 2010 to the spectrum rearranging in 2015.

Since the switch-off, the television market has experienced multiple changes on both sides of the screen. On the one hand, there have been decisive changes on the supply side. On the other hand, there have been a reorganization of private and public TV companies; for instance, mergers and acquisitions among the major players.

The influence of Web 2.0 has benefited the emergence of new forms of TV distribution, normally mediated by a broadband connection. However, it has also allowed the arrival of new broadcasters that complement the traditional ones. Moreover, the implementation of the Digital Divide has forced a reallocation of frequencies among the spectrum. 

The new devices for television multiply the means of access to TV content. Consequently, the companies producing devices have become major actors of the audiovisual panorama: connected and mobile TV have risen, allowing consumers to have a greater flexibility of access and choice. 

Finally, national and European public agencies have implemented multiple policies in the field of communication in order to support digitalisation and convergence. The aim is to allow the entrance of new consumption patterns and to enforce the common market in the frame of “Digital Agenda for Europe (2020 Strategy)” of the European Union.
\end{abstrct}

\begin{abstrct}
\begin{center}\textbf{Moving Beyond Graphene, Without Graphene}

Isaac Alcón Rovira

\emph{Universitat de Barcelona}
\end{center}

\textbf{Keywords: }2D-COFs, Graphene, applicability, Chemistry.

Since its discovery in 2004, Graphene has drawn the attention of very important areas within Science and Technology, becoming the most popular material of the world and allowing its finders to get the Nobel Price in 2010. However, this 2D material presents important drawbacks towards its applicability, which mainly reside on its inherent instability that makes it so difficult to utilize its amazing properties at realistic conditions. Very recently a new field in Chemistry has emerged: the so-called 2D covalent organic frameworks (or 2D-COFs). Within this field, 2D materials are constructed by the self-assembly of molecular building blocks which react in an ordered and controlled manner, giving rise to mono- and multi-layer networks with very specific properties. Due to the huge versatility of Organic Synthesis which has being growing up for hundreds of years, any type of molecule can be designed and prepared in the lab ready to be used to construct a certain 2D-COF which, in turn, will possess the most basic physic-chemical characteristics of the former building blocks. As time passes, without rising the so amazing properties of Graphene that keep physicists so fascinated (and busy), I strongly believe 2D-COFs will provide a much higher malleability and robustness that will open new possibilities much beyond of what Graphene can offer nowadays, without being Graphene at all. Time will say.
\end{abstrct}

\begin{abstrct}
\begin{center}\textbf{Checking the two water worlds hypothesis in a Mediterranean catchment}

Carles Cayuela Linares

\emph{Institute of Environmental Assessment and Water Research (IDAEA-CSIC)}
\end{center}

\textbf{Keywords: }ecohydrology, water stable isotopes.

Catchment scale hydrological models usually assume a complete mixing of subsurface water. Such assumption means that when water from precipitation infiltrates in the soil, it displaces the stored subsurface that is “pushed” and flows downslope to the stream; this mechanism is commonly known as translatory flow. Within this framework, trees would extract water similar to the one that enters the stream. Recent research using water-isotopes ($\delta$18O and $\delta$2H) has however questioned this assumption of a complete mixing of subsurface water. Although more research is needed, there are evidences that in seasonal environments, water from the firsts rainfall events following dry periods may remain locked into small pores at low matric potentials. This water might be used by trees, but would not participate to streamflow generation because it is not mixed with the mobile water, hence questioning the pertinence of the translatory flow concept during the transition from dry to wet conditions.  Following this theory, soil water should be in fact separated in two “water worlds”: mobile water and tightly bound water.

This work, which feeds on previous research conducted on the Vallcebre research catchment, aims to verify the existence of these two water worlds by analysing water-isotopes of rainfall water, throughfall water, stream water, tightly bound soil water, mobile soil water and xylem water, in combination with continuous measurement of  meteorological and hydrological variables. The sampling design consists in three spatially distributed intensive water sampling campaigns, between the dry-wet periods, combined with a regular sampling every 15 days at one location.

Results of the first spatially intensive sampling and of the first regular samplings will be soon available. They should be helpful to better understand the catchment’s hydrologic dynamics and to give some new insights to verify the existence of two water worlds in Mediterranean conditions.
\end{abstrct}

\begin{abstrct}
\begin{center}\textbf{Cross-fertilization of Key Enabling Technologies: Nanotechnologies in healthcare}

Cristina Paez Aviles

\emph{University of Barcelona}
\end{center}

\textbf{Keywords: }KETs, Nanotechnology, cross-fertilization, helthcare, innovation, H2020.

Cross-fertilization of Key Enabling Technologies (KETs) enables the production of new product properties and technology features which are important to achieve social impact and technological progress. At the healthcare domain, crosscutting KETs could improve the overall performance of the technological and biomedical systems given that the convergence of technologies in Nanotechnology, Biotechnology, Micro/Nano-electronics and Advanced Materials allow the development of new medical devices of small dimensions as well as new diagnostic and therapy approaches. The necessity of cross-KETs is increasingly growing especially when this is boosted by initiatives such as Horizon 2020. Given this scenario, this study analyses H2020’s projects taken from the European Union Open Data Portal database. The aim is to identify how different actors (universities, SMEs, foundations and research organizations) are crossing, connecting and joint efforts for developing new technologies that incorporates Nanotechnologies for improving healthcare. Therefore the diversity of knowledge, network structure and characteristic of the different actors are being analysed in order to understand how cross-fertilization of KETs can improve the development of emerging technologies. Findings contribute with the interdisciplinary debate and can allow practitioners, innovation managers and policymakers to effectively develop strategies of managing, developing and bringing innovative high-tech products closer to market.
\end{abstrct}

\begin{abstrct}
\begin{center}\textbf{How can marine sponges help us to fight sexually transmitted diseases?}

Jon Cantero Pérez

\emph{Mucosal Immunology (IGTP)}
\end{center}

\textbf{Keywords: }Chlamydia, Immunology, HIV, Vaccine.

Protection against sexually transmitted infections (STI) relies on the induction of a mucosal immune response at sites of potential exposure. Invariant Natural Killer T cells (iNKT) are major immune regulators, bridging innate and adaptive immunity, and responding within hours after activation. They have become an attractive target for vaccine development since, when activated, these cells trigger downstream activation of other immune cell populations. Further, among others pathogens, chlamydia and HIV have evolved strategies to evade iNKT cell responses supporting their relevance in antimicrobial mucosal immunity. With the generation of new agonists and iNKT modulators, the interest of using these compounds as adjuvants for both, preventive and therapeutic vaccines is rising. The aim of this project is to address the benefit of novel iNKT modulators as potential vaccine adjuvants against female reproductive tract infections.

The study of novel iNKT modulators in the context of chlamydia infection will not only inform on the protective role of activating these cells during vaccination, but also shed light on the immune adjuvant capacity of these molecules for vaccine development against other STIs.
\end{abstrct}

\begin{abstrct}
\begin{center}\textbf{Transcendental atmospheres: Exploring the place of imagination in the landscapes of home}

MIZAN RAMBHOROS

\emph{Universitat Pompeu Fabra}
\end{center}

\textbf{Keywords: }home, imagination, atmospheric, extra-sensory, embodiment, consciousness.

Our current condition of heightened mobility is riddled with socio-spatial complexities.  The existence of ‘home’ as a place-specific physical entity appears to be increasingly temporal for global networkers.  However, the ephemeral yet pervasive spaces of the mind may offer more permanence in our identification with feelings of home via an extra-sensory atmospheric dimension in which the immaterial space of imagination (that could include memories and dreams) may be mapped over a surrounding tangible reality.  And, whilst the social collective reflects a broader perspective of our interactions with place; the relevance of personal sensitivities, perceptions and interpretations, as influenced by individual identity and cultural affiliations, are necessary to the understanding of our embodied, existential and affective involvement with the worlds in which we physically live and mentally create.  The exploration of empirical socio-cultural situations in relation to theories of the consciousness / psyche are hoped to assist in our (re)considerations of the concept of home within our global, cultural and poetic landscapes.  Framed by emotional geographies, this research intends to engage the disciplines of architecture and the humanities by making reference to the arts (painting, literature and film) as a tool to explore and demonstrate the theories of imagination.  The philosophical paradigm of the study should assume a naturalistic ontology and an interpretivist epistemology, thereby informing an explorative research methodology. Filtered by phenomenology and metaphysics, the qualitative study will refer to Western and Eastern (specifically Indian) philosophy.\newline\textbf{Does not give the constent to be recorded}

\end{abstrct}

\begin{abstrct}
\begin{center}\textbf{Rational protein engineering by computational chemistry}

SANDRA ACEBES SERRANO

\emph{Barcelona Supercomputing Center}
\end{center}

\textbf{Keywords: }Computational chemistry, protein engineering, enzyme design.

In standard laboratory experiments sometimes is not clear what is taking place in the tube test. In this regards, computational chemistry can act as the glasses that help scientists to understand the reaction or even to model a new protein or material. 

Using computational chemistry methods we can visualize, for example, the form of the protein, analyze its cavities and evaluate how mutations affect the reaction or  the mobility of a protein in solution. This deep detail combined with the high improvement in speed and capacity of the computers nowadays allow   to engineer proteins rationally, saving money and time in the lab.

Combining both experimental and computational methods we enhanced by rational design an enzyme capable of degrading wood in an environmental friendly process. The degradation of wood waste is a promising solution for the generation of bio-fuels, a necessary alternative to the traditional fuels that cause contamination and greenhouse effect.
\end{abstrct}

\begin{abstrct}
\begin{center}\textbf{Games telling (hi)stories? Digital simulations mediating history}

Federico Peñate Domínguez

\emph{Universidad Complutense de Madrid}
\end{center}

\textbf{Keywords: }history, videogames, simulation, ludonarrative, digital media, digital humanities.

For the younger generations, media productions of the digital era have become one of the main foundations of their knowledge. Through movies, TV shows, and especially videogames, the so called “digital natives” build and shape the way they understand and perceive their surrounding realities. Therefore, pop culture characters such as Super Mario, Sonic and Ganondorf have become references that symbolize certain social behaviors, moral values and ideals. Furthermore: the worlds where they dwell are often emulations of our own.



Although ludofictional worlds tend to be set in fantasy landscapes or the far future, ancient epochs and historical events are also used as the background for digital adventures. The former words acquire a new dimension when fictional characters and settings are replaced with national heroes, infamous dictators and episodes that are remembered with solemnity. History no longer remains on the books, the classrooms, not even on the screens: history is being played and replayed, and has become an interactive experience.



Therefore, it is a matter of uttermost importance to understand how these new narratives, that place the playfulness over other, more traditional elements, mediate the past. This way of exploring historical issues offers a range of unsuspected possibilities that are yet to be discovered. Professional historians, instead of giving the could shoulder to this phenomenon, should embrace it and use it to its full potential. My current research aims to understand the nature of the video game medium as a system able to produce, through the player’s interaction, latent (hi)stories. Furthermore, it explores how historical simulations are influenced by both popular and academic historical texts, and how game mechanics change our relation with the past.
\end{abstrct}

\begin{abstrct}
\begin{center}\textbf{Los espacios del andar. Formación de recorridos en la ciudad fragmentaria}

MARIA FERNANDA LEON VIVANCO

\emph{UNIVERSIDAD POLITECNICA DE CATALUÑA}
\end{center}

\textbf{Keywords: }recorridos urbanos, fragmentos urbanos, espacios del andar.

Caminar en la ciudad, además de ser el medio de transporte más sostenible, que gasta menos recursos y afecta en menor medida al medio ambiente, permite a todos los grupos de población acceder equitativamente al espacio público y hacer uso de él; le proporciona al individuo un escenario físico que reivindica y conquista las condiciones para la apropiación consciente de la ciudad, le permite disfrutar de un paisaje imaginable, visible, coherente y claro, sobre todo, cuando cada vez es más frecuente encontrar en el tejido urbano áreas homogéneas, inaccesibles, contradictorias y divididas, cuya impersonalidad estimula el uso del vehículo y limita los desplazamientos a pie.



El exterior urbano sin embargo, es más que una simple muestra individual de obras arquitectónicas, lugares llenos de contrastes, de aglomeraciones y vacíos, etc., que se tejen entre sí a través de espacios por los que el andante traza sus trayectorias, las recorre, las espacializa y deja su huella. Existen entonces conexiones que se destacan por su alto grado de urbanidad, lugares que acumulan actividad e identidad, por los cuales circula o se congrega la población, que se convierten en recorridos que atraviesan los fragmentos y los ligan, y que, facilitan la transición de una pieza a otra. 



Considerar que estos elementos cumplen un rol fundamental en la vida urbana pone en relevancia aquellas características y valores que permiten su apropiación y uso. Recorrer la ciudad implica una conjunción de experiencias sensibles que resultan de los estímulos que recibe el andante del medio construido que atraviesa y con el que se siente plenamente identificado. El ir de un punto a otro resulta en sí mismo no sólo un acto de traspasar el espacio, sino también, un sistema complejo en el que a través del movimiento se superan los bordes internos de la ciudad  y al que recurren otros aspectos que influyen sobre el andante en la creación de su propia versión del espacio, de sus recorridos e itinerarios.
\end{abstrct}

\begin{abstrct}
\begin{center}\textbf{TV fiction and its power of cultural representation}

Isabel Villegas Simón

\emph{PhD student}
\end{center}

\textbf{Keywords: }Tv fiction; cultural representacion; adaptation; cross - cultural studies; television studies.

TV fiction can be an entertainment, a knowledge source, a passionate hobby or a profession. In an academic approach, TV fiction can be considered as a place where the interests and tastes, the concerns and desires, the problems and utopias are portrayed. My thesis concerns about this issues. In particularly, I try to deepen the differences and similarities in the social – cultural representation on the TV series between countries. I am studying the phenomenon of the adaptations of TV fictions around the world. Specifically, I have chosen two TV fictions created in Spain (Los misterios de Laura y Pulseras Rojas), which have been exported and adapted in EEUU and Italy with success. The analysis pretends to explain how the adaptation of each country changes according to its cultural, social and political framework. Regarding methodology, I am studying methodological proposal ranging from qualitative approach like semiotics, textual analysis and discourse analysis to more qualitative approach like content analysis. The results from this research may help to understand rigorously a phenomenon of the TV industry that is rising: the TV fiction adaptations worldwide, an issue very remarkable to the professionals and academics of this area; moreover, this research may contribute both methodological and theoretical level in recent television studies, an area of knowledge in which there are still many issues to explore; and finally, this study pretends to provide a set of knowledge to understand better ourselves as society and as individuals. studying one of the cultural and entertainment product that has become more popular in recent decade, TV fiction.
\end{abstrct}

\begin{abstrct}
\begin{center}\textbf{Drugs of the future}

Sanja Zivanovic

\emph{IRB Barcelona}
\end{center}

\textbf{Keywords: }medicine, drug discovery, computational, chemistry, innovation.

Have you ever wonder how many years and millions of euros we need to create a new medicine?

The human body is a miracle, but it is also extremely mysterious. Many illnesses disorders are still untreatable and unclear. The typical drug discovery and development cycle, from concept to market, takes approximately 14 years, and the cost ranges up to 1 billion \texteuro. Over 14 years about 100 research projects will eventually lead to just 1 drug to the market. The huge improvement would be to use scientific knowledge combined with powerful computational methods to develop new innovative drugs. The aim is to detect which damaged protein causes illness, as well finding drug like molecules and possible inhibitors, which will reduce costs of labor expensive and make the process of drug discovery more time efficient. But the most important thing, at the end it will give a new hope to patients, hope for cure,hope for life.
\end{abstrct}

\begin{abstrct}
\begin{center}\textbf{Electromecanichal heart simulation based on MRI images}

Mariña López Yunta

\emph{Barcelona Supercomputing Center}
\end{center}

\textbf{Keywords: }Cardiac simulations, electromechanics, finite element method, MRI images.

A multiphisic cardiac model gives accurate simulations of normal and pathologic behavior of the heart. This simulations can help to develop new pharmacological treatments and medical devices. The complexity rely on both mathematical and geometrical models, so that high performance computation is needed to obtain accurate results using the final element method. Different kind of problems are necessaries to face when solving fluid-electromechanical simulations: geometry and mesh generation, fiber orientation, model parametrization or boundary conditions. We will focussed on geometry and mesh generation. 

	Nowadays it is frequently to find electrophysiology simulations using simplified geometries. This is not enough once we introduce the mechanical problem. A complete heart geometry is needed, that is including atria and ventricle. This geometry is obtained from the MRI (magnetic resonance imaging) through  segmentation, then the corresponding CAD is generated. The CAD is useful to fix boundary conditions and the properties of each heart region and finally create the volume mesh.

	The heart mesh obtained form in vivo data does not correspond to a relaxed configuration because the heart is pre-stressed by the physiological conditions at the moment when MRI images were taken. Since the heart mesh is not stress-free, we will impose as initial condition the internal stresses calculated from the intraventricular pressure data. Finally we are in conditions to solve our cardiac simulation  using  finite element methods on a MRI-based mesh.
\end{abstrct}

\begin{abstrct}
\begin{center}\textbf{How online social networks grow and compete: a complex systems perspective}

Kaj Kolja Kleineberg

\emph{Universitat de Barcelona}
\end{center}

\textbf{Keywords: }Complex Systems, Complex Networks, Online Social Networks, Digital Ecology, Digital Revolution.

The overwhelming success of the Web 2.0, within which online social networks are key actors, has induced a paradigm shift in the nature of human interactions. The user-driven character of Web 2.0 services has allowed researchers to quantify large-scale social patterns for the first time. However, the mechanisms that determine the fate of networks at the system level are still poorly understood. Here, first we study how online social networks grow in an isolated environment. Their particular growth path allows us to quantify the relative importance of the two key dynamical processes, namely a viral spreading mechanism and mass media influence. However, the simultaneous existence of multiple digital services naturally raises questions concerning which conditions these services can coexist under. Analogously to the case of population dynamics, the digital world forms a complex ecosystem of interacting networks whose survival depends on their capacity to attract and maintain users’ attention, which constitutes a limited resource. We introduce an ecological theory of the digital world which exhibits stable coexistence of several networks as well as the dominance of a single one. Interestingly, our theory predicts that the most probable outcome is the coexistence of a moderate number of services, in agreement with empirical observations. Heterogeneity in the network intrinsic fitness can be applied to understand the competition between an international network, like Facebook, and local services. We find that above a certain threshold, the level of global connectivity can lead to the extinction of local networks. In addition, we reveal the complex role the tendency of individuals to engage in more active networks plays for the probability of local networks to become extinct and provide insights into the conditions under which they can prevail.
\end{abstrct}

\begin{abstrct}
\begin{center}\textbf{Accounting for life-history strategies and time-lags in marine restoration: a study case of a long-lived octocoral}

Ignasi Montero Serra

\emph{Departament d'Ecologia, Universitat de Barcelona}
\end{center}

\textbf{Keywords: }Marine Restoration, Mediterranean Sea, Life-history Theory, Tradeoffs, Corallium rubrum, Demographic Models.

Marine restoration has traditionally focused on survival rates as a measure of success. However, from a management perspective, success depends on further ecological properties such as the degree of structural complexity and long-term population viability, which is also determined by growth and reproduction. Although life-history theory suggests that tradeoffs may arise among these vital rates in natural conditions, their implications for marine restoration are unexplored. Using the threatened octocoral Corallium rubrum as a model species, we performed a transplantation experiment and assessed short and long-term restoration outcomes for this slow-growing and long-lived species. Four years of transplanting, C. rubrum colonies showed high survival, high reproductive potential and positive growth, suggesting high success at a relatively short-term. However, the transplanted population was dominated by small colonies and demographic projections revealed that a periods ranging from 25 to 30 years may be required for functional C. rubrum populations to develop. A comprehensive review of restoration experiments of marine habitat-forming species revealed a high diversity of life-history strategies. A negative relationship between survival after transplantation and growth rates demonstrated that demographic tradeoffs are present in marine restoration. Slow-growing species will tend to show higher survival rates after transplantation but the period required to obtain complex habitats will also tend be longer. In turn, faster-growing species can speed up the reef formation process but may require more transplantation efforts because their lower survival probability. Our results highlight the importance of bearing in mind the species-specific life history strategies and its implications in terms of initial effort, costs, and associated temporal scales when designing restoration actions and setting conservation goals.
\end{abstrct}

\begin{abstrct}
\begin{center}\textbf{Restatement of Kung-Traub Conjecture}

Fayyaz Ahmad

\emph{UPC, Barcelona}
\end{center}

\textbf{Keywords: }Nonlinear equations,  System of nonlinear equations,  Kung-Traub conjecture,  PDEs, ODEs.

According to Kung-Traub's conjecture,  a multi-point iterative method without memory, for solving nonlinear equations to find simple zero, 

could achieve a maximum convergence order $2^{d-1}$, where $d$ is the number of function evaluations.   We have shown that this cojecture fails for a particular class of nonlinear functions. A modification in this conjecture is presented.
\end{abstrct}

\begin{abstrct}
\begin{center}\textbf{Highthroughput crystallization of methamphetamine derivatives at low temperatures}

Walter Blanco; Jesus Homerosa

\emph{University of methland}
\end{center}

\textbf{Keywords: }meth, crystal, drugs, Highthroughput crystallization.

One of the most interesting questions in Highthroughput crystallization of methamphetamine derivatives is to understand the pattern formation of the different crystals. Since very recently, the high cost of the techniques prevented the small laboratorys to study this process. However, new techniques such as potinsideavan preparation allows the small labs to prepare their own crystals at low cost.

The aim of this talk is to give an overview of the different cook techniques to obtain the desired pattern for the meth crystals. Moreover, we discuss the most relevant problems: the spanish law regulations and the spontaneous eruption of Drosophila melanogaster in the labs during the process.
\end{abstrct}

\begin{abstrct}
\begin{center}\textbf{Waking up the beast: ultra-energetic phenomena in galaxy mergers}

Núria Torres Albà; Víctor Moreno de la Cita

\emph{ICCUB}
\end{center}

\textbf{Keywords: }astrophysics, black holes, jets, AGN, mergers, non-thermal emission.

Some events can only be triggered by the most catastrophic phenomena in the universe, such as the birth of Active Galactic Nuclei (AGN): relativistic jets of plasma powered by supermassive black holes. These objects, however, have remained invisible in the most energetic light (gamma rays) until recently. Our hypothesis is that this light is the product of the interaction between the jets and the population of giant stars surrounding them.



Here we present the work done to shed light onto this new realm divided in two steps: on one hand we characterise the stellar population in the nucleus, the number of stars and their properties. On the other hand we also study one single interaction between the jet and a typical giant star, obtaining an estimate of the non-thermal radiation produced. Combining these results we propose a new mechanism of gamma ray production in AGN, potentially able to explain this new light. 
\end{abstrct}

\begin{abstrct}
\begin{center}\textbf{Expanding Genomic Knowledge in Neurodevelopment disorders}

Carlos Ruiz Arenas

\emph{CREAL}
\end{center}

\textbf{Keywords: }Genetic, child health, neurodevelopment diseases, structural variation.

In 2001, a first draft of the human genome sequence was published. With this achievement, the scientific world expected that the understanding of biological complexity was close. Fourteen years later, results have not been so impressive. 

One of the reasons that justified the generation of a genome sequence reference was to find changes in the DNA that are correlated to diseases. Firstly, researchers explored genetic changes. Consequently, in the last decade, a lot of experiments have been done trying to match changes in a position of DNA sequence with diseases. Although for some diseases this approach worked well, for others this method was not able to explain their mechanisms. 

The next step was to study structural variants, another source of genomic variation. In these events, the DNA sequence remains unaltered but the chromosome structure changes. Copy number variants, i.e. repetitions of a DNA sequence, are an example of structural variants. This kind of variation has been found to be relevant in many diseases and it was a promising field in genomics. 

My research focuses on chromosomal inversions, another type of structural variant. Chromosomal inversions are fragments of a chromosome whose orientation has been changed. They have been related to susceptibility to diseases such as asthma or diabetes and, recently, it has been suggested that they are correlated to cognitive abilities. Taking into account their effect on brain, my research will try to study the role of chromosomal inversions in neurodevelopment disorders. To this end, we will use genomic data as well as other biological resources, such as expression or methylation arrays, in order to propose a possible mechanism that would explain the effect of inversions on these disorders.
\end{abstrct}

\begin{abstrct}
\begin{center}\textbf{Molecular Docking: development and applications for Drug Discovery}

Sergio Ruiz Carmona

\emph{PhD Student}
\end{center}

\textbf{Keywords: }Computers, pharmacy, drug discovery, chemistry.

Identification of chemical compounds with specific biological activities is an important step in both chemical biology and drug discovery. When the structure of the intended target is available, one approach is to use molecular docking programs to assess the chemical complementarity of small molecules with the target; such calculations provide a qualitative measure of affinity that can be used in virtual screening (VS) to rank order a list of compounds according to their potential to be active. 

Molecular Docking is a highly used computational tool in real projects for Drug Discovery, and a summary of different applications and development possibilities is highlighted.
\end{abstrct}

\begin{abstrct}
\begin{center}\textbf{Transmissions in Medieval Mediterranean: profane images from Iberian wooden painted ceilings}

Maria del Mar Valls Fusté

\emph{Universitat Rovira i Virgili (Tarragona)}
\end{center}

\textbf{Keywords: }Medieval Mediterranean, medieval ceilings, profane iconography, visual culture, Islamic Art, Christian Art..

The profane visual repertoire and the workshops of the artists around the Mediterranean area during Middle Ages came from different religious and cultural realities such as the Normand Sicily, the Maghreb and Al-Andalus. The cultural and political openness experienced by the Aragonese Crown with the territorial expansion of the king James I the Conqueror made the incoming and outgoing of iconographic models easy all around the geographical area. The figures of the painted ceilings during the XIII-XV centuries are a faithful prove.

The painted ceiling as a support with a wide  and complex repertoire of images, let us stablish a relationship among models and iconographic resources as they show the  cultural cohabitation and the influence exercised among the different ethnic and religious communities. At first, the order, the quantity and the diversity of representations had given us way to feel them as marginal. The growing interest of the investigators and the changing methodology of the Art History allows us prove the opposite. What is being developed and discussed in scientific meetings and published materials is that this specific figurative repertoire often has a vehicular message that supplies different data about the concept of work of art in itself, the ideologist of the piece, the functionality of the space where it’s located or the willing and taste of the promoter.

Scenes of courtesan and knightly life, feasts and dances are centred in the ceilings, both in holy and laic spaces. The dance, specifically, can be used as a mirror of this medieval society both complex and poliedric. The movement of some figures are above the mere staging of some gestures to give way to a whole set of representations with a significant aspect. 

The topic will be focused on the exhibition of these points, object of our investigation and its relevance for the science.
\end{abstrct}

\begin{abstrct}
\begin{center}\textbf{Protected thin film CIGS solar cells as flexible photocathodes for solar hydrogen production}

Carles Ros Figueras

\emph{IREC}
\end{center}

\textbf{Keywords: }Solar, Hydrogen, Photoelectrochemical, CIGS, thin film.

As society faces the problems derived from global warming and excessive pollution, harvesting solar energy and storing it into chemical bonds is one of the most promising paths in the so called solar fuels economy. Between them, photoelectrochemical (PEC) water splitting offers the possibility to directly convert water and solar energy into hydrogen and oxygen with competitive efficiencies.

Conformal photocathodes based on chalcopyrite (CIGS) thin film solar cells built-in on flexible stainless steel foils can become elements for developing new concepts for PEC cells to obtain solar fuels. The influence on the PEC characteristics of the protective anticorrosion layer based on TiO2 film deposited by Atomic Layer Deposition (ALD) technique at very low temperature are analyzed, taking into account the charge transport mechanisms through this layer.
\end{abstrct}

\begin{abstrct}
\begin{center}\textbf{Microwave hyperthermia for breast cancer treatment}

Aleix Garcia Miquel

\emph{Universitat de Barcelona}
\end{center}

\textbf{Keywords: }Microwave hyperthermia, thermal therapy, breast cancer treatment, antenna array, medical devices.

Breast cancer is the most frequent cancer among women worldwide. There is a clinical need to develop low-cost, minimally-invasive image-guided interventions for treating this disease. Hyperthermia, a form of moderate thermal therapy involving heating the tumor tissue to temperatures in the range of 41 to 45 $^o$C, has been demonstrated to increase the effectiveness of radiotherapy and chemotherapy. Hyperthermia also reduces the time of conventional treatment sessions, improving the patients’ quality of life and considerably decreasing the hospital costs. In our study, a compact, comfortable and easy-to-manufacture microwave applicator for the treatment of breast tumors has been developed. This device is based on the thermal increment produced by an array of microwave antennas focused on the breast. By modifying the amplitude, the phase and the frequency of the radiated microwaves we can achieve the desired heating in the tumoral tissue while maintaining the healthy surrounding tissue in an appropriate temperature.
\end{abstrct}

\begin{abstrct}
\begin{center}\textbf{LIGHTING UP THE DARK SIDE OF THE UNIVERSE}

José Luis Bernal Mera; Ignasi Pérez Ràfols

\emph{Institute of Cosmos Sciences (ICCUB)}
\end{center}

\textbf{Keywords: }Universe, Cosmology, Astronomy, Dark energy.

When we look up to the sky, we are looking through the mirror of time, since the light travels at a finite velocity. This has allowed us to observe the Universe at different epochs and to discover that the Universe has not been always like the one we live in now: it has evolved and expanded. The understanding of this expansion led to the well known Big Bang theory. Naively, one can think about the Big Bang as an explosion whose expansion wave (the expansion of the Universe) is stopped by gravity. However, contrarily to the common sense, the expansion of the Universe is accelerated. The acceleration is driven by an unknown energy which make up for almost the 70\% of the energy of the Universe today: the dark energy. To study how dark energy behaves, we follow the expansion of the Universe using standard candles and standard rulers. The former allow us to track distances to the sources (and thus the time at which we are observing). The later give an impression on 'how big' the Universe was at that time.
\end{abstrct}

\begin{abstrct}
\begin{center}\textbf{Wireless Networks with Battery-less IoT Devices: Utopia or Soon-to-be Reality?}

Prodromos-Vasileios Mekikis

\emph{UPC}
\end{center}

\textbf{Keywords: }telecommunications, internet of things, connectivity, wireless energy harvesting.

In view of the constantly growing need for wireless connectivity and the remarkable advancements of the Internet of Things (IoT),  more than 20 billion wireless devices are expected to be deployed until 2020. This significant increase in the density of wireless devices poses two critical design challenges: i) the aggregate interference worsens the communication performance, and ii) due to the massive number of IoT devices, traditional solutions for replenishing their consumed energy such as battery replacement or cable-charging are not practical and may not always apply, e.g., in the case of sensors embedded in buildings or human bodies. 



To confront the first issue, a mathematical tool called Stochastic Geometry is employed. Using Stochastic Geometry, it is possible to take into account both interfering and intended transmissions in the dense network and derive formulas that provide the probability of full connectivity, i.e., the probability that all devices can communicate with each other via at least one path.



In addition, although interference is generally undesirable, it is possible to exploit it using Wireless Energy Harvesting (WEH) to cope with the second major challenge of dense networks. The WEH technique has been recently introduced as a promising approach to confront the lifetime limitation of low-power wireless devices by converting the ambient radio frequency (RF) energy of the network transmissions into direct current (DC).



In my research, these two issues are jointly investigated using novel techniques that increase the network lifetime while guaranteeing the communication performance. The ultimate goal of my PhD is the modeling and characterization of fully-connected WEH-enabled dense wireless networks consisting of battery-less IoT devices.
\end{abstrct}

\begin{abstrct}
\begin{center}\textbf{Cultural Participation Oddity}

Julià Vicens

\emph{Universitat Rovira i Virgili}
\end{center}

\textbf{Keywords: }computacional social science, cultural consumer behaviour, human mobility,  social networks, recommender system, citizen science.

The cultural public and audiences have changed substantially over the last years. In order to study the cultural consumer's behaviour, specially focus on participation in live cultural events, museums and heritage, we present an approach to analyze and realize human behaviour by means of social networks and human mobility data.



We analyze geospatial metadata from social digital network datasets in which extract information such as preferred paths in the city (Flickr), community interactions (Twitter)…, particularly attending to cultural events and heritage sites. With the objective of achieving a better data quality, we carry out two projects (Museum Night Experiment and Cultuscope), currently in startup phase, which allow to capture collaboratively mobility data in cultural events following the citizen science's paradigm in terms of openness, transparency, collective learning, engagement…



Geospatial data can be useful to reveal cultural consumer’s behaviour. That involves build a model in such a way that can be possible implement a cultural recommender system.
\end{abstrct}

\begin{abstrct}
\begin{center}\textbf{Pakistani women in Barcelona: access to public services.}

Marina Arrasate Hierro-Olavarría; Komal Naz

\emph{Universitat Autònoma de Barcelona. Departament d'Antropologia Social i Cultural}
\end{center}

\textbf{Keywords: }Gender issues, Pakistani migration, acces to public services, cultural and linguistic barriers.

The Pakistani community is since year 2008 the major extra-communitarian population in Barcelona. The migration pattern of Pakistani families usually consists of a first all-male-settlement followed by the arrival of women and children through reunification processes. As they arrive in Barcelona, most women have little or no competence in Catalan and Spanish and due to the nature of their incorporation into the host society, this situation can last longer than expected. Our research addresses the linguistic and cultural barriers faced by Pakistani women in their access to public education and health services in the city of Barcelona. In order to establish the framework, an overview of some aspects of the Pakistani community in Barcelona will be presented, exploring social, demographic and organizational issues. Then, the study will focus in Pakistani women migration process, their access to public services and their perception of intercultural communication in public health care and educational settings. The data presented offer a preliminary approach to Pakistani women health and education conception and stress the importance of effective communication as a mean to provide quality public services and guarantee the rights of migrant populations.
\end{abstrct}

\begin{abstrct}
\begin{center}\textbf{3Dimensioning our bones from 2D}

Mirella López

\emph{Universitat Pompeu Fabra}
\end{center}

\textbf{Keywords: }Osteoporosis, statistical shape models, 3D reconstruction, medical imaging.

Bones allow us to move, to walk, to dance…. Even they protect our internal organs. Bones are formed by an outer compact shell and a porous inner, which looks like a sponge. So, everybody knows bones are really important for our life. But, does everybody know that they not only grow when we are children, if not they are constantly changing?  This is thanks to two types of cells which work together inside the sponge. One eats mature bone and the other creates new one.  But sometimes, the building cells fail to form enough new bone or the destructive cells eat too much. In these cases, osteoporosis occurs. 

In this bone disease, the sponge part has more pores than in normal bone. That implies a decrease in the mineral content of the bone, making it more fragile and with more probabilities to fracture. 

Osteoporosis is diagnosed comparing the BMD of a patient with a reference value. This comparison gives a score which confirm or rule out the bone fragility and the potential fracture risk. There are two main techniques to calculate the BMD : DXA  and CT. Both apply X-ray to the body area of interest, but the first applies ten times less radiation and it is cheaper. So, it is the most used in clinical practice. Its disadvantage?  It gives only a 2D image with no information about the general shape, size or mineral distribution in the bone. 

But, we know that our bones have a similar shape with a range of possible variation. For example, eyes are ellipsoidal. They could be bigger, smaller but not square! So, why not build a reference 3D model and use it for the 3D reconstruction from 2D?  Under this assumption, we use a database of 3D bones and calculate their mean shape and possible variations, obtaining a statistical model of the bone. Then, when we have a new patient, we test all possible variations in our model to fit with his 2D image. We calculate a series of clinical parameters and predict the potential risk of bone fracture!
\end{abstrct}

\begin{abstrct}
\begin{center}\textbf{Joint reflexive processes: An approach to professional development and understanding of the situations of practice}

Daniel Paredes

\emph{Universidad de Barcelona}
\end{center}

\textbf{Keywords: }Reflective practice, professional development, virtual representation, situation, cultural psychology.

Many researches suggests not only the importance of reflection for professional development, but also the need for clarification of what is understood by the term of reflection and the necessary conditions for reflection as a way of improve professional practice. The aim of this research is to explore the reflection process in workplace as a form of professional development that help teachers to make sense and solve new or uncertain situations. The study objectives are (1) to describe and understand the joint reflection process between teachers about the situation that they experience on practice, (2) to characterize and describe how teachers construct and use representations about these situations during joint reflection. Specifically, the research questions are: How joint reflection enables teachers to make sense or understand situations of practice? How are the dynamics of interaction between teachers during joint reflection, and how these dynamics evolve? How teachers elaborate and internalize a shared representation about the situations of practice during joint reflection? The study is based on two main theoretical perspectives. First, from a Deweyan pragmatism it is stressed the idea of transactionality, and from the Donald Schön’s perspective it is used the notion that people reflect by experimenting with the elements of the situation. Second, from a sociocultural perspective the interactive and mediational aspects of the joint reflective activity are described. The research presents a case study in which pairs of teachers visit each other’s classrooms to observe and reflect about their practice. The study use an ethnographic methodological framework based on two phases of analysis. The first one is oriented to describe and understand the joint reflection process through an interactivity model of analysis. And the second phase of content analysis is directed to characterize the process of construction and uses of virtual representations of situations.\newline\textbf{Does not give the constent to be recorded}

\end{abstrct}

\begin{abstrct}
\begin{center}\textbf{Health effect of medical radiation during childhood}

elisa pasqual

\emph{creal (upf)}
\end{center}

\textbf{Keywords: }radiation, child health, CT scan, epidemiology.

Clearly, the use of ionizing radiation (IR) tools in medicine has significantly improved clinical practice. However, the use of lR in medicine has grown greatly in the last 20 years and concern are rising in public health. Estimates for US indicate that the average annual per-capita effective dose from medicine has approximately doubled in the past 10–15 years. This topic is of particular concern in children as they have higher relative risks of IR induced disease per unit dose of IR than adults. Recent cohort studies of paediatric CT (computed tomography) patients in the UK, Australia and Taiwan have reported increased risks of leukaemia and brain tumours following CT exposure, with estimates higher than predicted from atomic bomb survivors’ studies. The validity of these results has been questioned, however, in particular because of absence of information on a number of potential confounding factors that may affect the risk estimates.

The aim of the present project is to give a better estimate of health effect from medical IR exposure during childhood. We will take into account the possible role of genetic predisposition and confounding factors allowing us to respond to concern raised about published cohort studies. Through a collection of biological sample, possible biomarkers of radiation sensitivity will be also evaluated.

The project is based on exploitation, through the conduct of nested case-control (NCC) study, of a large scale Spanish cohort (EPI-CT project).A NCC design overcomes the limitations of a cohort study, allowing, through contact with consenting study subjects, the collection of detailed individual information on potential confounding factors such as medical condition, other sources of IR exposure and possible environmental risk factors. A collection of salivary samples is planned in order to perform an exome sequencing and methylation analysis that would allow us to study possible cancer genetic and epigenetic predisposition.
\end{abstrct}

\begin{abstrct}
\begin{center}\textbf{Machines that Prove Theorems and Equations between Words.}

Albert Garreta Fontelles

\emph{Stevens Institute of Technology}
\end{center}

\textbf{Keywords: }Mathematics, algebra, groups, word equations, Tarski problems.

Can a machine state and prove all possible theorems in a given area of mathematics? Sometimes the answer is yes. We will discuss this question for a certain area of algrebra (free groups). A deep problem lies in its core: that of solving equations between words.
\end{abstrct}

\begin{abstrct}
\begin{center}\textbf{Money or Ethics. Multinationals and religious organisations in an era of corporate responsibility}

Katinka van Cranenburgh

\emph{University of Barcelona}
\end{center}

\textbf{Keywords: }corporate responsibility, responsible investing, religious organisations, multinational companies.

It is a general assumption that Religious Organisations (ROs) are driven by religious beliefs and values, whilst multinational corporations (MNCs) are considered to be concerned about their profits, their share price and their reputation. When ROs invest in capital markets, they participate in modern economy and thereby enter a sensitive spectrum of ethical dilemmas. Since ethics is the core business of ROs, they cannot maintain a situation in which their investment portfolio would be in contradiction with those ethics. MNCs on the other hand are operating in the same modern economy, whereby - in their aim for profits and growth - they too have to deal with the ethical dilemmas that occur due to the nature and expansion of their business and the different cultural contexts in which they operate. Business managers and religious investors struggle to define the roles and responsibilities of MNCs when the products and/or activities they provide or invest in have considerable impact on society. The common assumption would be that - in an era of Corporate Responsibility (CR) – the two types of organisations can be positioned at the different ends of the scale from money to ethics. This study provides insight into the struggles of MNCs and ROs and builds on theory that can be used by business and investment managers that have an influence on society through the means of their activities. The study does not compare MNCs with ROs but demonstrates how MNCs - being confronted with social issues - and ROs, being concerned with similar social issues and investing in MNCs, deal with these issues in an era of increased CR. This dissertation is a compilation of several sub-studies that - as a whole - provide insight into the black-box of decision-making of managers and investors in the context of business ethics. The popular assumptions are validated and a more sophisticated understanding of the two different actors manoeuvring in the fields of finance and ethics is provided.
\end{abstrct}

\begin{abstrct}
\begin{center}\textbf{Non-dualism in arts. Representations of the non-dualism way of thinking in visual arts}

Roger Ferrer Ventosa

\emph{University of Girona (FPU)}
\end{center}

\textbf{Keywords: }Non-dualism. Visual Arts. Tantra. Androgyne. Alchemy engraving books. Surrealism..

The aim of my presentation will be explain the intellectual basis of my dissertation: the non-dualism. Firstly, I will give some of the main ideas about this way of thinking. 

The history of mind can be divided in several ways. According to some philosophers, we can divide the history of culture in three phases: the first one is characterized by monism, a lonely idea that organize everything, and all the universe is connected with each person. The second phase is dualism. In this way of thinking, reality is divided in two principles with the same power. Consequently, we split the reality, we hurt it and put a wounded in existence. 

Non-dualism way of thinking pretends joining again the existence: the two principles are really linked in a relation. The polarity can be the different phases of one process (heat and cold), and alternation (theory of pendulum policy), or one necessity in order to create contrast, as light and darkness. In which case, the two principles of the polarity need the other. The polarity creates a friction that the world needs to move.  

After this, I show some images that we find in art history related with this way of thinking, such as the supreme ultimate diagram (yin-yang) of China, the tantric art of India and Tibet or, in the European cultures, the alchemy engraving books, above all during the XVI and XVII centuries, with representations of androgyne and sacred wedding (hierosgamos). Finally, we see one sacred wedding in Contemporary Art, by the surrealist artist Leonora Carrington.
\end{abstrct}

\begin{abstrct}
\begin{center}\textbf{Bearing the bear.  Experience, body and sexuality among gay bears in Barcelona}

ISABEL FERRANDIZ ARMERO

\emph{None}
\end{center}

\textbf{Keywords: }Gender, body, sexuality, gay, intersectionality.

This research takes place on Social and Cultural Anthropology and it is focused on the “bears”: a particular gay subculture. This group is composed of men having sexual and affective relationships with other men, but presenting a self-image opposed to the mainstream gay model. To some extent, “bears” claim for naturalness in terms of masculinity and body. While some people describe them bodily (hairier, bigger and older than the hegemonic gay image), some others focus on their attitude, which is seen as more tolerant and masculine than the average gay.

Attending the current discredit of gender essentialisms, bears remain controversial because of their understandings of masculinity. It is then necessary to reckon with other elements and features that could shed light on them, such as class, homophobic contexts or age. By considering some more categories, bears would be shown as a more diverse and complex group. Because little research has been done on this topic in Spanish context, my aim is to conduct fieldwork in the city of Barcelona. I would specially like to centre my attention on experience and affectivity from an intersectional perspective that allow grasping the diversity within bears.
\end{abstrct}

\begin{abstrct}
\begin{center}\textbf{Let me present you the Scientists Dating Forum}

Yoran Beldengrün

\emph{Insitute of Advanced Chemistry of Catalunya -CSIC}
\end{center}

\textbf{Keywords: }Scientists Dating Forum.

Some months ago I came up with the idea to build up the Scientists Dating Forum. 

“Scientists Dating Forum”. Yes, you read correctly. But no, I’m sorry, I have to disappoint you: It’s not a new dating/flirting platform for scientists. So what is it?



The idea is to bring out the scientist from their laboratories and show them the relation science has with politics, economy and the society. To motivate them to have “dates” with those fields. The Scientists Dating Forum will be in the future a platform for much more dates, but the objective for 2016 is to organise 3 big conferences here in Barcelona: “Science Dates Politics”, “Science Dates Economy” y “Science Dates Society”. 



The Scientists Dating Forum is organised by a team of over 30 scientists, coming from over 15 different institutes in Barcelona.  Many science related organisations, entreprises and universities are as well collaborating on this new exciting project emerging in the city of Barcelona.



Let me share you some thoughts about the Scientists Dating Forum, during those JIPI 2016.
\end{abstrct}

\begin{abstrct}
\begin{center}\textbf{Optical tweezers for manipulating and measuring forces at the micro-scale}

Frederic Català Castro; Dorian Treptow; Raúl Bola Sampol

\emph{Departament de Física Aplicada i Òptica, Universitat de Barcelona, Martí i franquès 1, 08028, Barcelona}
\end{center}

\textbf{Keywords: }Optical tweezers, force measurement, beam modulation, optical micromanipulation.

Optical tweezers are tightly focused laser beams capable of non-invasively trapping micron-sized dielectric particles through radiation pressure. Together with micro-manipulation abilities, light momentum detection enables quantitative analysis of trapping forces, making optical tweezers a powerful technique of the utmost interest in a large variety of fields, from physics to microbiology. This direct force measurement shows high robustness against fluctuations in optical and thermodynamic parameters like particle size, shape, refractive index, viscosity or local temperature. 



Interestingly, force measurements can be combined with beam shaping techniques, such as holography or acousto-optic modulation, in order to control the position and properties of an optical trap. On one hand, the holographic techniques offer us the possibility to create very complex trap arrays or exotic beams, allowing us to independently manipulate more than one object at the same time, or creating interesting force-field profiles. On the other hand, the fast switching capacity, in the kilohertz scale, of acousto-optic modulators, makes them a good choice for performing very fast experiments or creating multiple optical traps by time sharing the laser beam, which permits multiple-trap individual force measurements.



In this work, we will review the main state-of-the-art topics around trapping beam modulation and force measurements in optical tweezers, and their simultaneous combination to expand the boundaries of optical manipulation at the micro-scale.
\end{abstrct}

\begin{abstrct}
\begin{center}\textbf{Slicing the Wireless Internet}

Matías Richart Gutiérrez

\emph{Departament d'Enginyeria Telemàtica - UPC}
\end{center}

\textbf{Keywords: }wireless networks, internet, slicing, virtualization.

Wireless access networks are today the main way users access to the Internet. It is expected that in the next five years the traffic through wireless networks will increase by a factor of 10 and the number of wireless connections will be doubled. In this context new architectures and technologies are being developed to deal with these requirements.

One of the design solutions for dealing with this huge amount of traffic and connections is to split the network in different isolated slices. Slicing the network would allow to configure the network “edge-to-edge” and define specific functions for different scenarios while sharing the same infrastructure and avoiding higher costs. For example, network slices would offer efficient resource utilization as each slice could be customized for a specific service and on a dynamic on-demand way.

Because of the shared and dynamic nature of the wireless medium, slicing the wireless access is a complex problem. A slicing mechanism in the wireless domain has to deal with efficient resource utilization, inter-slice isolation, and customizable intra-slice resource allocation. 

In this talk, I will introduce on the context and problems on slicing a wireless network and briefly mention some of the solutions I am currently working on.
\end{abstrct}

\begin{abstrct}
\begin{center}\textbf{From the Depths of the Void: The poetics of the 'Sunken City' from William Blake to Alfred Kubin.}

Rocío Sola Jiménez

\emph{Department of Humanities, Universitat Pompeu Fabra}
\end{center}

\textbf{Keywords: }Sunken City; William Blake; Alfred Kubin; Illustrated Books; Romanticism..

This study has its goal on recovering two peculiar masterworks from oblivion, where image and text are inseparable –William Blake’s epic poem 'Milton: a Poem in Two Books' and Alfred Kubin’s fantastic novel 'Die Andere Seite'. These two works had not been connected until this moment, regarding how they switch register and create a new conception of book –organic and mutable– at the same time they depict one of the most important themes of western literature and philosophy: the ‘Sunken City’. The poetics involved by the Sunken City also touch most of the authors affected by the ‘bad of opium’ so it enables to set a marriage between the development of these specific works and a schizophrenic, alienated and drugged conception of the human condition –standing out the 19th Century and the Romantic period, which is well represented by the works selected for this research. It is also important to point out how this altered states of consciousness have been the incentive for a lot of artist and writers to create their works all over the century until our times. Likewise, what this study does is rebuilding one of the most deep-seated discourses in History of Ideas: Utopia, but in a very specific way.

This cities emerge now through an interdisciplinary approach –Art History, Literature, Philosophy, Psychology, and Pharmacology– establishing dialectics between the concepts of Heaven and Hell, Light and Darkness, Reason and Imagination, and Eastern and Western thoughts, with the purpose of placing this philosophic adventure on the conciliation of the human being with infinity –for so long sought. These connections of opposite concepts enable the artist-writer to create a sort of characters –inhabitants of the Sunken City– that perpetuate the first big antinomy conceived in the masterwork of one of the most important writers of all the times: John Milton’s 'Paradise Lost'.
\end{abstrct}

\begin{abstrct}
\begin{center}\textbf{New forms of elevation and transcendence in postmodern art}

Alison Moss Ferrer

\emph{Universitat Pompeu Fabra}
\end{center}

\textbf{Keywords: }Boltanski, abstraction, dematerialization, postmodernity, spirituality.

In the postmodern world, characterized by mass individualism and the end of narratives, the materiality of art has dissolved and glides like a “gas” in the words of Michaud, over all spheres of life. This dissolution of the aesthetic experience into the triviality of life itself radically transmutes the relationship between the spectator and the artwork. Indeed, it seems, at first glance, that the “aestheticization” of existence numbs the relationship of cult and adoration surrounding the artwork, thus minimizing its spiritual and mystical dimension. However, in a desacralized society in which God has disappeared, the artist’s role is, more than ever, profoundly esoteric and similar to that of the prophet. How, then, do postmodern artists convey the transcendental truths of human condition through dematerialized artworks? We may also wonder, considering the importance of spirituality in abstraction, if it is sustainable to view the dematerialization of the artwork as a culmination of the impulse of abstraction? If so, what would this reveal about man’s relationship with God in the postmodern era? In order to answer these questions, I will focus on the case of Christian Boltanski, a French contemporary artist whose artwork broaches the main metaphysical problems surrounding human condition: those of life and death. Although his artwork is grounded in life, it acts as an abstraction mechanism whose goal is to transcend the sensible realm in order to ascend to an intelligible reality. Therefore, instead of liberating himself from the contingencies of mimetic reality in the manner of  XXth century abstract artists, Boltanski frequently recurs to objects and photographs in his installations, paying special attention to the authenticity of those objects.
\end{abstrct}

\begin{abstrct}
\begin{center}\textbf{Popular music articulates our conceptions of romantic love}

Aida Roige

\emph{Universitat de Barcelona (invited student)}
\end{center}

\textbf{Keywords: }romantic love, metaphysics, popular music.

Which is the most common topic of popular music? You guessed it. But the music to which we are massively exposed not only reflects or captures things about romantic love; it actually shapes the way we think about love and relationships. This presentation will explain how hit songs can lead people to interpret certain feelings and actions in a particular manner, challenge certain conceptions, and present certain situations as inherently good or problematic.\newline\textbf{Does not give the constent to be recorded}

\end{abstrct}

\begin{abstrct}
\begin{center}\textbf{Better Learning: Leveraging the strength of children with dyslexia}

Maria Rauschenberger

\emph{PhD student UPF}
\end{center}

\textbf{Keywords: }Gamification; Dyslexia; Web; Serious Games; Learning.

The aim of this research is to show that a playful approach combined with music can help children with dyslexia to faster learn how to read and write than the current methods. 

Dyslexia is a learning disability which is caused by visual or auditory perception. To this day children with dyslexia are first detected by their bad school grades. Furthermore, overcoming dyslexia is a great effort for children with dyslexia and benefiting from their strength could support their learning process. 

Playing error-based exercises presented in a computer game was found to significantly improve the spelling skills of children with dyslexia in Spanish [1]. Since there are no similar error-based exercises for German, we adapted the method to German. We collected more than 1,000 errors written by children with dyslexia from different schools. After that, we created a classification of dyslexic errors in German and annotated the errors with different language specific features, such as phonetic and visual features. For the creation of the exercises we took into account the linguistic knowledge extracted from the analyses and designed more than 2,500 word exercises in German that have been integrated in a game available for iOS (https://itunes. apple.com/de/app/dyseggxia/id534986729?mt=8 ) [2, 3]. 

My envisioned web application will contribute to the development of assistive and motivating technology that improve the accessibility to the information society to around 10\% of the population.



[1] L. Rello, C. Bayarri, Y. Otal, and M. Pielot (2014). A Computer-based Method to Improve the Spelling of Children with Dyslexia. ASSETS, (pp. 153–160). 

[2] Rauschenberger, M., Füchsel, S., Rello, L., Bayarri, C., \& Gòrriz, A. (2015). A Game to Target the Spelling of German Children with Dyslexia. ASSETS, (pp. 445–446). 

[3] Rauschenberger, M., Füchsel, S., Rello, L., Bayarri, C., \& Thomaschewski, J. (2015). Exercises for German-Speaking Children with Dyslexia. INTERACT, (pp. 445–452).
\end{abstrct}

\begin{abstrct}
\begin{center}\textbf{ADAT: a promising target for protein synthesis regulation}

Helena Roura Frigolé; Àlbert Rafels Ybern; Marta Rodríguez Escribà

\emph{IRB Barcelona}
\end{center}

\textbf{Keywords: }Translation, tRNA, anticodon, ADAT, inhibitor.

Proteins are important building blocks of our body with a crucial function not only in our organism’s architecture, but also in metabolism and information processing. Protein synthesis is a tightly regulated process and its defects are linked to different diseases. The instructions to make proteins are contained in genes, composed of codons. To synthesize a protein, each gene is transcribed into a messenger RNA (mRNA), also composed of codons. Each codon is specifically recognized by its complementary anticodon in a transfer RNA (tRNA), which is charged with the codon-specific amino acid. Thanks this codon-anticodon recognition the gene’s message is translated into a sequence of amino acids that forms the protein. 

Modifications in tRNA molecules directly affect the translation process and play a key role in its regulation. Our research is focused on a specific enzyme called Adenosine Deaminase Acting on Transfer RNA (ADAT) that produces a modification in the anticodon of tRNAs. The resulting modified tRNAs can now recognize three codons instead of just one. This strategy has evolved only in a part of all the species (Eukarya) and has a crucial influence on protein synthesis.

We analyzed the distribution of codons dependent on ADAT activity in the human genome and found several gene candidates that are highly enriched in these codons. Strikingly, many of these genes encode for mucins, suggesting that mucins expression could be regulated by ADAT enzyme. To prove this, we are studying the effect of ADAT in cell lines highly expressing mucins, which should be more sensitive to the lack of ADAT.

Certain mucins are overexpressed in severe human diseases such as certain cancers, cystic fibrosis or asthma. Thus, ADAT inhibition could potentially modulate their translation and might constitute a promising tool to tackle these pathologies. For this reason, we are working on the development of new selective human ADAT inhibitors that act as modulators of translation.
\end{abstrct}

\begin{abstrct}
\begin{center}\textbf{Microelectronic biosensors: from the lab to the pocket}

Pablo Giménez Gómez; Maria Mallén Alberdi

\emph{Instituto de Microelectrónica de Barcelona (IMB-CNM), CSIC}
\end{center}

\textbf{Keywords: }biosensors, microelectronics, malolactic, wine fermentation, bacterial detection.

The measurement and control of different parameters is required in many fields as water quality control, health, food industry, etc 



Traditional analytical procedures for the determination of these parameters use costly equipment and are time consuming because the analysis must be carried out on external laboratories. In order to determine these parameters, real time and on site monitoring is required. In this context, biosensor devices emerge as a real alternative. Biosensors show some advantages compared to conventional methods, such as low detection limit, high selectivity and specificity, and the use of compact, cheap and portable equipment. On the other hand, the use of microelectronic technology allows working with very small volumes and thus reduces the reagents consumption for the analysis.



In this speech, two examples of biosensors are reported. The microelectrodes were fabricated at Instituto de Microelectrónica de Barcelona (IMB-CNM) under clean room conditions. Microfabrication allow us to obtain lab-on-chip devices for integrating all laboratory requirements in a portable chip (a pocket sized device).



One of the devices presented is used for the L-lactic acid determination in red wines during the malolactic fermentation process. The presence of L-lactic acid improves the sensorial qualities of wine and contributes to the chemical and microbiological stability. Therefore, the control of the L-lactic acid concentration can be used as a quality indicator of the final wine.

Finally, it is showed another type of biosensor based on impedance which permits the direct measurement of bacteria. It can be coupled to a flow of liquid to take continuous measurements and determine bacterial contamination.



Both devices are examples of how biosensors are revolutionizing the world and how they are making easier our lives.
\end{abstrct}

\begin{abstrct}
\begin{center}\textbf{HISTÉRESIS DE NUTRIENTES Y SÓLIDOS SUSPENDIDOS DURANTE EVENTOS DE LLUVIA EN UN RÍO INTERMITENTE DE CLIMA MEDITERRÁNEO DE CHILE}

Katherine Brintrup

\emph{Universidad at Concepción, Chile}
\end{center}

\textbf{Keywords: }Histeresis, nutrientes, sólidos suspendidos, eventos de lluvia y río intermitente..

Los eventos extremos de lluvia generan el transporte de grandes cantidades de nutrientes y sólidos suspendidos (SS) desde la cuenca circundante hacia los cuerpos de agua, así como también pueden alterar los procesos de transformación de los nutrientes dentro de los ríos.

Estos eventos pueden contribuir con una gran cantidad de carga de nutrientes y SS hacia las cuencas, ya que pueden ser transportados a través de los flujos superficiales y subsuperficiales, siendo liberados hacia los cuerpos de agua adyacentes. Los eventos de lluvia pueden dar lugar a episodios de contaminación que pueden permanecer desde minutos hasta días, lo que implica contaminación del agua con graves consecuencias ambientales, ya que se alteran las características físico-químicas de los cuerpos de agua y zonas costeras adyacentes.



La comprensión de las vías de transporte de nutrientes y SS durante los eventos de lluvia pueden profundizarse estudiando las relaciones que existen entre el caudal y los parámetros de calidad de agua que se investigan, esta relación se llama histeresis, la cual no sólo sirve para determinar el funcionamiento de las cuencas, sino que también puede ser una herramienta útil para el manejo de éstas.

En el presente estudio, se consideró la cuenca del río Lonquén, ubicada en la VIII región de Chile. La cuenca presenta clima mediterráneo, drena una superficie de 1075 km2, y es de tipo intermitente. Se realizaron muestreos de agua durante tres eventos de lluvia en la parte baja de la cuenca, las muestras fueron obtenidas cada dos horas, para determinar las concentraciones de fósforo total (TP), nitrógeno inorgánico y orgánico disuelto (DIN y DON) y SS. Los resultados indican que la histeresis es distinta en cada uno de los eventos de lluvia.
\end{abstrct}

\begin{abstrct}
\begin{center}\textbf{A brief introduction to the nanoworld}

Ana Conde Rubio; Laura Evangelio Araujo; Steven Gottlieb

\emph{IMB-CNM}
\end{center}

\textbf{Keywords: }nanotechnology, nanoscience, nanofabrication, block copolymers, plasmonics.

What is nanotechnology? Which implication has it in our lives? We can find it everywhere! Electronics, cosmetics, medicine, new materials etc. 

For instance, the miniaturization trend in many areas of science and technology demands the continuous improvement of the current patterning methods. Thanks to this progressive miniaturization, our society can enjoy multiple devices, such as mobile phones or computers that are increasingly faster, lighter and with more storage capacity and lower consumption. Nevertheless, this continuous trend to reduce the dimensions of a device is increasingly more difficult, because at the nanometer scale, the materials properties change. Block copolymers, which are polymeric materials formed by two organic molecules have the intrinsic property of self-assembly and form dense arrays of nanostructures with dimensions that are very difficult to get with the traditional lithographic techniques. Therefore, the aim of this research project is to combine this intrinsic property of these materials to self-assembly with the traditional lithographic techniques in order to achieve resolutions below 10 nanometers.

Another example is plasmonics, which refers to the science and applications of  metallic nanostructures which interact strongly with light giving rise to phenomena such as enhanced scattering or absorption or concentration of the electric field. Plasmonics has experienced a big growth thanks to the advances in nanofabrication methods, imaging tools and modeling resources which have allowed their implementation in applications such as photovoltaics, medicine, sensing, optics…We have been working in nanocup-shaped gold nanostructures fabricated by nanoimprint lithography. Our aim is to find nanostructures with outstanding optical properties and we work in simulations, tunning their geometry, combining different materials...to obtain the desired results.
\end{abstrct}

\begin{abstrct}
\begin{center}\textbf{The effect of network fluctuations on sensory processing}

Iñigo Romero Arandia

\emph{UPF}
\end{center}

\textbf{Keywords: }neuroscience, neural coding, variability, information, sensory processing.

Sensory neurons modulate their activity for different stimuli, and they are thought to encode information in this way. However, presenting the same stimulus several times also yields different responses in each trial. The origin and role of this trial-to-trial variability is unknown and a huge challenge in neuroscience. Learning about this noisy responses would be a large step towards understanding how information from the external world is encoded in the brain. An interesting property of this variability is that it is correlated among different neurons, meaning that at least part of this variability or "noise" is common. We wanted to exploit this fact studying the effect that fluctuations in the activity of the whole network (population activity) have on the sensory processing of individual neurons and ensembles of neurons.  

For example, we can study how pushing water from one side or the other changes the position of some balls in the sea. The response in each trial will vary, as the state of the sea is not always the same. We aim at getting some information about the movement of the balls analyzing the impact that common effects, like tide, have on them.

We faced this problem analyzing multi-electrode recordings (~100 neurons) from monkeys' primary visual cortex (V1) while they were stimulated with gratings in different orientations. In V1 there are neurons that modulate their response according to the orientation of the stimulus, so they are tuned to orientation. We found that this tuning is affected by the population activity in a multiplicative and additive manner, with different neurons showing different effects. Multiplicatively modulated neurons increased their information for large population activity, while the additive neurons lost information. However, overall the amount of information in the network is preserved, and thus the activity of the network only redistributes the information among different neurons.
\end{abstrct}

\begin{abstrct}
\begin{center}\textbf{Infrastructure-architecture overlaps}

Pablo Villalonga Munar

\emph{ETSAB. UPC}
\end{center}

\textbf{Keywords: }architecture, infrastructure, overlaps, intersection, scale, city.

In 1977 Charles and Ray Eames produced the film "Powers of Ten" for IBM. They focused on the relative size of things through an image sequence at multiple scales. This consideration of a frame containing more and more information, drawn from an increase of scale, can be transferred to architecture. 

Nowadays cities need to be thought taking into account not only the relative size of their elements but also their multiple information layers and the relationships between several factors. A way to observe this relative size of things and their contained heterogeneous information is to research encounters between infrastructure and architecture of the city: large-scale versus small-scale. In their intersection space is where there are overlapping dialogs between elements of different origins.

If we center on case studies from consolidated European cities −very different from those located in city's outskirts− the complexity of data relationships amassed in those intersections increases significantly. Moreover, if  we consider some selected places where segments of elevated linear infrastructures −such as bridges, viaducts, etc− cross consolidated city areas, the spatial and scalar contrast found define their urban mutations.

For example, Sants railway in Barcelona and Borough Market in London are clear case studies in which there is a cohabitation of infrastructure and architecture forced by an unplanned urban situation.  Research of the intermediate space between large and small scale is hold from architecture as an scenario of urban life that gathers other knowledge fields and interconnected dimensions which are compared simultaneously at different scales and in an interdisciplinary way. 

Hence, working with links within and between cases will generate a combined image of the different intersection's transformations between infrastructure and architecture in consolidated European city in order to find some project design strategies to face similar cases in the future.
\end{abstrct}

\begin{abstrct}
\begin{center}\textbf{Is religion at risk if explained by science? A new perspective from Japanese philosophy}

Carlos Andrés Barbosa Cepeda

\emph{Universitat Pompeu Fabra}
\end{center}

\textbf{Keywords: }Religion, science, existential knowledge, Nishitani Keiji.

Some religious people fear that a scientific explanation of science ends up undermining religion: if the latter has a natural origin instead of a supernatural one, would it not be proved that there is no God or gods, and more generally nothing but the natural world? Would not religion be then explained away, or at best reduced to a mere useful fantasy?



We can gain new insights into this problematic by going back to a 20th century Japanese philosopher: Keiji Nishitani. From him we can learn that religions face a big challenge, but do not need to fear. Indeed, they are challenged because science can keep finding natural explanations for phenomena that were traditionally considered as caused by gods or spirits, so that belief in such supernatural agents or forces will become less and less reasonable. But religions do not need to fear, since its essence is not, as usually supposed, belief of the supernatural. Rather, the core of religion is found in the pursuit existential knowledge: a knowledge of our own existence that puts us in contact with what reality is like and, this way, discloses a meaning to life. This is what mystics, sages and prophets have called wisdom and seeked throughout history.



Concerning all this, I would like to argue that, from Nishitani's philosophy, it is possible to argue for three points. One: This wisdom can be achieved even without beliefs in the supernatural. Two: in order to achieve such existential wisdom, we need to break through our the intellectual understanding of things that grounds ordinary as well as scientific knowledge. Three: Consequently, religion goes beyond what science can do, but far from clashing, they can cooperate fruitfully. In a word: religion can still thrive in the age of science and critical thinking.
\end{abstrct}

\begin{abstrct}
\begin{center}\textbf{Studying Solar Energetic Particle Events and their implications on Space Weather}

Daniel Pacheco Mateo

\emph{Universitat de Barcelona - ICC}
\end{center}

\textbf{Keywords: }Space Weather, Solar Physics, Heliophysics, Solar Storms.

I am a member of the Heliospheric Physics and Space Weather group of the University of Barcelona. My research is focused on understanding how Solar Energetic Particle (SEP) events evolve and extend in the interplanetary medium in order to model the particle radiation profiles due to those events at a different solar distances (from 0.2 to 1.6 astronomical units). For that reason, I compare several events observed from different positions seeking for general trends in their flux profiles regarding their solar origin, particle energy, intensity and duration.

This modelling is necessary, for instance, to determine how much radiation should the shielding of satellites resist or to know which radiation dose could the astronauts absorb during the whole duration of a manned mission to Mars.
\end{abstrct}

\begin{abstrct}
\begin{center}\textbf{Photons, electrons and vibrations in atomically thin flatlands}

Mathieu Massicotte; Kevin Schädler

\emph{ICFO - The Institute of Photonic Sciences}
\end{center}

\textbf{Keywords: }graphene, 2D materials, optomechanics, optoelectronics.

Since the discovery of graphene in 2004, two-dimensional (2D) materials have evolved from being an exciting new playground to study fundamental interactions of light and matter to a promising field for applications in many areas of technology.



Our work focuses on using these novel materials as an interface between light and electronics, allowing for the development of novel optical sensors. In particular, an interesting approach is to assemble different 2D materials in vertical heterostructures and thereby combine their properties in a single stack. Such devices can convert light into an electric signal with high speed and efficiency, making them promising candidates for next-generation optoelectronic applications.



On the other hand, these atomically thin systems can be used to make nano-mechanical resonators with ultra-low mass and high resonance frequency, allowing them to be used as extremely sensitive detectors of forces and displacement. We use nanoscale light as a probe to detect and study the motion of these tiny resonators. This kind of system is of interest for fundamental studies of the interaction between light and motion at the nanoscale and to study the quantum behaviour of a macroscopic object.
\end{abstrct}

\begin{abstrct}
\begin{center}\textbf{Let's go to the nano: Scanning Probe Microscopies}

Montse Lopez Martinez; Berta Gumí Audenis

\emph{IBEC-ESRF}
\end{center}

\textbf{Keywords: }SPM, nanotechnology, AFM, STM,  biosystems.

Scanning Probe Microscopies (SPMs) are a type of microscopy that can obtain images of a surface up to the atomic level. SPM covers lots of different techniques, all of them based on a tiny probe that scans the sample in a really precise way. The most commonly used SPMs are the Atomic Force Microscopy (AFM) and the Scanning Tunnelling Microscopy (STM).  AFMs work by measuring the interaction force between the probe and the sample, whereas STMs work by measuring the current between both parts. With both of these techniques, we can get not only high resolution images but also obtain information about the mechanical and electrical properties of a sample at the atomic level.

One of the most important advantages of these techniques is that both AFM and STM can work at different environmental conditions:  air, liquid, vacuum. Consequently, these techniques have a large variety of applications in very different fields: from biology to molecular electronics, passing by medicine or material science.

In our laboratory, we work with both AFM and STM applied to bio-systems. We mainly study lipid membranes and several proteins that play an essential role for life. We look at the tiniest details of them, and try to figure out how they work so nicely.
\end{abstrct}

\begin{abstrct}
\begin{center}\textbf{Drought drives DOM biodegradability in intermittent streams}

Astrid Harjung

\emph{Departament d'Ecologia, Universitat de Barcelona, Barcelona, Spain}
\end{center}

\textbf{Keywords: }Dissolved organic matter, intermittent rivers, fluorescence properties, aquatic ecosystem, interface.

In lotic ecosystems drought periods strongly influence the availability of dissolved organic matter (DOM) in terms of quantity and quality. It is essential to investigate the link between DOM properties and ecosystem functioning under this hydrologic condition.  It is expected that the increase of water residence time, as a consequence of drought, will enhance the transformation of DOM and the contribution of autochthonous relative to allochthonous DOM.

 An intensive sampling program of surface and hyporheic waters coupled to continuous measurements of dissolved oxygen (DO) and DOM was performed in a pristine Mediterranean intermittent stream in order to prove this hypothesis.



Drought in geomorphological diverse streams is not just causing a loss of water and hence source of DOM, but also enhancing other factors of production and transformation. Consequently, it is important to take different structures of the river bed into account when studying DOM.
\end{abstrct}

\begin{abstrct}
\begin{center}\textbf{Rummaging through your rubbish: an evolutionary approach to interpret the amphorae production in the Roman Empire}

Maria Coto-Sarmiento

\emph{Barcelona Supercomputing Center}
\end{center}

\textbf{Keywords: }Cultural evolution, Roman Empire, amphorae.

The aim of this study is to analyze the evolution changes of amphorae to understand the production dynamics in the Roman Empire. In particular, cultural evolution approach will be applied to the material culture study because it is considered an useful tool to understand the variability of the mechanisms of changes.



This analysis can be used to detect differences in the amphorae production through time that could explain this dynamic of change. However, one of the main problems of this research is the lack of a formal framework to apply on the conventional techniques for the analysis of the amphorae dataset. In this case, it will be presented a research project where cultural evolution provides a capacity to detect cultural changes in the production of olive oil amphorae. 



Specifically, our case of study has been focused to understand the dynamic of changes of the olive oil amphorae production found in Baetica (currently Andalusia) during the Roman Empire (Ist-IIIrd century AD). To achieve this goal, phylogenetic and statistical analysis were applied to distinguish pottery assemblages among different kinds of shapes that could be used to identify discontinuities in archaeological and historical sequences. The changes detected by these methods allow to quantify the rates of changes in the amphorae production mechanisms. In particular, we want to identify if these changes were produced by cultural reasons as they may be economical, political and social developments.



 Therefore, cultural evolution theories can be used successfully for the interpretation of the change processes in the material culture to difference to the classic taxonomy. The main results suggest that different factors can influence changes and that changes will be more or less likely depending on them.
\end{abstrct}

\begin{abstrct}
\begin{center}\textbf{Nanomaterials based miniaturized sensors for electroanalysis}

Martha Raquel Baez Gaxiola

\emph{Universidad Politécnica de Cataluña}
\end{center}

\textbf{Keywords: }Nanomaterials, Sensors, Electrochemistry, Water pollutants.

This work deals about the design of nanotechnology based miniaturized sensors for electroanalysis of water pollutants. Water analysis is a very important issue nowadays, due to the growing pollution of water bodies and also the international regulations about it, for this reasons it is currently very important to work in the development of devices that allow us to monitoring water pollutants in situ and in short periods of time. Combining the characteristics of nanomaterials and the versatility of the electrochemical techniques we are able to fabricate analytical systems for this purpose.  This devices are intended to be in a small scale, fabricated with simple and low cost techniques and materials.
\end{abstrct}

\begin{abstrct}
\begin{center}\textbf{Saving the wild bees!}

Jane Morrison

\emph{UPC}
\end{center}

\textbf{Keywords: }wild bees, pollinators, native flowers, biodiversity, agriculture, agro-ecosystems.

Recent concerns about the global decline in wild bees, as a result of agricultural intensification, has called for more knowledge about the drivers of bee diversity and abundance in agro-ecosystems. Flowering native plants in field margins play a significant role in supporting biodiversity and ecosystem services. This study investigates the role of flowering field margins in supporting bee abundance and diversity in Mediterranean cereal agro-ecosystems, with a landscape perspective. The experiment is carried out from 2014-2016, at approximately 30 margins in Catalonia, Spain, comprising bee sampling, observations of pollinator foraging activity and plant inventory. Relationships between margin characteristics, landscape and bee populations are modelled. Further functional relationships between the morpho-physiological features of bee species and associated floral traits present in the margin are analyzed. This research could provide important knowledge and strategies for maintaining wild bee populations in farmland, maintaining stable crop production without increasing costs for farmers. The importance of conservation programs to provide flowering habitat to support wild bees and other pollinators will be highlighted. In general this work aims to promote a shift from conventional agricultural management to more sustainable farming, and more robust agro-ecosystems.
\end{abstrct}

\newpage
\section{Posters}

\begin{abstrct}
\begin{center}\textbf{Meta-analysis reveals that transforming perennial rivers into intermittent rivers will be a major driver of biodiversity loss}

Maria Soria Extremera

\emph{Grup de Recerca F.E.M, Departament d'Ecologia de la Facultat de Biologia. Universitat de Barcelona}
\end{center}

\textbf{Keywords: }macroinvertebrates, temporary rivers, biodiversity, anthropogenic, flow intermittency, meta-analysis.

Despite research on intermittent rivers (IRs) has increased during the last decade, no studies have been done so far that provide a general overview of the biodiversity of IRs. Therefore, there is still controversy about whether IRs host more or less biodiversity than permanent rivers (PRs). Our aim was to determine if biodiversity in IRs differs from PRs, and how biodiversity in both river types is influenced by several factors, such as climate, catchment area, sampling season, taxonomic group, sampled habitat, and the level of anthropogenic disturbance. 

A meta-analysis was conducted on 68 published papers that compared biodiversity in PRs with that of IRs, 48 with replicated data and 20 with non-replicated data. Richness means and standard deviations were extracted from both river types in replicated studies, and effect sizes were obtained using Hedge’s g. Publication bias on the replicated studies was visually analysed by applying funnel plots. Because of the heterogeneity of the studies, a random effects model was applied on replicated studies to obtain the weighted mean effect size and its confidence interval. A forest plot was used to illustrate the individual and overall results of the model. Finally, publication bias and random effects models were also applied splitting studies by each factor and all the corresponding categories. 

Overall, biodiversity was significantly greater in PRs than IRs, which was a large effect size. Among the factors that did not show publication bias, the B, C and multiple general climates, multiple catchment areas, autumn, multiple and summer seasons, macroinvertebrates taxonomic group, multihabitat samples, and medium anthropogenic disturbance had a significant difference and a positive effect size with and without trim-and fill methods, showing a greater diversity in PRs than IRs. Our meta-analysis suggests a worrying scenario: global change is increasing the intermittency of PRs in many regions.
\end{abstrct}

\begin{abstrct}
\begin{center}\textbf{Intelligent Decision Support System for optimizing Rehabilitation Systems}

Alireza Mozaffari

\emph{UPC PHD researcher}
\end{center}

\textbf{Keywords: }Intelligent decision support system, User profiling, Data sensor, Adaptation, Rehabilitation, Gait.

In Rehabilitation Systems (RS), the amount of support to the patients should be adaptable according to users’ profile. This is because the clinical characteristics of RS users over time may have changed, so the amount support should be adapted accordingly. To perform the proper amount of aid, we present an Intelligent Decision Support System (IDSS) to clinicians providing exercise guidance. This IDSS is able to recognize the current user profile by evaluating and processing the raw data collected from stroke patients by RS sensors, and to suggest the proper RS configuration to clinicians in order to give a better and a more effective service to the patients. The estimations are based on analysis of medical history that medical personal cannot possibly process. Such systems cannot substitute the medical specialist but the information that system provide is extremely useful as an independent source of evidence concerning the correct diagnosis. In methodology part we propose a new technique aiming at building user profiles from the data recorded by sensors in user-system interaction sessions. These profiles would be constructed in order to see the most recent improvements/changes in stroke survivors` gait clinical status. The purpose is to establish a link as a platform, to use extracted data from sensors as input information in different events, and giving as output the corresponding user profile. This output will be used to estimate the expected needed aid and proper RS modification. Among rehabilitation systems, we pay attention to a specific RS, a walker type walking support system as the subject of this research. The domain of our proposed technology will be applied in an intelligent walker called i-Walker. The i-Walker is being used with post-stroke survivors in order to support the lack of power in movement and its configuration is already determined by therapists. The i-Walker rollator has been specially designed for people that can use it autonomously.
\end{abstrct}

\begin{abstrct}
\begin{center}\textbf{Public communication campaign for childhood obesity}

BEATRIZ GARCIA

\emph{UPF}
\end{center}

\textbf{Keywords: }media literacy, childhood obesity, emotions, advertising.

There is a disconnection between knowledge about the power (Heath, Motion and Leitch, 2009) and the praxis of different governments to tackle childhood obesity with public communication campaigns. Not only will be treated the concept of power, but also the vision: the ability wasted to observe powers and meaning systems, based on theories like framming, from an interpretive perspective. 

Forgetting in the projection of a public communication campaign for childhood obesity concepts such as self or identity and its relationship with society, means forget the audience for which the strategy is created.
\end{abstrct}

\begin{abstrct}
\begin{center}\textbf{Los perfiles profesionales de egresados en posgrados mexicanos}

TERESITA DE JESÚS MÉNDEZ REBOLLEDO

\emph{UNIVERSIDAD VERACRUZANA MÉXICO}
\end{center}

\textbf{Keywords: }graduados, transiciones, posgrado, competencias, capital humano, inserción laboral.

A nivel mundial durante el siglo XX, se experimentaron cambios estructurales en materia económica, bajo el concepto de la sociedad del conocimiento y en estos últimos años con el de sociedad de la innovación. Los países invierten en formación de capital humano e investigación, lo que se hace evidente en el número de publicaciones, descubrimientos, tecnología, patentes y bases de datos que generan. Las universidades e instituciones de educación superior comprometidas con el desarrollo social, desde todas las áreas de conocimiento, busca perfilar su oferta educativa fundamentándola con pertinencia y calidad. En tal sentido, basados en la política de ciencia, con un proyecto iniciado desde la Universidad Veracruzana en México, se analizaron las transiciones académicas y laborales de graduados: la inserción laboral de los egresados al trabajo académico y científico orientado a la producción y difusión del conocimiento, las acciones de vinculación y la formación de capital humano; su propia inversión en capital humano, en relación con la orientación profesionalizante o de investigación del plan de estudios de posgrado cursado. Para lo cual, se usó una metodología de corte cuantitativo con test estadísticos y se determinó con base en la inserción laboral al trabajo académico y científico el comportamiento institucional por niveles de posgrado, comparando las variables: perfil profesional (áreas de conocimiento, nivel de posgrado, orientación), con la inversión en su capital humano, la producción del conocimiento, la difusión del conocimiento, las acciones de vinculación y la formación de capital humano. Los resultados señalan: ¿En realidad los egresados desempeñan su práctica profesional desde una actividad laboral afín a su formación en cuanto a la inversión en capital humano e inserción al sector productivo? y ¿En qué medida reconocen los empleadores las competencias de los graduados?.
\end{abstrct}

\begin{abstrct}
\begin{center}\textbf{Narrativa transmedia y las experiencias inmersivas de la industria cultural de los videojuegos}

Maria Isabel Escalas Ruiz

\emph{Universitat de les Illes Balears/ Colaboradora del Grupo de investigación RIRCA (Representación, Ideología y Recepción en la Cultura Audiovisual, UIB)}
\end{center}

\textbf{Keywords: }narrativa transmedia, industria cultural videojuegos, experiencia inmersiva, mundo ludoficcional, audiencia, imaginario colectivo..

La narración se concibe como una actividad humana capaz de perpetrar en la creación de nuestro imaginario cultural a través de sus múltiples manifestaciones. Dentro de las nuevas formas narrativas contemporáneas se encuentran la Narrativa Transmedia (término acuñado por H. Jenkins, 2003 y recogido por Scolari, 2013), enmarcada dentro de la cultura de la convergencia, la sociedad cultural globalizada y los spreadable media (Jenkins, Ford y Green, 2013). La NT se entiende, de forma sintética, como una nueva forma de contar historias en el siglo XXI que se expande en medios distintos de forma independiente aunque nutriéndose de todos ellos y creando una experiencia inmersiva donde la audiencia asume un rol activo en el proceso.



La potencialidad y el crecimiento de la industria cultural de los videojuegos se enriquece de elementos de la literatura y del cine, además de promover y apostar por experiencias inmersivas, uno de los principios o identitik esenciales de las Narrativas Transmedia. Los videojuegos son construcciones culturales que generan vivencias narrativas y el sujeto participa como agente activo (Frasca, 2003; Juul, 2005; Ruiz-Collantes, 2008: Planells de la Maza, 2013 y Navarro, 2015). En los mundos ludoficcionales creados, los videojuegos otorgan un papel principal a los prosumidores accediendo a través de su imaginario colectivo (Masoliver y Solà Arguimbau, 2004 y Zipes, 2012) y el cerebro emocional (Ferrés i Prats, 2014), apelando a nuestros sentidos implicados (Pratten, 2014).



Se llevarán a cabo el análisis de dos estudios de caso desde una perspectiva metodológica interdisciplinar, tomando como base los cuentos de Caperucita (C. Perrault) y Alicia (L. Carroll) y, especialmente, los respectivos videojuegos Woolfe. The Red Hood Diaries (2015) y Alice: Madness Returns (2011), susceptibles de ser considerados como productos culturales transmediáticos que apelan al imaginario colectivo y pretenden proporcionar experiencias inmersivas en la audiencia.
\end{abstrct}

\begin{abstrct}
\begin{center}\textbf{Another urbanism is possible: a sustainable urban growth model for the current liquid society}

Daniel Navas Carrillo

\emph{Predoctoral Researcher. University of Seville}
\end{center}

\textbf{Keywords: }sustainable urban growth, urban planning, participation, social needs, liquid society..

The main object of this work is searching for a model of growth of the city of Malaga based on participation and real social needs, in comparison to a totally defined model, whose possibilities of maneuver are completely limited. At the same time, the new periphery of Malaga is taken to question the model of unsustainable use of land that has resulted in the destruction of much of the natural heritage that formed the city limits at the beginning of the century.

In order to get a clear and logical sequence and achieve these objectives it has established four specific phases of development work:

-	Analytical Framework: Synthesis of the dynamics that have defined the growth process of Malaga, as well as the conditions that have resulted in its urban configuration. This must be followed by an analysis of the aspects that define and articulate the identity of the study area from the "sum of all the cultural, economic, social and technological aspects influencing the quality and planning of the city"

-	Conceptual Framework: An approach to the characteristics of today's society in order to guarantee the success of the proposed model. Its liquid condition - characterized by instability that causes loss of reference and certainties - besides the complexity derived from multiple lifestyles require that the city would be able to adapt to short-term needs according to the specific context of each intervention.

-	Purposing Framework: Development of an inventory of possible intervention strategies, ultimately, to develop a theoretical model of growth, which could form the basis for drafting the specific planning for this area. That tries to establish a system that would be able to regulate the catalogue of possible alternatives.

The challenge is not easy, but at a time marked by a serious economic, social and environmental crisis, achieving an economy and a more efficient use of limited resources, is one of the main objectives that society must face it today.
\end{abstrct}

\begin{abstrct}
\begin{center}\textbf{Nejire contributes to regulate feeding, growth and molting in hemimetabolous metamorphosis}

Ana Fernandez Nicolas

\emph{UPF}
\end{center}

\textbf{Keywords: }CBP protein, Nejire, metamorphosis, Blattella germanica.

Nejire (Nej) (a term coined for Drosophila melanogaster) is a CREB-binding protein, or CBP, that acts as a transcriptional coactivator, interacting with a large number of developmentally important transcription factors. CBPs are recruited to DNA by several transcription factors, including CREB and cFos.

CBP/p300 were reported to have histone acetyltransferase (HAT) activity. HAT activity is associated with gene activation through the modification of chromatin structure. CBP are found in a complex with a third family of HATs, the hormone receptor coactivators SRC and ACTR. HATs can act in concert at or near single promoters.
\end{abstrct}

\begin{abstrct}
\begin{center}\textbf{Stable high-efficiency Silicon photocathodes protected with Atomic Layer Deposited TiO2 for solar hydrogen production}

Carles Ros Figueras

\emph{IREC}
\end{center}

\textbf{Keywords: }Silicon, CO2, Hydrogen, Solar Fuel, Photoelectrochemical, Photocathode.

An important approach towards an efficient and sustainable economy is storing solar energy into chemical fuels through photoelectrochemical (PEC) water splitting. For a cleaner CO2 recovery, hydrogenation is a feasible process to store the solar fuels into methanol or other related products, but it is very important to obtain hydrogen from a clean source instead of hydrocarbon steam reforming.

In this work, we present a route to protect from degradation a high throughput water splitting silicon photocathode. For tandem PEC cell, silicon is a great candidate to work as low light energy photocathode in a non-biased cell configuration. Using silicon with an electrolyte is a challenge which faces important stability drawbacks due to an easy spontaneous oxidation. We demonstrate that titanium dioxide grown by ALD can be used as a transparent protective and conductive layer on a p/n+ Silicon electrode.

It presented long term stability for over 24 hours while maintaining an excellent electron

transport for water reduction with negligible potential losses due to a proper conduction band alignment between n-type silicon, TiO2 and hydrogen evolution reaction. With 19 mA/cm2 (Ti underlayer is blocking 40\% of incident light) at 0.4 V vs RHE and almost 0.6 V open circuit potential with 100 mW/cm2 AM1.5 light, a Fill Factor of 0.72 can be obtained, meaning an excellent conversion from an industrial level photovoltaic cell into a robust efficient photocathode.
\end{abstrct}

\begin{abstrct}
\begin{center}\textbf{Molecular characterization in positively selected amino acid substitutions in mammalian rhodopsin evolution.}

Miguel Antonio Fernández Sampedro

\emph{UPC}
\end{center}

\textbf{Keywords: }Rhodopsin, protein, molecular, evolution..

Visual rhodopsin is a member of the G-protein coupled receptors superfamily. This membrane protein functions as dim light photoreceptors in the rods of vertebrate retina. Specific amino acids have been positively selected in visual pigments during mammal evolution, which, as products of adaptive selection, would be at the base of important functional innovations. We have analyzed the top candidates for positive selection at the specific amino acids and the corresponding reverse changes (F13M, Q225R and A346S) in order to unravel the structural and functional consequences of these important sites in rhodopsin evolution. We have constructed, expressed and immunopurified the corresponding mutated pigments and analyzed their molecular phenotypes. We find that position 13 is very important for the folding of the receptor and also for proper protein glycosylation. Position 225 appears to be important for the function of the protein affecting the G-protein activation process, and position 346 would also regulate functionality of the receptor by enhancing G-protein activation and presumably affecting protein phosphorylation by rhodopsin kinase. Overall, these findings provide a deeper insight into specific amino acids involved in rhodopsin molecular evolution and they may, at the same time, have implications for rhodopsin molecular evolution theories of early mammalian nocturnality, where these mammals would have changed their visual patterns, during the Mesozoic era, in order to avoid competition with diurnal reptiles. Our results represents a link between the evolutionary analysis, which pinpoints the specific amino acid positions in the adaptive process, and the structural and functional analysis, closer to the phenotype, making biochemical sense of specific selected genetic sequences in rhodopsin evolution.
\end{abstrct}

\begin{abstrct}
\begin{center}\textbf{Investigating dynamics and synchronisation in climate data}

Dario Zappalà

\emph{Universitat Politècnica de Catalunya}
\end{center}

\textbf{Keywords: }climate, network, cross-correlation, time series.

In this work we investigate climate interactions from the point of view of phase synchronisation. We analyse Surface Air Temperature (SAT) time-series in 10512 grid points over the Earth’s surface.[1] By using the Hilbert transform, from each time-series we first extract a phase and then calculate phase autocorrelations. The autocorrelation map reveals the geographical regions where the phase dynamics has longer memory. In a second step, we use the cross correlation (CC) as a measure of phase synchronisation. By using a uniform threshold, from the CC matrix we build an undirected network, which is compared to the climate network constructed in the usual way from the raw SAT data (without using the Hilbert transform to extract the phases). Our work is motivated by the recent demonstration of optimal network inference when the similarity analysis is performed over phase time-series.[2] In a third step, we consider nonuniform thresholds and keep in each geographical location only the links with the highest SSM values. In this way, we build a directed climate network that allows identifying in each geographical region the most relevant inward teleconnections.



[1] Monthly-averaged reanalysis data from the National Center for

Environmental Prediction/National Center for Atmospheric

Research (NCEP/NCAR).

[2] G. Tirabassi, R. Sevilla-Escoboza, J. M. Buld'u and C. Masoller,

Inferring the connectivity of coupled oscillators from

time-series statistical similarity analysis, Sci. Rep. 5, 10829

(2015).
\end{abstrct}

\begin{abstrct}
\begin{center}\textbf{A Virtual Environment for Fostering Socialization in Children with Autism}

Ciera Crowell

\emph{UPF}
\end{center}

\textbf{Keywords: }Autism, Full-Body interaction, virtual reality, special needs, interaction design, embodied cognition.

Autism Spectrum Disorder (ASD) is a neurodevelopmental condition that manifests itself in abnormalities in communication, social interaction, stereotypical or repeating behavior, and limited interests. In the lens of the embodied cognition theory, which places the body as a key part of the cognitive process, full-body interactive environments, such as the project Lands of Fog, could be particularly useful as technological learning environments for children with ASD. Lands of Fog is a full body interactive environment for fostering social interaction in high functioning children with ASD. While demonstrating motivation to play in this virtual environment with a typically developed partner, children with ASD were able to practice and learn positive visible social interaction attitudes, as shown by their increased level of social interaction over the course of playing sessions.
\end{abstrct}

\begin{abstrct}
\begin{center}\textbf{Clustering Optimization for Resource Efficiency}

Georgios Kollias

\emph{Iquadrat Informatica/UPC}
\end{center}

\textbf{Keywords: }short-range communication, optimization, clustering, resources, LTE-A.

The booming demands of cellular users due to introduction of smart devices, along with the ever increasing number of mobile subscriptions, bring operators’ capacity to their limits. In this framework, a debate on whether this amount of users could be considered as part of Radio Access Network (RAN) by operators, has initiated. In that sense, the usage of short-range communications (Device-to-Device (D2D)), among adjacently located users that communicate through good quality links, could be a decisive step towards facing the need for higher capacity and improved data rates.



Serving users located at the cell edge, can be a difficult task due to factors that affect the received service and demand the consumption of valuable resources. Therefore, we propose clustering of users under the coordination of a user, denoted as Cluster-Head (CH), which is characterized by a good connection with the overlaid base station. Inside these clusters, Cluster-Members (CM) transmit their uplink traffic to the CH, which subsequently forwards it to the overlaid base station. In the opposite link, the CH receives the downlink traffic of all the CMs from the base station, and then forwards it to the corresponding CM. The main objective of the proposed clustering algorithm is the reduction of the resources’ utilization, and hence the increase of the achievable capacity.



To evaluate the performance of our proposal, both intensive simulations and analytical models were used. Concerning the analytical analysis, optimization theory was used in order to form the optimal number of clusters of users. In order to validate our proposal, an algorithm was presented and verified through simulations in our system level simulator.



Our proposal can significantly decrease the utilization of resources thus enhancing capacity of cellular networks. Furthermore, the implementation of our clustering algorithm results in reduction of the traffic imbalance between the downlink and the uplink direction.
\end{abstrct}

\begin{abstrct}
\begin{center}\textbf{Network virtulization in next generation cellular networks}

Georgia Tseliou

\emph{PhD student at UPC}
\end{center}

\textbf{Keywords: }cellular networks; network virualization; 5G; telecommunications; heterogeneous deployments.

The management and provision of services in current mobile networks cannot match the demanding requirements imposed by the end user applications. In particular, next generation cellular deployments will have to become heterogeneous and denser so as to meet these demands, thereby posing new challenges (e.g., deployment cost, energy consumption, cooperation of stakeholders, etc.). It is precisely within this context that network virtualization has been proposed as an efficient approach for multiplexing resources across different networks and services.



Network virtualization aims at optimizing resource utilization according to the various requirements imposed by different applications. Meanwhile, it effectively reduces network operation costs as the virtual network providers operate over the physical ones. Thus, different network/service operators only need to focus on the management of their own processing logic.



Based on the expected future requirements, main goal of this research is to propose solutions for implementing network virtualization in cellular environments. In the first place we exploit radio resource management principles and propose solutions that can be applied into the current standard to achieve on-demand delivery of resources. By leveraging the multi-tenancy approach, we design an algorithm where physical radio resources can be transferred in isolation among multiple heterogeneous base stations belonging to distinct operators. In addition, we identify the need to rework the Radio Access Network, beyond incremental evolution of the current standards. Within this context we focus on the design of algorithms in hybrid scenarios that can act partially centralized/distributed as a function of both traffic distribution and the radio resources. To that end, we study how to allocate capacity in a dynamic way to cope with the level of expected demand (e.g., by reallocating resources from areas where they are not needed to meet the demands in busier areas).
\end{abstrct}

\begin{abstrct}
\begin{center}\textbf{Analysis of spike correlations in periodically forced Semiconductor Lasers with Optical Feedback.}

Carlos Quintero

\emph{Departament de Física, Universitat Politècnica de Catalunya}
\end{center}

\textbf{Keywords: }.

Semiconductor lasers with optical feedback are excitable devices when operate in low

frequency fluctuations regime. We investigate how the dynamic of a laser process weak forcing

through a direct modulation of the pump current. We used the ordinal symbolic analysis to

study how the time correlations (between several consecutive laser spikes) change with the spike

rate. Our results show that higher spike rates wash-out the effects of the modulation in time

correlations. The variation of the probabilities of the symbols with the modulation frequency

allows to identify different noisy phase locking regimes. Simulations using the Lang-Kobayashi

model have good qualitative agreement with experimental observations.
\end{abstrct}

\begin{abstrct}
\begin{center}\textbf{Traffic Offloading in Future Mobile Heterogeneous Networks}

Panagiotis Trakas

\emph{Open University of Catalonia (UOC)}
\end{center}

\textbf{Keywords: }Traffic Offloading, small cells, 5G, Economic Framework, Business Plan.

The introduction of numerous services by Over The Top (OTT) content providers has led to a perpetual increase in mobile data traffic. This in turn has been the fundamental factor for the improvement of the telecommunications technology. Huge investments have been made by both the industry and the academia, for the improvement of the existing networks (4G) and design of future ones (5G) in order to serve the increasing traffic, and satisfy the users’ demands. The answer to this issue lies in the densification of the networks with low power nodes, also known as small cells, which can increase significantly the networks’ capacity. Due to the network and financial potential of the small cells, numerous third parties have emerged, aiming for a share in the telecommunications market from the typical Mobile Network Operators (MNOs). In this multifaceted market (i.e. MNOs, third parties, and OTT providers), we study various traffic offloading scenarios, the connections among the market stakeholders, and their effects on the users. Our objective is to provide financial incentives for the market players, while guaranteeing a satisfactory user experience.
\end{abstrct}

\begin{abstrct}
\begin{center}\textbf{Evidence of Drought Stress Memory in the Facultative CAM, Aptenia cordifolia: Possible Role of Phytohormones}

Eva Fleta Soriano

\emph{UB}
\end{center}

\textbf{Keywords: }dorught stress, phytohormones, Aptenia cordifolia,  memory, abcisic acid, CAM.

Although plant responses to drought stress have been studied in detail in several plant species, including CAM plants, the occurrence of stress memory and possible mechanisms for its regulation are still very poorly understood. In an attempt to better understand the occurrence and possible mechanisms of regulation of stress memory in plants, we measured the concentrations of phytohormones in Aptenia cordifolia exposed to reiterated drought, together with various stress indicators, including leaf water contents, photosynthesis and mechanisms of photo- and antioxidant protection. Results showed that plants exposed to drought stress responded differently if previously challenged with a first drought. Gibberellin levels decreased upon exposure to the first drought and remained lower in double-stressed plants compared with those exposed to stress for the first time. In contrast, abscisic acid levels were higher in double- than single-stressed plants. This occurred in parallel with alterations in hydroperoxide levels, but not with malondialdehyde levels, thus suggesting an increased oxidation state that did not result in oxidative damage in double-stressed plants. It is concluded that (i) drought stress memory occurs in double-stressed A. cordifolia plants, (ii) both gibberellins and abscisic acid may play a role in plant response to repeated periods of drought, and (iii) changes in abscisic acid levels in double-stressed plants may have a positive effect by modulating changes in the cellular redox state with a role in signalling, rather than cause oxidative damage to the cell.
\end{abstrct}

\begin{abstrct}
\begin{center}\textbf{Estudio del mecanizado por electroerosión en materiales intermetálicos de base FeAl}

Miguel Villagómez Galindo

\emph{UPC}
\end{center}

\textbf{Keywords: }Electroerosión, Intermetálicos, Mecanizado.

Los materiales intermetálicos presentan excelentes propiedades de resistencia a la corrosión y presenta un gran potencial en aplicaciones en sistemas de generación de energía mediante procesos que involucran alta temperatura, en aplicaciones aeroespaciales y biomédicas como son los microdispositivos electromecánicos (MEMS); sin embargo su mecanizado es complicado mediante el uso de las tecnologías convencionales.



Para ello se propone desarrollar un centro de micromecanizado por electroerosión en virtud que se trata de una máquina capaz de mecanizar piezas geométricamente complejas, materiales difíciles de mecanizar por un método convencional, como lo son: los materiales compuestos, las súper-aleaciones, los materiales cerámicos, fases de carburos, etc. Ejemplos de los materiales anteriores están los aceros resistentes al calor que se utilizan ampliamente en las industrias de moldes, en las áreas aeroespacial y aeronáutica.



Una vez desarrollado el centro experimental de micromecanizado, se realizarán evaluaciones de los parámetros o condiciones de corte del proceso para mecanizar un material intermetálico de base FeAl. Para ello se utilizará la metodología para diseño experimentos de Taguchi. Con los resultados del diseño de experimentos antes mencionado, se obtuvieron las condiciones de mecanizado que permitan reducir la rugosidad superficial de las piezas mecanizadas.\newline\textbf{Does not give the constent to be recorded}

\end{abstrct}

\begin{abstrct}
\begin{center}\textbf{Discrete-continuum hybrid modelling of granular material in flowing and static regimes}

ILARIA IACONETA

\emph{CIMNE}
\end{center}

\textbf{Keywords: }Computational Mechanics, Finite Element Method, Material Point Method.

Granular materials, during the last few decades have been object of study by a great number of researcher at academic and industrial level. Granular matters can be found in every aspect of the human life and its use is fundamental for its sustenance.

Either, most of the products commonly used can be considered made of 'particles' (e.g. thinking to the pharmaceutical or food  industry) or originally, before being processed, the raw material they are composed of was granular.

This explains why particulate materials are interesting and deep knowledge of them is demanded.



Sometimes the lack of information and knowledge of material's behaviour is cause of waste of time and money.

For this reason laboratory tests are performed and many researcher team are focused on the material modelling.

However it is not always possible to solve every problem conducting real tests, because they are too expensive or the scale effect makes them extremely difficult to reproduce.

Nowadays, in most of the cases, the study of the material's behaviour at a macroscale is performed by means of appropriate computational methods and constitutive relations.



The main objective of this work lies in the development of a numerical technique designed to be applied to practical engineering problems, involving dense granular flows, such as, for example, particles moving inside silos or hoppers, where a flowing and a static regime coexist. 

After an accurate review of the state of the art, the Material Point Method (MPM) has been identified as a suitable numerical method to achieve such objective.
\end{abstrct}

\begin{abstrct}
\begin{center}\textbf{Signed languages and the construction of grammars: the case of modality}

Maria Josep Jarque

\emph{University of Barcelona}
\end{center}

\textbf{Keywords: }cognitive linguistics, grammar, grammaticalization, language, modality, signed languages.

What is a language? What constitutes a grammar of a human language? How is it created? How does it arise? Which concepts/categories tend to be included in the grammar of a language? Which cognitive and communicative factors underlie grammatical meanings? Which mechanisms and linguistic material are involved in the process of grammar construction? What about signed languages? Do they share the same linguistic categories with spoken languages? Which are their main linguistic resources to express grammatical categories? 

These questions constitute the point of departure of my dissertation. In the last twenty years, researchers interested in the question of 'how do languages acquire a grammar' have elaborated a theory of grammaticalization, the process by which grammar is created (Bybee, 2010; Heine, Claudi, \& Hünnemeyer, 1991; Hopper \& Traugott, 1993; E. C. Traugott \& Dasher, 2002). However, this theory needs to be contrasted with data coming from the signed modality.

The focus of this dissertation is the expression of the grammatical category of modality in Catalan Sign Language (LSC) and its interaction with other grammatical categories, namely aspect, negation, person, evidentiality and time. It deals with the linguistic resources expressing modal meanings (volition, ability, obligation, permission, epistemicity, etc.) and with their gestural and linguistic sources.\newline\textbf{Does not give the constent to be recorded}

\end{abstrct}

\end{document}
