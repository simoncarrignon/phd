\documentclass[a4paper]{article}
\usepackage[margin=1.2in]{geometry}


%%%%lualatex on
%\usepackage{luatextra}
%\usepackage{fontspec}
%%Ligatures={Contextual, Common, Historical, Rare, Discretionary}
%\setmainfont[Mapping=tex-text]{Linux Libertine O}
%%Phd Description after meeting with Xavi on the 27th of July

\usepackage{natbib}
\usepackage{graphicx}


\title{Research Proposal}
\author{Simon Carrignon}

\begin{document}


\subsection*{Simon Carrignon}
\subsection*{\\PhD Research Description}

As part of the Economy and Political Network project (EPNet - ERC 340828 ), the PhD will try, as a general objective :
\begin{quote}
	``[\ldots]to investigate the political and economical mechanisms that characterised the dynamics of the commercial trade system during the Roman Empire.''
\end{quote}

Within this framework, the thesis will study the conditions of the emergence of a free market in a pre-industrial context, with the underlying aim to shed new light on the old debate in history and archaeology about the nature of the roman economy. Was it a fully integrated free market economy or a heterogeneous network of small independent economics loosely connected? In which extend the control of the Roman Empire facilitate or avoid the emergence of a free market?

To do so we will study the interaction between trade mechanism and the evolution of cultural artifact (ie. co-evolution of trade and culture). The study of cultural artifact seems to us the most promising approach as other economical records has been lost.


The first step of the PhD will consist on exploring the interaction between trade and culture in a theoretical way using a computer simulation. In a second time, we will confront the theoretical finding to archaeological and historical evidences from the Roman Empire. In a third part, we want to explore further  the mechanisms that drive cultural evolution using more complete and precise set of data to precise the previous theoretical studies.


\section{Theoretical model of co-evolution trade \& culture}
The first part of the PhD will be dedicated to develop a theoretical model to analyse under which cultural and economical mechanisms a free market could evolve and stay stable:. 

More precisely, we want to study the impact of the trade mechanism on the distribution of the cultural artifact, and in the way other, what is the impact of the cultural behavior of the population on the economical system.

For that we want to explore different paths:

\subsection{Trade mechanism}
Different economical model, different utility function (rational assumption, behavioral economics, new institutional economics, research equilibrium, oligopoly,\ldots)

\subsection{Cultural mechanism}
Different imitation and innovation processes (random copy, frequency dependant\ldots)

\subsection{Structure of the trade network}
What properties (density/sparsity,\ldots) of the network (cultural and/or economical network) allow a free market to evolve and to stay stable? What happen in heterogeneous network where the speed of the transmission of the information is not the same on every edge?


\section{Roman case study}
In a second part the thesis will focus on some aspects of the Roman Empire, and using archaeological and historical evidence, try to articulate those aspect to the theoretical study made before.


Could we study roman empire as an ecological system where specialisation evolve (specialised production sites, precise consumptions sites) allowing the installation free market? Could Niche Construction Theory be a good framework to study that?

\section{Study of innovation process using a case study}
In a last experiment we want to explore more in detail the impact of the innovation process on the evolution of the trade mechanism using a case study with more abundant and precise data.

The need of this study come from the fact that in cultural evolution innovation is often taken as random, as is mutation in biology. But it seems obvious that in many case, and moreover in cases involving economical trade, that the innovation process could not be only random. 

We want here to study, using actual corpus, how the innovation behaviour could impact the evolution of a market system and vice versa, how a trade system could transform innovation behaviors.

\end{document}

