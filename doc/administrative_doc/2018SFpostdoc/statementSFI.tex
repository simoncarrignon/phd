\documentclass[10pt]{article}   
\usepackage{hyperref}
\usepackage{color}
\usepackage{fontspec} 
\hypersetup{
    colorlinks=true,
    linkcolor=blue,
    filecolor=magenta,      
    urlcolor=cyan,
}
\setmainfont{ebgaramond}


\usepackage[scale=0.75,bmargin=.2cm,tmargin=.2cm]{geometry}

\definecolor{grey}{gray}{0.6}
\title{\textcolor{grey}{\Large Statement of Interest in SFI}\\
\vspace{-.25cm}
{ \normalsize Simon Carrignon (October 2018)}}
\author{}
\date{}




%\color{grey}
\begin{document}
\maketitle

\vspace{-1cm}

\textcolor{grey}{Input from other fields:}  Given my formation and experience, I can reproduce a various range of social and natural phenomena by modeling them as set of algorithmic instructions. I can simulate them, explore the results and learn more about the original system modeled.  
Those techniques have great advantages: one can finely describe all elements of the system at various scale, integrate any kind of knowledge about the real system, etc. Nonetheless, extracting valuable information can get tricky. When more parameters, theories and knowledge are integrated, the \emph{curse of dimensionality} makes hard to follow all underlying relationships. High Performance Computing, good empirical data, good knowledge of studied system and precisely defined research questions helps, but working with people with strong theoretical, mathematical and statistical background is priceless. It allows to avoid loosing time exploring useless or easily predictable part of the model and increase the quality and the robustness of the analysis. This is even more true when the person knows well how to apply and reframe his knowledge to heterogeneous and unrelated systems. SFI provides a unique environment to work side by side with the world best researchers with this mindset. This will allow me to build the perfect network of colleagues and increase my own skills.

\vspace{.5cm}

\textcolor{grey}{Workshop:} One workshop I always wanted to organize is one about the epistemological and historical foundation of evolutionary theories. It will highlight how most of the problems faced when using evolutionary theories to understand social systems are related to problems faced by Darwin himself and evolutionary biology since its very beginning. It should also illustrates how different modelling practices (statistical model historically --\emph{cf} Weldon and Pearson\cite{weldon1893certaincorrelatedvariationsincrangonvulgaris}-- more recently Bayesian Inference and computer models) and alternative visions of Science, have helped and still help to solve those problems.

\vspace{.5cm}

%\textcolor{grey}{Traditional Academic Environment?} Though my main technical background and skills is about programming and the use of computers to model things, I wont fit in a laboratory of Computer Science. I am not able to write code that will improve the speed of a given compression algorithm or that will parallilized in the biggest supercomputer of the world. Neither my knowledg eof congitive science or evolutionary biology will make me a great evolutionary biologist or Cogntitives sceintist. The models I write as well as the hypothesis I test \emph{are} simple. Taken separately they will hardly interest traditional computer scientist or biologist nor historian.  only interesting when taking together. 

%\vspace{.5cm}

\textcolor{grey}{Open and inclusive Science:} As member of the Scientific Community, I feel as a duty that my work have to be available to all. That's why, following principle inherited from the Free Software movement in computer science, I make my work as open as possible. My code is written under open source licences and my presentations as well as the data generated by my models, are Creative Commons. I am deeply convinced Science as to be as open and shareable as possible. Those  are central requirements for the development of a knowledge robust, reproducible, and accessible to all, on which future generations can build. 
Another way to make the knowledge produced by science more accessible is to teach and popularize it. This is why I was we created an association in Paris, ``l'Academie du Robot'', where we organized workshops for young children to teach them basic principles of programming using robots. This ultimately led us to publish a paper on how Robots can be use to teach and learn: \cite{gaudiello2010representations}. In the same direction I co-animated a podcast on Artificial Life with my colleague and friend David Medernach: \href{http://vie-artificielle.com/}{vie-artificielle.com} (in french). We ended up writing in a collective book on Science: ``La Science \`a  contrepied''\cite{collectif2017science} where we wrote a short introduction to Artificial Life coupled with a Youtube Video available here:~\href{https://www.youtube.com/watch?v=HBhrmlXNqM4}{youtu.be/HBhrmlXNqM4}. 
All this aims to broader the audience of Science. Assuring that scientific knowledge and methods remain accessible and understandable to everyone is crucial if we want the scientific community to become more inclusive and diverse. The risk is high to stay stuck in a closed ring where only limited amount of people have the adequate background to enter academia, by deciphering its underlying codes, language and objectives.


%\vspace{-.5cm}

\bibliographystyle{unsrt}
\bibliography{../../biblio/bib/SimonCarrignon.bib}                   
\end{document}
