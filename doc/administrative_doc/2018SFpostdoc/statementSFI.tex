\documentclass[10pt]{article}   
\usepackage{hyperref}
\usepackage{color}
\usepackage{fontspec} 
\hypersetup{
    colorlinks=true,
    linkcolor=blue,
    filecolor=magenta,      
    urlcolor=cyan,
}
\setmainfont{ebgaramond}


\usepackage[scale=0.75,tmargin=.2cm]{geometry}

\author{Simon Carrignon, October 2018}
\title{Statement of Interest in SFI}
\date{}



\definecolor{grey}{gray}{0.6}

%\color{grey}
\begin{document}
\maketitle

\textcolor{grey}{Input from other fields:}  Thanks to my formation and practice, I can now reproduce a various range of social and natural  phenomena by model them as algorithmic instructions, simulate those model and explore the results of those simulation to learn more about the original system modeled.  
Algorithmic descriptions and computer simulations have great advantages: they allow to finely describe all elements of the system, at various scale, and integrate any kind of knowledge about the real system. Nonetheless, extracting valuable information from those finely tuned models can get quickly tricky. When more parameters, theories and knowledge is integrated to the model, the \emph{curse of dimensionality} makes it hard to control all details.  Good empirical data and precisely defined research questions can help avoid those pitfalls and limit the scope of enquiry, but I also learnt by experience that working with someone with strong theoretical, mathematical and statistical background helps a lot. It allows to avoid loosing time exploring useless part of the model that simple theories can predict, and it greatly increase the quality and the robustness of the analysis. The Santa Fe Institute provide the best environment to work side by side with this kind of people and by doing so increase my own  skills with this regard.

\vspace{.3cm}

\textcolor{grey}{Workshop:} One workshop I always wanted to organize would be on the epistemological and historical foundation of evolutionary theories. Moreover, how most of the problem faced when trying to use those theories to understand social system are not so different than the problem faced by Darwin himself and evolutionary biology in general.  I would love to discuss how different model practice (computer model in particular but also statistical model, \emph{cf} Weldon and Pearson) and different vision of science and theories have helped facing those problem and could better account for evolutionary theories and their application at different levels. 

\vspace{.3cm}

\textcolor{grey}{Open and inclusive Science:} As member of the Scientific Community, I feel as a duty that my work have to be available to all. This is why, following principle inherited from the Free Software movement in computer science, I make my work as open as possible. I realise my code under open source licences and my presentations under Creative Commons and similar licence, as well as the data generated by my models. I am deeply convinced that Science would greatly gain by following a path of more open and shareable science. Transparency and shared are central requirement for the development of a robust knowledge, accessible to all and on which future generation can build. 

Another way to make the knowledge produced by science more accessible is to teach and popularize it. To do so I was at the initiative of an association with colleagues of mine in Paris, where we organized workshops for young children to teach them the basic principle of programming in a engaging way, using robots. This ultimately led to a paper in the proceeding of one of the main conference of the field: \cite{gaudiello2010representations}.

I also co-animate a podcast on Artificial Life with my colleague and friend David Medernach: \href{http://vie-artificielle.com/}{vie-artificielle.com} (in french). We ended up participating to a collective book on Science: ``La Science \`a  contrepied'' (cf~\href{https://www.belin-editeur.com/la-science-contrepied}{here}) were we wrote a short introduction to Artificial Life and published a Youtube video with the text, available here:~\href{https://www.youtube.com/watch?v=HBhrmlXNqM4}{youtu.be/HBhrmlXNqM4}. 

All those activities allows to broader the audience of Science. Assuring that scientific knowledge and methods remain accessible and understandable to the most people is a crucial if we want the scientific community to become more inclusive and diverse. The risk is high for science to stay stuck in a closed ring where only a limited amount of people have the adequate background to be able to decipher academia, its codes, language and objectives.




\bibliographystyle{abbrv}
\bibliography{../../biblio/bib/SimonCarrignon.bib}                   
\end{document}
