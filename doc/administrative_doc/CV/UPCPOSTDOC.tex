\documentclass[10pt]{article}   
\usepackage{hyperref}
\usepackage{color}
\usepackage{fontspec} 
\hypersetup{
    colorlinks=true,
    linkcolor=blue,
    filecolor=magenta,      
    urlcolor=cyan,
}
\setmainfont{ebgaramond}


\usepackage[scale=0.75]{geometry}

\definecolor{grey}{gray}{0.6}
\title{\textcolor{grey}{\Large Research Interest and Projects}\\
\vspace{-.25cm}
{ \normalsize Simon Carrignon (November 2018)}}
\author{}
\date{}




%\color{grey}
\begin{document}
\maketitle

\vspace{-.5cm}

Throughout my studies I followed an interdisciplinary path, studying Biology, Computer Science, as well as Cognitives sciences, History and Philosophy of Science to end up doing a PhD where I apply Cultural Evolution to study large scale social and economic changes. Meanwhile I worked  in various laboratories, on fields as various as Philogenetic, Neuroscience, Psychology, Evolutionary Robotics or Philosophy of Sciences. The general scientific question that motivate me are : 
\begin{itemize}
    \item How various systems made of simple interacting elements can, based on few rules and without supervision, diversify, adapt and change themselves as well as their environment?
    \item What are the links between human production (languages, technologies, social organization, computer abstractions) and natural systems? 
    \item If we can use artificial tools to understand the complexity of natural systems, is it only a coincidence or does it reflect something more profound on the nature of both things?
\end{itemize}
Those questions are abstract and answer them is probably unrealistic but following them during the past 8 years, led me to various publications where I studied: cooperation \cite{zibetti2015acaciaesanagentbasedmodelingandsimulationtoolforinvestigatingsocialbehaviorsinresourcelimitedtwodimensionalenvironments}, division of labor and specialisation\cite{montanier2016behavioralspecializationinembodiedevolutionaryroboticswhysodifficult,bredeche2017benefitsofproportionateselectioninembodiedevolutionacasestudywithbehaviouralspecialization}, trade in past society \cite{carrignon2015modelingthecoevolutionoftradeandcultureinpastsocieties}, evolution of lichens \cite{carrignon2016lichen} or even purely theoretical evolutionary dynamics \cite{medernach2015evolutionary,medernach2016evolution}.


In general those studies remained mainly theoretical. How to make the link between them and the real system has always been a challenging source of reflection for me.  How to do it? How to justify it? Does it have a sense to try it? This is why I did a  master in History and Philosophy of Science, where I wrote a thesis on ``Evolutionary Robotics as a model to study Biology of Evolution''. Since then, I try to keep my work as tightly linked to data as possible. As I learnt during my PhD this don't add a layer of complexity. In fact, it often simplifies the modeling and analysis activity. The main difficulty to keep model linked with reality, is to deeply understand the system of study and to work closely with the specialists. This is the only way to formulate meaningful hypotheses that make sens both as research questions within the field studied and as system that can be modeled and tested using computers. But following that, the modeling \emph{per se} becomes easier, and the analysis straightforward. 

This is what I did during my PhD. As I started working with historical and archaeological dataset and theories, the link between models and data became more and more important. As there is no strong theories nor consensus to guide what can be modeled and how, keeping this link is the only way to built meaningful models. The data (archaeological or historical) even if heavily biased, sparse and noisy, is the only thing that we have to ultimately test the models.

That's why I started to use Bayesian Inference. More precisely: Approximate Bayesian Computation. It allows to use any kind of models and compute their likelihood to be true given an \emph{a priori} knowledge and a dataset. Evolutionary Biologists use it to compare and select between different competing scenarios explaining distributions of genes \cite{beaumont2009adaptiveapproximatebayesiancomputation}. Recently it has been successfully use in a similar way to explain socio-cultural history of human civilisations \cite{rubiocampillo2016modelselectioninhistoricalresearchusingapproximatebayesiancomputation,kandler2017inferringindividuallevelprocessesfrompopulationlevelpatternsinculturalevolution}.

Using that method, and thanks to highly qualified Archaeologists, Historians and powerful supercomputing architecture,  I was finally able to loop through the full circle of knowledge production, in a way that looked much similar to what traditional experimental science is. From the proposal and formulation of hypotheses, to the preparation of an experimental setup to test those hypotheses, to the comparison and rejection of those hypotheses with regard to known evidences. Most of the recent work I did, and my full PhD, go in that direction. Most is still to be published but have been subject of various talks and posters \cite{coto2016exploringamphorabetica,carrignon2018abmtrac,romanowska2018jerash,carrignon2017impactofdifferentsociallearningmechanismsontheemergenceofawalrasianequilibrium,carrignon2018hpcmodel,carrignon2018}. 

I have now a good experience on using computer models to explore various range of phenomena that cannot be tested through experimental or analytical exploration. I rely on evolutionary theories as a general framework within which implement and interpret those models, and Bayesian Inference to statistically test, compare and select them given known datasets and evidences.

If evolutionary biology is already successfully using this framework, much more has to be done in Humanities and Social Sciences. The potential of applications is huge. 

As it appear evident with this letter, I may not qualify as a traditional network/complex system scientist --though one of my co-supervisor Sergi Valverde from Universitat Pompeu Fabra where I am doing my PhD is one of them and I am closely working with the team of Albert Diaz at Unviersitat de Barcelona. I have no background in physics and I have not a really strong math/formal background. Nonetheless I already worked with networks: I published a paper where we evolved simulated robots interacting on simple networks and I though and build the main model I use for my PhD to be tested with different network of social interaction (we tried some simple one but didn't have time to develop this part yet).... Nonetheless, in each case the network part remain really basic (comparing different simple topologies and different basic metrics), it gave me some experience handling (read/write/generate/analysis) networks in different programming language (R/python/C++).

Thus I may not be the candidate that one may think of to work in a department of physic in a group of Complex System and Network Science. But I think I know enough to make the link with different approach and more transdisciplinary studies that aim at linked social and ecological phenomena.

I have various ongoing projects that could be framed as cooperation problem and that I would potentially like to follow, though I am open to any new projects that aims at studying ``Cooperative phenomena in social and ecological systems'' and I am more than willing to work on ecological dataset and problems. Nonetheless I mention quickly my actual project in the next paragraphs as illustrations.
 

One thing I want to follow in a close future is the analyse of spread of news in online social media. 
It's a path I started with Alex Bentley a few month ago and we presented the first results here \cite{carrignon2018}.  
We developed and tested simple models able to reproduce the distribution of true and false retweets on the online social media Twitter. Incorporating networks (the real ones or theoretical ones) is the next step. Though we discussed it a lot it's still a missing part. Public agencies start to be aware that understanding those phenomena is important, more attention will be brought to that and more datasets will become available. I think that network science and models have an important role to play here.

On another hand I would like to keep working with historical data. They are the only alternative to compare with actual analysis. At the same time, quantitative works on those dataset are just starting. As an example, I digitalized and started to analyse archive on commercial boats entering and going out of the port of Trieste, from 1850 to 1910. This dataset gives us a general view on which country was sending which kind of good and where, when the world was starting to industrialized and get more and more globalized. It's a perfect dataset to test assumption  on how social, political and economic interactions can shape societies. We presented some preliminary analysis in We presented the first results here \cite{carrignon2016patternsinglobalization}.



\bibliographystyle{unsrt}
\bibliography{../../biblio/bib/SimonCarrignon.bib,../../biblio/bib/phd.bib}                   
\end{document}
