\documentclass[10pt]{article}   
\usepackage{hyperref}
\usepackage{color}
\usepackage{fontspec} 
\hypersetup{
    colorlinks=true,
    linkcolor=blue,
    filecolor=magenta,      
    urlcolor=cyan,
}
\setmainfont{ebgaramond}


\usepackage[scale=0.75]{geometry}

\definecolor{grey}{gray}{0.6}
\title{\textcolor{grey}{\Large Cover Letter}\\
\vspace{-.25cm}
{ \normalsize Simon Carrignon (January 2019)}}
\author{}
\date{}




%\color{grey}
\begin{document}
\maketitle

\vspace{-.5cm}

Throughout my studies, from my very beginning at the university to my actual PhD, I always managed to follow an interdisciplinary path, mixing Computer Sciences with Evolutionary Biology, Cognitive Sciences and History \& Philosophy of Science. This interest in multi-disciplinarity started long before entering the University when I witnessed the arrival of personal desktop in most families. As soon as we got one in mine, I was amazed by the ability of numeric artifacts (games, internet,\ldots) to exhibit ‘life-like’ behaviors. 
It was as if, by finding the right combination of instructions, it was possible to recreate, within the small square of the screen, the complexity of the living things and social phenomena that surround us.

Quickly I wanted to find this \emph{right combination of instructions}. To do so I knew I will need to understand the rules of the living things as well as the way to program their artificial counterparts.  This is why I started to study the fields I mentioned before, and finally did a PhD in Cultural Evolution. In parrallel, I worked in various laboratories speciliazed in bacteria's phylogenies, monkey neural system, human psychology, Evolutionary Robotics or Philosophy of Biology. And the more I learnt, the more the importance of Evolutionary Theory became evident and my genuine inquiry evolved into concrete questions: 
\begin{itemize}
    \item How systems made of simple interacting elements can, based on few rules and without supervision, diversify, adapt and change themselves as well as their environment?
    \item What's the link between human productions (languages, technologies, social organization) and biological systems? 
    \item Is the fact that artificial tools can be used to understand the complexity of natural systems only the result of a coincidence or does it reflect something more profound on the nature of both things?
\end{itemize}
Those questions echo goals formulated by  Artificial Life, a field of research that inspired me even before going to the University. If they are abstracts and hardly answerable, they are the one driving my day to day research and interest. 
I spend the past 8 years exploring them using computer simulation and published on different related subject like: cooperation \cite{zibetti2015acaciaesanagentbasedmodelingandsimulationtoolforinvestigatingsocialbehaviorsinresourcelimitedtwodimensionalenvironments}, division of labor \& specialisation\cite{montanier2016behavioralspecializationinembodiedevolutionaryroboticswhysodifficult,bredeche2017benefitsofproportionateselectioninembodiedevolutionacasestudywithbehaviouralspecialization}, trade in past society \cite{carrignon2015modelingthecoevolutionoftradeandcultureinpastsocieties}, evolution of lichens \cite{carrignon2016lichen} or even purely theoretical evolutionary dynamics \cite{medernach2015evolutionary,medernach2016evolution}.
And while doing so I've never seen those questions as well formulated as within the field of Evolutionary Robotics. To illustrate this and explain how I am here, what I learnt on the way and why I apply to this postdoc position, I'm a afraid I have to detail the curriculum I briefly sketched before. This will probably be too long but I will try to summarize it as much as possible without loosing important information.

After my undergraduate study, where I learnt the basics of biology and computer science, I did a Master in Natural \& Artificial Cognition at the ``Ecole Pratique des Hautes Etudes'' in Paris.  There I started to work with Elisabetta Zibetti on a project using robots (Khepera III) to test human heuristics extracted from experimental psychology. Though this was not my first experience in a lab  % While I learnt a lot about the world of research during those two first internships%
I felt I was going a step closer to the problem I presented at the beginning. When before computers were simple \emph{tools} (powerful calculator for  phylogeneticists, precise controller for neuroscientists), here the robot (and the code controlling it) was more than that; it was a \emph{model} for human cognition, a model that one has to study as a cellular biologist study a mouse. For various reason out of my control this project didn't went well and ended without concrete results. Nonetheless it allowed me to learn a lot about robotics and its reactive school, sensor and motor control,\ldots and to publish with another PhD student protocols and experiments we developed to teach robotics to young children \cite{gaudiello2010representations}.

In the meantime I discovered Evolutionary Robotics during a class given by Nicolas Bredeche. I realised it was even more close to what I wanted to explore and I decided to finish my Master doing that. Nicolas Bredeche accepted to work with me and we started to collaborate. Together with his PhD student J.-M. Montanier we developed experiments to study the evolution of division of labor in swarm of autonomous agents. Using simple neural networks we left evolving online without direct fitness function on simulated Khepera-like robots we emulated different speciation scenarios inspired by Biology (sympatric vs parapatric). Each scenarios were implemented as different network topologies representing different mating networks. This work was published a few years after my Master in \cite{montanier2016behavioralspecializationinembodiedevolutionaryroboticswhysodifficult} and a follow up afterward in \cite{bredeche2017benefitsofproportionateselectioninembodiedevolutionacasestudywithbehaviouralspecialization}. %At the same time, and with the help of N. Bredeche we took back a project I started with E. Zibetti on the simulation of cooperation that I developed during the first terms of the master and that we published here \cite{zibetti2015acaciaesanagentbasedmodelingandsimulationtoolforinvestigatingsocialbehaviorsinresourcelimitedtwodimensionalenvironments}.

Then, convinced that Evolutionary Robotics was the best way to study the questions I presented at the beginning of this letter (and some other that we developed here \cite{bredeche11evolutionaryadaptationpopulationrobots}),  I decided to do a master in History and Philosophy of Science. My goal: get a  better understanding of the fundamental implication of our work. And while learning the history of Evolutionary Biology and the limits of applying computer models to study it, I wrote a thesis on ``Evolutionary Robotics as a model to study Evolutionary Biology''.  I then came back to N. Bredeche with whom we unsuccessfully tried to fund a PhD in Evolutionary Robotics. As I was not in the best financial position to wait more at this moment, I accepted the proposition made by J.M. Montanier to join his team as a PhD student at the Barcelona Supercomputing Center were he had moved for a postdoc meanwhile.

The topic of the project I joined may look far from Evolutionary Robotics, this ERC wanted to ``build formal model able to support a critical and alternative exploration of the Roman trade network'', but in practice, the methods were the same (and this is why JM. Montanier went there and convinced me to go). I spend the last 4 years developing and testing agent based model implementing evolutionary dynamics and game theory model  \cite{carrignon2015modelingthecoevolutionoftradeandcultureinpastsocieties}. 

One important thing changed though: when starting to work with historical and archaeological dataset and theories, the link between models and data became much more central. Where there is no strong theories nor consensus to guide what can be modeled and how, data is the only thing against which models can be tested. Unfortunately, in History, the relation between this data and the agent's behavior is far from being direct: the archaeological records is only very partially related to the original behavior. Moreover, dataset are heavily biased, sparse and noisy. Thus it is extremely difficult to have clear cut test to validate the models. 

This brought me to Approximate Bayesian Computation, a method that allows to compute the likelihood of any model given an \emph{a priori} knowledge and data. It is already widely used by Evolutionary Biologists to compare and select different competing scenarios to explain the actual distribution of genes \cite{beaumont2009adaptiveapproximatebayesiancomputation} and has been more recently successfully used to explain the socio-cultural history of human civilisations \cite{rubiocampillo2016modelselectioninhistoricalresearchusingapproximatebayesiancomputation,kandler2017inferringindividuallevelprocessesfrompopulationlevelpatternsinculturalevolution}. My actual PhD follow this path, using more precisely defined model that integrate game theory and cognitive psychology. 

Some of the last thing I did related to this are not yet publish but most of the code is already available online and has been presented in conferences:
\begin{itemize}
    \item \href{https://github.com/simoncarrignon/apemcc}{APEMCC}: A python model to study the change in pottery production techniques (presentation: \cite{COTOSARMIENTO2018}).
    \item \href{http://framagit.org/sc/twitter-spread/}{TWITTER-SPREAD}: A R model and ABC wrapper to study the spread of true and false information on online social media (presentation: \cite{carrignon2018}).
    \item \href{https://framagit.org/sc/abc-pandora}{CEEC-JERASH}: A python wrapper to use ABC methods with on a modified version of \cite{carrignon2015modelingthecoevolutionoftradeandcultureinpastsocieties} to explore change in ceramics use in the Roman East from 25BC to 170AD (presentation: \cite{carrignon2018abmtrac}).
\end{itemize}

Now that the end of my PhD is near I would love to come back to Evolutionary Robotics with my new baggage. If working on historical questions is great, as it questions our own past and offers hint on our own future, the degree of incertitude, though challenging, is often frustrating and can lead to weak conclusion. Moreover, the interdisciplinarity is not well developed yet. Various fields interact with incompatible objectives. Communication is difficult and the resolution of problem sometime impossible. In Evolutionary Robotics, since the first experiments at the beginning of the 90s and the explosion of the field in the early 00s, a great community with a coherent multi-disciplinary background has grown. People may have divergent opinions, they still share common views and great part of their background. This create an ideal context to work and develop good research.

However this doesn't mean I don't want to work anymore on those topics. I even think that Evolutionary Robotics is the perfect empirical playground where Cognitive and Evolutionary Theory can meet and be studied to solve problem that Cultural Evolution is facing.

On another totally different hand, I always stayed connected with the Alife community and Evolutionary Robotics. With my friend David Medenerach, with who we studied evolution of cellular automata \cite{medernach2015evolutionary,medernach2016evolution} we animate a \href{www.vie-artificielle.com}{podcast} on those topics, in which I mostly spoke about Evolutionary Biology and Evolutionary Robotics. We still work on new episodes and recently participated to a collective book ``La science à contrepieds'' where we wrote a short introduction to  Artificial Life and recorded a \href{https://www.youtube.com/watch?v=HBhrmlXNqM4}{video} (all podcast and video are in French).


All that been said, I am well aware of the limits of my candidature. I haven't worked in the field for some years and obviously have less related publications that any PhD student that stayed. Nonetheless I think and hope that my detailled background can illustrate how I still fit in, and how I can bring interesting new perspective firmly grounded epistemologically. Technically too, I guess my profile may be less interesting that the profile of a 'pure' engineer or computer scientist with a master degree in those discipline.  Nonetheless I am still writing my main models in C++ (so is the one of \cite{carrignon2015modelingthecoevolutionoftradeandcultureinpastsocieties}), I also write python models and have more than 6 years of experience working daily with grid and supercomputers, running bash and python scripts to manage, store and  explore millions of simulation while using fluently R to analyse and visualise the results. If it's also been a while since I haven't worked directly with physical Robots, I still work on my own projects that involve some raspberry pies, sensors, cameras and self-hosted servers.

Most of the code I write is split between \url{github.com/simoncarrignon} and \url{framgit.org/sc}, I already gave the main repositories but I will be more than happy to share anything else needed. 

It would also be my pleasure to answer any aditional question or precise any unclear point.\\

\vspace{.2cm}
\noindent Cordially,\\
Simon Carrignon


\bibliographystyle{unsrt}
\bibliography{../../biblio/bib/SimonCarrignon.bib,../../biblio/bib/phd.bib}                   
\end{document}

