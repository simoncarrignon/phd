\documentclass[10pt]{article}   
\usepackage{hyperref}
\usepackage{color}
\usepackage{fontspec} 
\hypersetup{
    colorlinks=true,
    linkcolor=blue,
    filecolor=magenta,      
    urlcolor=cyan,
}
\setmainfont{ebgaramond}


\usepackage[scale=0.75]{geometry}

\definecolor{grey}{gray}{0.6}
\title{\textcolor{grey}{\Large Research Interest and Projects}\\
\vspace{-.25cm}
{ \normalsize Simon Carrignon (November 2018)}}
\author{}
\date{}




%\color{grey}
\begin{document}
\maketitle

\vspace{-.5cm}

Throughout my studies I followed an interdisciplinary path, studying Biology, Computer Science, as well as Cognitives sciences, History and Philosophy of Science to end up doing a PhD where I apply Cultural Evolution to study large scale social and economic changes. Meanwhile I worked  in various laboratories, on fields as various as Philogenetic, Neuroscience, Psychology, Evolutionary Robotics or Philosophy of Sciences. The general scientific question that motivate me are : 
\begin{itemize}
    \item How various systems made of simple interacting elements can, based on few rules and without supervision, diversify, adapt and change themselves as well as their environment?
    \item What are the links between human production (languages, technologies, social organization, computer abstractions) and natural systems? 
    \item If we can use artificial tools to understand the complexity of natural systems, is it only a coincidence or does it reflect something more profound on the nature of both things?
\end{itemize}
Those questions are abstract and answer them is probably unrealistic but following them during the past 8 years, led me to various publications where I studied: cooperation \cite{zibetti2015acaciaesanagentbasedmodelingandsimulationtoolforinvestigatingsocialbehaviorsinresourcelimitedtwodimensionalenvironments}, division of labor and specialisation\cite{montanier2016behavioralspecializationinembodiedevolutionaryroboticswhysodifficult,bredeche2017benefitsofproportionateselectioninembodiedevolutionacasestudywithbehaviouralspecialization}, trade in past society \cite{carrignon2015modelingthecoevolutionoftradeandcultureinpastsocieties}, evolution of lichens \cite{carrignon2016lichen} or even purely theoretical evolutionary dynamics \cite{medernach2015evolutionary,medernach2016evolution}.


In general those studies remained mainly theoretical. How to make the link between them and the real system has always been a challenging source of reflection for me.  How to do it? How to justify it? Does it have a sense to try it? This is why I did a  master in History and Philosophy of Science, where I wrote a thesis on ``Evolutionary Robotics as a model to study Biology of Evolution''. Since then, I try to keep my work as tightly linked to data as possible. As I learnt during my PhD this don't add a layer of complexity. In fact, it often simplifies the modeling and analysis activity. The main difficulty to keep model linked with reality, is to deeply understand the system of study and to work closely with the specialists. This is the only way to formulate meaningful hypotheses that make sens both as research questions within the field studied and as system that can be modeled and tested using computers. But following that, the modeling \emph{per se} becomes easier, and the analysis straightforward. 

This is what I did during my PhD. As I started working with historical and archaeological dataset and theories, the link between models and data became more and more important. As there is no strong theories nor consensus to guide what can be modeled and how, keeping this link is the only way to built meaningful models. The data (archaeological or historical) even if heavily biased, sparse and noisy, is the only thing that we have to ultimately test the models.

That's why I started to use Bayesian Inference. More precisely: Approximate Bayesian Computation. It allows to use any kind of models and compute their likelihood to be true given an \emph{a priori} knowledge and a dataset. Evolutionary Biologists use it to compare and select between different competing scenarios explaining distributions of genes \cite{beaumont2009adaptiveapproximatebayesiancomputation}. Recently it has been successfully use in a similar way to explain socio-cultural history of human civilisations \cite{rubiocampillo2016modelselectioninhistoricalresearchusingapproximatebayesiancomputation,kandler2017inferringindividuallevelprocessesfrompopulationlevelpatternsinculturalevolution}.

Using that method, and thanks to highly qualified Archaeologists, Historians and powerful supercomputing architecture,  I was finally able to loop through the full circle of knowledge production, in a way that looked much similar to what traditional experimental science is. From the proposal and formulation of hypotheses, to the preparation of an experimental setup to test those hypotheses, to the comparison and rejection of those hypotheses with regard to known evidences. Most of the recent work I did, and my full PhD, go in that direction. Most is still to be published but have been subject of various talks and posters \cite{coto2016exploringamphorabetica,carrignon2018abmtrac,romanowska2018jerash,carrignon2017impactofdifferentsociallearningmechanismsontheemergenceofawalrasianequilibrium,carrignon2018hpcmodel,carrignon2018}. 

I have now a good experience on using computer models to explore various range of phenomena that cannot be tested through experimental or analytical exploration. I rely on evolutionary theories as a general framework within which implement and interpret those models, and Bayesian Inference to statistically test, compare and select them given known datasets and evidences.

If evolutionary biology is already successfully using this framework, much more has to be done in Humanities and Social Sciences. The potential of applications is huge. I have various ongoing project that I  could be seen   during my PhD that I want to follow during the coming years.

One started with the digitalisation and analyse of a new dataset on the industrialization of Europe before the World War I. We presented the first results here \cite{carrignon2016patternsinglobalization}, and we hope to successfully use the framework I described to show how globalization was one of the drivers of industrialization of Europe and how it ultimately led to war (I'll join our working paper on that with my application form).

I also hope to be able to link this with another industrialisation I studied during my PhD: the Roman Empire. My colleague Maria Coto-Sarmiento an I started to work on data and analysis she has done on amphora production\cite{COTOSARMIENTO2018117}. We hope to push forward this analyse. By using Agent Based Model and Bayesian Inference, we think we could infer more general properties on the path toward standardization of production in the Empire and the social structure that can make those changes possible.  This will allow to understanding better the links between industrialisation and globalization in a civilisation far from pre-WWI Europe and even more different from actual Europe.

In parallel, I am working with John Hanson on a model of urbanisation that we want to couple with the data he recently published on the scaling properties of cities in the Roman Empire \cite{Hanson20170367}. A comparative study under a general framework between Roman and Modern industrialisation would shed new light and bring more quantitatively and data driven arguments on important debates on urbanization, city growth and political organization.

Finally, a path I recently started and on which I want to focus during the coming years is the study of actual social phenomena.  One example is the spread of news online \cite{carrignon2018}. I started this with Alex Bentley a few month ago and we presented the first results here \cite{carrignon2018}. The data available for modern social dynamics are much more complete than the archaeological historical records.  I want to go ahead and apply my methods to understand modern social dynamics and compare them to past societies.  Moreover, as the actuality remind us every day, the rise and spread of fake news as well as the polarization of opinion are topics that need to be understand as soon as possible.  And I aim to do, the same way I aim to understand the rise and spread of cultures and the split and division of empires. 

All those projects are at different stage of advancement, but all have dataset and working models. All of them assume social change as an evolutionary process not so different to biological evolution. All of them rely on knowledge shared from various field that are not often brought together. All of them have networks, at various level and can be seen as cooperation problems.  It would be a great opportunity to follow on those projects and start new ones in a place such as IAST, which have been built for that,  where people from various fields can interact and learn from each other. I guess it would be one of the only place where I could do so.



\bibliographystyle{unsrt}
\bibliography{../../biblio/bib/SimonCarrignon.bib,../../biblio/bib/phd.bib}                   
\end{document}
