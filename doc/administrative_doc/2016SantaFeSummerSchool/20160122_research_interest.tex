\documentclass[10pt]{article}   
\usepackage{fontspec} 

\setmainfont{ebgaramond}

\usepackage[scale=0.75]{geometry}

\author{Simon Carrignon}
\title{Research Statement Interest}
\date{January 2016}




\begin{document}
\maketitle


\textbf{Throughout} my studies, from the very beginning of my time at the university continuing through to my actual PhD, I always managed to follow an interdisciplinary path, mixing computer sciences with evolutionary biology, cognitive sciences, philosophy of science and during the last year, history, economic and archeology.

This interest in multi-disciplinarity start long before entering the university, when young I look this amazing ability of numeric artifacts built by humans to exhibit ‘life-like’ or ‘human-like’ behaviors, and those incredible ‘natural' behaviors, so complex and well adapted. I quickly became determined to find a way to understand both entities: the living ones and their artificial counterparts. Since then all the courses I chose, all the schools I decided to attended, and all other choices I made were in pursuit of obtaining a better understanding it.

As I started to gather knowledge, this genuine inquiry quickly evolved into a concrete research quest: I wanted to understand what were the links between human production (language, social organization, computer program) and natural systems driven by Darwinian evolution. How they seem sometimes so similar? Why we can use artificial tools to understand the complexity of natural systems and why we can use the highly adapted design of natural systems to refine our artificial tools? How such systems can, based on simple rule and without supervision, diversify, adapt, change their environment as well as they change themselves? 

To answer those questions I did a first Master Degree in Natural \& Artificial Cognition in which my thesis focused on the study of the evolution of distribution of labor in swarm of autonomous agents. The main idea was to explore what are the properties of the reproductive network that permit the emergence of division of labor. We are finished the anlysis of this work which will soon be published. In parallel I studied the same kind of mechanisms but in simulated ‘cognitive’ agents designed with psychologists. The idea here was to explore what environmental conditions allow subpopulation of cooperative agents to co-exist with populations of selfish individual. This work has been published few months ago \cite{zibetti2015acaciaesanagentbasedmodelingandsimulationtoolforinvestigatingsocialbehaviorsinresourcelimitedtwodimensionalenvironments}.


In both case the results I found were valuable as themselves for me and provided a new understanding about crucial evolutionary processes in totally different contexts. But to convince people that this knowledge could be applied to real world entities and that it was valuable for the understanding of the world in general was not as straightforward. I realize that if I wanted myself to be able to produce meaningful, useful and concrete research projects and moreover if I wanted to be able to make the link between such projects and what more ‘traditional’ scientists were doing, I would need a deep understanding of the tool I use, the subject I study object, and the link between them. 

That decided me to engage myself in another Master Degree in History and Philosophy of Science. My work then focused on exploring what are evolutionary processes and how we can study them using computers. It allowed me to learn precisely the history and the building of Evolutionary Biology since Darwin and to dig deeper in the philosophical debates that occur throughout this history. I took also this opportunity to diversify the nature of the systems I wanted to study: I created the LaReMI Junior lab (http://en.laremi.net/), financially supported by the Ecole Normal Superieur (Lyon France) with the idea of applying similar approach (simulation coupled with simple cognitive experiments) to study the evolution of music melodies. It was my first contact with what is called ‘cultural evolution’, but moreover allow us to organized an international workshop about the topic in Lyon (http://en.laremi.net/actvity/meeting\footnote{Regarding to the auto-hosting solution we adopt for our server, the loading of those pages could take time}) and to present our approach to the community in Lisbon \cite{carrignon2013whyapply}.

After those two complementary Masters, I pursued the exploration of such questions in the scope of my actual PhD. The idea here is still to use simulation to try to understand the conditions of evolution of particular dynamics in a more or less decentralized system. But this time I choose to study the evolution of cultural and economic network during the Roman Empire. An historical question about a human activity in a project \footnote{ERC grant EPNET: www.roman-ep.net} involving historians, archaeologists as well as database specialists. My work focus on trying to understand the conditions of emergence of a decentralized market. The idea is to give tools to measure if such conditions were present during the Roman period and if not, what kind of economy we should expect to find. To do so I developed a simple cultural evolution simulation where cultural transmission mechanisms lead to economical changes, that in turn modify the cultural dynamics. I am exploring what kind of properties the cultural network need to exhibit in order that a stable decentralized economy evolved and I already presented the computational framework at the Winter Simulation Conference \cite{carrignon2015modelingthecoevolutionoftradeandcultureinpastsocieties} and will present in Oslo \cite{carrignon2016coevolutionofcultureandtradeimpactofculturalnetworktopologyoneconomicdynamics} the first results we have with networks.

 Those past 4 years and my actual research project can be summarized as the exploration of how systems made up of huge number of entities (homogeneous or not), highly decentralized and mostly unsupervised, can evolved powerful adaptive properties such as division of labor, cooperation, specialization and learning. And thanks to my master in Philosophy of Science I also developed in parallel to my scientific activity a deep concern about the epistemological relevance of what I am doing. Computer Simulation and Complex Systems Analysis are powerful tools but they could quickly mislead the research and bring the researcher to nonsense and useless explorations. Moreover, the questions raised by such approach quickly go behind the simple empiric inquisition and require skills that allow to rethink and redefine the traditional epistemic baggage into framework far from the ones scientists are use to work in.


In every projects I was I saw how mature such approach is getting. Trying to understand decentralized and unsupervised evolving system \emph{per se} is providing knowledge in a wide range of different area. It is not anymore a marginal object of distraction. With new generation of physicists, biologists, sociologists, economists and even historians able to truly understand such methods, concrete hypotheses are formulated and can be tested. With computer and program always more powerful, and with scientist like me with a strong transdisciplinary background able to use those tools and to make the link between the questions, the methods and the results, new discovery can be made about topic that were even unthinkable before. 


This gives us an amazingly exciting and wide open area of research where a lot remains to do.  I will continue to explore it. Among other things, I will continue to explore the properties of the networks that allow the emergence of particular properties (such as cooperation, division of labor, specialization\ldots) but I want to go farther and study: how such network can themselves evolve? Another huge track of research I want to pursue is to study in what extend the evolution of those properties depend on the abilities of the system to interact with its environment (using developmental mechanism, simple learning, cultural transmission,\ldots). 

In both case I want first of all to tighten again the link between what I am doing, the empirical data and the scientists working to try to extract meanings from those data. To do so I will to continue to find and work with people from different field but with concrete, real and complex case study. Because I think that the most valuable and beautiful striking knowledge don't lie in computational simulation we can do or in the mathematical model we extract from it, neither in the simple analysis of the raw data and the description of such analysis, but emerge from the well articulation of both side.



\bibliographystyle{abbrv}
\bibliography{../../biblio/bib/SimonCarrignon.bib}                   
\end{document}
