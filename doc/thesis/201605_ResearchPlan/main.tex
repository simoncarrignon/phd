\documentclass[a4paper]{article}
\usepackage[margin=1in]{geometry}


%%%%lualatex on
%\usepackage{luatextra}
%\usepackage{fontspec}
%%Ligatures={Contextual, Common, Historical, Rare, Discretionary}
%\setmainfont[Mapping=tex-text]{Linux Libertine O}
%%Phd Description after meeting with Xavi on the 27th of July

\usepackage{natbib}
\usepackage{graphicx}


\title{Research Plan}
\author{Simon Carrignon}

\begin{document}

\vspace{-1cm}
\hspace{-1cm}
\begin{minipage}{.5\textwidth}
    \noindent \includegraphics[width=5cm]{images/upfLogo.jpeg} 
\end{minipage}
\hfill
\begin{minipage}{.5\textwidth}
    \flushright
    \textbf{ \Large RESEARCH PLAN }
\end{minipage}

\vspace{3cm}

\hspace{-1cm}
\fbox{

    \begin{minipage}{\textwidth}
	\vspace{.2cm}
	\noindent\textbf{Student:} Simon Carrignon
	\vspace{.5cm}

	\noindent\textbf{Project Title:} ``Cultural evolution and long term economic dynamics: The case study of Rome.''
	\vspace{.5cm}

	\noindent\textbf{PhD thesis supervisors:} Xavier Rubio-Campillo \& Sergi Vavlerde
	\vspace{.5cm}

	\noindent\textbf{Research Group:} Barcelona Supercomputing \& Center Complex System Lab (UPF)  
	\vspace{.2cm}
    \end{minipage}

}
\vspace{2cm}

\noindent\textbf{\LARGE Research Plan: }


\section*{Introduction}
%\subsection*{Cultural Evolution}
In this thesis we are interested to study how social traits evolve. By social traits we mean every thing that are culturally transmitted (vs genetically) and/or can be socially learnt. Those traits often exhibits similar patterns of adoption and evolve given dynamics that are widely studied by people in what it is usually called: \emph{Cultural Evolution}.

People in that field of research try to understand:\\

\begin{minipage}{\textwidth}
    $\rightarrow$ What mechanisms drive the evolution of such traits?\\
    $\rightarrow$ What mechanisms generate such pattern?
\end{minipage}
\vspace{.5cm}

A lot has been written on such mechanisms and like \cite{bentley2005specialisationandwealthinequalityinamodelofaclusteredeconomicnetwork} have shown than a mechanism as simple as a random copy of what already explain can mimic the pattern of adoption of various social traits such as the names of the people, the dog's breed preferences, etc\ldots  

%\begin{figure}
%    \centering
%    \includegraphics[width=10cm]{images/powerlawrepartition.jpg}
%    \caption{Square: male names,Circle: female names, Dotted and plain lines: model result with different copy probabilities.From Bentley et al,~2004.}
%
%\end{figure}
%\subsection*{Co-Evolution of Culture and Trade}
In this thesis where are interested on what happen when those mechanism of adoption act upon social traits that are more or less indirectly linked to economical activities. If the social trait at sake is the taste of people for different kind of wine, the general cultural evolution of the most common taste of a population could have a deep impact at a production and trade level, as wine producer will have to follow the new habits of commotion of the people. But while adapting to new consumption's habit, the producers will change the market as some wine previously expensive will started to be less expensive and so one, leading to a tightly linked retroaction loop.

We call those ``co-evolution of culture and economy''. We resume this as system with one cultural level where social interaction happen (social learning and cultural transmission through social networks) and an economic level, wher trade, production and consumption activities happen through economics and trade networks. In our particular case, the main idea is that the activity at one level impact the activity at the other level and vice versa (cf Fig.~\ref{fig:inter}). 

\begin{figure}[ht]
    \centering
    \includegraphics[width=8cm]{images/interaction}	
    \caption{Co-evolution of culture and economy}
    \label{fig:inter}
\end{figure}

%\subsection*{Approach}

To study this mechanisms this PhD thesis we focus on on particular case study: the economy of the Roman Empire. We think that to often study of cultural dynamics focus on materials coming from internet social network that are not representative of general cultural dynamics. For that reason we want to explore a case study that relying not only on a technology used by a subset of a subset of the actual population but that can be compared throughout time and space: between different cultures and at different historical time.

We think that archaeological record and historical studies are the best suited approach in this regard. Moreover, as the PhD thesis is fully part of the EPNet Project \cite{remesal2014epnet}, that aims to ``[\ldots]to investigate the political and economical mechanisms that characterised the dynamics of the commercial trade system during the Roman Empire'',we have access to huge amount and eminent specialists about the Roman Economy, which make the Roman Empire as the perfect case study. 

To complement this historical and archaeological approach, this PhD thesis largely rely on computer model and simulation. They offer the perfect tool to gather knowledge coming from widely spread field of research and is the perfect way to encompass a lot of problems raised by archaeology (cf \cite{wurzer2015agentbasedmodelingandsimulationinarchaeology}) and make the kind of cross-cultural and cross-temporal comparison that we want to do. Moreover, they allow us to easily make a back and forth between a purely theoretical study of the system we described before and empirical and data-driven study of the Roman Empire.


To achieve that we divide this PhD in 3 part. The first one aims to develop a theoretical framework based on computer model to study the dynamics of co-evolution of economy and culture and how the introduction of social traits that bear economics value can transform the dynamics of cultural evolution and how the cultural dynamics can impact economy. The second part is twofold: we want to find the appropriate proxy in the archaeological record that could help us to characterize economical dynamics acting during the Roman Empire and once find the good proxy, to measure it throughout the cities of the empire. The last part heavily depend on the two first part and remains yet speculative but that main idea would be to try to articulate the dynamics empirically measured with the model developed in the first part in order to see if the dynamics implemented in our model can explain the empirical patterns observed.


\section{Theoretical Exploration: How economics value change the dynamics of Cultural Evolution}
A first version of this model has been already presented and published in the proceedings of the Winter Simulation Conference \cite{carrignon2015modellingthecoevolutionoftradeandcultureinpastsocieties}. The main idea behind our model is to use a simple Agent-Based Model that allow use to easily bring together knowledge from history and archaeology with knowledge build in other fields (economy, cognitive sciences, biology\ldots). 

This will allow us to (1) explore and measure how the linkage of a social traits with an economic and trade can transform the dynamics of cultural evolution with regard to traditional model of cultural evolution (random copy, frequency biased copy\ldots) and (2) explore and compare hypothesis made by historians on the nature of the economy during the Roman Empire. 

\section{Empirical Study: Characterize an economy through socio-economic artefacts distribution, a case study the Roman Empire}
In that section our intuition is that following \cite{ortman2014theprehistoryofurbanscaling}, the way the distribution of some particular socio-economic artefacts (such as the diversity of producers of a given good, the complexity of the infrastructure\ldots) change between cities of different size could help us to characterize the economy that generate the pattern we observe. This approach, comming from studies in biology (cf \cite{schmidtnielsen1984scalingwhyisanimalsizesoimportant}) has already been successfully used to characterize nowadays cities \cite{bettencourt2010aunifiedtheoryofurbanliving,batty2013theory}. 

This could be the perfect proxy to (1) qualify in a measurable way some economics properties of the Roman Empire and (2) compare those properties with nowadays society and between different cultures and historical periods. Nonetheless, this approach will mostly depend on the data available and we hope to be able rely on the huge amount of data made available within the EPNet project and by comparing it to other available database. Anyway any other source of information will be used and any other case study could be envisaged. 

\section{Articulate the Theoretical Exploration \& the Empirical Observations}
In this section we aim to see in what extend the dynamics implemented in the model developed in the first pattern can explain the properties observer during the empirical exploration of the second part. As this part mainly rely on the result of the two first part it remains highly speculative. 
\bibliographystyle{unsrt}
\bibliography{../../biblio/bib/phd.bib}
\end{document}

