\documentclass[9pt, handout=show,notes=show]{beamer}
\usetheme[width=0cm]{Goettingen}
\usecolortheme{rose}
\useoutertheme{default}

\usepackage{hyperref}
\usepackage{fontspec} 
\setsansfont{Futura LT} %%COMMENT ÇA SE LA RACCONTE!!


\usepackage{arydshln}
\usepackage{amsmath}
\usepackage{alltt}

\usepackage{mathptmx}
\usepackage{latexsym}
\usepackage{mathtools}
\usepackage{multirow}


\title{Brief summary of the past two month}
\date{21 Janvier 2016}

\begin{document}

%%\begin{frame}{ Winter Simulation Conference}
%%    A lot of\dots\\
%%    \includegraphics[width=5cm]{images/op.jpg}
%%    \includegraphics[width=5cm]{images/optimization.jpg}
%%
%%    
%%\end{frame}
%%
%%\begin{frame}{But..}
%%    \begin{itemize}
%%	\item random interesting stuff (reflection around $p-values$, philosophical issues, ecological studies\dots)
%%	\item Frederic Amblard, Audren Bouadjio et al. : \emph{Which Models Are Used in Social Simulation to Generate Social Networks? A Review of 17 years of Publications in JASSS}
%%
%%
%%	\item ANR project :
%%	    \begin{center}
%%		\begin{minipage}{.9\textwidth}
%%		    \scriptsize
%%
%%
%%		    Philippe RAMIREZ, Project Coordinator, Anthropologist, Director of the Centre d’Études Himalayennes 
%%
%%		    Frédéric AMBLARD, Multi-Agent Simulation, Associate Professor, Université Toulouse I Capitole, IRIT
%%
%%		    Arndt BENECKE, System biology, École des Neurosciences de Paris (CNRS-UPMC)
%%
%%		    Vittoria COLIZZA, Biological physics, Institut Pierre Louis d'Epidémiologie (INSERM-UPMC)
%%
%%		    Benoît GAUDOU, Multi-Agent Simulation, Associate Professor, Université Toulouse I Capitole, IRIT
%%
%%		    Laura HERNÁNDEZ, Statistical physics, Associate Professor, Université de Cergy-Pontoise, LPTM.
%%
%%		    Stéphane LEGENDRE, Population dynamics, IR, Dpt Écologie évolutive, CNRS-ENS, Paris
%%
%%		    \textbf{Laure NUNINGER, Spatial archaelogy, CR, Chrono-Environnement, CNRS, Besançon}
%%
%%		    \textbf{Marie-Jeanne OURIACHI, Ancient history, Associate Professor, CEPAM, CNRS-Université de Nice}
%%
%%		    Mehdi SAQALLI, Geography, CR1, GEODE (CNRS, Toulouse II)
%%
%%		    Clara SCHMITT, Spatial systems modelling, Postdoctoral fellow, Géographie-Cités (CNRS, Paris) 
%%
%%
%%		\end{minipage}
%%	    \end{center}
%%    \end{itemize}
%%
%%\end{frame}
%%\begin{frame}{Nuninger \& Ouriachi 2011:ArchaEpigrapho}
%%    \begin{quotation}
%%	``In this approach, the epigraphical document, bearing evidences concerning - at the same time - the  political, social and cultural fields is considered in itself and in its context and also in a serial way.''
%%	\\(\emph{website})
%%    \end{quotation}
%%    \begin{center}
%%	\includegraphics[height=3cm]{images/hierarchi.png}
%%	\hfil \includegraphics[height=3cm]{images/hierarchyTime.png}
%%    \end{center}
%%    
%%\end{frame}
%%
%%\begin{frame}{Montreal: NIPS 2015, network workshop}
%% Hanna Wallach : ``Computational Social Sciences'':\\
%%
%% \begin{itemize}
%%     \item Latent Network :
%%	 \begin{center}
%%	     \includegraphics[height=2.5cm]{images/mailnet.png}
%%	 \end{center}
%%     \item Dyadic event analysis : ``country A did action U to country C''\\
%%	 \begin{itemize}
%%	     \item extract groups of ``actors'' and type of group of actors interacting with other groups of ``receptors'' with different kind of ``receptors'' and different relation between different groups of actors/receptors.
%%	     \item reconstruct and analyse geopolitical history
%%	   
%%	 \end{itemize}
%%     \item Bayesian latent variable models: $P(Y,\Omega | X,H)$
%% \end{itemize}
%%
%%
%%    
%%\end{frame}
%%
%%\begin{frame}{Montreal: Mc Gill}
%%    Georgres Grantham: Retired researcher from the good old ``cliometrics''time who recently wrote:
%%    \begin{quote}
%%	G. Grantham, A Search-Equilibrium Approach to the Roman Economy. Latomus \& Peeters, 2015.
%%    \end{quote}
%%    in :
%%    \begin{quote}
%%	K. Verboven and P. Erdkamp, Structure and performance in the Roman economy: models, methods and case studies. Latomus \& Peeters, 2015.
%%    \end{quote}
%%
%%
%%    \begin{itemize}
%%	\item Really interested in the project and found the article going in the right direction \dots
%%    \item Cities as proxy for economical development 
%%    \item Clermont-Ferrand studies of hierarchy between villae using photos
%%    \end{itemize}
%%    \includegraphics[height=3cm]{images/clermont.png}
%%
%%    
%%\end{frame}
%%\begin{frame}
%%   \begin{center}
%%       \includegraphics[width=\textwidth]{images/clermontPhoto.png}
%%   \end{center}
%%\end{frame}
%%
%%\begin{frame}{Montreal: UdeM}
%%Frederic Bouchard : Philosopher of Biology, Chair of Phylosophy, Vice rector of UdeM.
%%    \begin{itemize}
%%	\item draft of the MS7 Abstract.
%%	\item Jean François Gauvin ( Chaire de recherche du Canada sur les transformations de la communication savante) \&   Pr. Vincent Larivière (director of the Collection of Historical Scientific Instruments (CHSI) @ Harvard Semitic Museum) :\\
%%	    use epigraphic data from Semitic museum to study cultural (scientific ) evolution.
%%    \end{itemize}
%%
%%    
%%\end{frame}
%%
%%\begin{frame}{Montreal: UQAM}
%%
%%    10 minutes talk in ``L'oeuf ou la poule''( the egg or the chicken?) about ``Model in science''
%%    \begin{center}
%%	\includegraphics[width=.8\textwidth]{images/lolp.png}
%%    \end{center}
%%
%%\end{frame}
%%
%%\begin{frame}{MS7}
%%    Conference ``Models \& Simulations'' in Barcelona May 18th-20th, 2016\\
%%    \begin{itemize}
%%	\item deadline abstract the 22 December,
%%	    \vfil
%%	\item submitted with Ale and Bernardo (and Remesal)
%%	    \vfil
%%	\item answer in february
%%	    \vfil
%%    \end{itemize}
%%\end{frame}
%%
%%\begin{frame}{Actual Work}
%%    \begin{itemize}
%%	\item Manchester DACAS bursary: Data and Cities as Complex Adaptive Systems, by the Manchester Metropolitan University,Manchester School of Architecture.
%%	    \vfil
%%	\item Sante Fe Summer School: 2 pages statements of reacher interest (deadline tomorrow)
%%	    \vfil
%%	\item JIPI, The 2nd of February (cf María)\\
%%	    \begin{center}
%%		\includegraphics[height=4cm]{images/jipi.png}
%%	    \end{center}
%%	\item Dijon CompleNet
%%    \end{itemize}
%%    
%%\end{frame}
%%
%%\begin{frame}{CompleNet 2016}
%%    \begin{center}
%%	\begin{columns}
%%	    \begin{column}{.5\textwidth}
%%		\begin{center}
%%		    Small-World
%%		\end{center}
%%		\begin{column}{.2\textwidth}
%%			\tiny
%%			200 Nodes,\\ 
%%			400 Edges,\\
%%			Density:.02
%%		\end{column}
%%		\begin{column}{.8\textwidth}
%%		    \begin{figure}
%%			\includegraphics[width=\textwidth]{images/graphSW.png}
%%		    \end{figure}
%%		\end{column}
%%		\begin{center}
%%		    \includegraphics[width=.6\textwidth]{images/scoreSW.png}
%%		\end{center}
%%	    \end{column}
%%	    \begin{column}{.5\textwidth}
%%		\begin{center}
%%		    Scale Free
%%		\end{center}
%%		\begin{column}{.2\textwidth}
%%			\tiny
%%			200 Nodes,\\
%%			396 Edges,\\
%%			Density:.02
%%		\end{column}
%%		\begin{column}{.8\textwidth}
%%		    \begin{figure}
%%			\includegraphics[width=\textwidth]{images/graphSF.png}
%%		    \end{figure}
%%		\end{column}
%%		\begin{center}
%%		    \includegraphics[width=.6\textwidth]{images/scoreSF.png}
%%		\end{center}
%%
%%	    \end{column}
%%	\end{columns}
%%    \end{center}
%%\end{frame}






\begin{frame}{CompleNet 2016}
    \begin{figure}
	\includegraphics[width=.4\textwidth]{images/smallworld.pdf}
	\includegraphics[width=.4\textwidth]{images/scalefree.pdf}
	\caption{100 nodes, ~400 edges}
    \end{figure}

    What kind of cultural network --the network that people use to copy the best strategies-- allows the evolution of a decentralized  economy ? previous work [WSC 15] showed that it is working with a fully connected network.
    
\end{frame}

\begin{frame}{Other works}

    \begin{itemize}
	\item ACCEPTED: S. Carrignon, J.-M. Montanier, J. Michaud, and X. Rubio-Campillo, “Co-evolution of culture and trade : impact of cultural network topology on economic dynamics,” in 44th Computer Applications and Quantitative Methods in Archaeology Conference (CAA 2016) , Avril 2016.
	\item ACCEPTED: I. Morer, S. Carrignon, and X. Rubio-Campillo, “Influence of the topology of cultural networks on the equilibrium of an exchange-based economy,” poster in 7th Workshop on Complex Networks (CompleNet 2016) , Mars 2016.
	\item ACCEPTED: Laureate of the Bursary and 3000\,words paper to come in Data and Cities as Complex Adaptive Systems Workshop 01 , February 2016. 
	\item SUBMITED: S. Carrignon, A. Mosca, B. Rondelli, R. Remesal, “ Computer modelling and simulation as heuristic tool to understand the past: the case of the EPNEt project.” Model and Simulation 7, May 2016.
    \end{itemize}

\end{frame}
\end{document}

