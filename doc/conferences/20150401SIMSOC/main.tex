%        File: main.tex
%     Created: lun. févr. 16 11:00  2015 C
% Last Change: lun. févr. 16 11:00  2015 C
%
\documentclass[a4paper]{article}


%%%%lualatex on
%\usepackage{luatextra}
\usepackage{fontspec}
%Ligatures={Contextual, Common, Historical, Rare, Discretionary}
%\setmainfont[Mapping=tex-text]{Linux Libertine O}

\usepackage{natbib}

\title{Note biblio}
\author{Simon Carrignon}
\date{15-2-2015}
\begin{document}

	Our aims here are to show a model able to handle cultural and economical change in a network of good exchange. In order to later explore the roman empire exchange network. An easy.

	We want to build a model general ennough to allow us to study economics mechanisms as it can be done with ACE (tesfatsion\ldots) as wel as cultural evolution (cf mesoudi, morin, etc\ldots). 


	\section{Roman Empire:}
	\begin{quote}
		Only a small alternative patterns for organize livelhood exist.
		\\\cite[introducary notes, p. xviii]{polanyi1957tradeandmarketintheearlyempireseconomiesinhistoryandtheory}
	\end{quote}

	\begin{quote}
		anciant economic life might better be understood if viewed from the perspective of primitive rather than modern society.
		\\(ibid. p.6)
	\end{quote}

	\begin{quote}
		Time and again it was urged that ``economics'' should be based upon the whole range of man's material want statisfaction -- hist material wants, on the one hand, the means of satisfying his wants, be these material or not, on the other.
		\\(ibid. p.241)
	\end{quote}

	\section{Agents Description}
	$\N$ agents.

\bibliographystyle{apalike}
\bibliography{/home/simon/Documents/Bibliographie/phd.bib}  
\end{document}



