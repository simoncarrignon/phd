\documentclass[a4paper]{article}

%%\title{Abstract MS7:\\ Transdisciplinarity : an ideal framework to develop and use computer simulation as a heuristic tool to understand the past, the example of EPNEt.}
\title{Abstract MS7:\\ Computer modelling and simulation as heuristic tool to understand the past: the case of the EPNEt project.}
\author{Simon Carrignon, Alessandro Mosca, and Bernardo Rondelli}
\date{October 2015}


\begin{document}
\maketitle 


\section{Introducion}
To Understand the social mechanisms behind the archaeological and historical artifacts that remain in our libraries and the archaeological sites we excavate is somehow tricky. It relies mostly on biased and sparse hints, and most of the knowledge of such mechanisms has to be inferred from indirect data sometime very far from the original social or cognitive process that originated it.

In this paper, we argue that \emaph{formal modelling} and computer simulation are valuable tools to overcome such obstacles. To sustain our argument, we show that most of the problems encountered are close to the ones that evolutionary biologists encompass when they try to reconstruct the history of living beings. We show how biologists faced that by embracing mathematical and computer modelling, and we argue that, at the informal level, the inferences they do are not very different to the \emph{narrative descriptions} usually made by historians.

Moreover, we  generalise this view by defending the use of computer modelling as a powerful heuristic tool in life science in general and we finish by introducing an interdisciplinary research setting where such an approach has been implemented, also saying that it has provided so far a innovative environment where formal modelling and computer simulation concretely support the understanding past societies.

\section{Model and Evolutionary Biology }
The goal of evolutionary biologists is to understand the mechanisms at the origin of the living world as we can see it today. Assuming the theory of evolution as it was first described by Darwin and as it is actually developed, they are committed to reconstruct the history of living beings. Having this aim in mind, evolutionary biologists infer and try to characterise the succession of past events (genetics, biologics, due to specific ecological and environmental contexts) that constitute this history. 

In following this way, as archaeologists and historians, evolutionary biologists can rely only on the knowledge they have about a small part of today's biological mechanisms, and on the scarce geological and palaeontological records.  Moreover, as shown by Gould (1980) this kind of understanding is strongly dependent on the contingency of the biological history: the today's world does not depend only on biological rules, but on the uniqueness of the succession of events, which put evolutionary biology at the same level as History (Beatty 1995).  The historicity of the biological evolution did not prevent evolutionary biologists to use the available data to expose a potential reconstruction of such history. Using maximum likelihood, Bayesian inference, and other formal/mathematical models, they can figure out different possible successions of events, and the likelihood of possible historical paths against the few data available.

Among those models, and given the more and more important emphasis put on ecological and individual-to-individual interactions, the artificial environment made of formal models and computer simulation is exploited to palliate the lack of data and the difficulty to recreate the right conditions of evolution in traditional laboratory (Peck 2004\ldots).

All this suggest that the problems encountered by evolutionary biologists are extremely similar than those archaeologists and historians have to face. The way inferences are made about the history of living beings seems to fall into epistemological framework that resembles that used by historians and archaeologists when they try to rebuild history of human societies. Mathematical and computer models are a way to make some of those inferences explicit in their premises and conditions of application, if not quantifiable: this can sometimes offer the possibility to infer, in a statistically plausible and transparent way, data that are missing.

\section{Computer Simulation: a heuristic tool}
But the use of computer simulation and modelling is not restricted to phylogenetic and evolutionary biology. Indeed it is now widely use in all branches of Science. People in Artificial Life (Bedau et al. 1998, 2000, Paolo et al. 2000, \ldots) argued that computer simulation are powerful heuristic tool that combine the exploratory power of thought experiment and the logical strength of mathematic. They allow to test quickly a lot of possible ``opaque though experiment'' that would be impossible to try mentally.
Moreover, HPC allow us nowadays to statistically try a wide range of parameter of such thought experiment otherwise technically impossible (thought theoretically yes).

We thus think that computer model and simulation are one of the best way to study social science and history, as they give  us a heuristic tool to test hypotheses made on very complex systems such as high scale economics activity in past society, where the interactions between every component of the system make the predictability of it very difficult to solve analytically and where those original component at the core of this activity are not anymore accessible.

Moreover, as said by Winsberg (2003) who follows Hacking, Cartwright and other, Computer Simulation gives us a semi-autonomy from theory that allow us to test theory-independent assumption. This epistemological freedom is a mandatory in such fields as economy and history, that hardly fit in traditional view of theories.


\section{Interdisciplinarity and the EPNet project}
The idea of building a computational model in such a way that it allows us to extract valuable knowledge from it, still remains a difficult task. Computer scientists have to be really careful about every assumption they implicitly made while building computer models (the programming paradigm used, the way time and continuity are implemented,\ldots). On the other side, historians are invited to formulate their hypotheses in new and innovative epistemological framework which still has to be clearly specified and investigated, and whose boundaries stay quite far away from those the historians have been used to live with until now. The communication between both side of the research is thus primordial. Moreover, knowledge in such a challenging journey does not lie in the mathematical models, neither in the historical data, but emerges from the well articulation of both side (``in blood and tears'', Winsberg 2009).
In this paper, we will provide examples and concrete experiences from the EPNet project, where the emphasis is on providing historians with computational tools to compare, aggregate, measure, geo-localise, and search data about latin and greek inscriptions on amphoras for food transportation.

In particular, the ERC Advanced Grant EPNet (