\documentclass[a4paper]{article}


\title{Abstract MS7:\\ Computer modelling and simulation as heuristic tool to understand the past: the case of the EPNEt project.}
\author{Simon Carrignon, Alessandro Mosca, Bernardo Rondelli and Jos\'{e} Remesal}
\date{October 2015}


\begin{document}
 


\section{Introducion}

Due to the incompleteness and uncertainty of the historical and archaeological record generates most of the historical interpretation has to be inferred from indirect data sometime very far from the original social or cognitive process that originated it (Madella et al., 2014).

In this paper, we argue that \emph{formal modelling} and computer simulation are valuable tools to overcome such limitations and (I) we show how evolutionary biologists do so and why it's close to what archaeologist and historian do, (II) we defend how computer modelling is a good heuristic tool in science in general and (III) we introduce an interdisciplinary research setting where the two previous points can be exploited in a meaningful way to investigate the political and economical mechanisms that characterized the dynamics of the commercial trade system during the Roman Empire.

\section{Model and Evolutionary Biology }
The goal of evolutionary biologists is to understand the mechanisms at the origin of the living world as we can see it today. Assuming the theory of evolution they try to characterise the succession of past events that constitute this history. 

Starting with Gould (1989), several biologists and philosophers have argued that their researc activity is closer to the activity of Historian than Physicist (Ereshefsky, 1992, Beatty, 1995) : the actual biological world does not depend \emph{only} on biological rules, but also on the uniqueness of the succession of events.

To encompass the issues raised by such historicity, evolutionary biologists use formal models (Maximum Likelihood, Bayesian Inference,\ldots) to figure out different possible successions of events and the likelihood of such possible historical paths and test it against the available data.

This suggest that (i) the problems encountered by evolutionary biologists are close to those archaeologists and historians have to face (ii) the way inferences are made about the history of living beings and the history of human societies and (iii) mathematical and computer models offer the possibility to infer, in a statistically plausible and transparent way, missing data and complex hypotheses in both.


\section{Computer Simulation: a heuristic tool}
The use of computer simulation and modelling is not restricted evolutionary biology. Indeed it is now widely use in all branches of Science. People in Artificial Life (Bedau et al. 1998, 2000, Paolo et al. 2000, \ldots) argued that computer simulation are powerful heuristic tools that combine the exploratory power of thought experiment and the logical strength of mathematic. They allow to test quickly a lot of possible ``opaque though experiment'' that would be impossible to try mentally.

We thus think that computer model and simulation are one of the best way to investigate social science and history research questions, as heuristic tool to test hypotheses made on very complex systems such as high scale economics activity in past society. In such system the interactions between every component of the system make the predictability of it very difficult to solve analytically and the heuristic power of computer modelling is one of the only way to explore it.


\section{Interdisciplinarity and the EPNet project}
The idea of building computational models that give us valuable knowledge about the modelled object still remains a difficult task. Computer scientists have to be aware off every assumption they could implicitly made and historians have to formulate their hypotheses in an epistemological framework yet not clearly specified and investigated, and far away from the one they have been used to now. The communication between both side of the research is thus primordial: knowledge in such a challenging journey does not lie in the mathematical models neither in the historical data, but emerges from the well articulation of both side (Winsberg 2009).

In this paper, we examples from the EPNet project, where the emphasis is on providing historians with computational tools for understanding the political and economical implications behind food production and distribution along the Roman Empire.

The computational infrastructure of the EPNet project takes the form of a ``Virtual Research Environment'' offering: (i) a conceptual layer (ontology) driving the access to datasets stored into fragmented, heterogeneous and distributed digital repositories; (ii) a platform for sharing of expert knowledge on characterisation, typology and dating of Roman Empire epigraphies/artefacts; (iii) dedicated data visualisations and analytics tools, such as statistical inference and computer-based simulation. By taking into consideration the design and development of such a computational infrastructure, the EPNet epistemological framework is aiming to address three main problems: (i) structuring and making accessible large collections of data through the Web, (ii) providing a formally defined, unambiguous, framework for analysing the data and exporting them in a way that can be further manipulated by computer simulation algorithms and complex network analysis, and (iii) making each collection of data integrable with other complementary data sources.



\end{document}


