We propose a framework to study the co-evolution of cultural change and trade. 

Cultural change comprises  processes that modify spread of information by social interaction within a population and numerous social scientists are using an evolutionary theory to model this.
Here we follow this trend and propose a model to study economics. Economics is seen here as a social activity that depends on particular cultural traits: the value attributed to goods used to trade during the economic activity. 

Multiple cultural processes could influence the way those values evolve through space and time leading to different trade dynamics.

We focus on the way those values are transmitted and vary form individual to individual, and on the bias that affect this transmission. We propose a framework that allow us to implement and test hypotheses and claims made about the nature of such transmission processes and bias and study how those claims and hypotheses affects a given economy.

The design aims for a trade-off between the flexibility necessary for the implementation of multiple models and the structure necessary for the comparison between the models implemented. To create this framework we propose an Agent-Based Model relying on agents producing, exchanging and associating values to a list of goods. 

We present the key concepts of the framework and two examples of its implementation which allow us to show the flexibility of our framework. Moreover, we compare the results obtained by the two models, thus validating the structure of the framework. Finally, we validate the implementation of a trading model by studying the price structure it produces.


tout ça c'est dla merde.
Dla grosse merde
Ça me soule comme c és tpas permit putain. Pas permis
