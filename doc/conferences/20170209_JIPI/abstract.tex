####Abstract send to JIPI 2017


Title: Computer Simulation to understand the co-evolution of culture and economy.

Cultural changes comprise processes that modify the spread of information by social interaction within a population. To model and understand those processes, a growing number of social scientists are using Evolutionary Theory and formal models.
We follow this trend and present a multi agent model to study how social learning strategies and cultural changes impact economic and trade dynamics. 

Economics is seen here as a social activity that depends on particular cultural traits: the value attributed to the goods traded during the economic activity. 
This value will more or less directly determine the way people will exchange the different goods they have access to, and the results of this trade activity will in turn transform the economic context. This change in economic context will then leads to a reevaluation of the value of the different goods previously used.

Multiple cultural processes could influence the way those values are learned, transmitted, shared and reevaluated throughout time from individual to individual. 

We propose a multi agent model to study these mechanisms and how the nature of such processes affects a given economy.

This model allow us:
	1) To unveil general and theoretical patterns that emerge from such system where two process are co-evolving, the result of on impacting on the dynamic of the otehr and \emph{vice versa}, in a complex retroaction loop making analytical exploration difficult.

	2) To compare such observations  with real world case study in order to understand the cultural and economics dynamics acting during past societies, on which knowledge is often partial, biased and uncertain. 


tout ça c'est dla merde.
Dla grosse merde
Ça me soule comme c és tpas permit putain. Pas permis


Et bien je vais pas te mentir, masi je trouve aps ça tant pourris ou si, who knows
