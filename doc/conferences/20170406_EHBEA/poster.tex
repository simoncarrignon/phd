\documentclass[a1paper,fontscale=.49]{baposter}

%%%%lualatex on
%\usepackage{luatextra}
\usepackage{fontspec}
%Ligatures={Contextual, Common, Historical, Rare, Discretionary}
\setmainfont[Mapping=tex-text]{Lora}

%%%lua off
%\usepackage[utf8x]{inputenc}
%\usepackage[T1]{fontenc} 
%\usepackage{lmodern}

\usepackage{enumerate}
\usepackage[english]{babel}
\usepackage{graphicx} %to insert pictures
\usepackage{color} %to set colors
\usepackage{latexsym}
\usepackage{caption}
\usepackage{multicol}
\usepackage{amsmath}

\usepackage{float}
\usepackage{booktabs}

\newcommand{\specialcell}[2][c]{%
  \begin{tabular}[#1]{@{}c@{}}#2\end{tabular}}


\makeatletter
\let\oldabs\abs
\def\abs{\@ifstar{\oldabs}{\oldabs*}}
\let\oldnorm\norm
\def\norm{\@ifstar{\oldnorm}{\oldnorm*}}
\makeatother


%\usepackage[top=1.5cm,bottom=2cm,left=2.5cm,right=2.5cm]{geometry}
%\linespread{1.5}\selectfont



\author{Simon Carrignon}
\definecolor{bordercol}{RGB}{255,255,255}

\definecolor{headercol1}{RGB}{3,51,123}
%\definecolor{bscol}{RGB}{33,57,112}
\definecolor{bscol}{cmyk}{.98,.58,0,.51}
\definecolor{upfcol}{cmyk}{.02,1,.85,.06}
\definecolor{headercol2}{RGB}{255,255,255}
\definecolor{headerfontcol}{RGB}{255,255,255}
\definecolor{boxcolor}{RGB}{255,255,255}
\definecolor{emphcol}{cmyk}{.71,.49,0,.56}

%%% Save space in lists. Use this after the opening of the list %%%%%%%%%%%%%%%%
\newcommand{\compresslist}{
	\setlength{\itemsep}{1pt}
	\setlength{\parskip}{0pt}
	\setlength{\parsep}{0pt}
}

\newcommand{\coloremph}[1]{
	\textcolor{emphcol}{\bf#1}
}


\begin{document}

\begin{poster}{
	borderColor=white,
	headerColorOne=upfcol,
	headerColorTwo=upfcol,
	headerFontColor=headercol2,
	% Only simple background color used, no shading, so boxColorTwo isn't necessary
	boxColorOne=white,
	boxColorTwo=upfcol,
	%roundedheadershape=roundedright,
	headerfont=\Large\sf\bf,
	%textborder=none,
	headerborder=open,
	background=plain,
	bgColorOne=white,
	boxshade,
	grid,
	columns=2
}
{
    eyecatcher andmp
}
{	
    \begin{flushleft}
	\color{upfcol}{Transmission, Innovation \& Economic Equilibrium}
    \end{flushleft}
}
{
    \begin{flushleft}
	Simon Carrignon$^{1,2}$, Xavier Rubio-Campillo$^{3}$\\
	{\small $^{1}$Barcelona Supercomputing Center, $^{2}$Universitat Pompeu Fabra, $^{3}$University of Edinburgh,}
    \end{flushleft}
}
{
%\setlength\fboxsep{0pt}
%\setlength\fboxrule{0.5pt}
\vspace{5mm}
\begin{minipage}[l]{14em}
	\includegraphics[height=4em]{../../logos/bscLogo.jpg}\\
	\vfill

	\includegraphics[height=4em]{../../logos/upf_word_imp.jpg}
    \end{minipage}
}

\headerbox{Introduction}{name=introduction,column=0,row=.01}{
    Economists provide numerous theories to understand human production, consumption and trade activity. Nonetheless, given the complexity and the huge amount of assumptions needed, those theories are hardly integrated in other fields studying related aspect of human activity.

	Following trend initiated by Cultural Evolution studies, we want to fill this gap by providing tools to smoothly articulate theories and hypothesis coming from various fields and quantify the likelihood of such combinations to explain practical case study.

%Cultural change comprises processes that modify spread of information by social interaction within a population~\cite{boyd_origin_2005} and numerous social scientists are using an evolutionary framework to model this~\cite{henrich_evolution_2003}.

	We already implemented such a tool in an agent based model~\cite{carrignon2015modelingthecoevolutionoftradeandcultureinpastsocieties} designed to study the co-evolution of economy and culture. Economics is here seen as a social activity that depends on particular cultural traits: the value attributed to goods. Multiple cultural processes (innovation, social learning,\ldots) could influence the way those values evolve through space and time leading to different trade dynamics. 
	
	In the present study we combine the ability of the model to go from culture to economics with \emph{Fitting to Idealized Outcomes} (FIO) in order to quantify the likelihood of different innovations rate to lead to a well studied theoretical economic equilibrium: the General Equilibrium.



}

\headerbox{Method}{name=ud,column=0,below=introduction}{
    \small
\subsection*{General Equilibrium}
In economic theory, a \emph{General Equilibrium} (GE) is reached when, given some initial endowment, consumers and producers agreed on prices that allow both of them to produce, exchange and consume goods of different markets in such quantities that nothing is missing nor left in the markets (prices are ``market-clearing prices'') and consumers are buying the exact quantities of goods they want (they are maximizing there utilities). 



\subsection*{Agent Based Model}
To test what kind of cultural processes allow the emergence of such GE, we use a model we previously developed to study the co-evolution of culture and economy. We have already shown that this model leads to market-clearing prices and maximal utility (cf Fig.~\ref{fig:ratioEvol}) for some particular sets of parameters. 
	\vspace{-.75cm}
\begin{figure}[H]
	\begin{tabular}{cc}
		\includegraphics[width=.5\textwidth]{img/ClearingPriceDistanceEvolutionForTrade-G3N500.pdf}&
		\includegraphics[width=.45\textwidth]{img/ScoreEvolutionForTrade-G3N500.pdf} \\
	\end{tabular}
	\vspace{-.5cm}
	\caption{
	    \small
	    Graphs from \cite{carrignon2015modelingthecoevolutionoftradeandcultureinpastsocieties}, left: evolution of prices toward clearing market prices. Right: evolution of agents score toward the optimal scores.
	}
	\label{fig:ratioEvol}
\end{figure}

In this original model, groups of agents produce, consume and exchange goods, and then adapt their trading strategies by \emph{innovating}, or by \emph{learning from someone else}. Those two later aspects are the cultural aspects we want to consider here. 

We compare two \emph{social learning mechanism}: a random one (called \emph{R-copy}), where agents copy the price of a randomly chosen agent and a selective one, where agents have a higher probability to copy the prices of the best agents (the \emph{B-copy}).

On the other side, the \emph{innovation process} is a probability $\mu$ that one agent randomly change the price he attributes to a good. This probability $\mu$ is what we call the innovation rate. For both copy mechanisms, we try values of $\mu$ randomly chosen between 0 and 1.

In the following sections we see what combinations of such parameters allow the model to end in a GE.

\subsection*{Fitting to Idealized Outcomes}

To do so we apply a variation of \emph{Approximate Bayesian Computation} (ABC). ABC relies on Bayesian inference to compute and compare the likelihood of models to explain a set of empirical evidences under different parameter distributions. It has been already fruitfully applied to study changes in socio-cultural construct such as battlefield strategies~\cite{rubiocampillo2016modelselectioninhistoricalresearchusingapproximatebayesiancomputation}.

Here we use a slight variation of ABC, called \emph{Fitting to Idealized Outcomes} (FIO)~\cite{gallagher2015transitiontofarmingmorelikelyforsmallconservativegroupswithpropertyrightsbutincreasedproductivityisnotessential}, as it compares the parameter space of a model to the output of known theoretical model, instead of empirical evidence.

\vspace{.2cm}
{
	\footnotesize 
	FIO steps:
\vspace{-.3cm}
\setlength{\columnsep}{1mm}
	\begin{multicols*}{2}
	    \begin{enumerate}
		    \compresslist
		\item sample of $\mu$ with $\mu\sim U(0,1)$
		\item run simulations with innovation rate = $\mu$ 
		\item compute distance $\epsilon$ to idealize outcome~(GE):  
		    \begin{align*}
			\epsilon = \frac{ \sum_{i=1}^{n} s_i-s_{ge} }{n}   
		    \end{align*}
		    {\tiny ($n$: total number of agents, $s_i$ score of agent i, $s_{ge}$: ideal score)}
		    \vfill
		    \columnbreak
		\item select $200$ simulations with $\epsilon<.25$, 
		\item draw \emph{posterior} distribution of $\mu$ for those simulations.
	    \end{enumerate}
	\end{multicols*}
 }

All simulations were run with 150 agents exchanging and producing 3 goods during $10\,000$ time steps. 

 }




\headerbox{Results \& Analysis}{name=res1,column=1,span=1,row=.01}{
    \small

    \subsection*{Transmission}
		\vspace{-.1cm}
    The Figures~\ref{fig:epsilon} show the distribution of $\epsilon$ for the two different selection processes. For both model 200 simulations where done with random value of $\mu \sim U(0,1)$. 
	\vspace{-.75cm}
\begin{figure}[H]
    \center
	\begin{tabular}{cc}
		\includegraphics[width=.4\textwidth]{img/trade.pdf}&
		\includegraphics[width=.4\textwidth]{img/rand.pdf} \\
	\end{tabular}
	\vspace{-.5cm}
	\caption{
	    \small
	    Distribution of final $\epsilon$ for ``B-copy'' (left) and ``R-copy'' (right)
	}
	\label{fig:epsilon}
\end{figure}
We see on the Figure~\ref{fig:epsilon}~(right), that the R-copy is unable to lead to general equilibrium ($\bar{\epsilon} = 0.936$): almost all exchange failed. On the other hand, a majority of the simulation with B-copy (Fig.~\ref{fig:epsilon}~left) leads to a state where more than half of the trade are succesful ($median(\epsilon)=0.423$) with some of them where almost no trade fails ($\epsilon < 0.25$, the red side on the left graph).

	\vspace{-.3cm}
    \subsection*{Innovation}
		\vspace{-.1cm}
    To see the value of innovation rate that have a higher probability to lead to a general equilibrium, we run simulations with B-copy until we get $200$ of them ending with $\epsilon < 0.25$.

    The Figure~\ref{fig:abc} shows the distribution of $\mu$ for those simulations vs the distribution of $\mu$ for the initial simulations of the Figure~\ref{fig:epsilon}.

\begin{figure}[H]
	\centering
		\includegraphics[width=.5\textwidth]{img/ABC.pdf} 
		\vspace{-.5cm}
		\caption{\small In green the prior distribution of $\mu$ for the 200 simulations show in Figure~\ref{fig:epsilon}~(left). In blue the posterior distribution of $\mu$ for $200$ simulations with $\mu>0.25$. }
		\label{fig:abc}
\end{figure}

We see in the Figure~\ref{fig:abc}, that for the simulation with B-copy leading to relatively good equilibrium ($\epsilon > 0.25$), the innovation rate has to be lower than $0.5$, which makes sense given that otherwise agents will be changing their prices almost once every two time steps. 

Interestingly, this innovation rate remains pretty high ($\bar{\mu} = 0.175$ $\sim$ one change every 6 time steps) compared to, for example, genetic mutation process acting in biological evolution. This may underlie the difference of speed between biological and social process and the need for  social agents to quickly adapt to complex situations. 

At the same time, we think that this high rate of random changes would be drastically reduced if our agents were able to made no-random innovation and use self-adapting strategy which are processes closer to what is observed in real social systems.

}


\headerbox{Concluding Remarks}{name=conclusion,column=1,below=res1}{
    \small

This preliminary study shows how the use of simulation and model testing tools such as Fitting to Idealize Outcome, can help us articulate economic theory with other fields. Coupled with empirical data and results from other experimental fields we could easily quantify if the right conditions allowing the establishment of a General Equilibrium were met.

}


\headerbox{References}{name=references,column=1,below=conclusion}{
	\scriptsize
	\renewcommand{\refname}{\vspace{-0.5em}}
	\setlength{\parskip}{0pt}
	\setlength{\itemsep}{0pt}
	\bibliographystyle{abbrv}
	\bibliography{../../biblio/bib/SimonCarrignon,../../biblio/bib/phd,biblio}
}
\headerbox{Acknowledgements}{name=acknowledgements,column=1,below=references}{
    \small
    Funding for this work was provided by the ERC Advanced Grant EPNet (340828).
    \begin{center}
	\includegraphics[width=3cm]{../../logos/epnetLogo.png}
	\includegraphics[width=2.5cm]{../../logos/LOGO-ERC.jpg}
    \end{center}
} 

\end{poster}

\end{document}
