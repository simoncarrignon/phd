\documentclass[a4paper,10pt]{report}
\usepackage{graphicx}


\begin{document}
\section*{Impact of cultural learning mechanisms on the emergence of a economix general equilibrium}

In this poster we explore a model previously build to study the interaction between cultural mechanisms and economy. In this original model groups of agents produce, consume and exchange goods. They learn and change the strategies they use to exchange goods by copying the strategies of the more successful agents in the population.

We use for this study a variation of Approximate Bayesian Computation (ABC) to explore the parameter space of the original model and to compare it to an ideal case where all exchange of goods are made under a well  know economic equilibrium: the  general equilibrium (also called Walrasian equilibrium). ABC allows us to compute how likely different social learning process are able to produce a system where the emergence of a Walrasian Equilibrum is possible. At the same time it allows us to quantify the likelihood of the emergence of such situation under different set of fixed and limited constraints derived from historical cases studies.



\end{document}
