%%Abstract based on the abstract use for the Young Researcher in Social Learning
\documentclass[a4paper,10pt]{paper}
\usepackage{graphicx}


\begin{document}

\section*{Authors: Simon Carrignon \& Xavier Rubio-Campillo}

\section*{Title: Impact of different social learning mechanisms on the emergence of a Walrasian Equilibrium}

\section{Objectives}
The aim of this study is to model different social learning mechanisms and see how they allow or not a simple society to find an economic equilibrium.

\section{Methods}
We use a model already developed to study the co-evolution of culture and economy. In this model, groups of agents produce, consume and exchange goods. They learn and adapt their trading strategy by randomly changing it on their own, or by learning new one from someone else. 

We explore this model using a variation of \emph{Approximate Bayesian Computation} (ABC) called: Fit to Idealized Outcome. This method allows us to compute how likely different social learning process, under a wide range of parameters, are able to lead to an ``Idealized Output'' -- in our case a situation where all exchanges of goods are made under a well know economic equilibrium: the  general equilibrium, also called Walrasian equilibrium.

The likelihood of leading to this equilibrium is measured with three different learning mechanisms (neutral, success biased, and frequency biased copy) under different sets of parameters (number of agents, number of goods, innovation and copy probabilities).

\section{Results}

Our simulations show that a neutral learning process cannot lead to any efficient equilibrium. In the other cases, an innovation process has to occur but at a relatively low rate in order to avoid harming the stability of the equilibrium. In most of the situations, the success biased learning mechanism remains the most likely to allow the emergence of the expected equilibrium.


\section{Conclusion}

This study shows that in a wide range of circumstances, a simple social learning process where people tend to copy the more successful, coupled with a relatively low innovation rate, are enough to lead a simple society toward an efficient economic equilibrium.


\end{document}

%%%%Preversion of he version send
and the more people are interacting altogheter into the economy the more likely an efficient equilibrium will be found.
Authors: Simon Carrignon \& Xavier Rubio-Campillo

Title: Impact of different social learning mechanisms on the emergence of a Walrasian Equilibrium

Objectives:The aim of this study is to model different social learning mechanisms and see how they allow or not a simple society to find an economic equilibrium.

Method:We use a model already developed to study the co-evolution of culture and economy. In this model, groups of agents produce, consume and exchange goods. They learn and adapt their trading strategy by randomly changing it on their own, or by learning new one from someone else.  
We explore this model using a variation of Approximate Bayesian Computation (ABC) that allow a fitting to idealized outcomes (FIO). We use this method to compute how likely different social learning process, under a wide range of parameters, are able to lead to an ideal situation where all exchange of goods are made under a well know economic equilibrium: the general equilibrium (also called Walrasian equilibrium).
The likelihood of leading to this equilibrium is measured with three different learning mechanisms (neutral, success biased, and frequency biased copy) under different sets of parameters (number of agents, number of goods, innovation and copy probabilities).

Results: Our simulations show that a neutral learning process cannot lead to any efficient equilibrium. In the other cases, an innovation process has to occur but at a relatively low  rate in order to avoid harming the stability of the equilibrium. In most of the situations, the success biased learning mechanism remains the most likely to allow the         emergence of the expected equilibrium.

Conclusion: This study demonstrates that in a wide range of circumstances, a simple social learning process where people tend to copy the more successful, coupled with a relatively low          innovation rate, are enough to lead a simple society toward an efficient economic equilibrium.
%%%%%%%%


%%%%%%%%%%%%%%%%%%%%%%%%%%%%%version envoyee:
Authors: Simon Carrignon & Xavier Rubio-Campillo

Objectives: model and test how different social learning mechanisms allow or not a simple society to find an economic equilibrium.

Methods:  We use a model already developed to study the co-evolution of culture and economy. In this model, groups of agents produce, consume and exchange goods and adapt their trading strategy by innovating, or by learning from someone else.
We explore this model using a variation of Approximate Bayesian Computation, that allows a fitting to idealized outcomes (FIO), to compute how likely different social learning processes lead to an ideal situation where all exchanges are made under the general equilibrium (``Walrasian equilibrium'').
The likelihood of leading to this equilibrium is measured for three different learning mechanisms (neutral, success biased, and frequency-biased copy) under different sets of parameters (number of agents and goods, innovation and copy probabilities,\ldots).

Results: We show that a neutral learning process cannot lead to any efficient equilibrium. In the other cases, an innovation process has to occur but at a relatively low rate to avoid harming the stability of the equilibrium. In most of the situations, the success biased mechanism is the most likely to lead to the expected equilibrium.

Conclusion: This study demonstrates that in a wide range of circumstances, a simple social learning process where people tends to copy the more successful, coupled with a low innovation rate, are enough to lead a society toward an efficient economic equilibrium.
