%%Abstract based on the abstract use for the Young Researcher in Social Learning
\documentclass[a4paper,10pt]{report}
\usepackage{graphicx}


\begin{document}
\section*{Impact of cultural learning mechanisms on the emergence of a Walrasian Equilibrium}

\section{Objectives}
The aims of this study is to test how different social learning mechanisms can lead a simple artificail society to an 

\section{Methods}
We explore here a model previously build to study the co-evolution between cultural mechanisms and economy. In this original model, groups of agents produce, consume and exchange goods. They learn and adapt the strategies they use to exchange goods by copying the strategies of more successful agents in the population.

In this present study we explore this model using a variation of \emph{Approximate Bayesian Computation} (ABC). This variation of ABC,a los called Fitted to Optimized Outcome,  to explore the parameter space of the original model and to compare it to an ideal case where all exchange of goods are made under a well  know economic equilibrium: the  general equilibrium (also called Walrasian equilibrium). ABC allows us to compute how likely different social learning process are able to produce a system where the emergence of a Walrasian Equilibrum is possible. At the same time it allows us to quantify the likelihood of the emergence of such situation under different set of fixed and limited constraints derived from historical cases studies.

\section{Results}


\section{Conclusion}




\end{document}



version envoyee:
Authors: Simon Carrignon & Xavier Rubio-Campillo

Objectives: model and test how different social learning mechanisms allow or not a simple society to find an economic equilibrium.

Methods:  We use a model already developed to study the co-evolution of culture and economy. In this model, groups of agents produce, consume and exchange goods and adapt their trading strategy by innovating, or by learning from someone else.
We explore this model using a variation of Approximate Bayesian Computation, that allows a fitting to idealized outcomes (FIO), to compute how likely different social learning processes lead to an ideal situation where all exchanges are made under the general equilibrium (``Walrasian equilibrium'').
The likelihood of leading to this equilibrium is measured for three different learning mechanisms (neutral, success biased, and frequency-biased copy) under different sets of parameters (number of agents and goods, innovation and copy probabilities,\ldots).

Results: We show that a neutral learning process cannot lead to any efficient equilibrium. In the other cases, an innovation process has to occur but at a relatively low rate to avoid harming the stability of the equilibrium. In most of the situations, the success biased mechanism is the most likely to lead to the expected equilibrium.

Conclusion: This study demonstrates that in a wide range of circumstances, a simple social learning process where people tends to copy the more successful, coupled with a low innovation rate, are enough to lead a society toward an efficient economic equilibrium.
