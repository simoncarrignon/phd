%
\documentclass[a4paper]{article}
\usepackage{fontspec}


\usepackage{fontspec}
\usepackage{natbib}

\title{EpNet Day lectures:\\\cite{valverde2015punctuatedequilibriuminthelargescaleevolutionofprogramminglanguages}}
\author{Simon Carrignon}
\date{19-3-2015}
\begin{document}
\maketitle

\section{Aims of the article}
The aim of the authors is to tackle the possibility to systematically test the hypothesis of the analogy between Cultural and man made evolution with biological and Darwinian evolution. To do so they choose to study programming language, arguing that, as natural language, their are good candidate to do systematic and quantitative studies.

\section{Method}
The idea behind the article is to find quantitative measure to automatise a tree construction of all programming language. To do so they build the influence network between each language based on a data given in Wikipedia : which language influence or has been influenced by which one. 

This network allows them to compute the relative ``impact'' of each language, taking that a language that influence lot of other languages has more impact that one which doesn't. Moreover, from this impact measurement, they compute a ``backbone'' that resumes the network of link between languages to a tree of directed acyclic links. 

Here I think, lay the core  of the article. As the tree is build up using this measurement one has to take all following studies in regards to it. It takes into account a measurement of ``influence''between language which is human measured. One should always remember that the tree will reflect that and may miss some properties more deeply hidden in the language creation process.


\section{Analysis}

After building their tree the author use probabilistic models to model the evolution of PL in order to understand what kind of mechanisms could lead to such a tree. Their results here don't seem really convincing.  The only thing they said is that the tree emerge from a more complicated model than those they used. On the other hand it shows well what kind of studies one could do once one has quantitative data.

This may be the power of the article. They propose a way to use quantitative method inspired by evolutionary biology in a cultural evolution framework. Nonetheless the paper seems to do half of the expected work. 

If it shows well that one could  indeed quantitatively \emph{measure} cultural evolution, it hardly \emph{explains} new things (or even validate already existing hypothesis) that one could not infer using traditional historical or sociological investigation. 

Moreover, their abilities to use quantitative measurement to build meaningful tree is deeply linked to the object studied. The Programming Language history is a recent tale on which the record is abundant and safely preserved. Data on its evolution are relatively well preserved and with as little noise than one could hope.

Given that, it seems hard to apply their methods on other materials. 



We see in the article of \cite{mesoudi2009randomcopyingfrequencydependencopyingandulturechange} that everything is cool.


\bibliographystyle{apalike}
\bibliography{/home/scarrign/Documents/biblio/bib/phd.bib}  
\end{document}


