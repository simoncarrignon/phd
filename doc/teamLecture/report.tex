\documentclass[]{article}
\begin{document}

\section{ Lycett 2015}
  
In this paper Lycett try to justify the use of cultural evolution to study archaeological data.

He puts a strong assumption saying that the evolution as described by 
\begin{quotation}
"a change in gene pool"
\end{quotation}
couldn't fit evolution of cultural artefact and will only lead to a risky "loose analogy" whereas the Darwin's original view of it make evolution of anything that share the needed properties (variation/transmission/selection) an "inevitable outcome and not a mere analogy".

I spend a lot of time reading and writing to defend this point so I could not disagree.

After introducing this idea he discusses in more detail the three Principe he lists as necessary for something to evolve in a cultural point of view.

\subsection{Inheritance}

Cultural artifact are inherited for him via what he call "social learning". Which is "any mean that allow someone to learn something from other". The author insists that they are numerous and that all of them take alone or combined, have to be considered in order to understand the archaeological record.

One thing that could be argued here is that willing to extract himself from the evolutionary biology the author totally put aside the evolutionary biology. He does not discuss the fact that biology itself could impact the learning abilities and cultural artifact production and that some artefactual evolution could reflect the evolution of biology and physiological/psychological constraints only without being impacted (or in a negligible account) by social learning of knowledge. One can take as an example a growing population due to environmental change which will lead population to change their house building habits not because of any cultural evolution but because au the physiological change.


\subsection{Variation}

For the author the variation are mainly introduce by copying error but could also be done intentionally although the author does not precise what he means by "intentionality" and how it brings variation into the cultural evolutionary process. 

 %We can do here the same critic as before : the mechanisms that lead to cultural variation are biological (all abilities need to learn : see, hear, re do the right complicated movement) and so are subject to evolution also evolution that can lead to variation in the mechanisms of variation themselves.

\subsection{differential replication}

Here the author present a series of bias that could lead to a differential reproduction of cultural artefact and gives some references that study those bias.


In developing those tenets the author tree to avoid the analogy with evolutionary biology following is starting point of view that, given those tenets are present, evolution of cultural artifact is an inevitable outcome. If I think that he's conclusion is good here the author is going to far in willing avoid the analogy. He tends to forget that biological mechanisms responsible for cultural artifact are themselves under evolutionary process. Thus evolution of biological system directly impact the evolution of culture evolution that in turns impact biological organisms : those two process \emph{are deeply connected} and one cannot study on without taking in account the other. Evolutionary biology \emph{is} giving to this those aspects an always growing attention through the rise of studies in epigenetic and the importance of developmental process and all those approach that tend to fill the gap between the ecological/cultural conditions in which the result of a (biological) evolutionary process is expressed and this evolutionary process.


\subsection{Ongoing development}
In this section the author, after justifying the well founded of evolutionary archeology showing that it exists a wide range of study following this approach, take back the analogy with biology to introduced the methodological tools in evolutionary biology as suitable tools for archeology. If the author come back to biology he do so cautiously because relying on an analogy he want to avoid previously. As told before evolutionary biology is changing, and so do the methodological tools it use. If the archaeology wants to benefit fully from evolutionary thinking it should not just picking cautiously some tools that fit a set of " well-suited-for-archaeology criterion" but conversely have a deep understanding of them and their meaning and change inside the evolutionary biology corpus in order to incorporate them well and modify them not only "to adjust them to archaeological tenets" but to improve them in such way that could even benefit to evolutionary biology and serve to answer all kind of "\emph{anthropologically important} questions", as the author says.



\section{Enrich \& Mc Elreat (2003)}
\cite
This paper take the problem of evolution of culture as seen in the previous one but in a broader point of view. The question here is not to discuss how and why evolutionary thinking could be applied to anthropological studies but to explore all the question the evolution of culture raise.
From the evolutionary mechanism that leads to the cognitives abilities able to support such evolution to a detailed review of those mechanisms.



\subsection{The replicator question in cultural evolution}
They try to answer critics from Sperber that says that cultural trait could not evolve in a darwinian way because they are continuous and that cognitives attractors are too strong and the evolution of such erroneous cognitively biased representation could only evolve by individual psychological means. They answer that it's true but it should not prevent to see cultural evolution as Darwinian. I would say that the Darwinian way of seeing evolution is itself often discussed in evolutionary biology itself, that lot of refinement have been add to the original Darwin's view and that a more loose/broad way of seeing darwinian evolution could be a good approach to reconciliate people in both side of the debate (ie. Godffrey-Smith darwinian population)


\end{document}

