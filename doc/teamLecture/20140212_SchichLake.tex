%        File: 20140211_SchichLake.tex
%     Created: lun. févr. 09 10:00  2015 C
% Last Change: lun. févr. 09 10:00  2015 C
%
\documentclass[a4paper]{article}
\usepackage{fontspec}


%%%%lualatex on
%\usepackage{luatextra}
\usepackage{fontspec}
%Ligatures={Contextual, Common, Historical, Rare, Discretionary}
%\setmainfont[Mapping=tex-text]{Linux Libertine O}

\usepackage{natbib}

\title{EpNet Day lectures}
\author{Simon Carrignon}
\date{12-2-2015}
\begin{document}
\maketitle
\section{\cite{lake2014trendsinarchaeologicalsimulation}}
The aim of the article is to give us a historical review of the usage of simulation in archeology.

The author follows more or less a ``traditional'' 4 phases chronological division:
\begin{enumerate}
	\item The pionner phase : late 60s-early 1970,
	\item the mature phase : between the mid 70s and the late 80s,
	\item the pessimist and inactive phase during the 80s
	\item and the renewal phase after the 90s.
\end{enumerate}

Within this temporal division the author use another more ``epistemological'' classification to sort the models depending on their \emph{purpose} :

\begin{itemize}
	\item \emph{Hypothesis testing} models: those who serve to validate or in valditate hypothesis done by more traditional archeological studies
	\item \emph{Heurtistic} simulations :  to explore and build up hypothesis.
	\item \emph{Tactical} simulations : help to build up archeaological methods/statistics tools.
\end{itemize}

We will find those  3 types of simulation hereafter the history of sim in archeology.

\subsection{The pionner phase, late 60s early 1970}
This period coincide with the raise of the ``New Archaeology'' also called ``Processual Archaelogy'' : a trend in archeology deeply marked by a ``dogmatic positivism''. This trend takes its roots in the late 50s and follow the more general idea of logicial postivism that all scientific fields need a ``scientification''. 

Archaeologists translate it during the 60 by the need to put the
\begin{quote}
	``
	emphasis [\ldots] on seeking general laws of crosscultural applicability.''
	\\\cite[p.~297]{redman1991indefenseoftheseventiestheadolescenceofnewarcheology}
\end{quote}
as well as a
\begin{quote}
	``
	demand for rigorous archeological methodology in formulating both research designs for the field and analytical strategies for interpretation of results ''
	\\(ibid.)
\end{quote}

Those directives, coupled with the spread of the computers outside of their original worlds, lead archaeologists 
\begin{quote}
	``
	 to construct algorithmic or mathematical models of processes involving the flow of information, energy, or materials through whole societies, sometimes in response to environmental change and often incorporating simple feedback representing, for example, the effect of population growth.  ''
	 \\\cite{lake2014trendsinarchaeologicalsimulation}
\end{quote}
They follow that path in such a way that during this pionner state : 
\begin{quote}
	``
	all the archaeological problems that even today make up the core ‘territory’ of archaeological simulation had been tackled.''
	\\ (ibid.)
\end{quote}


\subsection{The mature phase, mid 1970 and early 1980}

This phase is called ``mature'' mostly because of the publication of three majors books speaking about the coupling of archaeology and simulation. 
This period saw a growing diversity of models, from the subject of the simulation (hunter gather subsistence, trade \& exchange\ldots) to the purpose of those simulation (hypothesis generation/hypothesis validation/tools builidng, cf. intro). This grow in interest and maturity ``can be attributed to the waning optimism of the New Archeology'' \citep{lake2014trendsinarchaeologicalsimulation} that we will see more in details hereafter.


\subsection{The pessimiste/inactive phase, 80s}
This inactive phase is due to a more profound epistemological crisis in the archaeological world : a critic against the processional archaeology (that leads to call the resulting movements ``post processual archaeology''). This critics of porcessual archaelogy in turn comes from the wider critics the logical positivism view of science in general recieved during this period. The possibility to build an objective, cultural value-free science was questionned. 
Some archaeologists follow this lead and
\begin{quote}
	``
	 reject the notion that the past is directly accessible -an object to be read by the trained professional from the archaeological record. They argue instead that history is constructed by people rather than something that is handed down by nature.''
	 \\\cite[p.~562]{patterson1989historyandthepostprocessualarchaeologies}
\end{quote}

In this empistemological atmosphere the development of fullty deterministics models that aim to recreate an objective and logical truth seems difficult. According to the author, it would ``not fit the new inferential framework''\citep{lake2014trendsinarchaeologicalsimulation}.

Moreover, to build simulations that could handle and encompass those kind of problems require some technical skills and methods (in computer sciences obviously but also in statistical analysis and network studies) that seems to lack at this period. Thus, the simplicity of the models built leads archaelogists to see them as ``mildky untersteing bu on the whole not particularly useful''\citep{aldenderfer1981computersimulationforarchaeologyanintroductoryessay}.

\subsection{The renewal (1900) and onward (2001)}
First, the author soften the renewal/pessimiste dichotomy saying that it's partly due to a perceptive effect and that lot of paper/books describing simulation was published during the 90s whereas the studies were done during the late 80s.

Anyway a real renewal in term of the purpose of the simulation seems to happen, with the come back of more ``global'' simulations, that tend to describe large phenomenon. Moreover, the 90s simulations were more about \emph{the use of simulation} itself, and the value of those simulations to study archaeology. 

This renewal of simulations studies come along with the usage of new concept and tools, brought back from complex systems and cognitive sciences, as well as behavioral ecology and evolutionary biology. All those burgeoning fields gives to simulation the abilities to put human abilities in the center of the archaeological approach and to follow the Mithen's (2009, p~2) way to progress:
\begin{quote}
	``
	 building models of individual decision making, trying these with archaeological data, finding their faults, omissions and strengths, and then revising and developing the model. ''
	 \nocite[p.~2]{mithen2009thoughtfulforagersastudyofprehistoricdecisionmaking}	
\end{quote}

One of the assets of these new concepts and focus is that they allow to overpass numerous critics led by the post processual archeologists (environmental/cultural conditions,cognitive bias, non tractable/non deterministic ``explanation'' ie. more flexible view of scientific explanation), opening the path to ``the coming-of-age of agent-based modelling'' \citep{lake2014trendsinarchaeologicalsimulation} which ``might help reconcile processual interest in societal systems with postprocessual concern for human agency''(ibid.).

This renewal in the late 90, which is better describe later by the author as a ``methodological maturity'', led to an explosion, after the new millenium, of new studies in the differents categories of models we have already seen.

To illustrate this the author review recent simulations published in the litterature, trying to categorize and to build lineage between them. 
\begin{itemize}
	\item Reaction diffusion models
	\item Long-term social change
	\item human-environment interactions
	\item human evolution
	\item cultural evolution
	\item misc.
\end{itemize}

What comes off is that all those models share some methodological tools (ABM,dynamic systems theory, Game theory\ldots) but seldomly in a counscious and coherent way. Each groups take their inspirations from different oldest trends (ie reaction-diffusion equation cavalli-sforza style for the first, socio-ecological concerns for H-E I, evolutionary theory for Cult. Evol.) with which they seem to ble close (depending on the background of the team?).
\subsection{Conclusion}
This article allow use to understand the use of simulation in the world of Archaeology. The author gives use, for each periods of the archaeoly's history, the models and simulation builds at those time, with their purpose and methods. It helps to build a relatively precise kind of phylogeny and an ontology of the already published models.  But, as often when dealing with transdisciplinary fields, the lines between the different categories are difficult to draw and the lineage looks more like a messy bush than a one-way net and tidy tree.


\section{\cite{schich2014anetworkframeworkofculturalhistory}}
This article present us a methodological framework to study huge databases with network analysis. 

The authors shows us that using statistical and network analysis they can extract regularity from huge biaised databases and ``recreate'' the ``cultural flow''.

The problem remains that after spending some time looking a the paper and discussing it, I'm not yet sure to get the aim of it, except to show that authors can illustrate the content off a database using some nice looking (but discutable?) graphic visualisation. What do they show us that could not been seen using demographical datas? Do they found ne mecanisms/events that could help to understand something in the history of cultural cultural?

This lack of profound interest is well illustrated by the briefness of the way they validate what they found. Author tell us that they want to link quantitave with qualitative methode but by the end the ``qualitative'' relevance of the analysis is far from being clear so is the plus-value of the article\footnote{And that, without mentionning the negative impact those kind of eye catching visualisations could have on people that fill it with a meaning they don't bear, cf: the top comments on https://www.youtube.com/watch?v=4gIhRkCcD4U.}.



\bibliographystyle{apalike}
\bibliography{/home/simon/Documents/Bibliographie/phd.bib}  
\end{document}

