\documentclass[a4paper]{article}
\usepackage{fontspec}


%%%%lualatex on
%\usepackage{luatextra}
\usepackage{fontspec}
%Ligatures={Contextual, Common, Historical, Rare, Discretionary}
%\setmainfont[Mapping=tex-text]{Linux Libertine O}

\usepackage{natbib}

\title{Notes about clio}

\begin{document}
\section{Notes abouts the Interview of turchin by Piglluci}
	is it possible to study history scientificly?
	are their general laws and principle underlying history?

	strong empirical patterns relation to Jared Diamond work on find general pattern
but diamond got not go far, not bring the all power not quantitativ enough
=> objection good science field super string theories because it's impossible to test it
	history is easier to test than physical theory
	no science without empiracl test
	in principle string theory is testable
	in history it's easy
					
prediciton is a step of scientific explenation and hypothesis testing.
prediction in science is not about the future,

detect cycle and test into other region and see if it happen

Pigglucini : difference between detect pattern and speak about causal effect

turchin vs malthus
elite reproduction division of slice and vacante state


After the black death in england and france both kingdom shoudl grow back under matlusian expection. But no. Unvaliadtion of the hypothesis.

the black swan : how does unexpected event, which are those driving history, fit. => black swan : power law or pattern in any case.

How much data? stochastic factor : statistics is mandatory.

external vs internal factor:


historian don't call them scientist, they called them Humanist.

\section{\cite{turchin_arise_2008}}
The ``original'', short article of turchin in Nature. In this article Turchin urge the need to make History a science:


\section{\cite{turchin_war_2013}}

Article with the model evolving and after:

\cite{thomas_does_2014} : critics of the article and

\cite{turchin_reply_2014}: The answer

\bibliographystyle{apalike}
\bibliography{phd-journal-club}
\end{document}
