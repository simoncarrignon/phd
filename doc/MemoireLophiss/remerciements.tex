\section*{Remerciements}
{\footnotesize
En premier lieu, je tiens à remercier Philippe, sans qui je n'aurais probablement pas pris la peine de remercier quiconque à la fin de ce second mémoire. Ensuite je remercie toutes les personnes qui m'ont aidé et accompagné pendant ce long voyage : les étudiants du LOPHISS de P7, ma première vraie promo après 6 ans d'études, ma petite équipe de la BU UPMF avec qui on a reconquis EVE ; et à tous les autres étudiants, de Paris, Montréal, Grenoble, ailleurs, avec qui j'ai pu discuter, échanger, partager et rire un peu, à propos de science, de recherche, de philosophie, de biologie, de musique et autre.
Je ne peux pas oublier tous les supports techniques que j'ai reçu et les nombreux ordinateurs qu'on m'a prêté et sans lesquels je n'aurais rien pu rédiger. Les ordinateurs de mes sœurs, du lutin, de la taupe, de filio, de la BU des grands moulins et bien d'autres que j'oublie ou que je n'ai emprunté que quelques heures. Parmi eux un remerciement particulier à Gaël qui héberge avec professionnalisme et amour (j'espère) le serveur ayant servi de moule pour fondre ces quelques pages. Toujours du côté logistique, un grand merci aux différentes adresses et aux personnes les occupantes qui ont bien voulu me laisser un morceau de lit ou de canapé : l'appart de la rue de Noisy le Sec aux Lilas avec Dakoto, Pepito, Nina et les autres, le 62 Boulevard de Strasbourg avec Seb (\& Alessandra), Bastien et Alex, à qui revient de droit la palme des plus patients et accueillants. Au 2199 rue Harvard et à tous les membres plus au moins affiliés au MHCC, Damien et Xav en tête et les autres qui y sont passés (Andreas notamment), à Fab \& Anaïs pour leur canap rue Moreau, puis ensuite pour leur accueil dans la colloc du 2078 rue Saint Hubert avec le reste de ma famille outre-atlantique. Au 5 Rue Gaston Bachelard, étrangement bien nommé pour accueillir un étudiant en philosophie des sciences, où j'ai retrouvé \emph{la} famille, pour un prélude avant la reprise plus sérieuse de l'écriture d'une aventure qui ne s'est jamais arrêtée et n'est pas prête de le faire. \`A Babal et Alex pour le logement de ministre rue de Penthièvre, et par extension aux présidents, Nicolas et François, qui m'ont toléré comme voisin pendant que j'y séjournais. J'espère un jour pouvoir faire du vélo dans ces quartiers sans m'y perdre et babal, tu auras la tranquillité que tu mérites (les deux morceaux de phrase n'ont pas de lien causal). 
Enfin, à ma maison et mon village, où nul part ailleurs je ne me reposerais mieux.


Je remercie chaleureusement la Petite Cuillère, qui m'a offert pendant presque un été complet l'endroit rêvé pour commencer un journée de lecture et d'étude en attendant tranquillement que le soleil finisse de se lever complètement sur Montréal et que Damien se décide enfin à remonter Saint-Hubert pour me rejoindre. Damien, qu'il me faut remercier tout particulièrement pour avoir probablement été plus impliqué que moi dans ce mémoire et qui a toujours cru en ma réussite universitaire et en l'achèvement de ce travail (et de bien d'autres aussi). Pour ça, pour sa présence en témoin et auditeur de mes avancées, et pour ces nombreux verres trinqués, je le remercie, ce mémoire lui doit beaucoup. Toujours sur les rives du Saint Laurent, je tiens à remercier le CIRST pour m'avoir vraiment très bien accueilli ---avec une mention spéciale à Sengsoury--- pour les livres qui étaient laissés à porté de mains et que le hasard m'a permis de feuilleter, pour la gentillesse des gens de mon bureau malgré le fossé qui séparait nos recherches, et pour la splendide vu du centre ville et des défilés brodés de carrés rouges depuis mon bureau.

Je remercie encore le LUTIN, malgré ses défauts, et tous ses membres avec qui j'ai travaillé comme dans un joyeux village, plein de hauts, de bas, de rumeurs, de coups de gueules et de franches rigolades. \`A Alex donc, à Daniel, Christophe, Ilaria, Marco, Elisabetta, Bora, Zakia, Geoffrey, Thierry, Hamid, Lagha, Nadia, Gérard, ceux que j'oublie, et Charles, forcement.

Il me faut aussi remercier mes relectrices qui ont vainement essayé de rectifier un tir orthographique plus qu'hasardeux : mes s\oe urs, ma tante, ma mère et Geneviève.
Et à mon laboratoire miniature, LAREMI, à Blaise et Jérémy, mais ça va de soi.

Pour le contenu intellectuel je remercie Frédéric Bouchard, qui m'a aidé par bien des aspects, sur le fond comme sur la forme. Il m'a accueilli à Montréal en m'intégrant dans son équipe comme un de ses étudiants, malgré la brièveté de mon passage et m'a invité à de nombreux évènements qui m'ont permis de goûter un peu plus aux joies de la recherche et de la philosophie des sciences. Une fois encore je dois remercier Nicolas Bredèche, sans qui ce mémoire n'existerait tout simplement pas. 

Pour terminer, je me dois de remercier les seuls responsables et garants de chaque ligne, de chaque livre lu, de chaque année passée à étudier et m'épanouir : mes parents.
}
