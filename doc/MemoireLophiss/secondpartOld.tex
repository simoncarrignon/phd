\chapter{Méthodes d'études traditionnelles, modèles et simulations informatiques}\label{ch:methode}

\lettrine[lines=2]{C}{e} chapitre va être l'occasion de passer en revue différents outils d'investigation qu'utilisent les chercheurs pour étudier la théorie de l'évolution. Sans nécessairement prétendre être exhaustif, nous commencerons par analyser rapidement les méthodes plus <<\,traditionnelles\,>> que sont l'analogie et les expériences de pensée.

Dans une seconde partie nous nous attarderons sur les modèles et leur application dans le domaine de la biologie. Nous verrons comment ils peuvent être intégrés dans un vision générale de la science via notamment l'approche sémantique des théories \citep{vanfraassen1980thescientificimage,suppe1989thesemanticconceptionoftheoriesandscientificrealism} et l'intégration de la biologie de l'évolution dans cette conception de la science \citep{thompson1987adefenceofthesemanticconceptionofevolutionarytheory,lloyd1984asmanticapproachtothestructureofpopulationgenetics,beatty1980whatswrongwithreceivedwiew}.  

Nous analyserons enfin plus en détails les simulations informatiques, et notamment celles mises aux points par les chercheurs dans le domaine de la \emph{Vie Artificielle}. Nous verrons comment certains des acteurs de cette discipline \citep{barandiaran06alifemodelsasepistemicartefacts} perçoivent l'apport de ces simulations pour reprendre à notre compte certaines de leurs positions et leur classification.


\section{Sélection Artificielle \& expériences de pensées }
\subsection{Sélection Artificielle et vérifications expérimentales}\label{sec:cmpdr:sa}

\begin{quotation}
	Il est donc de la plus haute importance d'élucider quels sont les moyens de modification et de coadaptation. Tout d'abord, il m'a semblé probable que l'étude attentive des animaux domestiques et des plantes cultivées devait offrir le meilleur champ de recherches pour expliquer cet obscur problème. Je n'ai pas été désappointé ; j'ai bientôt reconnu, en effet, que nos connaissances, quelque imparfaites qu'elles soient, sur les variations à l'état domestique, nous fournissent toujours l'explication la plus simple et la moins sujette à erreur. Qu'il me soit donc permis d'ajouter que, dans ma conviction, ces études ont la plus grande importance et qu'elles sont ordinairement beaucoup trop négligées par les naturalistes.\footnote{\citet[Introduction]{darwin1859originspeciesbymeansnaturalselectionorpreservationfavouredracesstrugglelife}	D'après l'édition de 1896 (- SCHLEICHRE FRÊRES, EDITEURS -).Traduit sur l'édition anglaise définitive par ED. BARBIER.
}
\end{quotation}

C'est ainsi que dès l'introduction de son livre, Darwin place l'étude de la sélection artificielle, à travers l'étude des espèces domestiques, comme un élément central de son argumentation. Il va prendre le temps de développer en détail l'analogie entre cette sélection artificielle et la sélection naturelle pendant tout le premier chapitre de l'Origine des Espèces.  Il décrit de nombreuses études et exemples tirés de la littérature sur l'élevage et la culture qu'il connait très bien, pour analyser les processus évolutifs qu'il étendra ensuite à l'évolution du vivant dans son ensemble. Chiens, pigeons, chevaux, presque toutes les espèces domestiques de l'époque y passent.

La place de cette analogie dans la démarche de Darwin a été beaucoup débattue par les philosophes de la fin du XXe siècle. Lui a-t-elle permis de \emph{comprendre} (découvrir) la sélection naturelle ou bien l'a-t-il simplement utilisée pour convaincre du bien fondé de son hypothèse de la Sélection Naturelle~? 

Il est difficile de passer en revue tous les détails de ces débats dans ce mémoire, tout comme de juger la pertinence des différents arguments avancés. Il semble néanmoins clair pour de nombreux historiens que la Sélection Artificielle a eu un rôle crucial dans le développement de la théorie de l'évolution. 

Pour nous, comme pour \cite{waters86takinganalogicalinferenceseriouslydarwinsargumentartificialselection}~:
\begin{quote}
	Le raisonnement analogique a joué un rôle clef dans la justification de la théorie de Darwin \citep[p. 410]{waters86takinganalogicalinferenceseriouslydarwinsargumentartificialselection}
\end{quote}

Pour expliquer ce rôle clef, et justifier l'importance et la solidité de cette analogie, Charbonneau (2013), dans la seconde partie de sa thèse, décrit très bien comment la Sélection Naturelle et la Sélection Artificielle sont analogues d'un point de vue structurel. Il insiste sur le fait que la structure des deux mécanismes peut être mise en correspondance, à un élément près : l'intentionnalité présente dans la Sélection Artificielle qui n'apparaît pas dans la Sélection Naturelle. Ce point mis à part, Charbonneau, tout comme \cite{waters86takinganalogicalinferenceseriouslydarwinsargumentartificialselection}, considèrent que <<\,la projection est pratiquement isomorphique\,>> (Charbonneau, 2013, p. 71). Ceci va offrir à Darwin une correspondance solide pour assurer le transfert des observations faites sur un domaine vers l'autre. 

L'isomorphisme de l'analogie est assuré en grande partie par l'étroite correspondance entre les éléments impliqués de part et d'autre de l'analogie. En particulier, les deux entités centrales soumises à l'évolution sont des êtres vivants. D'un côté (la Sélection Naturelle), on a une force qui s'applique à \emph{tous} les êtres vivants, de l'autre (la Sélection Artificielle), on a une force qui s'applique à un échantillon limité d'êtres vivants : les espèces domestiques. Mais dans les deux cas, artificiel ou naturel, une force agit \emph{sur des êtres vivants}\footnote{Il est certain que dans le cadre artificiel, la limitation aux espèces domestiques a introduit certains biais dans la théorie de l'évolution. En limitant l'étude de l'évolution à l'évolution d'espèces domestiques, l'étude de la sélection artificielle a centré la théorie de l'évolution sur un certain type d'êtres vivants aux propriétés particulières : des organismes multi-cellulaires aux limites physiques clairement définies, se reproduisant sexuellement. Ce sont les <<\,populations paradigmatiques\,>> décrite par \cite{godfrey2009darwinian}. Nous avons déjà parlé dans la section \ref{sec:pbm} de ces limitations et des critiques qu'ont formulé \cite{godfrey2009darwinian,bouchard2011darwinismwithoutpopulationsamoreinclusiveunderstandingofsotf} à leur égard, et nous n'y reviendrons donc pas. Elles ne diminuent pas l'importance de l'avantage que la sélection artificielle offre à Darwin : offrir à l'étude des processus évolutifs qui agissent directement sur des êtres vivants.}. Ainsi, cette propriété qu'a la Sélection Artificielle de traiter avec le matériaux cible de la sélection naturelle a permis de consolider l'isomorphisme de l'analogie entre les deux systèmes étudiés, et ainsi de donner plus de poids au transfert qu'a fait Darwin de l'une à l'autre.

De plus, cette correspondance entre les entités cibles des deux mécanismes (SN et SA) autorise \citet[p. 140]{evans84darwinsuseanalogybetweenartificialnaturalselection} à conclure que :
\begin{quote}
	[\ldots] c'est seulement avec les [races] domestiquées qu'une approche aussi proche de la vérification expérimentale était possible. 
\end{quote}

Ainsi, l'utilisation de la Sélection Artificielle assure non seulement à Darwin une base solide pour construire son analogie dont il inférera un de ses résultats principaux, mais aussi le rapprochant de la vérification expérimentale. 

Le besoin de se rapprocher de cette vérification expérimentale et d'étudier les effets de l'évolution directement sur les êtres vivants --- et ainsi bénéficier des avantages que nous venons de décrire, ont toujours été un objectif pour les biologistes. Nous avons vu, dans la section \ref{sec:SN}, comment les biométriciens ont essayé d'introduire les statistiques pour réaliser ces rapprochements et proposer une vérification expérimentale sur des populations présentent dans la nature. Les paléobiologistes ont fait de même en étudiant les traces fossiles enregistrées dans les couches de sédiments. Mais dans ces deux cas des problèmes apparaissent : les traces fossiles sont fortement incomplètes et ne retiennent qu'une infime partie de l'histoire des êtres vivants ; les analyses statistiques sont difficiles à contrôler et leurs interprétations délicates et sujettes à débat. Et dans un cas comme pour l'autre, les études possibles dépendant de ce que la Nature a pu laisse et offre à l'observation.

Certain ont donc voulu coupler les avantages de la vérification expérimentale sur les êtres vivants avec certaines propriétés spécifiques de la sélection artificielle que les méthodes de biométrie ou d'études des fossiles avaient perdues~: la possibilité de contrôler et de documenter l'intégralité du processus évolutif. C'est ce que \cite{elena03evolutionexperimentsmicroorganismsdynamicsgeneticbasesadaptation,lenski94dynamicsadaptationdiversification10000generationexperimentbacterialpopulations} ont fait et ont appelé de l'<<\,évolution expérimentale\,>>.

Pour réaliser cette évolution expérimentale, \cite{lenski94dynamicsadaptationdiversification10000generationexperimentbacterialpopulations} proposent une expérience dans laquelle des bactéries évoluent dans un environnement artificiel contrôlé. Le dispositif expérimental permet aux auteurs d'enregistrer toutes les étapes de l'évolution, de tester certains hypothèses, d'ajuster les paramètres des conditions dans lesquelles évoluent les colonies de bactéries. Cette installation expérimentale est en réalité très proche et partagent de nombreux avantages avec certaines expériences de Vie Artificielle dont nous reparlerons dans la section suivante. Selon \citet[p. 457]{elena03evolutionexperimentsmicroorganismsdynamicsgeneticbasesadaptation} :
\begin{quote}
	Grâce au contrôle que l'on peut exercer sur un grand nombre de variables dans une expérience en laboratoire, et grâce à la puissance offerte par l'observation direct de n'importe quel processus dynamique, de nombreuses questions à propos de l'évolution peuvent être sondées avec une rigueur qui serait autrement impossible.
\end{quote}
Ils ajoutent aussi que <<\,la caractère reproductible du résultat évolutionnaire peut-être étudié dans des populations de microbes fondées par le même ancêtre mis dans un environnement identique\,>>(\emph{ibid.} p.~458). Il est donc possible de <<\,rejouer la cassette de l'histoire du vivant\,>> et de tester si, comme l'avait avancé \cite{gould1989wonderfullife} la répétition serait totalement différente l'original. 

Pour \cite{lenski94dynamicsadaptationdiversification10000generationexperimentbacterialpopulations}, ce dispositif est pour le biologiste de l'évolution : 
\begin{quote}
	un monde fantastique [\ldots] dans lequel on peut revenir dans le temps et modifier les populations, influer sur leur histoire évolutionnaire ou leur environnement et revenir au présent pour observer l'effet de ces modifications sur les dynamiques de l'adaptation et de la diversification.
\end{quote}

Parallèlement aux expériences de Lenski, et intiment liée à l'explosion des techniques et méthodes de biologie de synthèse des dix dernières années, est apparue l'<<\,évolution dirigée\,>> (\emph{Directed Evolution} en anglais, voir \cite{romero09exploringproteinfitnesslandscapesbydirectedevolution}). Cette méthode (suivant une démarche d'ailleurs assez similaire à l'Algorithmique \'Evolutionnaire et à la Robotique \'Evolutionnaire, très empreinte de problèmes d'ingénierie), reprend la théorie de l'évolution pour concevoir, en appliquant les pressions de sélection adéquates, des macro-molécules capables de se répliquer, proches de l'ADN, mais aux composés chimiques différents\footnote{Ils appellent ces réplicateurs des XNA pour XenoNucléoAcids, Acides Xéno-Nucléiques en français, puisque ce sont des Acides Nucléique dont le Desoxyribose (ADN) ou le Ribose(ARN) a été remplacé par une autre molécule \citep{pinheiro12xnaworldprogresstowardsreplicationevolutionsyntheticgeneticpolymers,pinheiro12syntheticgeneticpolymerscapableheredityevolution}.}, des protéines exotique ou encore des bactéries dont certains acides nucléiques ont intégralement été remplacé par des acides nucléiques non ADN .

Ces méthodes, en plus de permettre la conception de molécules et organismes originaux aux propriétés particulières (des réplicateurs XNA capables de résister à des PHs très élevés ou à des températures très différentes de celles auxquelles ADN et ARN résistent), offrent la possibilité, et c'est la partie qui nous intéresse le plus, d'étudier les processus et dynamiques évolutifs eux-mêmes, à des niveaux moléculaires. Tout comme Lenski étudie in-vivo l'évolution de ses bactéries, les chercheurs en Évolution Dirigée peuvent étudier in-vivo l'évolution de macro-molécules autoréplicatrices, de protéines. Ces expériences permettent ainsi de voir quelles routes évolutives sont possibles, lesquelles ne le sont pas, comment faire pour qu'évoluent des macro molécules capables de se répliquer. Cela tout en assurant :
\begin{quote}
	l'accès à <<\,enregistrement fossile\,>> complet des intermédiaires évolutifs, des séquences, structures et fonction qui peuvent être analysées pour essayer d'expliquer comment de nouvelles propriétés sont acquises.\\ \citep[p.~872]{romero09exploringproteinfitnesslandscapesbydirectedevolution}
\end{quote}

Néanmoins, lorsqu'ils questionnent la possibilité d'utiliser <<\, l'évolution dirigé pour comprendre l'évolution naturelle\,>> \citet[p.~874]{romero09exploringproteinfitnesslandscapesbydirectedevolution} insistent sur les précautions à prendre:
\begin{quotation}
	Étendre les leçons apprises via l'évolution dirigée à l'évolution naturelle [\ldots] requière des précautions car le processus de recherche opère, entre autres, à différentes échelles de temps, de tailles de populations, de taux de mutation et de force de la sélection. 

	De plus, l'évolution naturelle joue sur des paysages adaptatifs différents et il n'est pas clair comment la fitness d'une protéine obtenue par évolution dirigé peut être reliée à la fitness de l'organisme optimisée par l'évolution naturelle. Les différences reflètent les conséquences des interactions entre la protéine et l'environnement cellulaire et pourraient inclure des contraintes liées à la complexité du métabolisme, à la régulation, aux interactions non spécifiques, ou encore à d'autres facteurs.
	\\(\emph{ibid.})
\end{quotation}

La seconde partie de cette mise en garde est très intéressante car elle relève essentiellement du caractère particulier des substrats étudiés dans ces expériences d'évolution dirigé et de biologie synthétique : l'ADN et les protéines. Elle soulève certains problèmes, que nous avons déjà évoqué dans la section \ref{sec:pbm}, qui découlent de la complexité de considérer une protéine dans l'environnement de l'individu biologique à l'intérieur duquel elle est exprimée. Quelles fonctions dépendant uniquement de la protéine, quelles fonctions sont offertes par l'environnement, quelles fonctions émergent de l'interaction entre les deux? Néanmoins il nous semble que cela ne doit pas être un obstacle pour étendre les résultats à l'évolution naturelle. Au contraire, étudier l'évolution des protéines hors de l'environnement d'un <<\,individu\,>>, permet de souligner quelles fonctions peuvent évoluer hors d'un milieu cellulaire complexe et comment ; ou à l'inverse, quelles fonctions ne peuvent évoluer sans ces interactions.

La première partie quant à elle reprend des critiques qui peuvent être faites (et ont été faites) à toutes les autres approches énoncées dans cette section (l'analogie de Darwin, les analyses des biométriciens et les expériences de Lenski) et que nous avons déjà évoquées.

De façon général, les méthodes présentées dans cette section ont un avantage évident que nous avons essayé de souligner : elles s'attachent à étudier les mécanismes et dynamiques évolutifs directement sur des êtres vivants. Cette particularité fait d'elles les méthodes les plus proches de la méthode expérimentale traditionnelle comme avait pu la penser Claude Bernard. Néanmoins ces démarches ont de nombreux défauts. En dépit des élans enthousiastes de \cite{lenski94dynamicsadaptationdiversification10000generationexperimentbacterialpopulations}, elles imposent de nombreuses contraintes et limites sur les hypothèses que l'on peut tester. Travailler sur le matériau vivant rend difficile l'étude d'un grand nombre de paramètres différents, la durée des expériences est très dépendante des organismes observés et leur fragilité entraine de nombreuses précautions qui sont autant de coûts supplémentaires. On ne pourra pas vraiment mesurer des dynamiques évolutionnaire sur des échelles temporelles équivalente à celles sur lesquelles est sensée agir la Sélection Naturelle, ont ne pourra étudier que certain type d'êtres vivants (les espèces domestiques dans le cas de Darwin, les bactéries dans le cas de Lenski).

Pour s'affranchir de ces limites certains inventent de toutes pièces les situations pour mettre à l'épreuve leurs hypothèses, à travers des constructions mentales et théorique qu'il est convenu d'appeler des <<\,expériences de pensée\,>>

\subsection{Les expériences de pensée}\label{sec:cmpdr:te}

Comme nous venons de le suggérer, les expériences de pensées permettent de s'extraire de certains problèmes que posent la s`élection artificielle (où l'\emph{évolution expérimentale}). 
\cite{brown2011thoughtexperiments} les définissent comme :
\begin{quote}
	[\ldots] des outils de l'imagination utilisés pour explorer la nature des choses.
\end{quote}

Elles sont utilisées en philosophie et dans tous les domaines scientifiques. En biologie de l'évolution, Darwin déjà, proposait de nombreuses expériences de pensée pour illustrer et appuyer son propos. En guise d'exemple de Sélection Naturelle, Darwin décrit une situation hypothétique dans laquelle des populations de loups doivent s'adapter à des changements dans la composition des populations qu'elles prédatent \citep[p. 102]{darwin1859originspeciesbymeansnaturalselectionorpreservationfavouredracesstrugglelife}.

Cette usage par Darwin des expériences de pensée n'est pas étonnant compte tenu du fait (déjà souligné dans le chapitre \ref{ch:evolnat}) qu'il n'avait à l'époque aucun moyen de prouver empiriquement sa théorie, si ce n'est en l'illustrant et en s'appuyant sur l'analogie avec la sélection naturelle. Pour \cite{lennox91darwinianthoughtexperimentsafunctionforjustsostories}, qui étudie plus en détails l'utilisation de ces expériences de pensée chez Darwin, ces dernières <<\,ont une fonction claire dans le contexte d'explorer les possibilités qu'offre la théorie\,>>. Pour lui cette fonction est <<\, indépendante de la véracité empirique de la théorie\,>>. Elles illustrent, convainquent et justifient.

Ces expériences de pensée ont continué d'être utilisées par les biologistes pour questionner la théorie de l'évolution. Nous avons déjà parlé de celle utilisée par \cite{godfrey2009darwinian} pour illustrer sa notion <<\,S\,>>  (section \ref{sec:pgs}). Dans cette expérience deux individus jumeaux aux caractéristiques ``intrinsèques'' identiques, présentent pourtant une fitness différente : l'un est frappé par un éclair, meurt et ne peux donc plus se reproduire, l'autre non.

\cite{wilson1994reintroducinggroupselectiontothehumanbehavioralsciences} utilisent aussi une expérience de pensée dans laquelle des criquets développent des méthodes de coopération pour se déplacer de feuilles de nénuphars en feuilles de nénuphars. Ils utilisent cette expérience pour défendre l'existence de traits définis aux niveaux de groupes de criquets. Ces traits, qui n'existent que si les criquets sont deux, ont une valeur sélective et justifient, selon les auteurs, l'emploi d'un concept de sélection de groupes. \cite{dennett95darwinsdangerousideaevolutionmeaningslife} aussi utilise ce procédé et reprend de nombreuses expériences de pensée pour illustrer et expliquer la théorie de l'évolution.

La place de ces expériences de pensée reste néanmoins beaucoup débattue et pose de nombreuses questions :
\begin{quotation}
	Comment peut-on apprendre des choses sur la réalité (si seulement c'est possible), simplement par la pensée? Plus précisément, existe-t-il des expériences de pensée qui nous permettent d'acquérir des nouvelles connaissances à propos du royaume de l'investigation sans nouvelles données? Si tel est le cas, d'où viennent les nouvelles informations si ce n'est d'un contact direct avec le royaume de l'investigation considéré?\\
\citep{brown2011thoughtexperiments}
\end{quotation}
Autant de réflexions sur lesquelles de nombreuses personnes se sont penchées et qui contribuant à justifier l'importante littérature sur le sujet. 

On pourra prendre l'exemple de \citet[ch. I]{wilson1999biological}, qui, pour le cas de la biologie, explique que ces expériences de pensée impliquent de trop grands sacrifices qui les débarrassent de toute valeur scientifique. En revanche, même si \citet{hacking92dothoughtexperimentshavealifeoftheirown} leur donne un statut différent des expériences classiques, puisqu'elles n'ont pas, comme il le dit : <<\,de vie d'elle même\footnote{En anglais : \emph{Life on their own}}\,>> ; il insiste, tout comme Kuhn et \cite{lennox91darwinianthoughtexperimentsafunctionforjustsostories}, sur leur capacité à illustrer et à révéler des tensions entre des théories scientifiques et des visions différentes \citep[p. 304]{hacking92dothoughtexperimentshavealifeoftheirown}. 

Les expériences de pensée permettraient donc de contourner certaines limites imposées par les expériences empiriques traditionnelles. Elles pourraient révéler des tensions, illustrer et enclencher la réflexion autours de problèmes que soulèvent certaines hypothèses et certaines théories. Elle sont en ce sens, d'autant plus intéressantes en biologie de l'évolution que cette dernière impose des hypothèses qu'il est difficile d'explorer expérimentalement : nous avons vu certains de ces problèmes soulevés dès les formulations de Darwin (section \ref{sec:cmpdr}) et en avons repris quelques un dans la section précédente. Les échelles de temps sont très grandes, les êtres vivants sont tous très différents et interagissent beaucoup entre eux et leur environnement, les études en laboratoire, leurs interprétations et leur généralisation au milieu naturel restent extrêmement délicats.  Autant de problèmes dont les expériences de pensée peuvent en principe s'affranchir.

Néanmoins, leur place particulière, l'absence de données empiriques, les biais introduits par les fonctions cognitives qui les émettent, sont autant de points qui appellent les philosophes à la prudence. 
\section{Modèles et simulations informatiques}\label{sec:cmpdr:va} 

\subsection{Les modèles}
Depuis le milieu du XXe siècle, la biologie tout comme les sciences dans leur ensemble, ont vu se démocratiser l'utilisation et l'application de \emph{modèles}. Ces derniers représentent, via des abstractions, des schémas, des équations, certains objets du monde, certaines parties, certaines entités des théories scientifiques. Ces représentations permettent de mieux étudier et comprendre les objets physiques ou théoriques, d'affiner et explorer les théories scientifiques. 

La place particulière des modèles dans la science est un sujet de philosophie des sciences en général très vaste. Quel est leur lien vis à vis du monde réel, comment ils s'articulent avec les théories scientifiques et avec les expériences empiriques traditionnelles, quel statut ontologique leur donner, sont autant de questions complexes qui présupposent de nombreuses mises au point et partis pris philosophiques que nous ne pouvons traiter ici.

Mais malgré toutes ces questions, aujourd'hui, et qu'importe l'école de pensée philosophique, il nous semble que :
\begin{quote}
	elles s'accordent toutes pour dire que les modèles sont des unités centrales de la construction de théories scientifiques.\\ \citep{frigg2012modelsinscience}
\end{quote}

Nous choisirons de suivre cette constatation. Pour le reste de notre exposé nous allons présenter une version naïve et simplifiée de la vue sémantique des théories qu'ont développé, entre autres, \cite{suppe1989thesemanticconceptionoftheoriesandscientificrealism,vanfraassen1980thescientificimage}, pour comprendre les sciences en général et qui permet d'intégrer ces modèles à une vision globale de la science.

Cette conception des théories scientifiques soutient que ces dernières (les théories scientifiques) sont mieux décrites par des familles de modèles que par des axiomes mathématiques logiquement articulés qui décrivent directement le réel, des <<\,lois de la nature\,>> (\,comme les lois de Newton\,). 

\cite{thompson1989thestructureofbiologicaltheories} ---qui reprend les auteurs cités précédemment, résume le rôle des modèle au sein de cette conception sémantique ainsi :

\begin{quotation}
	[Dans la conception sémantique] une théorie (un modèle) est une entité mathématique qui n'est pas définie par référence à un système formel. En d'autres termes ; bien qu'un système formel dans lequel la théorie sera vraie puisse être construit, la théorie (le modèle) n'est pas construite comme une interprétation d'un tel système formel. Elle est définie directement en spécifiant le comportement du système. Et, plus important, les lois ne décrivent pas le comportement d'objets du monde ; elles spécifient la nature et le comportement d'un système abstrait. Ce système abstrait, indépendamment de sa spécification, est déclaré comme isomorphe avec un système empirique particulier. \'Etablir cet isomorphisme nécessite d'autres théories scientifique et l'adoption de méthodologies (comme les théories de design expérimental et de d'ajustement des modèles).\\
	\citep[p. 72]{thompson1989thestructureofbiologicaltheories}
\end{quotation}

Bas van Fraassen, lui, conclu :
\begin{quote}
	De ce point de vu, le travail essentiel d'une théorie scientifique est de nous fournir une famille de modèles, que l'on pourra utiliser pour représenter les phénomènes empiriques \citep{vanfraassen1972aformalapproachtothephilosophyofscience}.
\end{quote}

C'est cette naïve interprétation rapidement esquissée (chaque auteur en ayant en vérité une version qui lui est propre et bien plus complexe), que nous voulons reprendre pour la suite de notre exposé et qui nous semble adapté pour étudier la théorie de l'évolution.

Nous avons parlé rapidement, dans le chapitre précédant, des problèmes que la théorie de l'évolution peut poser. Nous avons notamment décrit dans la section \ref{sec:pgs} les commentaires que Godfrey-Smith a pu faire vis-à-vis de certaines descriptions et formulations de cette théorie (les <<\,recettes\,>>), et de ce besoin de construire un schéma général qui ne marche pas, ou des schémas particuliers qui ne généralisent plus. Il est à noter aussi, mais nous ne le détaillerons pas ici, que de nombreuses critiques ont souvent pointé du doigt la biologie comme une science qui ne possède pas de loi, puisque toutes les règles qu'elle à pu mettre au point (lois de Mendel, équilibre de Hardy-Weinberg), présentent de nombreuses exceptions. Nous avons vu comment PGS intègre ces exceptions dans son espace : il construit un modèle qu'il veut souple, qui peut être déformer pour intégrer les différentes composantes et leurs exceptions. 

C'est pour répondre à ces incohérences, pour justifier le besoin de multiples modèles et asseoir la valeur scientifique de ces modèles biologique dans un vision global de la science, que \cite{beatty1980whatswrongwithreceivedwiew,beatty1980ptimaldesignmodelsandstrategyofmodelbuildinginevolutionarybiology,beatty1987onbehalfofsemanticview,thompson1989thestructureofbiologicaltheories,thompson1987adefenceofthesemanticconceptionofevolutionarytheory,lloyd1984asmanticapproachtothestructureofpopulationgenetics,lloyd1988thesemanticapproachanditsapplicationtoevolutionarytheory}, ont repris la conception sémantique des théories pour l'appliquer à la biologie et en particulier à la biologie de l'évolution.

Ainsi, si \cite{beatty1980whatswrongwithreceivedwiew} considère qu'il ne peut effectivement pas y avoir de lois au sens classique des théories (des <<\,lois de la nature\,>>), puisque, par exemple, les lois de Mendel, ou l'équilibre de Hardy-Weinberg, sont le fruit du processus évolutionnaire et sont toujours soumis à son action :
\begin{quote}
	C'est à dire que les théories évolutionnaire peuvent changer en tant que [elles sont le] résultat du changement évolutionnaire \citep[p.~407]{beatty1980whatswrongwithreceivedwiew}.
\end{quote}
Ceci n'est plus un problème à la lumière de la conception sémantique des théories. Comme nous l'avons déjà dit, suivant cette conception, les faits empiriques sont indépendants de la théorie et des modèles qui la composent. Ainsi, qu'importe si certains systèmes empiriques ne suivent pas l'équilibre traditionnel de Hardy-Weinberg et nécessitent une version améliorée et différente de celui-ci : les populations qui respectent l'équilibre classique seront décrites par la théorie tout comme celles qui nécessitent un modèle plus complexe \citep[p.~410-411]{beatty1980whatswrongwithreceivedwiew}.

Cette vision des théories scientifiques s'articule très bien avec notre volonté d'étudier la biologie à travers le prisme de modèles évolutionnaires robotiques (ou plutôt, notre volonté s'articule très bien avec la conception sémantique). Différents modèles de Robotique Évolutionnaire pourront décrire différents systèmes biologiques, sans qu'il soit nécessaire que le modèle robotiques réponde à une \emph{loi de la nature}. Et même si la plupart des auteurs que nous avons cités considèrent souvent les modèles comme des entités mathématiques formels, il y en a pour penser qu'ils peuvent être <<\,des modèles caractérisés non mathématiquement\,>>\citep[p. 1]{lloyd1988thesemanticapproachanditsapplicationtoevolutionarytheory}. Nous pensons et nous allons essayer de montrer, que les simulations informatiques peuvent elles aussi être perçu comme des modèles solides et intéressant, qui présentent de nombreuses propriétés particulières utiles pour comprendre la théorie de l'évolution.

\subsection{Simulation et Vie Artificielle}
\subsubsection{Simulation informatique}
De nombreuses façons de construire un modèle ont été expérimentées, mais, depuis les années 70 avec l'avènement et la démocratisation de l'informatique, la modélisation par informatique est devenue une méthode centrale dans les sciences en générale. 

Dans un modèle informatique le <<\,monde\,>> est décrit par un algorithmique qu'il faut ensuite exécuter. Lors de cette exécution, une <<\,simulation\,>> du modèle a lieu dont on pourra observer la concordance avec le monde réel. Cette nouvelles façon de faire de la science diffère des méthodes classiques car l'expérience, le modèle, n'est plus centré sur l'objet d'étude, mais sur un processus informatique qui tente de le mimer. Un certain nombre de chercheurs considère que ces méthodes ne sont pas tant différentes que les expériences classiques. On citera par exemple \cite{winsberg03simulatedexperimentsmethodologyforavirtualworld} qui compare une expérience classique en physique dans laquelle les physiciens recréaient en laboratoire certains conditions du monde physique pour en tester les effets avec une simulation informatique. L'auteur explique que l'expérience en physique présente autant de simplifications et raccourcis potentiellement cruciaux, que celles faites pour la simulation informatique.

Les modèles informatiques (que nous confondons ici avec les simulations informatiques, comme le fait \cite{winsberg03simulatedexperimentsmethodologyforavirtualworld}) ont un certain nombre d'atouts. De plus en plus nombreux sont les philosophes et chercheurs qui pensent que les simulations informatiques doivent être considérées comme des expériences empiriques classiques et qu'elles doivent recevoir l'importance qu'elles méritent. Ainsi Winsberg reprends les idées de Hacking sur l'indépendance des expériences vis à vis des théories, et explique que ces dernières se transposent très bien dans le cas des simulations informatiques. 
Si, comme dit Hacking, les expériences <<\,ont une vie en elles même\,>>  les simulations aussi. En effet, le modèle est revu et corrigé sans cesse, dans un va-et-vient continu entre les outils numériques, mathématiques et les données empiriques nouvelles à disposition. Dans ce va-et-vient émergent <<\,dans les larmes et le sang\,>> \citet{winsberg09taletwomethods} de nouveaux aperçus du monde.

Ces aperçus permettent de porter un éclairage sur différentes parties du monde difficilement descriptibles via des équations générales, qu'on ne peut analyser analytiquement. Ainsi les simulations s'intègrent parfaitement dans un description du monde qui nécessitent des modèles précis, parfois complexes, telle que la décrit la vue sémantique des théories.

\subsubsection{Vie Artificielle}

Les simulations informatiques que nous venons de décrire sont utilisées dans tous les domaines scientifiques. Mais elles ont aussi depuis longtemps été utilisées pour étudier la biologie. Que ce soit pour vérifier des modèles mathématiques écologiques comme les équations proies prédateurs de Lotka-Volterra ou encore pour tester les modèles d'optimisation de Maynard-Smith. Dans ces cas précis elles sont utilisées comme des outils de calculs pour accélérer, automatiser ou améliorer les recherches. Mais elles ont aussi passionné tout une communauté des chercheurs à cheval entre informatique et sciences du vivant. Ces derniers, souvent perçus comme une branche héritière de la cybernétique, considèrent les simulations comme l'objet principal de leurs recherches. Cette communauté aux limites floues, à la croisé de nombreux domaines très différents, nous l'avons déjà plusieurs fois évoquée, est souvent désignée par le nom de Vie Artificielle (\emph{artificial life} en anglais, ou \emph{alife}, cf \citet{langton89alifeiproceedingsfirstinternationalworkshopsynthesissimulationlivingsystems}). 

Par bien des aspects, historiques, méthodologiques et scientifiques, cette communauté est très proche de la Robotique Évolutionnaire. C'est pourquoi nous allons maintenant insister sur la façon dont les chercheurs en Vie Artificielle abordent et traitent modèles et simulations.

\citet{barandiaran06alifemodelsasepistemicartefacts} divisent en trois catégories les modèles produits et étudiés par/dans cette discipline:
\begin{enumerate}
	\item Les modèles esthétiques, \label{it:est}
	\item les modèles d'ingénierie et,\label{it:ing}
	\item les modèles épistémiques. \label{it:epi}
\end{enumerate}

Ce sont avec les derniers, les modèles épistémiques, que nous voulons rapprocher la Robotique Évolutionnaire. L'intérêt pour beaucoup de chercheurs en RE et comme nous voulons le défendre, n'est plus simplement de construire (\emph{designer}) un objet technologique avec des caractéristiques particulières (ce qui correspondrait aux caractéristiques des modèles de la catégorie \ref{it:ing} de Barandiaran et Moreno)
mais de comprendre <<\,comment les systèmes naturels fonctionnent\,>>.

Parmi les modèles épistémiques, \citet{barandiaran06alifemodelsasepistemicartefacts} dégagent 4 classes bâties en fonction du but et de la portée épistémique du modèle :
\begin{enumerate}[\hspace{1cm}(a)]
	\item des modèles génériques, \label{it:gnx}
	\item des modèles conceptuels, \label{it:con}
	\item des modèles fonctionnels et,\label{it:fun}
	\item des modèles mécanistes. \label{it:mech}
\end{enumerate}

Les deux dernières classes décrivent des modèles qui s'évertuent à recréer des mécanismes présents dans le vivant pour en permettre l'étude (par exemple reconstruire une fourmilières artificiellement) et se veulent (surtout pour les modèles mécanistes) au plus proche possible des données empiriques. Leur but et de valider des modèles précis de fonctions et mécanismes du vivant (modèles d'une synapse, etc.). 

Les deux premières classes, elles, se veulent plus générales. Les modèles qui tombent dans la première se rapprochent plus des lois mathématiques (les auteurs donnent l'exemple des modèles NK de Kauffman, etc.) très génériques et applicables aussi bien aux réseaux sociaux qu'aux interaction protéiques. Les seconds, les modèles conceptuels, ont quant à eux un statut épistémique plus <<\,hétérodoxe\,>>. À cheval entre théorie et expérience empirique, ils servent d'outils pour questionner et réorganiser certaines assomptions théoriques. Ce sont dans cela que nous voulons classer la Robotique Évolutionnaire.

L'idée derrière ces modèles conceptuels est de concevoir les simulations comme véritables <<\,expériences de pensée\,>>. Ces dernières, appuyées par la puissance computationnelle des   ordinateur, peuvent être beaucoup plus élaborées que celles que le cerveau humain peut concevoir. Ces expériences de pensée peuvent être d'une grande utilité. Nous rejoignons en ce sens les conclusions de \citet{paolo00simulationmodelsasopaquethoughtexperiments} et sans prétendre qu'elles offriraient l'accès a des connaissances qui sans elles seraient inatteignables, il nous semble claire qu'elles peuvent <<\,[aider à] changer une attitude par rapport à un ensemble d'information déjà connues.\,>>, et qu'elles permettent la remise en question et la mise à l'épreuve de certains points théoriques flous ou mal compris.

Dans ce sens les modèle de vie artificielle possèdent bien les qualités que \citet{hacking92dothoughtexperimentshavealifeoftheirown}, dans la lignée de Kuhn, attribue aux expériences de pensée et dont nous avons parlé dans la section \ref{sec:cmpdr:te}: cette capacité a illustrer et a révéler des tensions entre des théories scientifiques et des visions différentes \citep[p. 304]{hacking92dothoughtexperimentshavealifeoftheirown}. Ou, comme le résume \citet{peck04simulationasexperiment} dans une revus de l'utilité des simulations en biologie de l'évolution et en écologie : 
\begin{quote}
	[La simulation] montre ce que le monde pourrait être, si vraiment il fonctionne de la façon dont nous l'imaginons fonctionner. 
	\citep[p. 533]{peck04simulationasexperiment}
\end{quote}

Mieux encore, comme le précise \cite{winsberg03simulatedexperimentsmethodologyforavirtualworld}, la simulation change un peu la donne par rapport aux traditionnelles expériences de pensée. Là où ces dernières ne sont que des illustrations statiques (d'après Hacking), les simulations informatiques en général, et les simulations en vie artificielle en particulier, sont lancées, ajustées, modifiées, au regard de la théorie, du résultat des simulations et des nouvelles données expérimentales. Elles offrent une puissance de calcul et une finesse bien plus grandes que ce que l'expérience de pensée traditionnelle permet. Elle ne sont pas limitées ni biaisées par les capacités cognitives humaines.

Notre but n'est pas non plus d'avancer, comme peuvent le faire \citet{bedau98philosophicalcontentmethodartificiallife} et les partisans d'une vie artificielle forte (\emph{strong Alife}), que les phénomènes observés dans des expériences de VA \emph{sont} biologiques. Cette approche d'une vie artificielle \emph{forte}, qui veut que les propriétés des parties étudiées suffisent à faire \emph{émerger} la vie, indépendamment du substrat physique, nous ne souhaitons pas nécessairement la défendre ici. Cela dépasse le cadre de ce mémoire d'autant qu'elle a beaucoup été critiquée par certains philosophes de la biologie \citep[Ch. 15]{sterelny99sexdeathintroductiontophilosophybiology}. Néanmoins même les plus critiques semblent s'accorder sur la valeur, si ce n'est explicative des modèles artificiels, au moins illustratrice de ceux-ci. 

C'est cette valeur illustratrice, cette capacité, défendu par \citet{paolo00simulationmodelsasopaquethoughtexperiments}, que les modèles artificiels ont de permettre de dérouler des mécanismes aux interactions multiples et qui rendent les systèmes difficiles à analyser analytiquement au premier abords, que nous voulons souligner ici. Ces propriétés naissent de ce statut hybride, entre expérience de pensée, expérience traditionnelle et modèle du monde qu'ont les simulations. Avec elles il devient possible de tester exhaustivement certains paramètres, de les ajuster, jusqu'à éclaircir une situation \emph{a priori} bien <<\,opaque\,>>, tout comme Lenski propose de le faire avec des bactéries. Cet intérêt des simulations de vie artificielle vues comme des <<\,expériences de pensée opaques\,>> semble faire consensus et découle en parti des propriétés énoncées précédemment. 

\'A titre d'exemple, et dans le cas précis nous intéressant de modèle de vie artificielle appliqués à l'évolution \citet{huneman12computersciencemeetsevolutionarybiologypurepossibleprocessesissuegradualism} à déjà montré que l'algorithmique évolutionnaire peut être utile à divers degré. Il avance un autre atout apporté par l'étude de l'évolution artificielle implémentée en Algorithmique \'Evolutionnaire et en Vie Artificielle. Selon lui, et nous et suivons en ce sens, elle permet d'étudier les processus évolutif de façon plus générale et de :
\begin{quote}
	dépasser l'assimilation classique de l'évolution comme l'évolution graduelle par sélection naturelle cumulative \citep{huneman12computersciencemeetsevolutionarybiologypurepossibleprocessesissuegradualism}.
\end{quote}
Elle permet ainsi de désigner des ``candidats potentiels'' de processus évolutifs et:
\begin{quote}
	autorise des schémas de macro-évolution pluralistes et plus plausibles, qui contribuent à l'intégration théorique en biologie : intégrer la micro, macro et méga-évolution (sensu Simpson, 1948);  et intégrer les différentes approches de l'intégration, allant de la théorie classique de l'évolution (Mayr, 1976) aux récentes tendances Evo-Devo (Müller, 2007; Gilbert, Opitz, Raff, 1996).\\
	\citep{huneman12computersciencemeetsevolutionarybiologypurepossibleprocessesissuegradualism}
\end{quote}

Il insiste aussi avec justesse sur une précaution à prendre et qui apparaît chaque fois que nous avons parlé d'étudier l'évolution dans un cadre artificiel, quoique dans le cadre présent cette critique est d'autant plus vrai : en biologie tout interagi avec tout. Lorsque l'on modélise, on <<\,choisi\,>> les caractères qui nous intéressent en en mettant donc certains de côté. Il en était de même lorsque nous discutions de la sélection artificielle chez Darwin, qui limitait les espèces étudiées aux espèces domestiquées, où chez Lenski qui se limite à l'étude de certains micro-orgnasimes. Mais dans le cas artificielle informatique, l'ensemble du système est construit et choisi, à la variable près. Les cas d'exclusions de paramètres importants sont donc d'autant plus grand qu'il peuvent même échapper totalement à la connaissance qu'on a du système que l'on veut étudier. Mais c'est aussi cette très grande liberté de conception qui permettra au chercheur de développer un modèle avec quelques molécules, une cellule entière ou un population de mammifères.

Il nous semble que cette étude des processus biologique gagne encore en pertinence lorsque les algorithmes génétiques sortent du carré de l'unité centrale de l'ordinateur et se garnissent de capteur et d'effecteurs, pour devenir à leur tour agent <<\,incarnés\,>>, dans ce que nous avons commencé à décrire et à appeler \emph{la Robotique évolutionnaire}. C'est ce que nous verrons dans le chapitre précédant.


\section{Conclusion}
Dans ce chapitre nous avons rapidement présenté quelques méthodes souvent utilisées par les chercheurs pour étudier la théorie de l'évolution et essayer de résoudre les problèmes qu'elle génère. Nous avons d'abord parlé de la sélection artificielle et de son utilisation par Darwin puis, comment cette idée d'étudier des entités physique soumises à des pressions sélectives dans des <<\,expériences\,>> de laboratoire à perduré et quel est l'intérêt de ce type d'expériences. Nous avons ensuite brièvement parlé des expériences de pensée, pour introduire des méthodes laissant plus de liberté aux expérimentations et tests possibles.

Dans la deuxième partie de ce chapitre nous avons parlé des modèles en essayant de justifier leur utilisation en sciences via la conception sémantique des théories scientifiques, puis nous avons soutenu l'intérêt de l'utilisation de simulation informatique dans ce cadre, en décrivant notamment les simulations étudiée dans le domaine de la Vie Artificielle, qui sont selon nous un des modèles offrant la liberté de mano\oe uvre la plus grande pour permettre d'étudier correctement la théorie de l'évolution.

