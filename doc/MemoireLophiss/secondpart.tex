\chapter{Méthodes d'études traditionnelles, modèles et simulations informatiques}\label{ch:methode}

\lettrine[lines=2]{C}{e chapitre} va être l'occasion de passer en revue différents outils d'investigation qu'utilisent les chercheurs pour étudier la théorie de l'évolution. Sans nécessairement prétendre être exhaustif, nous commencerons par analyser rapidement les méthodes plus <<\,traditionnelles\,>> que sont l'analogie et les expériences de pensée.

Dans une seconde partie nous nous attarderons sur les modèles et leur application dans le domaine de la biologie. Nous verrons comment ils peuvent être intégrés dans une vision générale de la science via notamment l'approche sémantique des théories \citep{vanfraassen1980thescientificimage,suppe1989thesemanticconceptionoftheoriesandscientificrealism} et l'intégration de la biologie de l'évolution dans cette conception de la science \citep{thompson1987adefenceofthesemanticconceptionofevolutionarytheory,lloyd1984asmanticapproachtothestructureofpopulationgenetics,beatty1980whatswrongwithreceivedwiew}.  

Nous analyserons enfin plus en détails les simulations informatiques et notamment celles mises au point par les chercheurs dans le domaine de la \emph{Vie Artificielle}. Nous verrons comment sont perçues ces simulations par les acteurs de cette discipline \citep{barandiaran06alifemodelsasepistemicartefacts} et nous reprendrons à notre compte leurs positions et leur classification.



\section{Modèles et simulations informatiques}\label{sec:cmpdr:va} 

\subsection{Les modèles}
Depuis le milieu du XXe siècle, la biologie tout comme les sciences dans leur ensemble, ont vu se démocratiser l'utilisation et l'application de \emph{modèles}. Ces derniers représentent, via des abstractions, des schémas, des équations, certains objets du monde, certaines parties, certaines entités des théories scientifiques. Ces représentations permettent de mieux étudier et comprendre les objets physiques ou théoriques, d'affiner et explorer les théories scientifiques. 

La place particulière des modèles dans la science est un sujet de philosophie des sciences \emph{en général} (de \emph{toutes} les sciences), et est donc très vaste. Quel est leur lien vis à vis du monde réel, comment s'articulent-ils avec les théories scientifiques et avec les expériences empiriques traditionnelles, quel statut ontologique leur donner,etc., sont autant de questions complexes qui présupposent de nombreuses mises au point et partis pris philosophiques que nous ne pouvons traiter ici.

Mais malgré toutes ces questions, aujourd'hui, il semble que les différentes écoles de pensée philosophiques :
\begin{quote}
	s'accordent toutes pour dire que les modèles sont des unités centrales de la construction de théories scientifiques.\\ \citep{frigg2012modelsinscience}
\end{quote}

Nous choisirons de suivre cette constatation. Pour le reste de notre exposé nous allons présenter une version naïve et simplifiée de la vue sémantique des théories qu'ont développé, entre autres, \cite{suppe1989thesemanticconceptionoftheoriesandscientificrealism,vanfraassen1980thescientificimage}, pour comprendre les sciences en général et qui permet d'intégrer ces modèles à une vision globale de la science.

Cette conception des théories scientifiques soutient que ces dernières (les théories scientifiques) sont mieux décrites par des familles de modèles que par des axiomes mathématiques logiquement articulés qui décrivent directement le réel, des <<\,lois de la nature\,>> (\,comme les lois de Newton\,). 

\cite{thompson1989thestructureofbiologicaltheories} ---qui reprend les auteurs cités précédemment, résume le rôle des modèle au sein de cette conception sémantique ainsi :

\begin{quotation}
	[Dans la conception sémantique] une théorie (un modèle) est une entité mathématique qui n'est pas définie par référence à un système formel. En d'autres termes ; bien qu'un système formel dans lequel la théorie sera vraie puisse être construit, la théorie (le modèle) n'est pas construite comme une interprétation d'un tel système formel. Elle est définie directement en spécifiant le comportement du système. Et, plus important, les lois ne décrivent pas le comportement d'objets du monde ; elles spécifient la nature et le comportement d'un système abstrait. Ce système abstrait, indépendamment de sa spécification, est déclaré comme isomorphe avec un système empirique particulier. \'Etablir cet isomorphisme nécessite d'autres théories scientifiques et l'adoption de méthodologies (comme les théories de design expérimental et de d'ajustement des modèles).\\
	\citep[p. 72]{thompson1989thestructureofbiologicaltheories}
\end{quotation}

Bas van Fraassen, lui, conclut :
\begin{quote}
	De ce point de vue, le travail essentiel d'une théorie scientifique est de nous fournir une famille de modèles, que l'on pourra utiliser pour représenter les phénomènes empiriques \citep{vanfraassen1972aformalapproachtothephilosophyofscience}.
\end{quote}

C'est cette naïve interprétation rapidement esquissée (chaque auteur en ayant en vérité une version qui lui est propre et bien plus complexe), que nous voulons reprendre pour la suite de notre exposé et qui nous semble adapté pour étudier la théorie de l'évolution.

Nous avons parlé rapidement, dans le chapitre précédant, des problèmes que la théorie de l'évolution peut poser. Nous avons notamment décrit dans la section \ref{sec:pgs} les commentaires que Godfrey-Smith a pu faire vis-à-vis de certaines descriptions et formulations de cette théorie (les <<\,recettes\,>>), et de ce besoin de construire un schéma général qui ne marche pas, ou des schémas particuliers qui ne généralisent plus. Il est à noter aussi, mais nous ne le détaillerons pas ici, que de nombreuses critiques ont souvent pointé du doigt la biologie comme une science qui ne possède pas de loi, puisque toutes les règles qu'elle a pu mettre au point (lois de Mendel, équilibre de Hardy-Weinberg), présentent de nombreuses exceptions. Nous avons vu comment PGS intègre ces exceptions dans son espace : il construit un modèle qu'il veut souple, qui peut être déformer pour intégrer les différentes composantes et leurs exceptions. 

C'est pour répondre à ces incohérences, pour justifier le besoin de multiples modèles et asseoir la valeur scientifique de ces modèles biologiques dans une vision globale de la science, que \cite{beatty1980whatswrongwithreceivedwiew,beatty1980ptimaldesignmodelsandstrategyofmodelbuildinginevolutionarybiology,beatty1987onbehalfofsemanticview,thompson1989thestructureofbiologicaltheories,thompson1987adefenceofthesemanticconceptionofevolutionarytheory,lloyd1984asmanticapproachtothestructureofpopulationgenetics,lloyd1988thesemanticapproachanditsapplicationtoevolutionarytheory}, ont repris la conception sémantique des théories pour l'appliquer à la biologie et en particulier à la biologie de l'évolution.

Ainsi, si \cite{beatty1980whatswrongwithreceivedwiew} considère qu'il ne peut effectivement pas y avoir de loi au sens classique des théories (des <<\,lois de la nature\,>>), puisque, par exemple, les lois de Mendel, ou l'équilibre de Hardy-Weinberg, sont le fruit du processus évolutionnaire et sont toujours soumis à son action :
\begin{quote}
	C'est à dire que les théories évolutionnaire peuvent changer en tant que [elles sont le] résultat du changement évolutionnaire \citep[p.~407]{beatty1980whatswrongwithreceivedwiew}.
\end{quote}
Ceci n'est plus un problème à la lumière de la conception sémantique des théories. Comme nous l'avons déjà dit, suivant cette conception, les faits empiriques sont indépendants de la théorie et des modèles qui la composent. Ainsi, qu'importe si certains systèmes empiriques ne suivent pas l'équilibre traditionnel de Hardy-Weinberg et nécessitent une version améliorée et différente de celui-ci : les populations qui respectent l'équilibre classique seront décrites par la théorie tout comme celles qui nécessitent un modèle plus complexe \citep[p.~410-411]{beatty1980whatswrongwithreceivedwiew}.
Cette vision des théories scientifiques s'articule très bien avec notre volonté d'étudier la biologie à travers le prisme de modèles évolutionnaires robotiques (ou plutôt, notre volonté s'articule très bien avec la conception sémantique). Différents modèles de Robotique \'Evolutionnaire pourront décrire différents systèmes biologiques, sans qu'il soit nécessaire que le modèle robotique décrive une \emph{loi de la nature}. Et même si la plupart des auteurs que nous avons cités considèrent souvent les modèles comme des entités mathématiques formels, il y en a pour penser qu'ils peuvent être <<\,des modèles caractérisés non mathématiquement\,>>\citep[p. 1]{lloyd1988thesemanticapproachanditsapplicationtoevolutionarytheory}. Nous pensons et nous allons essayer de montrer, que les simulations informatiques peuvent elles aussi être perçu comme des modèles solides et intéressants, qui présentent de nombreuses propriétés particulières utiles pour comprendre la théorie de l'évolution.

\section{Conclusion}
Dans ce chapitre nous avons rapidement présenté quelques méthodes utilisées par les chercheurs pour étudier la théorie de l'évolution et essayer de résoudre les problèmes qu'elle génère. Nous avons d'abord parlé de la sélection artificielle et de son utilisation par Darwin puis, comment cette idée d'étudier des entités physique soumises à des pressions sélectives artificielles dans des <<\,expériences\,>> de laboratoire a perduré et quel est l'intérêt de ce type d'expériences. Nous avons ensuite brièvement parlé des expériences de pensée, pour introduire des méthodes laissant plus de libertés aux expérimentations et tests possibles.
Dans la deuxième partie de ce chapitre nous avons parlé des modèles en essayant de justifier leur utilisation en sciences via la conception sémantique des théories scientifiques. Nous avons soutenu l'intérêt de l'utilisation de la simulation informatique comme modèle et donc leur importance au sein la conception sémantique. 

Pour mieux cerner l'intérêt de ces simulations afin d'étudier la biologie, nous avons repris les analyses des chercheurs en Vie Artificielle en détaillant les modèles épistémiques conceptuels et leur liens avec les expériences de pensée. Nous pensons que ces modèles offrent la plus grande liberté de mano\oe uvre pour travailler sur la théorie de l'évolution, et nous verrons dans la dernière partie de ce mémoire un type de simulations qui tombent, selon nous, dans cette catégorie tout en essayant de s'affranchir de certaines limites évoquées dans ce chapitre.

L'objectif principal de cette partie a été de dégager certains avantages et inconvénients des méthodes décrites. Nous avons vu que certaines méthodes, comme l'analogie avec la sélection artificielle des éleveurs ou l'évolution dirigé de Lenski, essaient de se rapprocher le plus possible des expériences empiriques traditionnelles en fournissant des résultats directement applicables à la théorie de l'évolution des êtres vivants. Néanmoins ces méthodes souffrent de certaines difficultés techniques dues notamment aux contraintes imposées par le vivant. Ainsi, d'autres méthodes essayent d'abstraire les propriétés des êtres vivants via des expériences de pensées ou des simulations, dans le but de s'offrir une plus grand liberté d'études. En contrepartie, l'application des résultats obtenues par ces méthodes à la biologie requiert plus de précautions et peut parfois être un véritable problème, voir impossible.
