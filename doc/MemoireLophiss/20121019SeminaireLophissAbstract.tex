\documentclass[a4paper,10pt]{article}

%%%%lualatex on
%\usepackage{luatextra}
\usepackage{fontspec}
%Ligatures={Contextual, Common, Historical, Rare, Discretionary}
\setmainfont[Mapping=tex-text]{Linux Libertine O}

%%%lua off
%\usepackage[utf8x]{inputenc}
%\usepackage[T1]{fontenc} 
%\usepackage{lmodern}

\usepackage{enumerate}
\usepackage{graphicx}
\usepackage[frenchb]{babel}
\usepackage[authoryear,round]{natbib}
\usepackage[top=2.5cm,bottom=3cm,left=2.5cm,right=2.5cm]{geometry}

\usepackage{lettrine}
%%\usepackage{paris7}

\title{Robotique Évolutionnaire et Biologie de l'Évolution} 

\author{Simon Carrignon}
\begin{document}

\maketitle

\begin{abstract}
	Nous allons voir dans cet article comment la Robotique peut être un nouvel outil pour étudier la théorie de l'évolution. Pour cela nous reprendrons les grandes lignes de la théorie de l'évolution et les problèmes qu'elle soulève. Nous verrons notamment le débat sur les unités et niveaux de sélection. Puis nous terminerons en présentant la Robotique \'Evolutionnaire comme outil pour explorer ces débats en illustrant notre propos avec une étude de \cite{waibel09geneticteamcompositionlevelselectionevolutioncooperation} sur ces questions de niveaux de sélection.
\end{abstract}

\section{La théorie de l'évolution}\label{sec:TE}
En 1859, Charles Darwin proposait une théorie pour essayer de rendre compte de la diversité des êtres vivants dans son livre l'Origine des Espèces\nocite{darwin1859originspeciesbymeansnaturalselectionorpreservationfavouredracesstrugglelife}. Il la résumait comme:
\begin{quote}
	%%Theory of descent with modification by variation and Natural Selection.
	Une théorie de la descendance avec modification par la variation et par la Sélection Naturelle.\\
	\cite[dernière édition, trad. \cite{gayon1991darwinetlapresdarwin}]{darwin1859originspeciesbymeansnaturalselectionorpreservationfavouredracesstrugglelife}.
\end{quote}
et elle fut plus couramment appelé la théorie de l'évolution par Sélection Naturelle.

Pour bâtir cette théorie \cite{gayon1991darwinetlapresdarwin} nous explique que Darwin propose et admet deux mécanismes:
\begin{itemize}
	\item L'hypothèse de la Sélection Naturelle : autrement dit ``la survie du plus apte''. %%(pour reprendre la formulation de Spencer).
	\item La descendance avec variation, autrement dit ``l'hérédité des différences individuelles'' \citep{gayon1991darwinetlapresdarwin}.
\end{itemize}

L'admission de ces hypothèses, et donc de la théorie qui en découle, permet de rendre compte de : l'évolution des espèces dans le temps (avec chez Darwin une notion de ``progression'', ``d'amélioration'' des espèces) et de leur divergences les unes des autres. Mais cette théorie (et les hypothèses qu'elle présuppose), de par sa structure et son contenu, n'est simple ni à observer ni à démontrer. Surmonter ces difficultés et essayer de convaincre du bien fondé de ces théories fut l'une des tâches principales de Darwin, qu'il mena dans ses livres depuis l'Origine des Espèces jusqu'à ses derniers ouvrages. Néanmoins nous verrons dans la partie \ref{sec:lvl}, après avoir brièvement rappelé et éclaircit les composantes de la théorie et les difficultés qu'elles soulevèrent au début de cette partie, que ces difficultés sont toujours d'actualité et qu'elles continuent d'alimenter les réflexions des scientifiques et des philosophes. 

\subsection{Descente avec modifications}\label{sec:hered}
Avant toute chose, et pour que la sélection Naturelle puisse agir et donc l'évolution selon Darwin avoir lieu, il faut que les caractères qui déterminent la fitness\footnote{La fitness est difficilement définissable en anglais donc encore plus difficilement traduisible en français. On parle parfois de \emph{valeur sélective} mais nous conserverons souvent le terme anglais de \emph{fitness}.} des individus (c'est à dire leur degré d'adaptation à leur environnement) soient transmis à leurs descendants. Et non seulement ils doivent être transmis, mais pour qu'il y ai du changement, ils doivent aussi varier. Hors, ces variations et cette hérédité (la transmission), Darwin (il le concède lui même) ne possède pas de théorie satisfaisante pour en rendre compte. Comment les parents transmettent-ils leurs caractères à leurs descendants et pourquoi ces caractères varient-ils, sont des phénomènes encore assez mal compris à l'époque. Néanmoins Darwin battît sa théorie de l'évolution en présupposant un certain type de transmission et un certains type de variations. Selon lui pour que l'évolution ai lieu il faut que les caractères transmis \emph{varient} de façon aléatoire et \emph{graduellement} chez les descendants. Par exemple : une population d'individus de taille X aura des descendants avec des tailles comprises entre X-n et X+n qui varient de façon quasi continue entre -n et n. Ainsi pour supporter sa théorie générale de l'évolution Darwin doit donc construire et accepter une théorie de l'hérédité qui présente de telles propriétés et qui de plus est conforme aux nombreuses observations qu'il a faites dans la nature (c'est ce qu'il fera dans \cite{darwin1868variation}). Mais l'adoption de ces propriétés et de cette théorie de l'hérédité pose plusieurs problèmes.

Dans un premier temps des problèmes théoriques. Un des détracteurs de Darwin, Jenkin, soutiendra en l'illustrant mathématiquement (ce sera d'ailleurs pour Darwin la critique la plus sérieuse faite à sa théorie) qu'une sélection agissant sur des variations continues ne pourra que stabiliser ces variations autours de valeurs moyennes et n'aboutira pas à l'apparition de nouvelles valeurs moyennes (de nouveaux phénotypes). Cette critique Darwin ne réussira pas vraiment à la surmonter puisque elle est une conséquence des choix qu'il a fait vis de l'hérédité et des variations. 

Dans un second temps le problème empirique de la redécouverte des lois de Mendel  au début du XIX va venir se poser. Quelques décennies après la disparition de Darwin, les biologistes cellulaires montreront que, si il y a bien une transmission héréditaire de certains caractères via ce qu'il sera convenu d'appeler les gènes, ces caractères ne varient pas de façon continue chez les descendants, mais sont des caractères discrets qui ne peuvent qu'être hybridés selon certaines lois statistiques régulières. Ils ne peuvent pas véritablement se ``mélanger'', ce qu'avait imaginé Darwin dans \cite{darwin1868variation}. Paradoxalement, c'est en réconciliant cette génétique avec Darwin que disparaitra le problème soulevé par Jenkin. 

\subsection{Hypothèse de la Sélection Naturelle}\label{sec:SN}
Une fois une théorie de l'hérédité des variations individuelles admise, la sélection naturelle peut jouer son rôle. On peut la résumer ainsi :
\begin{quote} les individus avec les variations offrant à ceux qui les portent la meilleure adaptation (ie les individus qui possèdent la meilleure fitness) vivrons plus longtemps et/ou laisserons en moyenne plus de descendants.\end{quote}

	Pour \cite[p. 22]{gayon1991darwinetlapresdarwin} c'est ``l'hypothèse organisatrice'' de la \emph{théorie} de Darwin. Mais Darwin ne possède aucun \emph{fait} justifiant cette hypothèse. Pour arriver à convaincre il va utiliser l'analogie avec la Sélection Artificielle et s'appuyer sur les très nombreuses études d'éleveurs, qu'il connait très bien, pour montrer comment les humains ont été capables de transformer et faire diverger de beaucoup les espèces domestiques. Selon lui, si les êtres humains, en appliquant ce genre de pression sélective explicite et direct, sont capable de modifier les espèces vivantes, alors il doit exister une force analogue capable de modifier les espèces dans la nature. Une force qui pourrait rendre compte de la diversité et des adaptations du vivant. Cette force, qui agirait donc sur des variations transmises par hérédité, comme il l'a définit et discuté en \ref{sec:hered}, il l'appelle la sélection naturelle, et il la décrit grossièrement comme : les individus les mieux adaptés se reproduiront plus et auront plus de descendant dans les générations suivantes. 


Il est intéressant de noter qu'en définissant la Sélection Naturelle, Darwin insiste sur le fait que les variations qui vont, ou non, augmenter les chances de survie et de reproduction (la fitness) sont portées par les individus (individus qu'il identifie aux ``organismes biologiques, identité dont nous reparlerons). Pourtant ce n'est pas si évident qu'il n'y parait, et à l'époque déjà, Wallace par exemple, co ``découvreur'' de la théorie de l'évolution par sélection naturelle, avançait que les variations étaient des différences entre des populations d'individus, et que la sélection portait sur ces populations, et non sur les individus qui les composent. 

Mais ni l'un ni l'autre ne pouvait prouver son point de vue, ni théoriquement, ni empiriquement. Il faudra attendre les biométriciens, et les nouvelles méthodes statistiques apportées par Galton, puis Pearson et Weldon, pour avoir des premières preuves mathématiques et empiriques de l'action de la Sélection Naturelle. Néanmoins ces preuves ne suffiront pas vraiment à départager un sélection ``individuelle'' d'une sélection ``populationnelle''.

Pearson et Weldon par exemple vont mesurer des variations morphologiques chez des crabes pour montrer qu'il y a bien déplacements des moyennes de certains traits morphologiques, signes selon eux de l'action de la sélection naturelle. Pour autant ces résultats ne répondent pas directement aux critiques formulées par Jenkin. Si ils montrent que la sélection peut agir au niveau des individus et faire se déplacer certains caractères morphologiques, ils ne démontrent pas que les espèces peuvent durablement se modifier par se biais et diverger véritablement. D'où les arguments de l'époque, faits par les ``mutationistes'', qui avanceront qu'il est nécessaire d'avoir des mutations brutales et importantes qui vont beaucoup modifier certains caractères morphologiques pour obtenir une évolution. Ces critiques seront d'autant plus fortes que la redécouverte des lois de Mendel leur offrira l'appuie empirique dont ne jouissait pas la théorie plus ``gradualiste'' que Darwin avait envisagé.

\subsection{Théorie synthétique de l'évolution}\label{sec:SM}
Au début du XXe siècle la théorie de l'évolution affronte donc un problème. D'un côté Weldon et Pearson démontrent statistiquement et empiriquement qu'une évolution de caractères aux variations graduelles est possible, d'un autre côté les études empiriques montrent que ces caractères graduels n'apparaissent que rarement voir pas du tout dans la nature, et qu'en réalité il n'y a que des caractères discrets qui s'hybrident selon les lois de Mendel. Par conséquent, pour les partisans de ce \emph{mendelism}, la sélection naturelle de caractères qui varient graduellement (ce qu'avait défini Darwin) ne peut pas faire évoluer les espèces. Ce sont les mutations ponctuelles de caractères discrets qui doivent être à ``l'origine des espèces''. 

Mais la donne change lorsqu'en 1918 Fisher démontre que le \emph{gradualisme} nécessaire à la sélection naturelle de Darwin n'est pas incompatible avec des caractères mendéliens. Avec Wright et Haldane ils vont, pendant la première moitié du XXe siècle, finir de réconcilier théoriquement Mendel et Darwin. S'en suivirent plusieurs décennies de synthèse, au cours desquelles les différents domaines de la biologie vont être rattachés empiriquement et théoriquement à cette théorie de l'évolution ``réconciliée''. Cette période, et le paradigme scientifique qui en découla, fut appelé la synthèse moderne (ou théorie synthétique de l'évolution en français) d'après le livre d'un de ses artisans : \cite{huxley1942evolution}. Elle fut témoin de nombreuses avancées et a eu (et a toujours) un impact profond sur la biologie dans toute sa diversité. Nous ne rentrerons pas dans les multiples détails de cette riche époque ; ce que nous retiendrons est qu'un certain consensus sur une théorie de l'évolution par sélection naturelle a émergé. Cette théorie a armé les scientifiques de preuves et de nombreux outils formels et empiriques qui peuvent s'appliquer sur, et dont le bien fondé a été confirmé par, tous les champs de la biologie (paléontologie, écologie, biologie cellulaire,~\ldots). 

Néanmoins nous allons voir que cette synthèse n'a pas résolu tous les problèmes théoriques soulevés par les propositions de Darwin, et nous allons voir comment ceux-ci sont réapparus et ont été repris par les scientifiques et philosophes depuis les années 1970.

\subsection{Niveaux et unités de sélection}\label{sec:lvl}
Dans la section \ref{sec:SN} nous avons dit que, pour peu qu'ils transmettent leurs caractères à leur descendants et que ces caractères varient, les ``individus'' les mieux adaptés vivront plus longtemps et auront plus de descendants. Ainsi générations après générations les populations se verront composées de plus en plus d'individus mieux adaptés, les caractères divergeront, les espèces évolueront. Nous avons vu que cette définition a posé quelques problèmes quant à la nature des variations nécessaires, problèmes qui furent partiellement résolus par la synthèse moderne.

Mais la théorie de Darwin soulève d'autre questions. Dans un article majeur sur le sujet, \cite{lewontin70unitsselection} reprend la théorie de Darwin pour la formuler et la résumer dans des termes plus actuels de la façon suivante :
\begin{enumerate}
	\item Dans une population des individus différents ont une morphologie, une physiologie et des comportements différents (variation phénotypique).
	\item Des phénotypes différents ont des taux de survie et de reproduction différents dans des environnements différents (fitness différentielle).
	\item Il y a une corrélation entre parents et descendants à chaque génération future (hérédité de la fitness).
\end{enumerate}

L'auteur explique que le niveau d'abstraction de cette définition n'impose pas l'organisme comme ``individu''. N'importe quelle entité répondant aux trois critères que Lewontin énonce peut remplir ce rôle d'individu. Pour lui cette définition s'applique aux différents niveaux d'organisation biologique, les ``individus'' pouvant aussi bien être les cellules que les chromosomes, les gènes, les organes au sein d'un organisme, les organismes entre eux ou même les espèces d'organismes entres elles. 

Déjà à l'époque, Wallace considérait que les populations et non les individus pouvaient être sélectionnés, suggérant ainsi que la sélection pouvait être interprétée à différents niveaux. Et même si pour Darwin la pression de sélection ne peut porter que sur les variations des individus (pour lui des organismes mutlicelllulaires), il considérera dans son livre \emph{The descent of man} \citep{darwin1871thedescentofman}, que les sociétés humaines (donc des \emph{populations} d'individus biologiques) peuvent être soumises à une force similaire à la sélection naturelle qu'il avait défini pour les organismes. Weismann, au début du XXe siècle fut un de ceux qui théorisa le mieux ces idées avant qu'elles ne soient éclipsées par les travaux de la synthèse moderne.

Dans la foulée de l'article de Lewontin beaucoup se penchèrent à nouveau sur ces questions. D'autant qu'au sortir de la synthèse moderne un certain consensus était adopté, avec le gène comme unité principale (voir unique) de sélection. Ce consensus, appelé aussi ``gene eye view'', d'abord proposée par \cite{williams1966adaptationandnaturalselection} puis popularisée par \cite{dawkins76selfishgene} eu un certains succès dans la communauté scientifique et auprès du large publique. Cette vision réductionniste a une valeur heuristique forte et a permis de simplifier bon nombre de problèmes mais elle fut très vite l'objet de nombreuses critiques. \cite{wimsatt1980theunitsofselectionandthestructureofthemultilevelgenome}, par exemple, remettra en cause la vision de Williams (et de Dawkins), en argumentant qu'elle ne peut rendre compte de tous les phénomènes évolutifs en biologie. Pour lui la fitness des individus n'est pas réductible à la fitness des gènes qu'il possède, ces derniers interagissant de façon totalement non linéaire entre eux. \cite{gould2002thestructureofevolutionarytheory} écrira même, dans l'imposante synthèse de sa réflexion qu'est ``la structure de la théorie de l'évolution'', que :
\begin{quote}
	[\ldots] la théorie de l'évolution centrée sur le gène était indéfendable.\\
	\citep[p. 855]{gould2002thestructureofevolutionarytheory}
\end{quote}

Ces problèmes et visions divergentes sont nombreux et, comme l'écrit Gould dans le même ouvrage,
\begin{quote}
	[o]n pourrait organiser la discussion sur ce sujet très difficile et très important d'une centaine de façons différentes.\\
	(ibid, p. 833).
\end{quote}
Il est donc difficile d'en faire une revue exhaustive mais on peut parler par exemple de l'approche de \cite{dawkins76selfishgene} et \cite{hull1974philosophyofbiologicalscience} qui, pour clarifier le débat, ont proposé de distinguer les intéracteurs (qui agissent dans l'environnement et interagissent entre eux) des réplicateurs (qui se répliquent de générations en génération). Cette séparation fut très discutée, modifiée, affinée et a eu le mérite d'offrir de quoi réfléchir aux scientifiques et philosophes. En revanche il reste toujours difficile et controversé de savoir à quels niveaux d'organisation biologique ce situent ces entités. Pour pallier certains des problèmes de cette distinction dite de Hull-Dawkins, Griesemer (2000) proposera le concept de ``reproducteur'', qui intègre le caractère développementale des individus. D'autre pour discuter ces problèmes, choisissent d'étudier la transmission des caractères en prenant des colonies (des superogargnismes) comme individus \citep{wilson1989revivingsuperorganism}, certains choisissent les espèces ou même les idées culturelles \citep[p. 147]{godfrey2009darwinian} comme individus.


Parmi ces différents courants, des chercheurs avancent que l'évolution peut être comprise et réduite à l'étude d'une entité simple comme le gène\cite{dawkins76selfishgene,dennett95darwinsdangerousideaevolutionmeaningslife}, d'autres soutiennent que c'est impossible et qu'il faut étudier des niveaux supérieurs \citep{gould2002thestructureofevolutionarytheory,wilson1989revivingsuperorganism}, d'autres encore avancent que ces visions sont similaires. Quoiqu'il en soit le problème est loin d'être résolu et le débat continue. 

Pour essayer de répondre aux questions que ce débat soulève et pour tester les limites, les atouts et les faiblesses des différentes solutions avancées, les scientifiques et les philosophes développent tout un arsenal d'expériences de pensées, de modèles mathématiques, informatiques ou encore d'expériences empiriques. Présenter ces méthodes, analyser leurs avantages et inconvénients serait utile mais dépasse le cadre de cette étude. 
%d'autant que cela nécessiterais la présentations des approches épistémologiques plus larges dans lesquelles elles trouvent, ou non, leur valeur.
Nous nous contenterons ici de présenter un modèle récent qui nous semble idéal pour étudier ce type de problèmes. Ce modèle les chercheurs l'ont appelé la Robotique \'Evolutionnaire. Après avoir expliquer les principes et l'histoire de cette méthode, nous présenterons une étude qui l'utilise pour tenter d'explorer la question des unités et niveaux de sélection dont nous venons de dessiner les grandes lignes.

\section{Robotique \'Evolutionnaire}
\subsection{Algorithmique évolutionnaire}\label{sec:ae}
La Robotique \'Evolutionnaire est une descendante direct de ce qu'il est convenu d'appelé l'Algorithmique \'Evolutionnaire (ou AE, en anglais \emph{Evolutionnary Computation, EC})\footnote{Nous reprenons ici la traduction utilisé par \cite{schoenauer09lesalgorithmesevolutionnaires} et ne faisons pas non plus de distinction entre le terme \emph{évolutionnaire} et le terme \emph{évolutionniste}, que ce soit lorsque l'on parle d'Algorithmique Évolutionnaire ou de Robotique Évolutionnaire, l'un comme l'autre aurait pu être utilisé, le premier ayant été semble-t-il choisi par les biologistes lors de la traduction de Darwin, le terme évolutionnaire, ayant plutôt été utilisé par les informaticiens et philosophe de la biologie contemporain francophone, qui ont traduit directement les textes en anglais sans avoir reçu de formation en biologie en français auparavant. Pour certaines raisons qu'il n'est pas utile de développer ici et surtout par préférences personnelles nous préfèrerons le second terme.}.
Ce terme désigne un ensemble de méthodes informatiques originales apparues dans les années 60-70, \footnote{Pour une brève introduction en français sur l'Algorithmique Évolutionnaire, son histoire et ses diverses sous-composantes, voir \citet{schoenauer09lesalgorithmesevolutionnaires}, pour une introduction complète et en anglais voir \citet{eiben03introductiontoevolutionarycomputing}.} dont le point commun est que toutes reprennent l'idée de Darwin \citet{darwin1859originspeciesbymeansnaturalselectionorpreservationfavouredracesstrugglelife} de l'évolution par sélection naturelle que nous avons présenté dans la première partie. L'intuition est la suivante : si la force adaptative de l'évolution telle que nous l'avons décrite a permis l'émergence d'espèces et d'organes extrêmement adaptés aux contraintes et conditions environnementales pourquoi ne pourrait-elle pas servir pour programmer des outils et calculer des solutions à des problèmes informatiques ? 

C'est ce que les informaticiens ont fait, en tentant ainsi d'implémenter un <<~solveur universel~>> (en anglais \emph{universal solver}) capable de trouver automatiquement les solutions à tout ---ou au moins une large gamme--- de problèmes possibles. 
Nous n'illustrerons pas ici les réussites de l'AE et vous renvoyons une fois encore aux introductions de \cite{schoenauer09lesalgorithmesevolutionnaires} et \cite{eiben03introductiontoevolutionarycomputing}, un des exemples de problème bien connu résolu par l'AE est le problème du voyageur de commerce, mais nous ne le détaillerons pas non plus ici.
%mais croyez nous, \c{c}a à marché. L'exemple le plus connu est probablement celui du voyageur de commerce mais nous ne le détaillerons pas non plus, GIYF comme disent les jeunes.

Brièvement résumé le principe de l'Algorithmique \'Evolutionnaire est le suivant : pour trouver la solution à un problème, on génère un ensemble de solutions aléatoirement, on mesure la proximité de ces solutions avec la solution optimale, on sélectionne les meilleures et on les modifie, les mixe, et on recommence.

Il a ensuite fallu attendre une vingtaine d'années que ces méthodes d'informatique évolutionnaire fassent leurs preuves et qu'en parallèle la robotique évolue (ou du moins change d'approche, changement que nous ne verrons pas ici mais voir notamment \citet{brooks91intelligencewithoutreason} ou plus récemment \cite{pfeifer2006howthebodyshapesthewaywethink}), pour qu'apparaisse la Robotique Évolutionnaire. 
Il n'est plus question ici de trouver des solutions à des problèmes en utilisant des représentations où l'analogie avec la biologie est lointaine et se borne souvent, comme pour l'exemple du voyageur de commerce, à des emprunts de vocabulaire (l'individu étant une suite de villes bien différentes de ce que peut être un individu biologique). Avec la robotique la donne change. 

\subsection{Robotique \'Evolutionnaire}\label{sec:re}
Pour les roboticiens le but est de construire des robots amenés à se déplacer dans un environnement <<~réel~>>, ouvert et changeant. Les contraintes qui s'appliquent sur les systèmes à concevoir sont donc \emph{a priori} les mêmes que celles auxquelles sont soumis les êtres vivants. De plus, les t\^{a}ches que ces <<~agents>> doivent effectuer ressemblent beaucoup aux comportements que peuvent avoir les êtres vivants dans leur activité quotidienne. Par exemple, le chercheur en robotique veut que son robot soit capable de se déplacer dans l'environnement tout en évitant les obstacles ; ou bien il veut que le robot puisse transformer et échanger avec son environnement pour assurer son autonomie énergétique et sa sécurité ; ou encore que le robot communique et se synchronise avec d'autres agents pour résoudre des tâches complexes qu'il ne pourrait pas accomplir seul.

L'analogie est donc forte. Les caractéristiques des entités étudiées sont les mêmes et l'environnement dans lequel elles sont étudiées aussi. Il parait donc judicieux d'utiliser une théorie de l'évolution biologique pour construire des robots, et c'est bien ce que les pionniers de la discipline \citep{nolfi00evolrobobiolintetechselfmach} ont pensé et dont ils ont démontré, depuis les années 90, l'efficacité.

Mais si l'analogie est si forte, ne pourrait-on pas renverser les rôles et utiliser la Robotique \'Evolutionnaire comme un modèle pour étudier la biologie ? Cette approche semble d'autant plus justifiée que beaucoup de scientifiques et de philosophes avancent que la biologie, de par sa nature, à beaucoup plus de chance d'être mieux comprise si elle est étudiée à travers des modèles ayant une porté sémantique locale, plutôt qu'à travers des lois générales. Nous ne pourrons détailler cette vision de la biologie mais voir par exemple \cite{beatty1987onbehalfofsemanticview}. Il serait aussi intéressant dans cette optique de voir ce que les simulations informatiques apportent aux modèles scientifiques par rapport aux outils traditionnel de modélisation, mais là encore nous n'auront pas le temps de traiter ici cet aspect du problème, pour plus d'information voir le texte de Youna dans ce recueil ou encore \cite{winsberg03simulatedexperimentsmethodologyforavirtualworld}.

Pour résumer rapidement les principes de la Robotiques \'Evolutionnaire, les chercheurs vont faire évoluer les contrôleurs des robots (le programme en charge du comportement du robots\footnote{Ce sont les comportements des robots qui sont évolués et non la morphologie de ces derniers, ce qui fait une différence très importante avec la biologie où les deux composantes évoluent conjointement. Cette différence est un problème et une limitation d'envergure en Robotique \'Evolutionnaire, limitation essentiellement du aux technologies existantes et dont les chercheurs ont conscience et sur laquelle de nombreux travaux sont faits\citep{pollack2000thegolemproject}.}). Ce sont ces contrôleurs les solutions de leur algorithme évolutionnaire. Pour commencer ils vont générer aléatoirement des contrôleurs qu'ils vont injecter dans les robots, puis ils vont tester les robots sur la tâche à résoudre, sélectionner les meilleurs contrôleurs, les muter, les mixer, générer de nouveaux contrôleurs (une nouvelle génération) puis re-injecter et re-tester les robots. 

Sans rentrer dans les détails et subtilités mises aux points par les chercheurs pour faire celà, nous allons plutôt, à titre d'illustration reprendre rapidement l'étude de \cite{waibel09geneticteamcompositionlevelselectionevolutioncooperation} qui explore les questions de niveau de sélection abordées dans la section \ref{sec:lvl}.


\subsection{Niveau de sélection et composition des groupes \citep{waibel09geneticteamcompositionlevelselectionevolutioncooperation}}\label{sec:nscg}

Dans cette étude les auteurs veulent tester différentes méthodes de sélection sur différents groupes de robots. Ils le font dans l'optique de quantifier l'efficacité de différentes approches pour programmer automatiquement des robots. Nous allons essayer de montrer que ces résultats peuvent être analysés et interprétés différemment si on les éclaire à lumière des débats discutés précédemment (cf \ref{sec:lvl}).

Pour réaliser leur étude les auteurs proposent trois tâches dans lesquelles différents niveaux d'altruisme (1. pas d'altruisme,2. coopération,3.  altruisme) sont nécessaires.
Dans les trois tâches des groupes de robots doivent déplacer des jetons vers une zone particulière (peinte en blanc) d'une arène. Dans la tâche où aucune coopération n'est nécessaire, les robots doivent déplacer des petits jetons qu'un robot seul peut pousser. Chaque fois qu'un robot ramène un de ces jetons, sa fitness est augmentée de un. Dans la tâche où la coopération\footnote{Nous considérons l'altruisme comme un acte de coopération qui n'offre aucun avantage sélectif à l'individu qui coopère mais en donne à l'autre. Lors d'un acte de coopération ``classique'' les deux protagonistes gagnent à coopérer.} est requise, il faut déplacer des gros jetons qui ne sont déplaçables que part au moins deux robots. Lorsqu'un gros jeton est ramené, la fitness de \emph{tous} les robots de l'expérience est augmentée de un, que les robots aient ou non participé au déplacement du jeton. Dans la dernière tâche les deux types de jetons sont présents, les mêmes systèmes de distribution des points de fitness que les tâches précédentes sont repris.

Les comportements seront contrôlés par un réseau de neurones artificiels assez simple\footnote{Un perceptron mutli couches.} reliant des capteurs infrarouges qui permettent au robot d'éviter les obstacles, une caméra qui permet différencier les jetons et de voir la zone blanche (un mur), et les moteurs qui contrôlent les roues pour déplacer le robot.
Les auteurs vont ensuite, par évolution artificielle, faire évoluer les paramètres de ces contrôleurs pour que les robots soient capables de résoudre les tâches imposées, en étudiant deux aspects particuliers de leur stratégie de sélection : le niveau auquel la sélection s'effectue et la composition des groupes sur lesquels elle opère. 

Par niveau de sélection les auteurs entendent la chose suivante : dans un algorithme d'évolution artificielle (cf section \ref{sec:ae}) des comportements sont générés aléatoirement, puis testés et évalués sur des robots dans l'environnement (réel ou simulé). \`A l'issue de ces tests le robot obtient une valeur de \emph{fitness} qui servira pour sélectionner les meilleurs individus. Mais le problème est de savoir si, lors d'une tâche dans laquelle une coopération est nécessaire, il ne vaut pas mieux sélectionner une populations au complet ayant bien résolu la tâche plutôt qu'un individu seul ayant une fitness élevé, au risque de ne sélectionner que des individus égoïstes. 

Pour comprendre ce que les auteurs entendent par composition des groupes il faut essayer de mieux comprendre l'expérience. Pour faire évoluer les comportements des robots les chercheurs doivent générer un certain nombre de groupes et tester l'efficacité de ces groupes à résoudre la tâches. Il faut ensuite constituer les générations suivantes en générant de nouveaux groupes en fonction des résultats des groupes initiaux. Mais comment doivent être constitués les groupes initiaux et les groupes des générations suivantes ? Avec des individus similaires ou différents ? Les générations suivantes doivent-elles être constituées de groupes homogènes composés de descendants des meilleurs individus des générations précédentes, ou bien d'un mix des individus des meilleurs groupes? Voilà la question derrière ``composition des groupes''. Que se passe-t-il si nous avons des groupes uniformes ``génétiquement'', dont les individus sont les ``clones'' de la même ``souche'' (ce qui est par exemple le cas pour les cellules d'un organisme pluricellulaire ou d'une colonie de fourmis) ou si les populations des générations N+1 sont composé de mixtes aléatoires des descendants des meilleurs individus de la génération N.

En couplant ces questions à celles sur les niveaux de sélection les auteurs veulent voir si certains types de sélection (groupe vs individu) favorisent l'évolution de certaines caractéristiques (altruisme) suivant les groupes sur lesquels la sélection opère (groupes homogènes vs hétérogènes). On peut imaginer par exemple que, dans le cas de la tâche où l'évolution de l'altruisme est nécessaire, si les meilleurs individus sont sélectionnés dans des groupes hétérogènes, il sera possible de sélectionner des individus égoïstes d'autant plus si ils ont été testés dans des groupes avec beaucoup d'altruistes. Les groupes des générations suivantes seront donc peu performants à la tâche puisque composés uniquement d'égoïstes et donc incapables de pousser les gros jetons qui maximiseraient leur fitness. Alors que si la sélection s'opère sur les groupes, les groupes des générations suivantes, composé de mix d'individus choisis aléatoirement dans les meilleurs groupes d'individus, auront des chances d'être composés d'altruistes et donc de réussir la tâche.

Bien qu'ici posées dans un cadre d'optimisation et d'ingénieurie, ces questions sont exactement celles que biologistes et philosophes se posent. L'idée est donc de transférer les résultats des ces expériences à la biologie. Mais en le faisant un certain nombre de précautions doivent être prises et il faut garder à l'esprit que ces expériences se font dans un cadre et selon un protocoles très particulier, qu'il convient de considérer avec attention. Il faut donc bien vérifier que le système biologique auquel on veut appliquer l'analogie corresponde au protocole expérimental si on veut pouvoir étendre les résultats de l'expérience en robotique à un système biologique particulier.  


Nous ne rentrerons pas dans le détails des résultats de l'expérience et n'en présenterons que l'essentiel. Dans tout les cas les algorithmes d'évolution artificielle sont capable de faire évoluer les comportements adéquats pour résoudre les t\^aches demandées, ce qui irait dans le sens de certains pluralistes qui considèrent que la sélection de groupes peut exister conjointement à la sélection individuel, les deux n'étant que le reflet d'un même phénomène observé différemment. 
Néanmoins des différences significatives dans la \emph{vitesse de convergence}\footnote{Rapidité de l'algorithme à trouver des comportements adéquats} ou dans la qualité des solutions obtenues sont observables. En règle générale les tâches ne nécessitant pas de coopération sont mieux réussies par des groupes hétérogènes, tandis que la sélection (individuelle ou de groupe) au sein de groupes homogènes est plus efficace pour les tâches où la coopération est nécessaire. Les auteurs analysent les raisons de ces différences pour essayer d'en comprendre les causes, et c'est là que le biologiste doit être vigiliant. Car avant d'étendre ces résultats à un système biologique le chercheur doit faire très attention à ce que le protocole expérimental appliqué par les roboticiens corresponde à son système biologique. La façon dont la ``sélection de groupes'' a été implémentée peut-elle correspondre à la façon dont l'objet biologique est sélectionné? le mode de propagation, les tailles de populations, etc.. toutes ces variables doivent être étudiées et comparées si l'on veut situer l'étendu et la valeur des informations que le modèle peut fournir au biologiste. D'autant plus que ces protocoles n'ont pas été pensés initialement pour répondre à des problèmes biologiques, mais d'ingénieurie.



\section{Conclusion}
\'Etudier la théorie de l'évolution via la Robotique \'Evolutionnaire nous semble donc non seulement souhaitable, mais aussi idéal. N'ayant pas pris le temps d'énoncer les méthodes classiques utilisées pour étudier la théorie de l'évolution il est difficile d'argumenter en la faveur de la robotique mais nous pensons qu'elle est \emph{au moins} aussi pertinente que les méthodes de recherche et d'inférence traditionnelles. De par sa proximité avec la biologie elle permet de pointer du doigt certaines possibilités (ou impossibilités) qui, remises dans le contexte de la biologie théorique, peuvent indiquer des axes à suivre pour étudier des problèmes posée par la théories de l'évolution. Une fois ces possibilités mises en avant c'est dans un échange continuel avec la biologie que le modèle peut et doit être affiné et que seront approchées les réponses aux problèmes soulevés. Pour que cet échange soit fructueux il faut que les analogies soient finement décortiquées et que les concepts transposés de l'une à l'autre discipline soient judicieusement transposés. Par judicieusement nous entendons que l'analogie avec la biologie doit sans cesse être questionnée et les différences et limites clairement mises en avant ; mais aussi qu'il ne faut pas rejeter hâtivement des conclusions parce que contraires à certains modèles biologiques, généralement admis mais qui peuvent s'avérer largement discutables et dont le modèle artificiel ne fait que pointer les faiblesses.

\bibliographystyle{apalike}
\bibliography{/home/simon/Documents/biblio/memoireLophiss}

\end{document}
