\documentclass[a4paper,10pt]{report}

%%%%lualatex on
%\usepackage{luatextra}
\usepackage{fontspec}
%Ligatures={Contextual, Common, Historical, Rare, Discretionary}
\setmainfont[Mapping=tex-text]{Linux Libertine O}

%%%lua off
%\usepackage[utf8x]{inputenc}
%\usepackage[T1]{fontenc} 
%\usepackage{lmodern}

\usepackage{enumerate}
\usepackage{paralist}
\usepackage{graphicx}
\usepackage[french,ruled,vlined]{algorithm2e}
\usepackage[authoryear,round]{natbib}
\usepackage[frenchb]{babel}

%\usepackage[top=1.5cm,bottom=2cm,left=2.5cm,right=2.5cm]{geometry}
%\linespread{1.5}\selectfont

\usepackage{lettrine}
\usepackage{paris7}

%%Fichier Latex utilisé pour le mémoire envoyé le 26 août 2013 à Philippe Huneman et stephane schmitt frederic bouchard et nicolas bredèche.
\usepackage[colorlinks=true,urlcolor=black,citecolor=black,linkcolor=black,bookmarks=true]{hyperref}
\title{La Robotique Évolutionnaire comme modèle pour étudier la Biologie de l'Évolution }

\author{Simon Carrignon}

\begin{document}

\maketitle

%%Abstract based on the abstract use for the Young Researcher in Social Learning
\documentclass[a4paper,10pt]{report}
\usepackage{graphicx}


\begin{document}
\section*{Impact of cultural learning mechanisms on the emergence of a Walrasian Equilibrium}

In this poster we explore a model previously build to study the interaction between cultural mechanisms and economy. In this original model groups of agents produce, consume and exchange goods. They learn and change the strategies they use to exchange goods by copying the strategies of the more successful agents in the population.

We use for this study a variation of Approximate Bayesian Computation (ABC) to explore the parameter space of the original model and to compare it to an ideal case where all exchange of goods are made under a well  know economic equilibrium: the  general equilibrium (also called Walrasian equilibrium). ABC allows us to compute how likely different social learning process are able to produce a system where the emergence of a Walrasian Equilibrum is possible. At the same time it allows us to quantify the likelihood of the emergence of such situation under different set of fixed and limited constraints derived from historical cases studies.



\end{document}


\tableofcontents

\addcontentsline{toc}{chapter}{Introduction} 

\include{introduction}

\include{firstpart}  %%Théorie de l'evolution, histoire principes, problèmes

\chapter{Méthodes d'études traditionnelles, modèles et simulations informatiques}\label{ch:methode}

\lettrine[lines=2]{C}{e chapitre} va être l'occasion de passer en revue différents outils d'investigation qu'utilisent les chercheurs pour étudier la théorie de l'évolution. Sans nécessairement prétendre être exhaustif, nous commencerons par analyser rapidement les méthodes plus <<\,traditionnelles\,>> que sont l'analogie et les expériences de pensée.

Dans une seconde partie nous nous attarderons sur les modèles et leur application dans le domaine de la biologie. Nous verrons comment ils peuvent être intégrés dans une vision générale de la science via notamment l'approche sémantique des théories \citep{vanfraassen1980thescientificimage,suppe1989thesemanticconceptionoftheoriesandscientificrealism} et l'intégration de la biologie de l'évolution dans cette conception de la science \citep{thompson1987adefenceofthesemanticconceptionofevolutionarytheory,lloyd1984asmanticapproachtothestructureofpopulationgenetics,beatty1980whatswrongwithreceivedwiew}.  

Nous analyserons enfin plus en détails les simulations informatiques et notamment celles mises au point par les chercheurs dans le domaine de la \emph{Vie Artificielle}. Nous verrons comment sont perçues ces simulations par les acteurs de cette discipline \citep{barandiaran06alifemodelsasepistemicartefacts} et nous reprendrons à notre compte leurs positions et leur classification.



\section{Modèles et simulations informatiques}\label{sec:cmpdr:va} 

\subsection{Les modèles}
Depuis le milieu du XXe siècle, la biologie tout comme les sciences dans leur ensemble, ont vu se démocratiser l'utilisation et l'application de \emph{modèles}. Ces derniers représentent, via des abstractions, des schémas, des équations, certains objets du monde, certaines parties, certaines entités des théories scientifiques. Ces représentations permettent de mieux étudier et comprendre les objets physiques ou théoriques, d'affiner et explorer les théories scientifiques. 

La place particulière des modèles dans la science est un sujet de philosophie des sciences \emph{en général} (de \emph{toutes} les sciences), et est donc très vaste. Quel est leur lien vis à vis du monde réel, comment s'articulent-ils avec les théories scientifiques et avec les expériences empiriques traditionnelles, quel statut ontologique leur donner,etc., sont autant de questions complexes qui présupposent de nombreuses mises au point et partis pris philosophiques que nous ne pouvons traiter ici.

Mais malgré toutes ces questions, aujourd'hui, il semble que les différentes écoles de pensée philosophiques :
\begin{quote}
	s'accordent toutes pour dire que les modèles sont des unités centrales de la construction de théories scientifiques.\\ \citep{frigg2012modelsinscience}
\end{quote}

Nous choisirons de suivre cette constatation. Pour le reste de notre exposé nous allons présenter une version naïve et simplifiée de la vue sémantique des théories qu'ont développé, entre autres, \cite{suppe1989thesemanticconceptionoftheoriesandscientificrealism,vanfraassen1980thescientificimage}, pour comprendre les sciences en général et qui permet d'intégrer ces modèles à une vision globale de la science.

Cette conception des théories scientifiques soutient que ces dernières (les théories scientifiques) sont mieux décrites par des familles de modèles que par des axiomes mathématiques logiquement articulés qui décrivent directement le réel, des <<\,lois de la nature\,>> (\,comme les lois de Newton\,). 

\cite{thompson1989thestructureofbiologicaltheories} ---qui reprend les auteurs cités précédemment, résume le rôle des modèle au sein de cette conception sémantique ainsi :

\begin{quotation}
	[Dans la conception sémantique] une théorie (un modèle) est une entité mathématique qui n'est pas définie par référence à un système formel. En d'autres termes ; bien qu'un système formel dans lequel la théorie sera vraie puisse être construit, la théorie (le modèle) n'est pas construite comme une interprétation d'un tel système formel. Elle est définie directement en spécifiant le comportement du système. Et, plus important, les lois ne décrivent pas le comportement d'objets du monde ; elles spécifient la nature et le comportement d'un système abstrait. Ce système abstrait, indépendamment de sa spécification, est déclaré comme isomorphe avec un système empirique particulier. \'Etablir cet isomorphisme nécessite d'autres théories scientifiques et l'adoption de méthodologies (comme les théories de design expérimental et de d'ajustement des modèles).\\
	\citep[p. 72]{thompson1989thestructureofbiologicaltheories}
\end{quotation}

Bas van Fraassen, lui, conclut :
\begin{quote}
	De ce point de vue, le travail essentiel d'une théorie scientifique est de nous fournir une famille de modèles, que l'on pourra utiliser pour représenter les phénomènes empiriques \citep{vanfraassen1972aformalapproachtothephilosophyofscience}.
\end{quote}

C'est cette naïve interprétation rapidement esquissée (chaque auteur en ayant en vérité une version qui lui est propre et bien plus complexe), que nous voulons reprendre pour la suite de notre exposé et qui nous semble adapté pour étudier la théorie de l'évolution.

Nous avons parlé rapidement, dans le chapitre précédant, des problèmes que la théorie de l'évolution peut poser. Nous avons notamment décrit dans la section \ref{sec:pgs} les commentaires que Godfrey-Smith a pu faire vis-à-vis de certaines descriptions et formulations de cette théorie (les <<\,recettes\,>>), et de ce besoin de construire un schéma général qui ne marche pas, ou des schémas particuliers qui ne généralisent plus. Il est à noter aussi, mais nous ne le détaillerons pas ici, que de nombreuses critiques ont souvent pointé du doigt la biologie comme une science qui ne possède pas de loi, puisque toutes les règles qu'elle a pu mettre au point (lois de Mendel, équilibre de Hardy-Weinberg), présentent de nombreuses exceptions. Nous avons vu comment PGS intègre ces exceptions dans son espace : il construit un modèle qu'il veut souple, qui peut être déformer pour intégrer les différentes composantes et leurs exceptions. 

C'est pour répondre à ces incohérences, pour justifier le besoin de multiples modèles et asseoir la valeur scientifique de ces modèles biologiques dans une vision globale de la science, que \cite{beatty1980whatswrongwithreceivedwiew,beatty1980ptimaldesignmodelsandstrategyofmodelbuildinginevolutionarybiology,beatty1987onbehalfofsemanticview,thompson1989thestructureofbiologicaltheories,thompson1987adefenceofthesemanticconceptionofevolutionarytheory,lloyd1984asmanticapproachtothestructureofpopulationgenetics,lloyd1988thesemanticapproachanditsapplicationtoevolutionarytheory}, ont repris la conception sémantique des théories pour l'appliquer à la biologie et en particulier à la biologie de l'évolution.

Ainsi, si \cite{beatty1980whatswrongwithreceivedwiew} considère qu'il ne peut effectivement pas y avoir de loi au sens classique des théories (des <<\,lois de la nature\,>>), puisque, par exemple, les lois de Mendel, ou l'équilibre de Hardy-Weinberg, sont le fruit du processus évolutionnaire et sont toujours soumis à son action :
\begin{quote}
	C'est à dire que les théories évolutionnaire peuvent changer en tant que [elles sont le] résultat du changement évolutionnaire \citep[p.~407]{beatty1980whatswrongwithreceivedwiew}.
\end{quote}
Ceci n'est plus un problème à la lumière de la conception sémantique des théories. Comme nous l'avons déjà dit, suivant cette conception, les faits empiriques sont indépendants de la théorie et des modèles qui la composent. Ainsi, qu'importe si certains systèmes empiriques ne suivent pas l'équilibre traditionnel de Hardy-Weinberg et nécessitent une version améliorée et différente de celui-ci : les populations qui respectent l'équilibre classique seront décrites par la théorie tout comme celles qui nécessitent un modèle plus complexe \citep[p.~410-411]{beatty1980whatswrongwithreceivedwiew}.
Cette vision des théories scientifiques s'articule très bien avec notre volonté d'étudier la biologie à travers le prisme de modèles évolutionnaires robotiques (ou plutôt, notre volonté s'articule très bien avec la conception sémantique). Différents modèles de Robotique \'Evolutionnaire pourront décrire différents systèmes biologiques, sans qu'il soit nécessaire que le modèle robotique décrive une \emph{loi de la nature}. Et même si la plupart des auteurs que nous avons cités considèrent souvent les modèles comme des entités mathématiques formels, il y en a pour penser qu'ils peuvent être <<\,des modèles caractérisés non mathématiquement\,>>\citep[p. 1]{lloyd1988thesemanticapproachanditsapplicationtoevolutionarytheory}. Nous pensons et nous allons essayer de montrer, que les simulations informatiques peuvent elles aussi être perçu comme des modèles solides et intéressants, qui présentent de nombreuses propriétés particulières utiles pour comprendre la théorie de l'évolution.

\section{Conclusion}
Dans ce chapitre nous avons rapidement présenté quelques méthodes utilisées par les chercheurs pour étudier la théorie de l'évolution et essayer de résoudre les problèmes qu'elle génère. Nous avons d'abord parlé de la sélection artificielle et de son utilisation par Darwin puis, comment cette idée d'étudier des entités physique soumises à des pressions sélectives artificielles dans des <<\,expériences\,>> de laboratoire a perduré et quel est l'intérêt de ce type d'expériences. Nous avons ensuite brièvement parlé des expériences de pensée, pour introduire des méthodes laissant plus de libertés aux expérimentations et tests possibles.
Dans la deuxième partie de ce chapitre nous avons parlé des modèles en essayant de justifier leur utilisation en sciences via la conception sémantique des théories scientifiques. Nous avons soutenu l'intérêt de l'utilisation de la simulation informatique comme modèle et donc leur importance au sein la conception sémantique. 

Pour mieux cerner l'intérêt de ces simulations afin d'étudier la biologie, nous avons repris les analyses des chercheurs en Vie Artificielle en détaillant les modèles épistémiques conceptuels et leur liens avec les expériences de pensée. Nous pensons que ces modèles offrent la plus grande liberté de mano\oe uvre pour travailler sur la théorie de l'évolution, et nous verrons dans la dernière partie de ce mémoire un type de simulations qui tombent, selon nous, dans cette catégorie tout en essayant de s'affranchir de certaines limites évoquées dans ce chapitre.

L'objectif principal de cette partie a été de dégager certains avantages et inconvénients des méthodes décrites. Nous avons vu que certaines méthodes, comme l'analogie avec la sélection artificielle des éleveurs ou l'évolution dirigé de Lenski, essaient de se rapprocher le plus possible des expériences empiriques traditionnelles en fournissant des résultats directement applicables à la théorie de l'évolution des êtres vivants. Néanmoins ces méthodes souffrent de certaines difficultés techniques dues notamment aux contraintes imposées par le vivant. Ainsi, d'autres méthodes essayent d'abstraire les propriétés des êtres vivants via des expériences de pensées ou des simulations, dans le but de s'offrir une plus grand liberté d'études. En contrepartie, l'application des résultats obtenues par ces méthodes à la biologie requiert plus de précautions et peut parfois être un véritable problème, voir impossible.
 %%étudier la théorie de l'évolution : expériences de pensée, modèles, simulation

\include{thirdpart}  %%la robotique evolutionnaire


\chapter*{Conclusion}

\lettrine[lines=2]{L}{'objectif} principal de ce travail a été de légitimer l'utilisation de la Robotique \'Evolutionnaire comme modèle pour étudier la théorie de l'évolution. Si nous espérons avoir avancé dans cette direction, la diversité et l'extrême transversalité des thèmes abordés rendent difficile l'analyse en profondeur de cette relation entre théorie biologique et modèles robotiques, et dans se contexte, légitimer scientifiquement et philosophiquement la capacité de la RE à apporter de nouvelles connaissance en biologie est une entreprise de longue haleine. Mener à bien cette justification prendrait de nombreuses analyses comparatives entre les recherches en robotique et en biologie, il faudrait probablement étudier les fondements philosophique de la relation de façon plus approfondie, complète et détaillé, voir pourquoi pas entreprendre des expériences empiriques parallèles pour tester certaines hypothèses avancées, et cela dépasse de loin le cadre de ce mémoire. 

Plutôt que d'entreprendre cette justification, nous avons préféré dresser un panorama global des domaines que cette approche mobilise et proposer une introduction, <<\,en largeur\,>>, pour qui s'intéresse à ces questions. L'idée est de fournir au chercheur, qu'il soit philosophe, informaticien ou biologiste, les bases pour comprendre l'ensemble du problème en reprenant les concepts centraux dans toutes les disciplines impliquées. Nous avons néanmoins essayé de mettre en avant certaines interrogations qui nous semblent importantes ainsi que les pistes pour y répondre qui nous paraissent les plus pertinentes.

Dans cette optique d'introduire, <<\,en largeur\,>>, l'utilisation de modèles artificiels pour étudier la biologie, nous avons commencé par décrire l'élément central de la question : la théorie de l'évolution. Le but de cette description est de fournir les clefs (historique, conceptuels) pour comprendre \emph{pourquoi} la théorie de l'évolution est telle qu'elle est, en essayant de savoir dans quels buts elle a été conçue, pour répondre à quelles questions, etc.. Ces interrogations nous poussent aussi à comprendre comment cette théorie a été modifiée, transformée et repensée pour s'adapter aux incessantes découvertes de la biologie.  

Cette approche est nécessaire et idéale pour être certain de savoir ce que nous manipulons aujourd'hui lorsque nous parlons d'étudier ou re-utiliser (dans le cadre des algorithmes génétiques, par exemple) cette théorie. \cite{gayon1991darwinetlapresdarwin} a très bien fait ce travail dans son livre <<\,Darwin et l'après Darwin\,>> et nous avons fait de notre mieux pour le reprendre et l'adapter à nos besoins. Il est intéressant de voir comment tout au long de son histoire la théorie a été sujette aux conceptions de ceux qui l'ont étudiée et des époques qu'elle a traversé. Nous avons vu comment les biométriciens l'ont drapée de statistiques, comment les physiologistes mendéliens l'ont poussée dans ses retranchements, où encore, comment la Synthèse Moderne a réussi à réunir les différents domaines de la biologie sous l'enseigne de cette théorie. Réaliser cette perméabilité de la théorie de l'évolution nous semble tout à fait nécessaire pour quiconque souhaite l'étudier, qu'importe l'approche et les moyens qu'il choisit d'utiliser. Repenser la théorie de l'évolution à la lumière de son histoire et des débats et controverses qui l'animent permet selon nous de justifier mieux encore l'utilisation de modèles artificiels pour l'étudier.

Dans un second temps, nous avons présenté des méthodes utilisées pour étudier cette théorie. Nous avons montré qu'il en existait un certain nombre, chacune ayant ses avantages et ses inconvénients. Nous avons parlé d'expériences de pensée, de sélection artificielle, d'analogie, d'évolution dirigée, le tout dans le but premier de montrer que les modèles artificiels et les simulations informatiques ne diffèrent pas tant des approches plus traditionnelles, qu'elles en partagent bon nombre des caractéristiques tout en s'affranchissant de certaines limites imposées par ces méthodes classiques. Dans cette partie nous avons aussi présenté un cadre épistémologique plus général: l'approche sémantique des théories. Cette approche nous paraît plus adéquate pour appréhender la théorie de l'évolution et les façons dont il est possible d'acquérir de nouvelles connaissances à propos de cette théorie. Elle offre une vision du savoir scientifique qui s'accorde avec l'utilisation de modèles de robotique évolutionnaire (et de modèles artificiels en général) pour étudier la biologie (et la théorie de l'évolution en particulier).

Pour terminer nous avons présenté la Robotique \'Evolutionnaire. Notre objectif a été d'offrir une vue d'ensemble du domaine pour permettre à des non roboticiens, à des philosophes ou biologistes, de comprendre mieux cette technique et d'avoir une idée du potentiel qu'elle offre lorsqu'elle se veut appliqué à l'étude de la théorie de l'évolution. \`A travers ce panorama général, nous  avons vu que la RE synthétise nombreux avantages d'autres méthode. Pour résumer : 

C'est un modèle artificiel, une simulation, mais plus proche de l'expérience classique que les simulations traditionnelle sur ordinateur car la relation entre l'objet d'étude (l'évolution des êtres vivant) et le modèle sur lequel sont faites les expériences (des robots) est bien plus étroite, les propriétés et caractéristiques des entités quasiment isomorphes. Là où les simulations sur ordinateur permettent l'étude de processus évolutivement \emph{possible}, la robotique réduit ces possibilités à des processus naturellement \emph{plus probables}, et se rapproche ainsi de la sélection artificielle chez les éleveurs qu'avait utilisé Darwin dans son analogie, ou de l'évolution dirigé de Lenski. Elle diminue les précautions à prendre et rend plus simple le transfert de connaissance du champ artificiel vers la biologie.

Et là réduction de ces précautions, contrairement aux expériences de Lenski où à la Sélection Artificielle pratiquée par les éleveurs, n'enlève pas la liberté qu'offrent les simulations comparées aux travaux menés sur les êtres vivants. La cassette d'une vie factice peut être rejouée mille fois, avec de multiples paramètres différentes à des vitesses différentes, tout en enregistrant chacune des pistes de la façon la plus détaillée possible, sans souffrir de lacunes historiques, ou de biais environnementaux. Ainsi, la Robotique \'Evolutionnaire offre au chercheur la même liberté qu'offre les expériences de pensée, et les simulations informatiques en générale. 

Il pourra donc étudier la biologie sans qu'il soit nécessaire pour lui de choisir \emph{une} approche en particulier parmi celles que nous avons vu dans la partie \ref{sec:pbm}. Au contraire, le roboticien sera à même construire des protocoles expérimentaux  pour tester et comparer ces différentes hypothèses. C'est ce qu'on fait par exemple \cite{waibel09geneticteamcompositionlevelselectionevolutioncooperation}, pour tester différentes hypothèses sur le niveaux de sélection possibles dans une étude dont nous n'avons pas parlé mais qui est selon nous une des voies à suivre.

C'est aussi dans cette optique que nous avons présenté l'approche de \cite{godfrey2009darwinian}. Par sa souplesse et son originalité elle nous semble la plus adaptée pour servir de cadre commun capable d'harmoniser les recherches faites en Biologie et en Robotique. En effet, combien d'études en Robotique \'Evolutionnaire essayent de parcourir exhaustivement certains paramètres de leur système artificiel comme par exemple : combien faut-il de générations pour que dans tel système, tel propriété évolue, quel taux de reproduction permet une convergence évolutive la plus rapide, quel degré de similarité entre les individus assure une coopération efficace\dots? Autant de questions qui peuvent être directement calquées sur des espaces mutli-dimensionnels ``façon pgs''. 

En unifiant la terminologie des biologistes et des informaticiens, le cadre de PGS permettrait aux premiers de mieux appréhender les résultats des roboticiens, en les positionnant dans un zone de l'espace de PGS particulière, qui correspondra, ou non, à une réalité biologique et que les biologistes pourront manipuler. \`A l'inverse, l'espace des populations darwiniennes, précisé par les recherches en biologie, pourra donner aux chercheur en robotique un agenda et une direction pour les paramètres à étudier afin de se déplacer dans l'espace de PGS et atteindre des zones ou les populations biologiques possèdent des propriétés qui les intéressent.

Et pour répondre aux critiques qui avanceront que faire des modèles pour étudier des caratéristiques précise de populations particulière n'apporte rien en science nous répondrons comme \cite{beatty1980ptimaldesignmodelsandstrategyofmodelbuildinginevolutionarybiology} que la biologie peut difficilement se passer de ces approches et que certaines visions de la science, telle celle de \cite{vanfraassen1972aformalapproachtothephilosophyofscience} s'en accomodent très bien.

Pour terminer cette étude nous voudrions revenir sur un point brièvement abordé mais sur lequel nous souhaitons insister. Si la direction donnée à ce mémoire a été de justifier et promouvoir l'utilisation de la Robotique \'Evolutionnaire comme \emph{modèle} de la théorie de l'évolution, nous pensons que cette dichotomie modèle/objet d'étude, n'est pas nécessaire. En réalité, étudier les processus évolutifs à l'\oe uvre sur un système robotique ou étudier les processus évolutif à l'\oe uvre sur un système biologique revient à étudier les mêmes processus. L'étude de l'un et l'étude de l'autre ne sont au final que l'étude de <<\,deux branches d'un même arbre\,>> \citep{huneman12computersciencemeetsevolutionarybiologypurepossibleprocessesissuegradualism}. Ainsi les deux peuvent s'enrichir mutuellement sans que le sens du transfert de connaissance n'ait à avoir de direction privilégiée.


\addcontentsline{toc}{chapter}{Conclusion} 

\section*{Remerciements}
{\footnotesize
En premier lieu, je tiens à remercier Philippe, sans qui je n'aurais probablement pas pris la peine de remercier quiconque à la fin de ce second mémoire. Ensuite je remercie toutes les personnes qui m'ont aidé et accompagné pendant ce long voyage : les étudiants du LOPHISS de P7, ma première vraie promo après 6 ans d'études, ma petite équipe de la BU UPMF avec qui on a reconquis EVE ; et à tous les autres étudiants, de Paris, Montréal, Grenoble, ailleurs, avec qui j'ai pu discuter, échanger, partager et rire un peu, à propos de science, de recherche, de philosophie, de biologie, de musique et autre.
Je ne peux pas oublier tous les supports techniques que j'ai reçu et les nombreux ordinateurs qu'on m'a prêté et sans lesquels je n'aurais rien pu rédiger. Les ordinateurs de mes sœurs, du lutin, de la taupe, de filio, de la BU des grands moulins et bien d'autres que j'oublie ou que je n'ai emprunté que quelques heures. Parmi eux un remerciement particulier à Gaël qui héberge avec professionnalisme et amour (j'espère) le serveur ayant servi de moule pour fondre ces quelques pages. Toujours du côté logistique, un grand merci aux différentes adresses et aux personnes les occupantes qui ont bien voulu me laisser un morceau de lit ou de canapé : l'appart de la rue de Noisy le Sec aux Lilas avec Dakoto, Pepito, Nina et les autres, le 62 Boulevard de Strasbourg avec Seb (\& Alessandra), Bastien et Alex, à qui revient de droit la palme des plus patients et accueillants. Au 2199 rue Harvard et à tous les membres plus au moins affiliés au MHCC, Damien et Xav en tête et les autres qui y sont passés (Andreas notamment), à Fab \& Anaïs pour leur canap rue Moreau, puis ensuite pour leur accueil dans la colloc du 2078 rue Saint Hubert avec le reste de ma famille outre-atlantique. Au 5 Rue Gaston Bachelard, étrangement bien nommé pour accueillir un étudiant en philosophie des sciences, où j'ai retrouvé \emph{la} famille, pour un prélude avant la reprise plus sérieuse de l'écriture d'une aventure qui ne s'est jamais arrêtée et n'est pas prête de le faire. \`A Babal et Alex pour le logement de ministre rue de Penthièvre, et par extension aux présidents, Nicolas et François, qui m'ont toléré comme voisin pendant que j'y séjournais. J'espère un jour pouvoir faire du vélo dans ces quartiers sans m'y perdre et babal, tu auras la tranquillité que tu mérites (les deux morceaux de phrase n'ont pas de lien causal). 
Enfin, à ma maison et mon village, où nul part ailleurs je ne me reposerais mieux.


Je remercie chaleureusement la Petite Cuillère, qui m'a offert pendant presque un été complet l'endroit rêvé pour commencer un journée de lecture et d'étude en attendant tranquillement que le soleil finisse de se lever complètement sur Montréal et que Damien se décide enfin à remonter Saint-Hubert pour me rejoindre. Damien, qu'il me faut remercier tout particulièrement pour avoir probablement été plus impliqué que moi dans ce mémoire et qui a toujours cru en ma réussite universitaire et en l'achèvement de ce travail (et de bien d'autres aussi). Pour ça, pour sa présence en témoin et auditeur de mes avancées, et pour ces nombreux verres trinqués, je le remercie, ce mémoire lui doit beaucoup. Toujours sur les rives du Saint Laurent, je tiens à remercier le CIRST pour m'avoir vraiment très bien accueilli ---avec une mention spéciale à Sengsoury--- pour les livres qui étaient laissés à porté de mains et que le hasard m'a permis de feuilleter, pour la gentillesse des gens de mon bureau malgré le fossé qui séparait nos recherches, et pour la splendide vu du centre ville et des défilés brodés de carrés rouges depuis mon bureau.

Je remercie encore le LUTIN, malgré ses défauts, et tous ses membres avec qui j'ai travaillé comme dans un joyeux village, plein de hauts, de bas, de rumeurs, de coups de gueules et de franches rigolades. \`A Alex donc, à Daniel, Christophe, Ilaria, Marco, Elisabetta, Bora, Zakia, Geoffrey, Thierry, Hamid, Lagha, Nadia, Gérard, ceux que j'oublie, et Charles, forcement.

Il me faut aussi remercier mes relectrices qui ont vainement essayé de rectifier un tir orthographique plus qu'hasardeux : mes s\oe urs, ma tante, ma mère et Geneviève.
Et à mon laboratoire miniature, LAREMI, à Blaise et Jérémy, mais ça va de soi.

Pour le contenu intellectuel je remercie Frédéric Bouchard, qui m'a aidé par bien des aspects, sur le fond comme sur la forme. Il m'a accueilli à Montréal en m'intégrant dans son équipe comme un de ses étudiants, malgré la brièveté de mon passage et m'a invité à de nombreux évènements qui m'ont permis de goûter un peu plus aux joies de la recherche et de la philosophie des sciences. Une fois encore je dois remercier Nicolas Bredèche, sans qui ce mémoire n'existerait tout simplement pas. 

Pour terminer, je me dois de remercier les seuls responsables et garants de chaque ligne, de chaque livre lu, de chaque année passée à étudier et m'épanouir : mes parents.
}



\bibliographystyle{apalike}
\bibliography{/home/simon/Documents/biblio/memoireLophiss}
\addcontentsline{toc}{chapter}{Bibliographie} 





\end{document}

