\begin{abstract}
	Nous voulons, à travers cette étude, justifier l'utilisation de la Robotique \'Evolutionnaire comme modèle pour étudier la théorie de l'évolution. Pour atteindre cet objectif nous reprenons les principes généraux, l'histoire et la structure de la théorie darwinienne afin d'être certain de bien comprendre l'objet à modéliser. Nous nous penchons sur certaines difficultés qu'elle soulève, et présentons des approches actuelles : celles issues de la synthèse moderne et les populations darwiniennes de Peter Godfrey-Smith. Dans un second temps nous décrivons des outils et approches utilisés pour étudier la biologie de l'évolution. Nous insistons sur la pertinence de l'application de modèles et surtout de simulations informatiques de ces modèles, pour permettre cette étude du vivant. Nous appuyons l'intérêt de cette approche dans le cadre de la conception sémantique des théories scientifiques (de Bas van Fraassen, Frederick Suppe et d'autres). Nous montrons que cette conception s'articule parfaitement avec les outils que nous souhaitons promouvoir. Ceci pour introduire notre objet d'étude principal : la Robotique \'Evolutionnaire. Nous en décrivons les déclinaisons qui nous semblent pertinentes et nous montrons qu'en tant qu'expériences de Vie Artificielle Embarquée, elle combine de nombreux atouts majeurs, qui font d'elle un modèle idéal pour étudier la théorie de l'évolution.
	\begin{center}
		\textbf{Abstract}
	\end{center}

	\noindent With this study we want to justify the use of Evolutionary Robotics as a model to study the theory of evolution. In order to reach this goal, we explain the general principles and the history of the darwinan theory of evolution to better understand its structure, to look closely at some difficulties it raises and to present some approaches used to study current evolutionary biology: those borned after the Modern Sythesis and the darwinian populations approach by Peter Goddfrey-Smith. In a second time, we underline the pertinance of the application of models, and simulations of those models, to allow the study of life. To justify the usage of this approach we present the semantic view of scientific theories (by Bas van Fraassen, Frederick Suppe and other), which makes perfect sens with the tools we want to promote. It allows us to introduce the main object of our study : Evolutionary Robotics. We describe some branches we think are accurate and we show that, as an embodied artificial life experiment, ER combines numerous advantages that makes it an ideal model to study the theory of evolution. 
\end{abstract}

