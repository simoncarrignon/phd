
\documentclass[8pt]{beamer}
\usepackage[authoryear,round]{natbib}
\usepackage{graphicx}
\usepackage{fancybox}
\usetheme[width=2cm]{Goettingen}
\usepackage[utf8x]{inputenc}
\usepackage[T1]{fontenc}
\usepackage[french]{babel}
\usepackage[small]{caption}
%%%Laremi Bis

\author{Simon Carrignon\\-\\ENS - P7 - LAREMI - DEoM}
\title{Why apply Darwinian Evolution to Music}
\begin{document}

\begin{frame}
	\titlepage
\end{frame}
\section{Introduction}
\begin{frame}{Introduction}
	Hypothesis :Melodies are darwinian populations?
	Focus on one, ``simple'' case.
\end{frame}

\begin{frame}{Begin With Data}
	First, why not to look at what's happened?
	\begin{itemize}
		\item Record and analyze.
		\item Rythmes,
		\item Frequencies
	\end{itemize}

	Draw it and observe!

\end{frame}

\begin{frame}{And so?}
	Try to apply Darwinian Evolution to cultural entities, ok, but Why? (other theories and tools works well)
	\vfill
	\begin{itemize}
		\item Musicological studies?
		\item Universal Darwinism.
		\item Darwinian Theory of Evolution (Biology)?
	\end{itemize}
\end{frame}


\section{Musicology}
\begin{frame}{Musicology}
	\emph{The} hard question! (cf. O. Morin yesterday). 

	Can we use Darwinian Evolution to \emph{help} Musicologists, Historians, Anthropologists. To be answered!
\end{frame}

\section{Evolutionary Process}
\begin{frame}{Universal Darwinism \& Evolutionary Process}
	To study evolutionary process themselves. 
	\begin{itemize}
		\item Broader approach,
		\item Alife-like (Baradiaran et al. 2006),
			\begin{enumerate}
				\item Les modèles esthétiques, \label{it:est}
				\item les modèles d'ingénierie et,\label{it:ing}
				\item les modèles épistémiques. \label{it:epi}
			\end{enumerate}

		\item \ldots
	\end{itemize}

\end{frame}

\section{Biology}

\begin{frame}{Biology}
	Bring back lessons learnt from musical evolution to biological evolution:
	\begin{itemize}
		\item Developmental approach.
		\item Rethink ontologies.
	\end{itemize}

\end{frame}


\end{document}
