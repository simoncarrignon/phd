
%%\section{Comprendre l'Évolution des Êtres Vivants}

\section{Histoire et principes de la théorie de l'évolution}\label{sec:cmpdr}

En 1859, Charles Darwin proposait une théorie pour essayer de rendre compte de la diversité des êtres vivants dans son livre l'Origine des Espèces\nocite{darwin1859originspeciesbymeansnaturalselectionorpreservationfavouredracesstrugglelife}. Il la résumait comme:
\begin{quote}
	%%Theory of descent with modification by variation and Natural Selection.
	Une théorie de la descendance avec modification par la variation et par la Sélection Naturelle.\\
	\cite[dernière édition, trad. \cite{gayon1991darwinetlapresdarwin}]{darwin1859originspeciesbymeansnaturalselectionorpreservationfavouredracesstrugglelife}.
\end{quote}
et elle fut plus couramment appelé la théorie de l'évolution par Sélection Naturelle.

Pour bâtir cette théorie \cite{gayon1991darwinetlapresdarwin} nous explique que Darwin propose et admet deux mécanismes:
\begin{itemize}
	\item L'hypothèse de la Sélection Naturelle : autrement dit ``la survie du plus apte''.
	\item La descendance avec variation, autrement dit ``l'hérédité des différences individuelles'' \citep{gayon1991darwinetlapresdarwin}.
\end{itemize}

L'admission de ces hypothèses, et donc de la théorie qui en découle, permet de rendre compte de : l'évolution des espèces dans le temps (avec chez Darwin une notion de ``progression'', ``d'amélioration'' des espèces) et de leur divergences les unes des autres. Mais cette théorie (et les hypothèses qu'elle présuppose), de par sa structure et son contenu, n'est simple ni à observer ni à démontrer. Surmonter ces difficultés et essayer de convaincre du bien fondé de ces théories fut l'une des tâches principales de Darwin, qu'il mena dans ses livres depuis l'Origine des Espèces jusqu'à ses derniers ouvrages. Néanmoins nous verrons dans la partie \ref{sec:lvl}, après avoir brièvement rappelé et éclaircit les composantes de la théorie et les difficultés qu'elles soulevèrent au début de cette partie, que ces difficultés sont toujours d'actualité et qu'elles continuent d'alimenter les réflexions des scientifiques et des philosophes. 

\subsection{Descente avec modifications}\label{sec:hered}
Avant toute chose, et pour que la sélection Naturelle puisse agir et donc l'évolution selon Darwin avoir lieu, il faut que les caractères qui déterminent la fitness\footnote{La fitness est difficilement définissable en anglais donc encore plus difficilement traduisible en français. On parle parfois de \emph{valeur sélective} mais nous conserverons souvent le terme anglais de \emph{fitness}.} des individus (c'est à dire leur degré d'adaptation à leur environnement) soient transmis à leurs descendants. Et non seulement ils doivent être transmis, mais pour qu'il y ai du changement, ils doivent aussi varier. Hors, ces variations et cette hérédité (la transmission), Darwin (il le concède lui même) ne possède pas de théorie satisfaisante pour en rendre compte. Comment les parents transmettent-ils leurs caractères à leurs descendants et pourquoi ces caractères varient-ils, sont des phénomènes encore assez mal compris à l'époque. Néanmoins Darwin battît sa théorie de l'évolution en présupposant un certain type de transmission et un certains type de variations. Selon lui pour que l'évolution ai lieu il faut que les caractères transmis \emph{varient} de façon aléatoire et \emph{graduellement} chez les descendants. Par exemple : une population d'individus de taille X aura des descendants avec des tailles comprises entre X-n et X+n qui varient de façon quasi continue entre -n et n. Ainsi pour supporter sa théorie générale de l'évolution Darwin doit donc construire et accepter une théorie de l'hérédité qui présente de telles propriétés et qui de plus est conforme aux nombreuses observations qu'il a faites dans la nature (c'est ce qu'il fera dans \cite{darwin1868variation}). Mais l'adoption de ces propriétés et de cette théorie de l'hérédité pose plusieurs problèmes.

Dans un premier temps des problèmes théoriques. Un des détracteurs de Darwin, Jenkin, soutiendra en l'illustrant mathématiquement (ce sera d'ailleurs pour Darwin la critique la plus sérieuse faite à sa théorie) qu'une sélection agissant sur des variations continues ne pourra que stabiliser ces variations autours de valeurs moyennes et n'aboutira pas à l'apparition de nouvelles valeurs moyennes (de nouveaux phénotypes). Cette critique Darwin ne réussira pas vraiment à la surmonter puisque elle est une conséquence des choix qu'il a fait vis de l'hérédité et des variations. 

Dans un second temps le problème empirique de la redécouverte des lois de Mendel  au début du XIX va venir se poser. Quelques décennies après la disparition de Darwin, les biologistes cellulaires montreront que, si il y a bien une transmission héréditaire de certains caractères via ce qu'il sera convenu d'appeler les gènes, ces caractères ne varient pas de façon continue chez les descendants, mais sont des caractères discrets qui ne peuvent qu'être hybridés selon certaines lois statistiques régulières. Ils ne peuvent pas véritablement se ``mélanger'', ce qu'avait imaginé Darwin dans \cite{darwin1868variation}. Paradoxalement, c'est en réconciliant cette génétique avec Darwin que disparaitra le problème soulevé par Jenkin. 

\subsection{Hypothèse de la Sélection Naturelle}\label{sec:SN}
Une fois une théorie de l'hérédité des variations individuelles admise, la sélection naturelle peut jouer son rôle. On peut la résumer ainsi :
\begin{quote} les individus avec les variations offrant à ceux qui les portent la meilleure adaptation (ie les individus qui possèdent la meilleure fitness) vivrons plus longtemps et/ou laisserons en moyenne plus de descendants.\end{quote}

	Pour \cite[p. 22]{gayon1991darwinetlapresdarwin} c'est ``l'hypothèse organisatrice'' de la \emph{théorie} de Darwin. Mais Darwin ne possède aucun \emph{fait} justifiant cette hypothèse. Pour arriver à convaincre il va utiliser l'analogie avec la Sélection Artificielle et s'appuyer sur les très nombreuses études d'éleveurs, qu'il connait très bien, pour montrer comment les humains ont été capables de transformer et faire diverger de beaucoup les espèces domestiques. Selon lui, si les êtres humains, en appliquant ce genre de pression sélective explicite et direct, sont capable de modifier les espèces vivantes, alors il doit exister une force analogue capable de modifier les espèces dans la nature. Une force qui pourrait rendre compte de la diversité et des adaptations du vivant. Cette force, qui agirait donc sur des variations transmises par hérédité, comme il l'a définit et discuté en \ref{sec:hered}, il l'appelle la sélection naturelle, et il la décrit grossièrement comme : les individus les mieux adaptés se reproduiront plus et auront plus de descendant dans les générations suivantes. 


Il est intéressant de noter qu'en définissant la Sélection Naturelle, Darwin insiste sur le fait que les variations qui vont, ou non, augmenter les chances de survie et de reproduction (la fitness) sont portées par les individus (individus qu'il identifie aux ``organismes biologiques, identité dont nous reparlerons). Pourtant ce n'est pas si évident qu'il n'y parait, et à l'époque déjà, Wallace par exemple, co ``découvreur'' de la théorie de l'évolution par sélection naturelle, avançait que les variations étaient des différences entre des populations d'individus, et que la sélection portait sur ces populations, et non sur les individus qui les composent. 

Mais ni l'un ni l'autre ne pouvait prouver son point de vue, ni théoriquement, ni empiriquement. Il faudra attendre les biométriciens, et les nouvelles méthodes statistiques apportées par Galton, puis Pearson et Weldon, pour avoir des premières preuves mathématiques et empiriques de l'action de la Sélection Naturelle. Néanmoins ces preuves ne suffiront pas vraiment à départager un sélection ``individuelle'' d'une sélection ``populationnelle''.

Pearson et Weldon par exemple vont mesurer des variations morphologiques chez des crabes pour montrer qu'il y a bien déplacements des moyennes de certains traits morphologiques, signes selon eux de l'action de la sélection naturelle. Pour autant ces résultats ne répondent pas directement aux critiques formulées par Jenkin. Si ils montrent que la sélection peut agir au niveau des individus et faire se déplacer certains caractères morphologiques, ils ne démontrent pas que les espèces peuvent durablement se modifier par se biais et diverger véritablement. D'où les arguments de l'époque, faits par les ``mutationistes'', qui avanceront qu'il est nécessaire d'avoir des mutations brutales et importantes qui vont beaucoup modifier certains caractères morphologiques pour obtenir une évolution. Ces critiques seront d'autant plus fortes que la redécouverte des lois de Mendel leur offrira l'appuie empirique dont ne jouissait pas la théorie plus ``gradualiste'' que Darwin avait envisagé.

\subsection{Théorie synthétique de l'évolution}\label{sec:SM}
Au début du XXe siècle la théorie de l'évolution affronte donc un problème. D'un côté Weldon et Pearson démontrent statistiquement et empiriquement qu'une évolution de caractères aux variations graduelles est possible, d'un autre côté les études empiriques montrent que ces caractères graduels n'apparaissent que rarement voir pas du tout dans la nature, et qu'en réalité il n'y a que des caractères discrets qui s'hybrident selon les lois de Mendel. Par conséquent, pour les partisans de ce \emph{mendelism}, la sélection naturelle de caractères qui varient graduellement (ce qu'avait défini Darwin) ne peut pas faire évoluer les espèces. Ce sont les mutations ponctuelles de caractères discrets qui doivent être à ``l'origine des espèces''. 

Mais la donne change lorsqu'en 1918 Fisher démontre que le \emph{gradualisme} nécessaire à la sélection naturelle de Darwin n'est pas incompatible avec des caractères mendéliens. Avec Wright et Haldane ils vont, pendant la première moitié du XXe siècle, finir de réconcilier théoriquement Mendel et Darwin. S'en suivirent plusieurs décennies de synthèse, au cours desquelles les différents domaines de la biologie vont être rattachés empiriquement et théoriquement à cette théorie de l'évolution ``réconciliée''. Cette période, et le paradigme scientifique qui en découla, fut appelé la synthèse moderne (ou théorie synthétique de l'évolution en français) d'après le livre d'un de ses artisans : \cite{huxley1942evolution}. Elle fut témoin de nombreuses avancées et a eu (et a toujours) un impact profond sur la biologie dans toute sa diversité. Nous ne rentrerons pas dans les multiples détails de cette riche époque ; ce que nous retiendrons est qu'un certain consensus sur une théorie de l'évolution par sélection naturelle a émergé. Cette théorie a armé les scientifiques de preuves et de nombreux outils formels et empiriques qui peuvent s'appliquer sur, et dont le bien fondé a été confirmé par, tous les champs de la biologie (paléontologie, écologie, biologie cellulaire,~\ldots). 

Néanmoins nous allons voir que cette synthèse n'a pas résolu tous les problèmes théoriques soulevés par les propositions de Darwin, et nous allons voir comment ceux-ci sont réapparus et ont été repris par les scientifiques et philosophes depuis les années 1970.

\section{Philosophie de la biologie}
\subsection{Niveaux et unités de sélection}\label{sec:lvl}
Dans la section \ref{sec:SN} nous avons dit que, pour peu qu'ils transmettent leurs caractères à leur descendants et que ces caractères varient, les ``individus'' les mieux adaptés vivront plus longtemps et auront plus de descendants. Ainsi générations après générations les populations se verront composées de plus en plus d'individus mieux adaptés, les caractères divergeront, les espèces évolueront. Nous avons vu que cette définition a posé quelques problèmes quant à la nature des variations nécessaires, problèmes qui furent partiellement résolus par la synthèse moderne.

Mais la théorie de Darwin soulève d'autre questions. Dans un article majeur sur le sujet, \cite{lewontin70unitsselection} reprend la théorie de Darwin pour la formuler et la résumer dans des termes plus actuels de la façon suivante :
\begin{enumerate}
	\item Dans une population des individus différents ont une morphologie, une physiologie et des comportements différents (variation phénotypique).
	\item Des phénotypes différents ont des taux de survie et de reproduction différents dans des environnements différents (fitness différentielle).
	\item Il y a une corrélation entre parents et descendants à chaque génération future (hérédité de la fitness).
\end{enumerate}

L'auteur explique que le niveau d'abstraction de cette définition n'impose pas l'organisme comme ``individu''. N'importe quelle entité répondant aux trois critères que Lewontin énonce peut remplir ce rôle d'individu. Pour lui cette définition s'applique aux différents niveaux d'organisation biologique, les ``individus'' pouvant aussi bien être les cellules que les chromosomes, les gènes, les organes au sein d'un organisme, les organismes entre eux ou même les espèces d'organismes entres elles. 

Déjà à l'époque, Wallace considérait que les populations et non les individus pouvaient être sélectionnés, suggérant ainsi que la sélection pouvait être interprétée à différents niveaux. Et même si pour Darwin la pression de sélection ne peut porter que sur les variations des individus (pour lui des organismes mutlicelllulaires), il considérera dans son livre \emph{The descent of man} \citep{darwin1871thedescentofman}, que les sociétés humaines (donc des \emph{populations} d'individus biologiques) peuvent être soumises à une force similaire à la sélection naturelle qu'il avait défini pour les organismes. Weismann, au début du XXe siècle fut un de ceux qui théorisa le mieux ces idées avant qu'elles ne soient éclipsées par les travaux de la synthèse moderne.

Dans la foulée de l'article de Lewontin beaucoup se penchèrent à nouveau sur ces questions. D'autant qu'au sortir de la synthèse moderne un certain consensus était adopté, avec le gène comme unité principale (voir unique) de sélection. Ce consensus, appelé aussi ``gene eye view'', d'abord proposée par \cite{williams1966adaptationandnaturalselection} puis popularisée par \cite{dawkins76selfishgene} eu un certains succès dans la communauté scientifique et auprès du large publique. Cette vision réductionniste a une valeur heuristique forte et a permis de simplifier bon nombre de problèmes mais elle fut très vite l'objet de nombreuses critiques. \cite{wimsatt1980theunitsofselectionandthestructureofthemultilevelgenome}, par exemple, remettra en cause la vision de Williams (et de Dawkins), en argumentant qu'elle ne peut rendre compte de tous les phénomènes évolutifs en biologie. Pour lui la fitness des individus n'est pas réductible à la fitness des gènes qu'il possède, ces derniers interagissant de façon totalement non linéaire entre eux. \cite{gould2002thestructureofevolutionarytheory} écrira même, dans l'imposante synthèse de sa réflexion qu'est ``la structure de la théorie de l'évolution'', que :
\begin{quote}
	[\ldots] la théorie de l'évolution centrée sur le gène était indéfendable.\\
	\citep[p. 855]{gould2002thestructureofevolutionarytheory}
\end{quote}

Ces problèmes et visions divergentes sont nombreux et, comme l'écrit Gould dans le même ouvrage,
\begin{quote}
	[o]n pourrait organiser la discussion sur ce sujet très difficile et très important d'une centaine de façons différentes.\\
	(ibid, p. 833).
\end{quote}
Il est donc difficile d'en faire une revue exhaustive mais on peut parler par exemple de l'approche de \cite{dawkins76selfishgene} et \cite{hull1974philosophyofbiologicalscience} qui, pour clarifier le débat, ont proposé de distinguer les intéracteurs (qui agissent dans l'environnement et interagissent entre eux) des réplicateurs (qui se répliquent de générations en génération). Cette séparation fut très discutée, modifiée, affinée et a eu le mérite d'offrir de quoi réfléchir aux scientifiques et philosophes. En revanche il reste toujours difficile et controversé de savoir à quels niveaux d'organisation biologique ce situent ces entités. Pour pallier certains des problèmes de cette distinction dite de Hull-Dawkins, Griesemer (2000) proposera le concept de ``reproducteur'', qui intègre le caractère développementale des individus. D'autre pour discuter ces problèmes, choisissent d'étudier la transmission des caractères en prenant des colonies (des superogargnismes) comme individus \citep{wilson1989revivingsuperorganism}, certains choisissent les espèces ou même les idées culturelles \citep[p. 147]{godfrey2009darwinian} comme individus.


Parmi ces différents courants, des chercheurs avancent que l'évolution peut être comprise et réduite à l'étude d'une entité simple comme le gène\cite{dawkins76selfishgene,dennett95darwinsdangerousideaevolutionmeaningslife}, d'autres soutiennent que c'est impossible et qu'il faut étudier des niveaux supérieurs \citep{gould2002thestructureofevolutionarytheory,wilson1989revivingsuperorganism}, d'autres encore avancent que ces visions sont similaires. Quoiqu'il en soit le problème est loin d'être résolu et le débat continue. 

Pour essayer de répondre aux questions que ce débat soulève et pour tester les limites, les atouts et les faiblesses des différentes solutions avancées, les scientifiques et les philosophes développent tout un arsenal d'expériences de pensées, de modèles mathématiques, informatiques ou encore d'expériences empiriques. Présenter ces méthodes, analyser leurs avantages et inconvénients serait utile mais dépasse le cadre de cette étude. 
%d'autant que cela nécessiterais la présentations des approches épistémologiques plus larges dans lesquelles elles trouvent, ou non, leur valeur.
Nous nous contenterons ici de présenter un modèle récent qui nous semble idéal pour étudier ce type de problèmes. Ce modèle les chercheurs l'ont appelé la Robotique \'Evolutionnaire. Après avoir expliquer les principes et l'histoire de cette méthode, nous présenterons une étude qui l'utilise pour tenter d'explorer la question des unités et niveaux de sélection dont nous venons de dessiner les grandes lignes.

En 1859, Darwin, (Wallace, lamarck,\&c) propose une théorie pour expliquer la diversité des espèces sur la terre (comment les êtres vivants ont pu se diversfier et s'ameliorer). Bayon, notamment, vois dans Dans darwin deux objets : l'explication d'un processus qui permet de modifier la répartition  des différent individus dans la population, et l'explication d'une 'histoire évolutiove' qui fait dériver l'ensemble des être viviant à une unique anêtre commun.

Lorsqu'il émettait cette théorie, que Gayon dçoupe en réalité en une théorie complété par une hyptohthèse, Il manquait à Darwin beaucoup des élements qui ont ensuite permis d'asseoir ça théorie empiriquement. Pour l'illustrer et la rendre convaicquante il lui a donc fallut utiliser certains outils, dont l'analogie que nous verrons beaucoup plus en détails dans le chapitre suivant. Malgrés ces arguments et outils que Darwin utilisa pour convaincre du bien fondé de sa théorie, il fallut quelques temps avant que ``l'explication darwinienne'' soit admise comme la ``meilleure'' théorie pour comprendre l'histoire des êtres vivants. Il a fallut les neo-darwinien de la fin du XIXe (Weisman), la redécouverte des lois de Mendel et leur intégration a la sélection pour que le darwinisme soit admis. Puis dans la foulé des néo-darwinien, la synthèse moderne, comme nous l'avons déjà évoqué en introduction, affina cette explication et lui apporta de nombreux confirmation empirique.  

Néanmoins un certains nombre de problèmes subsistent. 


Expliquer et comprendre l'évolution des êtres vivant est une chose très complexe et délicate. La question de l'explication scientifique est déjà en elle-même un sujet philosophique très débattu et quand il s'agit de l'application aux sciences du vivant le problème devient plus épineux encore.

Y'a-t-il des lois en Biologie? Science historiques?.

Comment l'évolution doit-elle être expliquée? faut-il trouver des lois universelles de la biologie, telle les lois de mendel, ou bien faut-il essayer de comprendre quelles évenements on condtuis 

Dans la suite de Charlse Darwin qui a pavé la route, beuacoup on essayé, et boeaucoup a été appris sur les mécanismes évolutif, tout comme sur l'histoire. Cette partie ne ce veux pas un résumé exhaustif des méthdoes déxperimentation en biologi e de l'évolution mais plutôt liste certaines qui nous semble pertinante selon une certaines approche que nous apellerons d'illustration et test
Nous verrons donc analogies, expériences de pensée et simulations informatiques comme différentes techniques pour illustrer et tester les processus évolutionnaires.  Nous verrons les intérêts de ces techniques et qu'apportent-elle avant de parler plus en détails de la robotique évolutionnaire, de sa posisiont et ses particularits par rapport à ces techniques.
\cite{beatty1980whatswrongwithreceivedwiew}, thomson,lloyd\ldots
\cite{brandon78adaptationevolutionarytheory}

\begin{quote}
	all we need is that reproduction lead to parental-offspring similarity at the level at which evolution is to occur. It does not matter what particular emchanism underlies this pattern of similarity, so long as the pattern is present.\\
	\citep[p. 34]{godfrey2009darwinian}
\end{quote}

\section{Population Darwinienne}
