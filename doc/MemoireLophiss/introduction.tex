 \chapter*{Introduction}
\section*{Algorithmique Évolutionnaire, histoire et influences}\label{sec:intro:ae}

\lettrine[lines=2]{D}{epuis} maintenant une quarantaine d'années, l'informatique assiste à l'expansion et la popularisation de toute une famille de techniques et méthodes désignées par le terme d'Algorithmique Évolutionnaire (ou AE, en anglais \emph{Evolutionnary Computation, EC})\footnote{Nous reprenons ici la traduction utilisée par \cite{schoenauer09lesalgorithmesevolutionnaires} et ne faisons pas non plus de distinction entre le terme \emph{évolutionnaire} et le terme \emph{évolutionniste}, que ce soit lorsque l'on parle d'Algorithmique Évolutionnaire ou de Robotique \'Evolutionnaire, l'un comme l'autre aurait pu être utilisé.}.
Sous cette étiquette sont regroupées des méthodes apparues plus ou moins parallèlement dans les années 60-70, dont les plus connues sont les Algorithmiques Génétiques \citep{holland75adaptationnaturalartificialsystem} et les Stratégies Évolutionnaires \citep{rechenberg73evolutionsstrategieoptimierungtechnischersystemenachprinzipienderbiologischenevolution}\footnote{Pour une brève introduction en français sur l'Algorithmique Évolutionnaire, son histoire et ses diverses sous-composantes, voir \citet{schoenauer09lesalgorithmesevolutionnaires}, pour une introduction complète et en anglais voir \citet{eiben03introductiontoevolutionarycomputing}.}.

Le point commun de ces techniques est qu'elles reprennent toutes l'idée proposée par \citet{darwin1859originspeciesbymeansnaturalselectionorpreservationfavouredracesstrugglelife} de l'évolution par sélection naturelle. L'intuition est la suivante : si la force adaptative de l'évolution telle que l'a décrite Darwin a permis l'émergence d'espèces et d'organes extrêmement adaptés aux contraintes et conditions environnementales, allant du bec du pivert parfait pour percer les arbres et attraper les insectes, à la graine couverte d'un duvet que la moindre brise transporte\footnote{Pour reprendre certains des exemples de \citet[ch. 3 notamment]{darwin1859originspeciesbymeansnaturalselectionorpreservationfavouredracesstrugglelife}.}, pourquoi la sélection naturelle ne pourrait-elle pas servir pour programmer des outils et calculer des solutions à des problèmes aussi différents que de trouver le plus court chemin entre des villes (le fameux problème du voyageur de commerce dont nous reparlerons plus tard) ou l'agencement optimal de composants électroniques sur une puce~?

\subsection*{Turing et <<\,recettes algorithmiques\,>>}
Cette intuition Turing l'avait dès 1950, dans son article que beaucoup considèrent comme fondateur de l'intelligence artificielle moderne~: <<\,Computing Machinery and Intelligence\,>>. Dans cette tentative de concevoir une machine intelligente, il propose de ne pas s'attaquer de front au <<\,développement d'un programme pour simuler l'esprit humain, mais [\ldots] plutôt d'essayer de produire l'esprit d'un enfant\,>> \citet[p. 456]{turing50computingmachineryintelligence}. L'enfant-robot produit devra avoir les mécanismes d'apprentissage capables de le faire passer correctement au stade adulte. Ainsi il n'y aurait pas besoin de programmer toutes les capacités des adultes mais simplement les méthodes d'apprentissage qui permettront aux <<\,enfants\,>> de les acquérir. Le problème est donc, selon Turing, décomposé en deux tâches : le processus d'apprentissage et la conception de la machine-enfant. Pour trouver la bonne \guillemotleft machine enfant\guillemotright, il imagine une suite d'essais/erreurs comparable, toujours selon lui, à l'évolution. Dans un premier temps une machine enfant serait conçue, à laquelle un enseignement serait dispensé. En fonction des résultats de cet enseignement <<\,il serait possible d'essayer une autre [machine-enfant] et voir si elle est mieux ou moins bien.\,>> Ainsi, par itération et accumulation de tests, de jugements de qualité et de modifications de la machine initiale, il serait possible de trouver la bonne machine capable de se développer en une <<\,machine adulte\,>> à la complexité proche de l'esprit humain.

Turing voit <<\,un lien évident entre ce processus et l'évolution\,>>. Il l'illustre par les égalités suivantes~:
\begin{quote}
   \begin{tabular}{lcl}
       Structure de la machine enfant & = &Matériel acquis par hérédité\\
       Changement dans la machine enfant&= & Mutations\\
       Sélection Naturelle &=& Jugement de l'expérimentateur\\
   \end{tabular}\\    
   \citep[p. 456]{turing50computingmachineryintelligence}
\end{quote}
Cette évolution, ajoute-t-il, <<\,l'expérimentateur, par l'exercice de son intelligence, serait capable de l'accélérer\,>>. Et voici déjà réunis tous les ingrédients que les acteurs de l'algorithmique évolutionnaire reprendront vingt ans plus tard.  

Nous reviendrons plus en détails sur la structure de la théorie de l'évolution dans le chapitre \ref{ch:evolnat} mais il est intéressant dès maintenant de noter comment cette théorie, ici résumée par Turing, peut être sujette à des entreprises <<\,d'axiomatisation\,>>. Bon nombre de biologistes à la suite de Darwin ont entrepris cette démarche et ont réfléchi à une <<\,recette idéale\,>> de l'évolution, qui résumerait les propriétés suffisantes et/ou nécessaires que doivent posséder les êtres vivants (voir pour certains à n'importe quels <<\,systèmes\,>>, vivants ou non) pour évoluer.

Cette propriété de pouvoir être résumée en courts énoncés, ainsi que le caractère formel de ces résumés (de ces recettes) fait de la théorie de l'évolution et des recettes utilisée pour la décrire, des candidates idéales pour une transcription algorithmique \citep[ch.~4, p.~48]{dennett95darwinsdangerousideaevolutionmeaningslife}. Elles apparaissent comme une suite d'instructions simples pour résoudre des problèmes complexes. Il n'est pas surprenant que, dès la naissance de l'informatique, les pères de la discipline, adeptes d'automatisation, aient pressenti le potentiel de ces recettes et aient eu envie de les reprendre à leur compte. Ils avaient à leur portée un <<\,solveur universel\,>> (en anglais \emph{universal solver}), un <<\,algorithme naturel\,>> capable de trouver automatiquement les solutions à tous ---ou au moins à une large gamme--- les problèmes possibles. Il n'y aurait plus à construire les solutions à chaque problème rencontré, mais juste à trouver un moyen d'informatiser le processus évolutif. Turing l'avait bien vu et c'est ce que les chercheurs en Algorithmique Évolutionnaire ont fait.  

Il serait tentant de conclure que Darwin, Turing et les chercheurs en Algorithmique Évolutionnaire aient tous eu la même intuition et confiance en  
\begin{quote}
   [\ldots] cette force qui adapte lentement et merveilleusement bien chaque forme aux plus complexes relations de la vie.\\
   \citep[ch. 15]{darwin1859originspeciesbymeansnaturalselectionorpreservationfavouredracesstrugglelife}
\end{quote}
Mais savoir s'ils parlent effectivement de la même <<\,force\,>>, du même processus est une question toujours débattue. Sans y répondre directement nous essayerons d'apporter quelques clefs pour y réfléchir tout au long de ce mémoire.

\subsection*{L'apport de la synthèse moderne}
Parallèlement à cette intuition turingienne quasi-visionnaire de l'applicabilité de la <<\,force darwinienne\,>> à l'informatique, qui traçait la voie à l'Algorithmique \'Evolutionnaire, l'éclosion de ces méthodes dans les années 70 s'explique mieux si elle est mise en relation avec l'histoire de la Biologie de l'époque. Nous reviendrons plus en détails sur cette histoire dans la section \ref{sec:SM}, mais pour comprendre l'avènement de l'AE, il nous paraît déjà judicieux de nous arrêter quelques instants sur cette période.

En effet, pendant les années 60 (et donc en même temps que l'apparition de l'AE), la Biologie était marquée par la fin, l'adoption et la reconnaissance de la Synthèse Moderne (SM) de la théorie de l'évolution par l'ensemble de la communauté scientifique. Cette Synthèse Moderne, appelée aussi Théorie Synthétique de l'Évolution (TSE) d'après le titre du livre de \citet{huxley1942evolution}, l'un des acteurs de l'époque, désigne une période entre les années 1930 à 1960 \footnote{Les dates exactes de la Synthèse Moderne peuvent varier suivant les auteurs et dépendent souvent de la prise en compte ou non de la phase <<\,théorique\,>>. Pour une revue critique de l'histoire de cette période voire \citep{reif00synthetictheoryevolutiongeneralproblemsgermancontributiontosynthesis}, pour une très bonne et plus large introduction à la Synthèse Moderne et à l'histoire, en français, de la pensée évolutionniste en générale voir \cite{gayon1991darwinetlapresdarwin}.} pendant laquelle un ensemble de découvertes, de modèles et de théories biologiques ont vu le jour. Cette époque a été témoin de la réconciliation des lois de Mendel (lois de transmission de caractères génétiques discrets) avec l'évolution (graduelle et continue) des êtres vivants selon Darwin et son principe de Sélection Naturelle. Cela a aussi été l'occasion d'affiner et d'agrémenter la théorie de tout un arsenal d'outils mathématiques et de modèles statistiques et conceptuels, qui n'auront qu'à être repris par les informaticiens.

Au c\oe ur de la biologie de l'évolution, là où il n'y avait que des énoncés et des hypothèses, venait désormais se greffer des équations, des chiffres et des tables statistiques, bien plus précis et simples à implémenter avec un ordinateur que des intuitions abstraites (telles celles qu'avait pu avoir Turing). Les équations de \citet{fisher1930geneticaltheorynaturalselection} pouvaient être reprises directement, les paysages adaptatifs de \cite{wrights1932rolesmutationinbreedingcrossbreedingandselectioninevolution} offraient l'outil parfait pour illustrer les espaces de recherche des solutions avec lesquels jonglent les ordinateurs \footnote{Ces paysages adaptatifs, plus souvent appelés <<\,fitness landscapes\,>> ou espaces de recherches en sciences computationnelles, se révélèrent si pertinents conceptuellement qu'ils sont devenus un élément incontournable du vocabulaire de l'Algorithmique Évolutionnaire et sont encore et toujours présentés dès l'introduction des <<\,textbooks\,>> de la discipline \citep[p. 12]{eiben03introductiontoevolutionarycomputing}} et la \emph{gene eye view} offrait les abstractions et concepts clairs et simples que pouvaient implémenter \cite{holland75adaptationnaturalartificialsystem} et les autres.


D'autres éléments de la Synthèse Moderne ont aidé à ce rapprochement et à la naissance de l'AE. Par exemple, l'introduction du langage de l'optimisation et de la théorie de l'information en biologie \citep{maynardsmith78optimizationtheoryinevolution} a permis l'application et l'échange de concepts et de procédés. Ainsi, de nouvelles méthodes d'explication ont pu être explorées\footnote{Nous reverrons plus en détails les critiques, la valeur et la portée explicative de ces modèles d'optimisation et des modèles en général, notamment à la lumière de l'approche sémantique des théories scientifiques de van Frassen comme l'a fait (entre autres) \citet{beatty1980whatswrongwithreceivedwiew} pour la biologie. Nous ne les citons ici que pour souligner le rapprochement qui s'opéra pendant le milieu du XXe siècles entre biologie et sciences computationnelles.}.

Le rapport entre cette Synthèse Moderne arrivée à maturité et l'émergence de l'Algorithmique Évolutionnaire est donc étroit. La première a offert à la seconde les outils parfaits pour que puisse être calqué le vocabulaire de la biologie sur des problèmes diverses et abstraits afin d'essayer de les résoudre en suivant une méthode, un algorithme <<\,universel\,>>. Cette superposition de concepts a rendu possible le rapprochement des vues des deux protagonistes d'une phrase de \citet[p. 179]{maynardsmith00theconceptofinformationinbiology} : <<\,[l]à où un ingénieur voit du \emph{design}, un biologiste y voit la sélection naturelle.\,>>. Pendant les années 1970, l'ingénieur a commencé à pouvoir imaginer faire du \emph{design} avec (une imitation artificielle) la Sélection Naturelle.

Néanmoins et nous essayerons de le justifier tout au long de ce mémoire, il ne faudrait pas que ce rapprochement marque trop profondément les disciplines d'Algorithmique Évolutionnaire. Si la Synthèse Moderne a été un guide et une base de départ idéale, elle ne doit pas être une entrave et les critiques qui lui sont faites doivent être étudiées et testées avec sérieux.

De plus, bien que fructueux, ce pont jeté entre problèmes informatiques et biologie de l'évolution est avant tout conceptuel et abstrait. De surcroît, il est, la majeure partie du temps, perçu comme à sens unique par les chercheurs en informatique. L'Algorithmique Évolutionnaire <<\,puise son inspiration du processus de l'évolution naturelle\,>>, annoncent \citet[p. 1]{eiben03introductiontoevolutionarycomputing} et presque tous les auteurs de la discipline. Le lien semble souvent se réduire à de simples emprunts de langage. C'est pourquoi avant d'introduire la discipline qui nous intéresse et dans laquelle l'abstraction s'estompe un peu, l'exemple classique en informatique théorique et en optimisation qu'est le problème du voyageur devrait nous aider à mieux comprendre et éclairer la nature de cette superposition linguistique entre informatique et biologie, tout en nous donnant l'occasion d'illustrer la catégorie de problèmes que peut aider à résoudre les techniques d'Algorithmique Évolutionnaire.

\section*{Le voyageur de commerce}\label{sec:intro:vc}

Dans ce problème bien connu des mathématiciens et informaticiens, un voyageur doit passer par un ensemble de villes en traversant chacune d'elles une seule fois. Le but est pour lui de parcourir le moins de kilomètres possible (d'optimiser au mieux son trajet). Comme souvent en informatique, la meilleure et plus sûre façon de trouver la solution à ce problème et de tester tous les trajets possible pour trouver le plus court parmi eux. L'inconvénient dans ce cas est que le nombre de trajets possible peut vite devenir très, très important. Ajouter une ville par laquelle notre voyageur doit passer ce n'est pas ajouter \emph{un} nouveau trajet possible mais, $M \times N$ trajets (où $M$ et $N$ sont respectivement le nombre de solutions et le nombre de villes du problème avant l'ajout de la nouvelle ville). Autrement dit : si vous aviez 6 villes à parcourir, en ajouter une seule à l'itinéraire c'est obtenir 4320 nouveaux trajets à tester. Le problème est dit NP-complet, et même des ordinateurs des milliards de fois plus puissants que le plus puissant des ordinateurs d'aujourd'hui mettraient plusieurs fois l'\^{a}ge de l'univers pour énumérer toutes les solutions possible avec une 50aine de villes.

C'est là que l'Algorithmique \'Evolutionnaire entre en jeu. Pour comprendre comment nous allons présenter sans entrer dans les détails, un algorithme génétique simple pour résoudre ce problème.

La première étape avant de lancer l'algorithme et de faire correspondre les éléments du problème aux concepts biologiques. Dans notre situation le but est de trouver le trajet le plus court, la <<\,solution\,>> du problème sera donc un trajet, autrement dit une suite de villes à traverser (par exemple Lyon, Marseille, Paris et Grenoble). Dans le monde biologique il n'y a pas de <<\,solution\,>> à proprement parler mais plutôt un <<\,résultat\,>> : des individus adaptés à leur environnement. Pour simplifier et dans la lignée des architectes de la Synthèse Moderne, l'individu est considéré comme réductible à son génome. Nous dirons ainsi que ---dans un grossière simplification qui ne veut pas rendre compte d'une réalité du monde du vivant mais simplement en extraire une heuristique utile : le résultat (les <<\,solutions\,>>) de l'évolution est un ensemble de génomes qui code pour des individus adaptés à leur environnement\footnote{Nous tenons à insister encore ici sur la valeur heuristique de cette simplification, qui, et nous le reverrons, peut poser des problèmes d'ailleurs illustrés de façon particulièrement probante en algorithmique évolutionnaire, cf \citet{huneman12computersciencemeetsevolutionarybiologypurepossibleprocessesissuegradualism}.}. Notre solution numérique, notre trajet, sera donc identifié dans la biologie à ces génomes. Dans ce cas précis il est facile de faire correspondre une représentation de ces deux entités (le génome et le trajet). Notre trajet peut être simplement encodé sous la forme d'une chaîne de caractères très similaire aux séquences de nucléotides que l'on retrouve dans l'ADN. Un trajet Lyon$\rightarrow$Marseille$\rightarrow$Paris$\rightarrow$Grenoble pourra être représentée par une chaîne LMPG, tout comme un gène peut être résumé par sa séquence de nucléotide (ex: AATGGTA).

Si on continue notre analogie avec la biologie : que s'est-il passé dans la nature pour que soient produites ce que nous venons de définir comme les <<\,solutions\,>>, à savoir des génomes adaptés à leur environnement~? D'après Darwin et la Synthèse Moderne c'est la Sélection Naturelle, autrement dit le fait qu'en moyenne les individus mieux adaptés ont plus de descendants, qui se charge du travail. C'est cet énoncé qu'il va falloir transcrire dans notre système artificiel. D'abord, que peut signifier \emph{mieux adaptés} dans le cas d'un trajet du voyageur de commerce~? Il s'agit simplement d'être plus proche de la meilleure solution possible ; à savoir, du plus court trajet. Ainsi <<\,être adapté\,>>, pour un trajet, signifiera <<\,être le plus court possible\,>>. On identifie cette fois-ci la \emph{fitness} d'un individu biologique à la longueur du trajet codé par notre suite de villes. Les entités élémentaires sont là, reste à définir les mécanismes qui vont les manipuler. La théorie de l'évolution nous dit que les génomes les mieux adaptés ont plus de chances de se retrouver dans les générations futures ; l'algorithme évolutionnaire va donc donner plus de chance aux listes de villes les plus courtes d'être présentes à la génération suivante. L'algorithme (celui qui l'a conçu) <<\,sélectionnera\,>> de préférence les trajets les plus courts\footnote{Le dosage de cette sélection est d'une importance capitale en Algorithmique Évolutionnaire. Ne sélectionner que les meilleurs et l'évolution s'arrête très vite, loin de la solution optimale, ne sélectionner qu'aléatoirement et l'évolution n'atteindra jamais de solution intéressante dans un temps humainement concevable.}.

Dans la nature, une fois cette sélection faite et lors de l'apparition de ce que les biologistes appellent une \emph{nouvelle génération}, les individus descendants présentent des variations par rapport à leurs parents. Ces variations sont dues à de nombreux facteurs : environnementaux, développementaux, génétiques\ldots Au niveau des individus nous n'avons transféré par analogie que le <<\,génome\,>>, dans notre exemple. Nous ne retiendrons donc que les facteurs de variation génétique. Ceux-ci sont essentiellement : des croisements entre individus de sexe opposé (si reproduction sexuée il y a, de transfert horizontaux sinon) et des mutations. Ce sont ces deux mécanismes que l'Algorithme Évolutionnaire retiendra et reprendra pour brasser, muter et  croiser les listes de villes\footnote{Cette limitation de la variation aux facteurs génétiques est sans doute une source de problème que rencontre le projet de l'AE, mais nous reviendrons dessus dans la partie \ref{sec:RE:EE}.}. Deux trajets pourront ainsi être croisés pour n'en former qu'un seul, une ville remplacée par une autre de façon aléatoire\footnote{On peut noter que tout comme en biologie, certaines contraintes s'imposent. En effet certains croisements de trajets comme certains remplacements de villes ne sont pas <<\,viables\,>> puisque le voyageur doit passer par toutes les villes. Si par exemple, lors d'un croisement nous reprenons les cinquante premiers pour-cent du génome d'un parent et les cinquante derniers pourcents du génome du second parent pour obtenir un nouvel individu, et si les génomes respectifs des parents sont LMPG et PGML, j'obtiendrai un individu LMML qui ne sera pas <<\,viable\,>>.}, et toutes les autres <<\,manipulations génétiques\,>> possibles envisagées.

Ainsi au fil des générations un certain nombre de <<\,solutions\,>> seront testées, et moyennant le réglage adéquat des différents paramètres de l'algorithme (évolution du taux de mutations, nombre de croisements, pressions sélectives, ré-injection de nouveaux individus\ldots le réglage de ces paramètres est d'une importance cruciale) et il suffira de quelques itérations du processus pour qu'une solution satisfaisante soit trouvée. Ainsi en partant d'un ensemble limité de listes de trajets choisies aléatoirement, on obtiendra par <<\,évolution artificielle\,>>, en usant des différents mécanismes découverts par les biologistes comme actifs dans la nature, un très bonne solution. Cette solution ne sera peut-être pas \emph{la} meilleure solution mais s'en approchera suffisamment pour offrir un résultat satisfaisant\footnote{En réalité il a été prouvé mathématiquement que certains réglages de ce genre d'algorithmes garantissent de trouver \emph{la} meilleure solution, il s'agit ensuite d'un compromis entre temps de calcul et <<\,satisfaisabilité\,>> de la solution, mais ceci dépasse le cadre de ce travail.}.

Il ne s'agit, ni plus ni moins, que d'effectuer un parcours <<\,dirigé\,>> dans le paysage adaptatif de tous les trajets possibles en reprenant les mécanismes utilisés par la nature pour parcourir les paysages adaptatifs des êtres vivants.
Ce parcours <<\,dirigé\,>> illustre très bien le rapprochement entre théorie de l'optimisation et biologie de l'évolution que nous avons évoqué plus haut et souligne comment il peut donner des résultats fructueux en révélant la distance qui peut séparer les entités manipulées\footnote{Pour une introduction complète à l'Algoritmique Évolutionnaire et à ses divers résultats et applications cf : \citet{eiben03introductiontoevolutionarycomputing}.}.

Néanmoins, l'analogie reste très <<\,théorique\,>> et heuristique. Les problèmes sont très éloignés dans la pratique et les emprunts d'un domaine vers l'autre se font dans un sens unique (de la biologie vers l'informatique), et sont surtout linguistiques. On comprend bien qu'un génome, qui encode les caractéristiques d'un être vivant n'a que très très peu à voir avec une chaîne de caractéristiques représentant une suite de villes à parcourir.

\section*{La Robotique \'Evolutionnaire} \label{sec:intro:re}
Il a ensuite à nouveau fallu attendre une vingtaine d'années que ces méthodes d'informatique évolutionnaire fassent leurs preuves, que leurs fondements mathématiques s'affinent et qu'en parallèle la robotique évolue (ou du moins change d'approche comme nous le verrons avec \citet{brooks91intelligencewithoutreason}), pour qu'apparaisse dans les années 90 ce qui \emph{a posteriori} apparaît comme probablement plus proche des <<\,intuitions\,>> de Turing : la \emph{Robotique \'Evolutionnaire} (RE). Il n'est plus question ici de trouver des solutions à des problèmes en utilisant des représentations où l'analogie avec la biologie est lointaine et se borne souvent, comme nous l'avons vu précédemment avec l'exemple du voyageur de commerce, à des emprunts de vocabulaire (l'individu étant une suite de ville bien différentes de ce que peut être un individu biologique). Avec la robotique la donne change.

Pour les roboticiens le but est de construire des robots qui seront amenés à se déplacer dans un environnement <<\,réel\,>>, ouvert et changeant. Les contraintes qui s'appliquent sur les systèmes à concevoir sont donc \emph{a priori} les mêmes que celles auxquelles sont soumis les êtres vivants. De plus, les t\^{a}ches que ces <<\,agents\footnote{Nous utiliserons le terme \emph{agent} dans sa version naïve et intuitive utilisée notamment en intelligence artificielle où un agent peut être : toute entité capable de percevoir son environnement et d'agir dessus. Cette définition a le mérite d'inclure aussi bien les robots, des programmes simulés ou les êtres vivants. Pour une discussion plus approfondie sur la définition d'\emph{agent} voir \citet{barandiaran09definingagencyindividualitynormativityasymmetryspatiotemporalityaction}.}\,>> doivent effectuer ressemblent beaucoup aux comportements que peuvent avoir les êtres vivants dans leur activité quotidienne. Le chercheur en robotique veut que son robot soit capable de se déplacer dans l'environnement tout en évitant les obstacles, de transformer et d'échanger avec cet environnement pour assurer son autonomie énergétique, sa sécurité, de communiquer et se synchroniser avec d'autres agents pour résoudre des taches distribuées et complexes qu'il ne pourrait pas accomplir seul, etc..

L'analogie est cette fois-ci clairement plus forte entre les individus biologiques et les robots. La phrase de \cite{maynardsmith00theconceptofinformationinbiology} <<\,[l]à où un ingénieur voit du \emph{design}, un biologiste y voit la sélection naturelle.\,>> peut ici être reprise et précisée par :
<<\,Là où les biologistes veulent comprendre comment a pu évoluer cette multitude d'entités autonomes et adaptées à de nombreux environnements que sont les êtres vivants, les roboticiens veulent savoir comment construire des entités autonomes et adaptées à de nombreux environnements.\,>>. L'analogie apparaît clairement dès l'énoncé du problème et à deux niveaux : les caractéristiques des entités étudiées sont les mêmes (autonomie et adaptation), et l'environnement dans lequel agissent ces entités est lui aussi le même (le monde physique)\footnote{Il y a toute de même une lacune difficilement comblée par la Robotique \'Evolutionnaire : l'évolution de la morphologie. En effet nous ne traiterons ici que de l'autonomie et de l'adaptation \emph{comportementale} des robots. Les seuls buts à atteindre par évolution artificielle sont cette autonomie et cette adaptation comportementale. Or chez les êtres vivants la morphologie de ceux-ci évolue aussi. Nous reviendrons sur cette limitation, qui est souvent citée par les chercheurs comme un des grands défis de la discipline, dans la section \ref{ch:RE}.}.  

Une fois fait le constat de cette proximité entre les problèmes que veulent résoudre les roboticiens via la Robotique \'Evolutionnaire et les mécanismes qu'essayent de comprendre les biologistes de l'évolution, ne serait-il pas intéressant de voir dans quelle mesure les résultats obtenus par les uns pourraient être utiles aux autres~? L'affirmation d'un chercheur en Robotique \'Evolutionnaire démontrant que <<\,pour obtenir des agents autonomes et adaptés à l'environnement réel il faut X\,>> peut-elle être transcrite en <<\,pour obtenir des être vivant adaptés à l'environnement il faut aussi X\,>>~?

En effet, si comme l'écrit \citet{eiben03introductiontoevolutionarycomputing} en conclusion de l'ouvrage de référence en AE :
\begin{quote}
   [\ldots] c'est avec la Robotique \'Evolutionnaire que les ingénieurs humains et les scientifiques s'approchent le plus près de l'évolution naturelle.\\
   \citep[p. 264]{eiben03introductiontoevolutionarycomputing}
\end{quote}
ne pourrait-on pas revoir la nature du lien qui connecte Biologie Théorique et Robotique \'Evolutionnaire pour que cette dernière devienne non seulement \emph{méthode inspirée par} mais aussi \emph{méthode inspirante pour} l'étude de l'évolution naturelle~? En d'autres termes, si les mécanismes à l'\oe uvre dans les expériences de Robotique \'Evolutionnaire sont proches des mécanismes à l'\oe uvre dans le monde du vivant, ces expériences ne pourraient-elle pas servir de modèle pour comprendre et expliquer l'évolution telle qu'elle se passe dans la nature~?

Répondre à cette question n'est pas chose aisée quand on s'y penche avec attention. Avant d'affirmer que la RE peut servir d'outil (ou de <<\,modèle\,>>) pour mieux comprendre l'évolution, d'abord faut-il être certain de ce que c'est que <<\,comprendre l'évolution\,>>. Qu'est-ce que cette théorie, qu'implique son utilisation, comment peut-on l'étudier, quelles approches et quels outils permettent de le faire~?

Dans cette optique, et afin d'être bien certain du contenu scientifique et des problèmes que l'on souhaite résoudre en étudiant la théorie de l'évolution, il nous parait primordial de commencer ce mémoire en retraçant les grandes lignes de l'histoire de cette théorie depuis sa formulation par Darwin. En reprenant grossièrement l'approche de \citep{gayon1991darwinetlapresdarwin}, nous décrirons la structure de cette théorie, les problèmes que cette structure a soulevé et les corrections qui ont été apportées.

Nous nous attarderons ensuite sur certaines visions plus actuelles de cette théorie et nous présenterons un des débats qui agite toujours la biologie de l'évolution : \emph{les unités et niveaux de sélection}. Nous verrons (sans être exhaustif) certaines positions prises par les protagonistes de ce débat puis terminerons cette première partie en décrivant plus en détails l'approche des \emph{populations darwinienne} de \cite{godfrey2009darwinian} qui nous semble adéquate pour réfléchir à ce débat et pour lier les études biologiques avec l'outil que nous voulons présenter, à savoir : la Robotique \'Evolutionnaire.

Dans un second temps nous présenterons des outils et méthodes qui ont permis et permettent toujours d'étudier et de réfléchir à la théorie de l'évolution. Nous commencerons par décrire rapidement la sélection artificielle et les expériences de pensée, pour ensuite prendre le temps de parler de l'utilisation des modèles en science. Nous traiterons en particulier les simulations informatiques et insisterons sur leur utilisation qu'en on fait les chercheurs en VA et sur l'approche qu'ils ont de ces simulations. 

Nous décrirons ensuite la Robotique \'Evolutionnaire dans le détail, les principes qu'elle mobilise et les obstacles qu'elle rencontre. Nous verrons quelques unes des voies récentes suivies par les chercheurs du domaine en insistant sur les approches dites <<\,Embarquées\,>> et <<\,Open-Ended\,>>, qui nous semblent les plus prometteuses et les meilleures candidates pour l'étude de la biologie.

Nous terminerons en essayant de faire le lien entre les différentes notions et disciplines abordées dans ce mémoire, pour clarifier la position de le Robotique \'Evolutionnaire. Notre but est de justifier son utilisation en tant que modèle d'étude de la théorie de l'évolution mais aussi pour réaffirmer l'inspiration biologique de la RE en insistant sur la nécessité de toujours réévaluer les concepts biologiques à la lumière des réflexions et avancées proposées par les philosophes et chercheurs dans ce domaine, afin que ces concepts soient appliqués aux mieux et apportent un maximum à la RE.

Ce mémoire veut intéresser aussi bien le biologiste avide d'approfondir sa connaissance de la théorie de l'évolution que le philosophe des sciences curieux d'étudier de nouvelles approches de la biologie de l'évolution ou encore le roboticien qui souhaite élargir ses recherches aux sciences du vivant.
