
\chapter*{Conclusion}

\lettrine[lines=2]{L}{'objectif} principal de ce travail a été de légitimer l'utilisation de la Robotique \'Evolutionnaire comme modèle pour étudier la théorie de l'évolution. Si nous espérons avoir avancé dans cette direction, la diversité et l'extrême transversalité des thèmes abordés rendent difficile l'analyse en profondeur de cette relation entre théorie biologique et modèles robotiques, et dans se contexte, légitimer scientifiquement et philosophiquement la capacité de la RE à apporter de nouvelles connaissance en biologie est une entreprise de longue haleine. Mener à bien cette justification prendrait de nombreuses analyses comparatives entre les recherches en robotique et en biologie, il faudrait probablement étudier les fondements philosophique de la relation de façon plus approfondie, complète et détaillé, voir pourquoi pas entreprendre des expériences empiriques parallèles pour tester certaines hypothèses avancées, et cela dépasse de loin le cadre de ce mémoire. 

Plutôt que d'entreprendre cette justification, nous avons préféré dresser un panorama global des domaines que cette approche mobilise et proposer une introduction, <<\,en largeur\,>>, pour qui s'intéresse à ces questions. L'idée est de fournir au chercheur, qu'il soit philosophe, informaticien ou biologiste, les bases pour comprendre l'ensemble du problème en reprenant les concepts centraux dans toutes les disciplines impliquées. Nous avons néanmoins essayé de mettre en avant certaines interrogations qui nous semblent importantes ainsi que les pistes pour y répondre qui nous paraissent les plus pertinentes.

Dans cette optique d'introduire, <<\,en largeur\,>>, l'utilisation de modèles artificiels pour étudier la biologie, nous avons commencé par décrire l'élément central de la question : la théorie de l'évolution. Le but de cette description est de fournir les clefs (historique, conceptuels) pour comprendre \emph{pourquoi} la théorie de l'évolution est telle qu'elle est, en essayant de savoir dans quels buts elle a été conçue, pour répondre à quelles questions, etc.. Ces interrogations nous poussent aussi à comprendre comment cette théorie a été modifiée, transformée et repensée pour s'adapter aux incessantes découvertes de la biologie.  

Cette approche est nécessaire et idéale pour être certain de savoir ce que nous manipulons aujourd'hui lorsque nous parlons d'étudier ou re-utiliser (dans le cadre des algorithmes génétiques, par exemple) cette théorie. \cite{gayon1991darwinetlapresdarwin} a très bien fait ce travail dans son livre <<\,Darwin et l'après Darwin\,>> et nous avons fait de notre mieux pour le reprendre et l'adapter à nos besoins. Il est intéressant de voir comment tout au long de son histoire la théorie a été sujette aux conceptions de ceux qui l'ont étudiée et des époques qu'elle a traversé. Nous avons vu comment les biométriciens l'ont drapée de statistiques, comment les physiologistes mendéliens l'ont poussée dans ses retranchements, où encore, comment la Synthèse Moderne a réussi à réunir les différents domaines de la biologie sous l'enseigne de cette théorie. Réaliser cette perméabilité de la théorie de l'évolution nous semble tout à fait nécessaire pour quiconque souhaite l'étudier, qu'importe l'approche et les moyens qu'il choisit d'utiliser. Repenser la théorie de l'évolution à la lumière de son histoire et des débats et controverses qui l'animent permet selon nous de justifier mieux encore l'utilisation de modèles artificiels pour l'étudier.

Dans un second temps, nous avons présenté des méthodes utilisées pour étudier cette théorie. Nous avons montré qu'il en existait un certain nombre, chacune ayant ses avantages et ses inconvénients. Nous avons parlé d'expériences de pensée, de sélection artificielle, d'analogie, d'évolution dirigée, le tout dans le but premier de montrer que les modèles artificiels et les simulations informatiques ne diffèrent pas tant des approches plus traditionnelles, qu'elles en partagent bon nombre des caractéristiques tout en s'affranchissant de certaines limites imposées par ces méthodes classiques. Dans cette partie nous avons aussi présenté un cadre épistémologique plus général: l'approche sémantique des théories. Cette approche nous paraît plus adéquate pour appréhender la théorie de l'évolution et les façons dont il est possible d'acquérir de nouvelles connaissances à propos de cette théorie. Elle offre une vision du savoir scientifique qui s'accorde avec l'utilisation de modèles de robotique évolutionnaire (et de modèles artificiels en général) pour étudier la biologie (et la théorie de l'évolution en particulier).

Pour terminer nous avons présenté la Robotique \'Evolutionnaire. Notre objectif a été d'offrir une vue d'ensemble du domaine pour permettre à des non roboticiens, à des philosophes ou biologistes, de comprendre mieux cette technique et d'avoir une idée du potentiel qu'elle offre lorsqu'elle se veut appliqué à l'étude de la théorie de l'évolution. \`A travers ce panorama général, nous  avons vu que la RE synthétise nombreux avantages d'autres méthode. Pour résumer : 

C'est un modèle artificiel, une simulation, mais plus proche de l'expérience classique que les simulations traditionnelle sur ordinateur car la relation entre l'objet d'étude (l'évolution des êtres vivant) et le modèle sur lequel sont faites les expériences (des robots) est bien plus étroite, les propriétés et caractéristiques des entités quasiment isomorphes. Là où les simulations sur ordinateur permettent l'étude de processus évolutivement \emph{possible}, la robotique réduit ces possibilités à des processus naturellement \emph{plus probables}, et se rapproche ainsi de la sélection artificielle chez les éleveurs qu'avait utilisé Darwin dans son analogie, ou de l'évolution dirigé de Lenski. Elle diminue les précautions à prendre et rend plus simple le transfert de connaissance du champ artificiel vers la biologie.

Et là réduction de ces précautions, contrairement aux expériences de Lenski où à la Sélection Artificielle pratiquée par les éleveurs, n'enlève pas la liberté qu'offrent les simulations comparées aux travaux menés sur les êtres vivants. La cassette d'une vie factice peut être rejouée mille fois, avec de multiples paramètres différentes à des vitesses différentes, tout en enregistrant chacune des pistes de la façon la plus détaillée possible, sans souffrir de lacunes historiques, ou de biais environnementaux. Ainsi, la Robotique \'Evolutionnaire offre au chercheur la même liberté qu'offre les expériences de pensée, et les simulations informatiques en générale. 

Il pourra donc étudier la biologie sans qu'il soit nécessaire pour lui de choisir \emph{une} approche en particulier parmi celles que nous avons vu dans la partie \ref{sec:pbm}. Au contraire, le roboticien sera à même construire des protocoles expérimentaux  pour tester et comparer ces différentes hypothèses. C'est ce qu'on fait par exemple \cite{waibel09geneticteamcompositionlevelselectionevolutioncooperation}, pour tester différentes hypothèses sur le niveaux de sélection possibles dans une étude dont nous n'avons pas parlé mais qui est selon nous une des voies à suivre.

C'est aussi dans cette optique que nous avons présenté l'approche de \cite{godfrey2009darwinian}. Par sa souplesse et son originalité elle nous semble la plus adaptée pour servir de cadre commun capable d'harmoniser les recherches faites en Biologie et en Robotique. En effet, combien d'études en Robotique \'Evolutionnaire essayent de parcourir exhaustivement certains paramètres de leur système artificiel comme par exemple : combien faut-il de générations pour que dans tel système, tel propriété évolue, quel taux de reproduction permet une convergence évolutive la plus rapide, quel degré de similarité entre les individus assure une coopération efficace\dots? Autant de questions qui peuvent être directement calquées sur des espaces mutli-dimensionnels ``façon pgs''. 

En unifiant la terminologie des biologistes et des informaticiens, le cadre de PGS permettrait aux premiers de mieux appréhender les résultats des roboticiens, en les positionnant dans un zone de l'espace de PGS particulière, qui correspondra, ou non, à une réalité biologique et que les biologistes pourront manipuler. \`A l'inverse, l'espace des populations darwiniennes, précisé par les recherches en biologie, pourra donner aux chercheur en robotique un agenda et une direction pour les paramètres à étudier afin de se déplacer dans l'espace de PGS et atteindre des zones ou les populations biologiques possèdent des propriétés qui les intéressent.

Et pour répondre aux critiques qui avanceront que faire des modèles pour étudier des caratéristiques précise de populations particulière n'apporte rien en science nous répondrons comme \cite{beatty1980ptimaldesignmodelsandstrategyofmodelbuildinginevolutionarybiology} que la biologie peut difficilement se passer de ces approches et que certaines visions de la science, telle celle de \cite{vanfraassen1972aformalapproachtothephilosophyofscience} s'en accomodent très bien.

Pour terminer cette étude nous voudrions revenir sur un point brièvement abordé mais sur lequel nous souhaitons insister. Si la direction donnée à ce mémoire a été de justifier et promouvoir l'utilisation de la Robotique \'Evolutionnaire comme \emph{modèle} de la théorie de l'évolution, nous pensons que cette dichotomie modèle/objet d'étude, n'est pas nécessaire. En réalité, étudier les processus évolutifs à l'\oe uvre sur un système robotique ou étudier les processus évolutif à l'\oe uvre sur un système biologique revient à étudier les mêmes processus. L'étude de l'un et l'étude de l'autre ne sont au final que l'étude de <<\,deux branches d'un même arbre\,>> \citep{huneman12computersciencemeetsevolutionarybiologypurepossibleprocessesissuegradualism}. Ainsi les deux peuvent s'enrichir mutuellement sans que le sens du transfert de connaissance n'ait à avoir de direction privilégiée.
