\begin{abstract}
	Nous voulons, à travers cette étude, justifier l'utilisation de la Robotique \'Evolutionnaire comme modèle pour étudier la théorie de l'évolution. Pour atteindre cet objectif nous reprendrons les principes généraux et l'histoire de la théorie darwinienne pour mieux comprendre sa structure, se pencher sur certaines difficultés qu'elle génère, et présenter certaines approches de la biologie actuelle comme celles issues de la synthèse moderne et les populations darwiniennes de Peter Godfrey-Smith. Dans un second temps nous décrirons des outils et approches utilisés pour étudier la biologie de l'évolution. Nous insisterons sur la pertinence de l'application de modèles et de simulations informatiques de ces modèles pour permettre cette étude du vivant et appuierons l'intérêt de cette approche avec la conception sémantique des théories scientifiques (de Bas van Fraassen, Frederick Suppe et d'autres). Nous montrerons que cette conception s'articule parfaitement avec les outils que nous souhaitons promouvoir. Ceci nous permettra d'introduire notre objet d'étude principal : la Robotique \'Evolutionnaire. Nous en décrirons les déclinaisons qui nous semblent pertinentes et nous montrerons qu'en tant qu'expérience de Vie Artificielle Embarquée, elle combine de nombreux atouts majeurs, qui font d'elle un modèle idéal pour étudier la théorie de l'évolution.
	\begin{center}
		\textbf{Abstract}
	\end{center}

	\noindent With this study we want to justify the use of Evolutionary Robotics as a model to study the theory of evolution. In order to reach this goal, we will explain the general principles and the history of the darwinan theory of evolution to better understand its structure, to look closely at some difficulties it raises and to present some approaches used to study current evolutionary biology ---as those borned from the Modern Sythesis and the darwinian populations approach of Peter Goddfrey-Smith. In a second time we will underline the pertinance of the application of models and simulations of those models to allow the study of life. To justify the usage of this approach we will present the semantic view of scientific theories (by Bas van Fraassen, Frederick Suppe and other), which makes perfect sens with the tools we want to promote. It will allow us to introduce the main object of our study : Evolutionary Robotics. We will describe some branches we think are accurate and we will show that, as an embodied artificial life experiment, ER combines numerous advantage	making it an ideal model to study the theory of evolution. 
\end{abstract}

