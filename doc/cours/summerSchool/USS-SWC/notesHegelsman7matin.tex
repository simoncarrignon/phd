\documentclass[a4paper]{article}


%%%%lualatex on
%\usepackage{luatextra}
\usepackage{fontspec}
%Ligatures={Contextual, Common, Historical, Rare, Discretionary}
%\setmainfont[Mapping=tex-text]{Linux Libertine O}

\usepackage{natbib}

\title{USS-SWC - 2015}
\author{Simon Carrignon}
\begin{document}
\section{Schelling}
Schelling model, seggregation blabla.

\section{Sokoda unkown model}
Well known in the world of origami


\subsection{Schelling And Sokoda}
Neighbor are weighted by there distant  (euclidian distance).

behavior encoded in an attitude matrix, for different matric off preferences difference results.
=> segregation. The same word? schelling is just an instance of the sokoda model


Schelling maximising? it depend on the utility function. It could be maximizing or only ``satisfyong.
Is it possible to describe schelling model with sokoda model? Yes if you modify sokoda

Behaviorly equivalent model. CCL: sokoda is ``better'' ?

\subsection{Sokoda and Minikoda}
Minidoka : japanese relocation center afeter WWII one of them was SAKODA he write book about that

demecroatizing the enemy (about all that relocation cener)
\subsection{The original writing of Sokoda (1949)}
Checkerboard conceptual model.

The first agent based model? with a real application.
\subsection{Programming language}
Dystal, a programming language for non professional programmer. Based on fortran.

But in the 60 - 70 Sokoda were know , shelling not

Because Schelling didn't use a computer. and warn against it, even if he used it. But at those time, no scren, so ``no view of the process''.
Sokoda's model was to complex to be run by hand. Moreover, sokoda was known in a small network of computer scientist.

A bit of Merton and the Mathew effect to explain Sodoka ``failure''

\section{Question}
What do you get? A feeling of what happen. Candidate of answer.

Being an autonomous agent?


Not a natural explanation. Different ideas.



What is understanding in a model?

How decide the power of the model? Again, ``getting a feeling'', a qualitative output build upon a quantitative tool








\end{document}
