\documentclass[a4paper]{article}


%%%%lualatex on
%\usepackage{luatextra}
\usepackage{fontspec}
%Ligatures={Contextual, Common, Historical, Rare, Discretionary}
%\setmainfont[Mapping=tex-text]{Linux Libertine O}

\usepackage{natbib}

\title{USS-SWC - 2015\\
K. Zollman}

\author{Simon Carrignon}
\begin{document}
\section{Model of communication network in science}
\section{Testing model with model : the case of game theory}
\begin{enumerate}
	\item 	find a phenomenon 
	\item 	build a model
	\item 	analyze the model
	\item 	test the model against the data
\end{enumerate}


\subsection{introduction to game theory}
A, NAsh equilibrium PD (not iterated)

N-player game Nash equilibrium = 0, divise par 2/3 a chaque fois jusqu'a tendre vers 0
Evolutionary GT? 
epistemic GT? 
\subsubsection{evolutionary GT}

different way Nat Selec, try and trial,, reinforcemen, diff imi, learning\ldots


Natural selection \& imitation
VS differential equation

go to Nash equilibrium? ESS?

For the IPD : it goes to nash .

RPScissors : evolutionary version cannot reach nash E using replication dynamics.


It is not the REAL WORLD as lot of false assumption.
\subsubsection{epistemic GT}

Logical language.

Rationalizable strategies (include nash equilibrium)


galilean de-dealization.

Galilean idealiwation : we make idealizationbecause we annot do better.

Galilean deidealization : more realistitc, but more idealization

Robustness testing : test the model under different condition, new idealization

Novel modeling programs :

Not the same model.


And what about data?


De-idealization bring new problem, new factor that impact the system, so more difficult to understang what happened. Is it a problem of the way biologist need to understand the system? or a problem that come from the way de idealization is done? de idealization should be ``empiracally guided''?





\end{document}

