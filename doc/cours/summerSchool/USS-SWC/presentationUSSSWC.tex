\documentclass[8pt, handout=show,notes=show]{beamer}

\usetheme[width=2cm]{Goettingen}
\usecolortheme{rose}


%%%%lualatex on
%\usepackage{luatextra}
\usepackage{fontspec}
\usepackage{amsmath}
%Ligatures={Contextual, Common, Historical, Rare, Discretionary}
%\setmainfont[Mapping=tex-text]{Linux Libertine O}

\usepackage{natbib}
\usepackage{mathptmx}
\usepackage{latexsym}
\usepackage{mathtools}


\DeclarePairedDelimiter\abs{\lvert}{\rvert}%
\DeclarePairedDelimiter\norm{\lVert}{\rVert}%


\makeatletter
\let\oldabs\abs
\def\abs{\@ifstar{\oldabs}{\oldabs*}}
\let\oldnorm\norm
\def\norm{\@ifstar{\oldnorm}{\oldnorm*}}
\makeatother

\title{USS-SWC -- 2015\\
	Presentation:\\
	ABM \& History
}

\date{Vienna, July 13, 2015}
\author{Simon Carrignon$^1$}
\institute[]{
	$^1$Barcelona~Supercomputing~Center \\

}
\begin{document}
\begin{frame}
	\maketitle

\end{frame}

\section{Monte Testaccio}
\begin{frame}{The Monte Testaccio}

	An amphora garbage in Roma.\\

	\begin{center}
		\includegraphics[height=0.3\textwidth]{./Mount-Testaccio.jpg}
		\hfil \includegraphics[height=0.3\textwidth]{./Mount-Testaccio2.jpg}\\
		\vfill
		\includegraphics[height=0.3\textwidth]{./titulus.png}

	\end{center}

\end{frame}
\begin{frame}{Data}

	About 47000 amphora from CEIPAC database and other data in other databases (places in Pleiade, Greek names in Oxford...)

	\begin{center}
		\includegraphics[width=.9\textwidth]{./fortGreekPlaceAndAmphora.png}
	\end{center}
	
\end{frame}

\section{Roman Economy}
\begin{frame}{Historical question}
	\begin{center}
		\Huge
		What was the nature of the Roman Economy?\\
	\end{center}
	\vfill
	\begin{block}
		{The primitivism/modern debate}
		The Roman Economy was already a free-market similar as today vs all price were fixed by the state, no free market, us of slave.
	\end{block}
\end{frame}
\section{Computer Model}
\begin{frame}{Computer Side, a Starting Point}

	An Agent Based Model mixing to main aspects (WSC -- 2015):

	\vfil
	\begin{enumerate}
		\item a	simple bargain mechanism,
		\item and (cultural) evolutionary dynamics.
	\end{enumerate}

	\vfill
	$\rightarrow$ Implement a ``simple'' theoretical abstract model, \emph{to be ``complexified''}.
\end{frame}



	
\begin{frame}{Bargain Mechanisms}
	\begin{block}{Bargaining}
		\begin{itemize}
			\item Agents have :
				\begin{itemize}
					\item Goods
					\item Value they attribute to goods
				\end{itemize}
			\item Agents produce 1 good and use it to exchange for the other goods, given the value they associate to each good.
			\item After the exchange, agents consume the goods and get a ``score'' (utility?) depending on the amount of good they gather and a scale of ``universal intrinsic value'' for each good.
		\end{itemize}

	\end{block}
\end{frame}

\begin{frame}{Evolutionary Dynamics}
	\begin{block}{Evolving}
		After 10 steps of exchange :
		\begin{itemize}
			\item  The less successful (in term of utility) agents copy the set of value of the most successful agent (Biased-Copy/selection).
			\item Given a probability $\mu$ the value attributed to some goods are modified (Innovation/Mutation)
		\end{itemize}

	\end{block}
\end{frame}

%	\begin{center}
%		\includegraphics[width=1.5\textwidth]{simsoc-stage1-small}
%	\end{center}


\begin{frame}{Parameter Exploration \&  Epistemic Opacity}
	Illustrate the opacity :
		\begin{itemize}
			\item 	\alert{One simulation : 57min}
			\item	100 simulations (statistical need) : 5700min $\approx$ 4 days
		\end{itemize}

	Lets try with :
	\begin{itemize}
		\item 	10 different probability exchange right. (0.001 to 0.20)
		\item 	3 size of population (250 , 500 , 1000)
		\item 	And different number of goods : (3, 6 , 9)
	\end{itemize}

	\begin{center}
		$= 10 \times 3 \times 3 = 90 $ ``environments'' (experimental setups).\\
		\vfill
		$\rightarrow$ \alert{360 days of continuous simulations.}


	\end{center}

\end{frame}


\section{Price Equilibrium}
\begin{frame}{Price Equilibrium}

	\begin{block}{Result for 3 goods and 500 agents}
		Without surprise, the system evolves toward an equilibrium where all agents adopt optimal prices (clearing-market prices). 
	\end{block}
	\begin{columns}
		\column{.5\textwidth}
		\includegraphics[height=\textwidth]{./ClearingPriceDistanceEvolutionForTrade-G3N500.pdf}
		\column{.35\textwidth}
		\includegraphics[width=\textwidth]{./scoreEx1.png}\\
		 \includegraphics[width=\textwidth]{./scoreEx2.png}
	\end{columns}
	
\end{frame}



\begin{frame}{Underlying code}
	\begin{center}
		\includegraphics[width=1\textwidth]{./codePrices.png}\\
	\end{center}
\end{frame}

\begin{frame}{Let change that}
	\begin{center}

		\includegraphics[width=1\textwidth]{./codeNeeds.png}\\
	\end{center}
\end{frame}
\begin{frame}{Let change that}
	\begin{center}

		\includegraphics[width=1\textwidth]{./codeNeeds.png}\\
		\includegraphics[width=.5\textwidth]{./NonEquilibrium.png}
	\end{center}
\end{frame}


\begin{frame}{Let change that}

	\begin{columns}
		\column{.5\textwidth}
		\includegraphics[height=\textwidth]{./NonEquilibrium.png}
		\column{.35\textwidth}
		\includegraphics[width=\textwidth]{./scoreEx1b.png}\\
		\includegraphics[width=\textwidth]{./scoreEx2b.png}
	\end{columns}
	
\end{frame}

\begin{frame}{Back To Rome}
	What does all that mean?\\

	Epistemological uncertainty\dots
	\vfill
	\begin{center}
		\Huge
		What was the nature of the Roman Economy?\\
	\end{center}
	\vfill
	
	De-idealization needed, yes, but how?
	\begin{itemize}
		\item A ``guided'' de-idealization? 
	\end{itemize}
	
\end{frame}

\begin{frame}
	\begin{center}
	Thanks for you attention.\\
	\vfil

		\includegraphics[width=.6\textwidth]{./bsc.jpeg}
	\end{center}

\end{frame}

\begin{frame}{The fitness/utility/consumption function}
	\begin{equation}\label{eq:score}
		s^i_j = \begin{cases}
			 s_{max}=1 & \text{if $q^i_j = n_j$}\\
			 1 -\dfrac{\abs{q^i_j - n_j}}{ \sqrt{\abs{(q^i_j)^2-(n_j)^2}}} & \text{if $q^i_j \neq   n_j$}
		 \end{cases}
	 \end{equation}


	\begin{figure}[htp]
		\begin{center}
			\includegraphics[width=.6\textwidth]{fitness.pdf}
		\end{center}
		\caption{The utility for different value in the ``universal scale''}
		\label{fig:fit}
	\end{figure}
	
\end{frame}
\end{document}


