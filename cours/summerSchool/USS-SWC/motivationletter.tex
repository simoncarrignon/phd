\documentclass[a4paper]{lettre}


%%%%lualatex on
%\usepackage{luatextra}
%\usepackage{fontspec}
%Ligatures={Contextual, Common, Historical, Rare, Discretionary}
%\setmainfont[Mapping=tex-text]{Linux Libertine O}

\title{USS-SWC Motivations}
\begin{document}
Simon Carrignon
\\mail: simon.carrignon@bsc.es
\\tel: +33~675~932~066


\vfill

object: previous works \& motivation.

\vspace{1.5cm}
	Dears,



Throughout my university studies, from the very beginning of my time at the university continuing through to my actual PhD, I always managed to follow an interdisciplinary path, mixing computer sciences with evolutionary biology, cognitive sciences and more recently history and philosophy of science.

Even before entering the university my interests in those fields were strong.  I was so impressed by the ability of numeric artifacts built by humans to exhibit ‘life-like’ or ‘human-like’ behaviors that I became determined to find a way to understand artificial artifacts and living beings. Since then all the courses I chose, all the schools I decided to attended, and all other choices I made were in pursuit of obtaining a better understanding of life as well as computers, and the link between them. 

I did a first Master Degree in Natural \& Artificial Cognition, in which I wrote a thesis where I study the evolution of specialisation process in swarm of autonomous agent trying to link it with biological process of speciation \& specialisation.

But there I understood that I would need more than scientific knowledge to deeply understand the link between the process I want to explain and the tools I used to do it. And if I want myself to be able to produced meaningful, useful and concrete research projects I would need a really deep understanding of such a link. That's why I decided to engage myself in another Master Degree in History and Philosophy of Science where I focused on evolutionary processes and their study using computer sciences. It allows me to dig deeper in the philosophical debates that occur throughout the history of evolutionary and to learn the history of the building of the field itself since Darwin.

Now I have found a Ph.D. and joint a team who wants to couple computer simulation and archaeological data to test and validate (or not) historical hypothesis on the evolution of cultural and economical network during the Roman Empire. I will be part of the people who have to develop the computer model. So the subject is at the exact crossing between historical and evolutionary process, human behavior, economy and agent based modelling. It seems to me that more than ever I will need a strong epistemological background in order to do something meaningful. Trying to build computer model of human activity is something that one should take really carefully if she want to do so, then trying to model such activities in a historical context is something again more fastidious.

This summer school would be perfect for me to go further into my reflexion on the link between computer simulation and cognitive sciences, but moreover and given that I'm still in the subject-building stage of my PhD it could be an unexpected and invaluable occasion to help me build this subject following the best, epistemologically accurate way. 


\vspace{1.5cm}

Simon Carrignon




\end{document}



