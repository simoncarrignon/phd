%        File: 20140211_SchichLake.tex
%     Created: lun. févr. 09 10:00  2015 C
% Last Change: lun. févr. 09 10:00  2015 C
%
\documentclass[a4paper]{article}


%%%%lualatex on
%\usepackage{luatextra}
\usepackage{fontspec}
%Ligatures={Contextual, Common, Historical, Rare, Discretionary}
%\setmainfont[Mapping=tex-text]{Linux Libertine O}

\usepackage{natbib}

\title{EpNet Day lectures}
\author{Simon Carrignon}
\date{5-3-2015}
\begin{document}
\maketitle

\cite{enquist2011modellingtheevolutionanddiversityofcumulativeculture} present a model of cultural evolution where they put the emphasis on emergence of what they call ``limitless appearance of new and increasingly complex cultural elements''. The core idea of their model is that cultural elements interact altogether and impact the appearance or disappearance of new cultural artefact. 


\paragraph{Network:\\}
Those networks describe a set of cultural facts in which combinations of artefacts can enhance or inhibit the apparition of sub-elements of the network. For the authors those interaction networks can explain why societies can grow complex cultural system but also can grow \emph{different} cultural systems. By complex the authors means ``with a huge amount of elements''. 


\paragraph{Complexity growth problem:\\}
This bring us again to the definition of what are ``cultural elements''. Because if, for instance and for sure, technological elements could grow in complexity\footnote{Even here knowing if ``the number of different elements'' is a good indicator the complexity remains an open question} a huge part of what one could describe as cultural element doesn't seem to be subject to this kind of ``complexity growing'' effect (all artistic artefact for instance). 

Anyway, even taking account of that, the idea behind these interaction networks remains a good one, indeed, even if, let say, Music does not grow in complexity but \emph{change} in a more ``randomly-driven-and-neutral'' way, one can expect that a given kind of music or trend of music will need a set of other pre-existing music or trends to emerge.

\paragraph{From Bayesian approach\ldots:\\}
This kind of network seems really close to Bayesian network used in traditional IA. Given a set of \emph{a priori probabilities} one could infer how the culture will evolve and which kind of elements will emerge or which not. This proximity, even if not explicitly exposed in the article, could nonetheless be a good thing given all tools already developed to study these objects. Those are also widely used in cognitive sciences and AI in general. In fact, lot of studies try to recreate this kind of ``creative behavior'' using probabilistic (or not) inference to model creativity and/or to solve computational problems (see for instance \cite{pereira2007creativity}).

\paragraph{\ldots to cognitive mechanisms:\\}
And even if this particular instance of their approach lacks the individually based focus needed to implement these cognitives abilities, I think that it bring back what \cite{pereira2002conceptual} call the ``holy quest for creativity'' which seems to me a good way to bring back interesting cognitive elements into the study of cultural evolution. It seems an even more interesting shift as this quest of creativity is an already widely studied question in Artificial Intelligence, meanings that lot of tools are already available that will include more or less cognitively accurate elements that could fit well a study of historical phenomena.

\paragraph{\emph{A priori} Knowledge of Belief State problem:\\}
However, this approach raise a difficult question not answered by the author. In these Bayesian network the probabilities and the relationship between the cultural elements have to be known \emph{a priori}. If for some cultural elements the link are easily drawn (you need to know how to build a wheel before create a bike), for some other the link could remain unknowable even if existing. 

Moreover, in an historical perspective, it could be really difficult even impossible to recreate the belief state of an epoch, and so on for the initial network. 

But even if the authors don't discuss this questions it should not prevent one to use this method to test hypothesis and challenge such model (even using scarce or even randomly generated informations) to alternative cultural evolution hypothesis built using alternative methods.



\bibliographystyle{apalike}
\bibliography{/home/scarrign/Documents/biblio/bib/phd.bib}  
\end{document}

